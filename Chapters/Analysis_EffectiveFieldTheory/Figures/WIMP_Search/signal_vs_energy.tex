\begin{figure}[!htbp]%
\centering
    \begin{tikzpicture}
    \centering
        \begin{groupplot}[view={0}{90},
            group style = {group size = 2 by 1,
            horizontal sep=0.6cm}]
            \nextgroupplot[
            width=0.48\textwidth, height=8cm,
            xlabel={Recoil Energy [keV$_{nr}$]},
            ylabel={S1$_{c}$ [phd]},
            mark size=0pt,
            xmin=0, xmax=100,
            ymin=0, ymax=150]

            \addplot[yellow, name path = psig2] table[x=energy, y=psig2]
                      {Data/HENR/sr1_ws/Signal/data_cuts/s1_vs_recoil.dat};
            \addplot[yellow, name path = nsig2] table[x=energy, y=nsig2]
                      {Data/HENR/sr1_ws/Signal/data_cuts/s1_vs_recoil.dat};
            
            \addplot[green, name path = psig1] table[x=energy, y=psig1]
                      {Data/HENR/sr1_ws/Signal/data_cuts/s1_vs_recoil.dat};
            \addplot[green, name path = nsig1] table[x=energy, y=nsig1]
                      {Data/HENR/sr1_ws/Signal/data_cuts/s1_vs_recoil.dat};
                      
            \addplot[yellow, forget plot] fill between[of=nsig2 and psig2]; 
            \addplot[green, forget plot] fill between[of=nsig1 and psig1];
            
            \addplot[black] table[x=energy, y=mean]
                    {Data/HENR/sr1_ws/Signal/data_cuts/s1_vs_recoil.dat};
            
            \addplot[blue, dashed] coordinates { (0,80)  (100,80)};
            \addplot[black, dashed] coordinates { (72,0)  (72,200)};
                  
            \nextgroupplot[
            width=0.48\textwidth, height=8cm,
            xlabel={Recoil Energy [keV$_{nr}$]},
            ylabel={log$_{10}$(S2$_{c}$ [phd])},
            yticklabel pos=right,
            mark size=0pt,
            xmin=0, xmax=100,
            ymin=2.5, ymax=4.5]
            
            \addplot[yellow, name path = psig2] table[x=energy, y=psig2]
                      {Data/HENR/sr1_ws/Signal/data_cuts/logs2_vs_recoil.dat};
            \addplot[yellow, name path = nsig2] table[x=energy, y=nsig2]
                      {Data/HENR/sr1_ws/Signal/data_cuts/logs2_vs_recoil.dat};
            \addplot[yellow, forget plot] fill between[of=nsig2 and psig2];          
            
            \addplot[green, name path = psig1] table[x=energy, y=psig1]
                      {Data/HENR/sr1_ws/Signal/data_cuts/logs2_vs_recoil.dat};
            \addplot[green, name path = nsig1] table[x=energy, y=nsig1]
                      {Data/HENR/sr1_ws/Signal/data_cuts/logs2_vs_recoil.dat};
            \addplot[green, forget plot] fill between[of=nsig1 and psig1];
            
            \addplot[black] table[x=energy, y=mean]
                    {Data/HENR/sr1_ws/Signal/data_cuts/logs2_vs_recoil.dat};
                    
            \addplot[black, dashed] coordinates { (72,2.5)  (72,4.5)};
        
        \end{groupplot}
    \end{tikzpicture}
    \caption{Detector response in S1 (\textbf{Left}) and S2 (\textbf{Right}) space for a given recoil in the LZ detector for SR1 detector parameters.
    The black line is the mean response, the green band is 1$\sigma$ and the yellow is 3$\sigma$. The blue dashed line is the S1$_c$ cut. The dashed black lines are the largest recoil energy signal within 3$\sigma$ of that cut.
    }
    \label{fig:sr1_detector_model_response_for_flat_nr}
\end{figure}