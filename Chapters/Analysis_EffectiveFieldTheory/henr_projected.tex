\section{Projected Sensitivity}
\par
As was seen in \autoref{fig:HENR_recoil_spectra_m1000}, the response from EFT operators extends significantly above the region of interest that the SI WIMP-nucleon interaction exists in.
In this section the sensitivity to these responses is set for the planned full exposure of LZ; 1000 live-days with 5.6 tonne of xenon.
The approach adopted in this section is analogous to \cite{LZ_projected_sensitivity_paper_ref}, but with with an extended region of interest.
\par
The projected detector parameters are shown in \autoref{tab:projected_sensitivity_detector_parameters}.
The extended region of interest was set such that the it was below the end point of any calibration of NEST ($\backsim$ 300 keV from AmBe).
As such it was defined as S1$_c$ [3, 500].
This was determined from simulations of a flat NR background, which is shown in \autoref{fig:projected_detector_model_response_for_flat_nr}.
With the projected detector parameters, an S1$_c$ of 500 phd will be from pulses around 250 keV, but the largest recoil (within 3 $\sigma$) is from 278 keV recoils.

\begin{table}[]
    \centering
    \begin{tabular}{c|c}
        Parameter   & Value  \\ \hline
        $g_{1}$     & 0.119 \\
        $g_{2}$     & 79.1  \\
        Drift field & 310 Vcm$^{-1}$ \\
        electron lifetime & 850 $\mu$s
    \end{tabular}
    \caption{Key detector parameters for the LXe-TPC parameters in the projected detector performance case.}
    \label{tab:projected_sensitivity_detector_parameters}
\end{table}

\begin{figure}[!htbp]%
\centering
    \begin{tikzpicture}
    \centering
        \begin{groupplot}[view={0}{90},
            group style = {group size = 2 by 1,
            horizontal sep=0.6cm}]
            \nextgroupplot[
            width=0.48\textwidth, height=8cm,
            xlabel={Recoil Energy [keV$_{NR}$]},
            ylabel={S1$_{c}$ [phd]},
            mark size=0pt,
            xmin=0, xmax=300,
            ymin=0, ymax=600]

            \addplot[yellow, name path = psig2] table[x=energy, y=psig2]
                      {Data/HENR/Projected_Sensitivity/data_cuts/s1_vs_recoil.dat};
            \addplot[yellow, name path = nsig2] table[x=energy, y=nsig2]
                      {Data/HENR/Projected_Sensitivity/data_cuts/s1_vs_recoil.dat};
            
            \addplot[green, name path = psig1] table[x=energy, y=psig1]
                      {Data/HENR/Projected_Sensitivity/data_cuts/s1_vs_recoil.dat};
            \addplot[green, name path = nsig1] table[x=energy, y=nsig1]
                      {Data/HENR/Projected_Sensitivity/data_cuts/s1_vs_recoil.dat};
                      
            \addplot[yellow, forget plot] fill between[of=nsig2 and psig2]; 
            \addplot[green, forget plot] fill between[of=nsig1 and psig1];
            
            \addplot[black] table[x=energy, y=mean]
                    {Data/HENR/Projected_Sensitivity/data_cuts/s1_vs_recoil.dat};
            
            \addplot[blue, dashed] coordinates { (0,500)  (325,500)};
            \addplot[black, dashed] coordinates { (278,0)  (278,700)};
                  
            \nextgroupplot[
            width=0.48\textwidth, height=8cm,
            xlabel={Recoil Energy [keV$_{NR}$]},
            ylabel={log$_{10}$(S2$_{c}$ [phd])},
            yticklabel pos=right,
            mark size=0pt,
            xmin=0, xmax=300,
            ymin=2.0, ymax=5.0]
            
            \addplot[yellow, name path = psig2] table[x=energy, y=psig2]
                      {Data/HENR/Projected_Sensitivity/data_cuts/logs2_vs_recoil.dat};
            \addplot[yellow, name path = nsig2] table[x=energy, y=nsig2]
                      {Data/HENR/Projected_Sensitivity/data_cuts/logs2_vs_recoil.dat};
            \addplot[yellow, forget plot] fill between[of=nsig2 and psig2];          
            
            \addplot[green, name path = psig1] table[x=energy, y=psig1]
                      {Data/HENR/Projected_Sensitivity/data_cuts/logs2_vs_recoil.dat};
            \addplot[green, name path = nsig1] table[x=energy, y=nsig1]
                      {Data/HENR/Projected_Sensitivity/data_cuts/logs2_vs_recoil.dat};
            \addplot[green, forget plot] fill between[of=nsig1 and psig1];
            
            \addplot[black] table[x=energy, y=mean]
                    {Data/HENR/Projected_Sensitivity/data_cuts/logs2_vs_recoil.dat};
                    
            \addplot[black, dashed] coordinates { (278,2)  (278,5)};
        
        \end{groupplot}
    \end{tikzpicture}
    \caption{Detector response in S1 (\textbf{Left}) and S2 (\textbf{Right}) space for a given recoil in the LZ detector assuming the projected detector parameters.
             The values have been extrapolated from simulations of a flat NR spectrum.
    }
    \label{fig:projected_detector_model_response_for_flat_nr}
\end{figure}

\subsection{Backgrounds}
\par
The background model considered was made up of 11 components which represent the most significant contributors discussed in \autoref{sec:lz_detector}.
Contributions from ``Detector components", ``Surface contamination" and ``Environmental" sources are summed together, but kept separate as ER and NR components.
This is performed as generally speaking the shape contribution is similar\footnote{this is particularly true in low energy recoil searches (<30 keV), where they all appear as flat contributions.}.
Neutrinos are se


\par
One background that was not considered are $\gamma$-X events.
These were excluded because in order to effectively model them the positional information of events need to be considered, and we have limited the observable parameters to \{S1$_c$,log$_{10}$(S2$_c$)\}. 
When LUX included 5-dimensions to in their PLR for a RUN-4 EFT analysis the resultant PLR required $\approx$15,000 CPU-hours per mass point \cite{billyboxer_thesis_ref} and so it was desirable to avoid a similar situation here.
There are novel ways of over coming this such detailed in \cite{flamenest_ref} and \cite{lux_ml_plr_ref}, but these would have required a significant effort to intergrate into the LZ computing framework.
In future searches, LZ is planning on using a boosted decision tree machine learning technique that is based upon LUX Run-4 EFT analysis \cite{LUX_RUN4_EFT_2021}.





\begin{enumerate}
    \item \textbf{SS}: Select events which have only scattered once, ``single scatter". An event is a single scatter in the TPC if the energy-weighted standard deviation of the deposits is less than the detector resolution. This is taken to be $\sigma_r <$ 3.0 cm and $\sigma_z <$ 0.2 cm.
    \item \textbf{ROI}: Select events where the recoil energy is in the range expected from a WIMP scatter and set on S1$_c$, S2 (uncorrected). This is cut is dependent upon which model of dark matter we are using. Here S1$_c$ must be less than 500 phd and have at least a 3-fold coincidence in the TPC PMTs. S2 must be greater than 415, the value required for at least 5 emitted electrons. This electron requirement is to ensure that the S2 size is large enough for position reconstruction.
    \item \textbf{FID}: The inner volume, or fiducial volume of the TPC is taken, removing events near the edges. The FID is defined as a cylinder extended from the centre of the TPC to 4 cm from the TPC walls, 2 cm above the cathode grid, and 13 cm below the gate grid. This inner volume contains 5.6 tonnes of LXe, meaning that there is 1.4 tonnes of xenon used for self-shielding.
    \item \textbf{Veto}: TPC scatters where there is a time-coincident deposit in either of the veto detectors are removed. In the Skin detector the signal must be within 800 $\mu$s of the TPC scatter and be at least 3 phd in size. In the OD the deposit must be at least 200 keV in size and within 500 $\mu$s of the TPC scatter. This OD selection was chosen to maintain consistency with \cite{LZ_projected_sensitivity_paper_ref} with an older simulation framework.
\end{enumerate}
The events that are left are then defined as as a differential rate per recoil energy can then be calculated for each component.
Other sources of backgrounds which would only scatter once anyway do not need to be simulated in this way, instead the rate can be determined directly from the flux and scattering cross-section.
The spectral contributions from each of of the background components is shown in \autoref{fig:sensitivity_paper_backgrounds}.
The expected number of events from each of these components in the WIMP-SI search region of interest and in the extended region used here are shown in 
\autoref{tab:projected_lz_backgrounds}.
A simulated dataset of these backgrounds is shown in  \autoref{fig:sensitivity_paper_backgrounds}.

\begin{table}[]
    \centering
    \begin{tabular}{c|c|c|c}
        \multirow{2}{*}{Background}                  & \multicolumn{2}{|c|}{N}                          & \multirow{2}{*}{$\sigma$/N}  \\ 
                                                     &  (S1$_c <$ 80 phd)     & (S1$_c <$ 500 phd)      &              \\ \hline
        \textbf{ER contributions}                    &                        &                         &   \\
        Detector contaminants                        & 171                    & 1166                    & 20\% \cite{LZ_projected_sensitivity_paper_ref}        \\
        pp + ${}^{7}$Be + ${}^{13}$N solar neutrinos & 615                    & 2950                    & 2\% \cite{pp_solar_neutrinos_rate_ref}       \\
        ${}^{222}$Rn                                 & 1915                   & 12514                   & 10\% \cite{lz_predicted_radon_rate_ref}        \\
        ${}^{220}$Rn                                 & 316                    & 1902                    & 10\% \cite{lz_predicted_radon_rate_ref}        \\
        ${}^{136}$Xe 2$\mu\beta\beta$                & 495                    & 19183                   & 50\% \cite{double_beta_decay_rate_ref}        \\
        ${}^{85}$Kr                                  & 83                     & 557                     & 20\% \cite{kr85_rate_ref}         \\ \hline
        \textbf{NR contributions}                    &                        &                         &   \\
        Detector contaminants                        & 0.81                   & 1.91                    & 20\% \cite{LZ_projected_sensitivity_paper_ref}         \\
        ${}^{8}$B solar neutrinos                    & 36                     & 36                      & 4\%  \cite{b8_neutrino_rate_ref}       \\
        hep solar neutrinos                          & 0.9                    & 0.9                     & 15\% \cite{solar_neutrinos_rate_ref, pp_solar_neutrinos_rate_ref}        \\
        Diffuse supernova neutrinos                  & 0.15                   & 0.15                    & 50\% \cite{dissuse_supernova_neutrinos_rate_ref}        \\
        Atmospheric neutrons                         & 0.65                   & 0.65                    & 25\% \cite{atmospheric_neutrinos_rate_ref}      
    \end{tabular}
    \caption{Backgrounds considered in the PLR for a projected sensitivity to EFT operator coupling. The values for the WIMP-SI search region values differ from previous studies \cite{LZ_projected_sensitivity_paper_ref,LZ_Ibles_LZStats_Thesis_ref}. This is due to an upgraded NEST package version used here.}
    \label{tab:projected_lz_backgrounds}
\end{table}

\begin{figure}[!htbp]%
\centering
    \begin{tikzpicture}
    \centering
        \begin{groupplot}[view={0}{90},
            group style = {group size = 2 by 1}]
            \nextgroupplot[
            width=0.5\textwidth, height=8cm,
            xlabel=Recoil Energy (keV),
            ylabel=Differential Rate (kg/day/keV),
            legend pos=north east,
            mark size=0pt,
            xmin=0, xmax=2700,
            ymode=log]
                \addplot
                    table[x=Energy,y=Rate]
                    {Data/HENR/Projected_Sensitivity/Background_Rates/detector_er.dat};
      %          \addlegendentry{Det. + Sur. + Env.};
      %         \addplot
      %             table[x=Energy,y=Rate]
      %             {Data/HENR/Projected_Sensitivity/Background_Rates/Xe136.dat};
      %          \addlegendentry{${}^{136}$Xe}
      %         \addplot
      %             table[x=Energy,y=Rate]
      %             {Data/HENR/Projected_Sensitivity/Background_Rates/Rn222.dat};
      %          \addlegendentry{${}^{222}$Rn};
      %         \addplot
      %             table[x=Energy,y=Rate]
      %             {Data/HENR/Projected_Sensitivity/Background_Rates/Rn220.dat};
      %          \addlegendentry{${}^{220}$Rn};
      %         \addplot
      %             table[x=Energy,y=Rate]
      %             {Data/HENR/Projected_Sensitivity/Background_Rates/Solar.dat};
      %          \addlegendentry{Solar $\nu$};
      %         \addplot
      %             table[x=Energy,y=Rate]
      %                {Data/HENR/Projected_Sensitivity/Background_Rates/Kr85.dat};
      %          \addlegendentry{${}^{85}$Kr};
                  
            \nextgroupplot[
            width=0.5\textwidth, height=8cm,
            xlabel=Recoil Energy (keV),
            yticklabel pos=right,
            legend pos=north east,
            mark size=0pt,
            xmin=0, xmax=250,
            ymin=1e-12, ymax=,
            ymode=log]
                \addplot
                    table[x=Energy,y=Rate]
                    {Data/HENR/Projected_Sensitivity/Background_Rates/detector_nr.dat};
                \addlegendentry{Det. + Sur. + Env.};
                \addplot
                    table[x=Energy,y=Rate]
                    {Data/HENR/Projected_Sensitivity/Background_Rates/atm.dat};
                \addlegendentry{Atm};
                \addplot
                    table[x=Energy,y=Rate]
                    {Data/HENR/Projected_Sensitivity/Background_Rates/DSN_DiffRate.dat};
                \addlegendentry{DSN};
                \addplot
                    table[x=Energy,y=Rate]
                    {Data/HENR/Projected_Sensitivity/Background_Rates/hep.dat};
           %     \addlegendentry{hep};
           %     \addplot
           %         table[x=Energy,y=Rate]
           %         {Data/HENR/Projected_Sensitivity/Background_Rates/B8.dat};
           %     \addlegendentry{${}^{8}$B}
        
        \end{groupplot}
    \end{tikzpicture}
    \caption{Backgrounds considered in the projected sensitivity}
    \label{fig:sensitivity_paper_backgrounds}
\end{figure}

\begin{figure}
    \centering
    \includegraphics[width=15cm]{Figures/EFT/Projected_backgrounds/projected_backgrounds_s1_s2.png}
    \caption{Simulated data set for a background-only 1000 live day run with a 5.6 tonne fiducial mass. The ER and NR bands are shown in blue and red respectively; the solid lines are the mean and the dashed are 10\% and 90\% quantiles.}
    \label{fig:my_label}
\end{figure}



\subsection{Signal Model}
\par
As mentioned in \autoref{chap:detection_theory}, the parameters for the Standard Halo Model (SHM) used for generated signal models within direct dark matter experiments has been standardised since mid-2021 \cite{standard_halo_model_conventions_ref}.
However, this study began before that date and so used the parameters previously used within the LZ collaboration, in \cite{LZ_projected_sensitivity_paper_ref,LZ_TechnicalDesignReview_ref,LZ_Ibles_LZStats_Thesis_ref}.
The parameters used are shown in \autoref{tab:projected_DMFormFactor_parameters}.
One interesting benefit of this however is that the parameters are exactly the same as those used by both Xenon100 \cite{xenon100_eft_ref} LUX RUN-4 EFT analysis \cite{LUX_RUN4_EFT_2021}.

\begin{table}[]
    \centering
    \begin{tabular}{c|c}
        Parameter         & Value  \\ \hline
        $\nu_0$           & 220$km s^{-1}$ \\
        $\nu_{esc}$       & 544$km s^{-1}$ \\
        $\rho_{\chi}$     & 0.3 $GeV/cm^{3}$ \\
        $|\nu_E|$         & 245 $km s^{-1}$ 
    \end{tabular}
    \caption{Standard Halo Model parameters used for projected sensitivity study.}
    \label{tab:projected_DMFormFactor_parameters}
\end{table}

\par
With the projected detector parameters, the observable quantities \{$S1_c,log(S2_c)$\}, for a selection of operators at a mass of 100 GeV / c$^2$ are shown in \autoref{fig:projected_detector_model_signal_pdfs}.

\begin{figure}[!htbp]%
\centering
    \begin{tikzpicture}
    \centering
        \begin{groupplot}[view={0}{90},
            group style = {group size = 1 by 3,
            horizontal sep=0.6cm}]

        \nextgroupplot[
        width=15cm, height=8cm,
        xlabel={S1$_{c}$ [phd]},
        ylabel={log$_10$(S2$_{c}$ [phd])},
        mark size=0pt,
        xmin=0, xmax=500,
        ymin=2.5, ymax=5.5,
        colormap={blackwhite}{color=(white) color=(black)}]
        
        \addplot3[
              surf,
              shader=flat corner,
        	  mesh/cols=40,
        	  mesh/ordering=rowwise,
            ] file {Data/HENR/Projected_Sensitivity/Signal/pdf/o1_m100_pdf.dat};
            
        \addplot[blue, ]
            table [x=bin, y=mean]
            {Data/HENR/Projected_Sensitivity/Signal/pdf/er_band.dat};     
        \addplot[blue, dashed]
            table [x=bin, y=high]
            {Data/HENR/Projected_Sensitivity/Signal/pdf/er_band.dat};     
        \addplot[blue, dashed]
            table [x=bin, y=low]
            {Data/HENR/Projected_Sensitivity/Signal/pdf/er_band.dat};     

        \addplot[red, ]
            table [x=bin, y=mean]
            {Data/HENR/Projected_Sensitivity/Signal/pdf/nr_band.dat};    
        \addplot[red, dashed]
            table [x=bin, y=high]
            {Data/HENR/Projected_Sensitivity/Signal/pdf/nr_band.dat};     
        \addplot[red, dashed]
            table [x=bin, y=low]
            {Data/HENR/Projected_Sensitivity/Signal/pdf/nr_band.dat};  
            
        \nextgroupplot[
        width=15cm, height=8cm,
        xlabel={S1$_{c}$ [phd]},
        ylabel={log$_10$(S2$_{c}$ [phd])},
        mark size=0pt,
        xmin=0, xmax=500,
        ymin=2.5, ymax=5.5,
        colormap={blackwhite}{color=(white) color=(black)}]
        
        \addplot3[
              surf,
              shader=flat corner,
        	  mesh/cols=40,
        	  mesh/ordering=rowwise,
            ] file {Data/HENR/Projected_Sensitivity/Signal/pdf/o6_m100_pdf.dat};
            
        \addplot[blue, ]
            table [x=bin, y=mean]
            {Data/HENR/Projected_Sensitivity/Signal/pdf/er_band.dat};     
        \addplot[blue, dashed]
            table [x=bin, y=high]
            {Data/HENR/Projected_Sensitivity/Signal/pdf/er_band.dat};     
        \addplot[blue, dashed]
            table [x=bin, y=low]
            {Data/HENR/Projected_Sensitivity/Signal/pdf/er_band.dat};     

        \addplot[red, ]
            table [x=bin, y=mean]
            {Data/HENR/Projected_Sensitivity/Signal/pdf/nr_band.dat};    
        \addplot[red, dashed]
            table [x=bin, y=high]
            {Data/HENR/Projected_Sensitivity/Signal/pdf/nr_band.dat};     
        \addplot[red, dashed]
            table [x=bin, y=low]
            {Data/HENR/Projected_Sensitivity/Signal/pdf/nr_band.dat};  
            
        \nextgroupplot[
        width=15cm, height=8cm,
        xlabel={S1$_{c}$ [phd]},
        ylabel={log$_10$(S2$_{c}$ [phd])},
        mark size=0pt,
        xmin=0, xmax=500,
        ymin=2.5, ymax=5.5,
        colormap={blackwhite}{color=(white) color=(black)}]
        
        \addplot3[
              surf,
              shader=flat corner,
        	  mesh/cols=40,
        	  mesh/ordering=rowwise,
            ] file {Data/HENR/Projected_Sensitivity/Signal/pdf/o15_m100_pdf.dat};
            
        \addplot[blue, ]
            table [x=bin, y=mean]
            {Data/HENR/Projected_Sensitivity/Signal/pdf/er_band.dat};     
        \addplot[blue, dashed]
            table [x=bin, y=high]
            {Data/HENR/Projected_Sensitivity/Signal/pdf/er_band.dat};     
        \addplot[blue, dashed]
            table [x=bin, y=low]
            {Data/HENR/Projected_Sensitivity/Signal/pdf/er_band.dat};     

        \addplot[red, ]
            table [x=bin, y=mean]
            {Data/HENR/Projected_Sensitivity/Signal/pdf/nr_band.dat};    
        \addplot[red, dashed]
            table [x=bin, y=high]
            {Data/HENR/Projected_Sensitivity/Signal/pdf/nr_band.dat};     
        \addplot[red, dashed]
            table [x=bin, y=low]
            {Data/HENR/Projected_Sensitivity/Signal/pdf/nr_band.dat};  
        
        
        \end{groupplot}
    \end{tikzpicture}
    \caption{Detector response in S1 (\textbf{Left}) and S2 (\textbf{Right}) space for a given recoil in the LZ detector assuming the projected detector parameters.
             The values have been extrapolated from simulations of a flat NR spectrum 
    }
    \label{fig:projected_detector_model_signal_pdfs}
\end{figure}


\subsection{Limits}
\par
Presented in \autoref{fig:EFT_Result_Projected_Sensitivity_1} and \autoref{EFT_Result_Projected_Sensitivity_2} are the limits from a one-sided PLR test statistic.
This was chosen over using a two-sided test as the purpose is to determine the sensitivity of LZ to the couplings, $c^{a}_i$, not the discovery significance. 
This approach is also in line with that used for the SI and SD projected sensitivity studies \cite{LZ_projected_sensitivity_paper_ref} as well a other sensitivity studies within LZ \cite{LZ_Ibles_LZStats_Thesis_ref, umituktu_thesis_ref}.
Included in the figures are the XENON100 \cite{xenon100_eft_ref} and LUX \cite{LUX_RUN4_EFT_2021} limits.
Only a single data point is shown from LUX as their analysis was originally done in an \{$neutron,proton$\} basis, and given the huge quantity of work it took they decided to reduce the problem to a single WIMP mass per operator.
\par
The limits are presented in terms of the dimensionless values $({c}^{s}_{i}\times{m}^{2}_{w})^{2}$ where $m_w$ is the Higg's vacuum expectation value.
${c}^{s}_{i}$ has dimensionality of [mass]$^{-2}$, which originates from the decision of the authors of \textit{DMFormFactor} to normalise spinors to unity and to use dimensionless representations of the operators.
Another decision by the authors is to scale the couplings by $m^{-2}_w$.
The limits shown here are scaled by $m^2_w$ in order to be reported without dimensions.
Reporting results in this format mirrors the LUX and XENON100 approach, allowing for a direct comparison.
\par
The parameter space which the LZ detector is a significant step forward compared to current limits.
The planned LZ exposure is orders of magnitude greater than the both LUX and XENON100 results, with 5.6 tonnes $\times$ 1000 live days for LZ compared to 34 kg $\times$ 224.6 live days with XENON100 and 100 kg $\times$ 311.2 live days for LUX.
The level of $\gamma$-X events will be the primary hold back to any potential discovery in the EFT region, but assuming that they can be successfully modelled in a more complex PLR, LZ has a very high projected sensitivity to EFT signatures.

\begin{figure}[!htbp]%
\centering
\begin{tikzpicture}
\centering
  \begin{groupplot}[view={0}{90},
    group style = {group size = 2 by 4,
                   vertical sep=1.0cm,
                   horizontal sep=2.0cm}]
    
    \pgfplotsforeachungrouped \x in {1,3,4,5,6,7,8,9}{
     \edef\tmp{
        \noexpand \nextgroupplot[
                                xlabel={Mass [GeV/c$^2$]},
                                ylabel=$({c}^{s}_{\x}\times{m}^{2}_{w})^{2}$,
                                mark size=0pt,
                                width=0.45\textwidth,
                                height=5.5cm,
                                xmode=log,
                                ymode=log,
                                yminorticks=true,
                                x label style={at={(axis description cs:0.75,-0.1)},anchor=near ticklabel},
                                y label style={at={(axis description cs:-0.13,.75)},anchor=near ticklabel},
                                ]
            
            \noexpand \addplot[blue, name path = xenon100] table[]
                      {Data/HENR/Xenon100/O\x.dat};
                        
            \noexpand \addplot[only marks, mark size=1, error bars/.cd,
                               y dir=both, y explicit, error bar style={color=black}]
                               table[x=mass,y=median, y error plus index=3, y error minus index=2] {Data/HENR/Projected_Sensitivity/LUX/O\x.dat};
            
            \noexpand \addplot[green, name path = nsig1] table[x=mass, y=nsig1]
                      {Data/HENR/Projected_Sensitivity/Results_method1/O\x.dat};
                      
            \noexpand \addplot[green, name path = psig1] table[x=mass, y=psig1]
                      {Data/HENR/Projected_Sensitivity/Results_method1/O\x.dat};
                      
            \noexpand \addplot[yellow, name path = psig2] table[x=mass, y=psig2]
                      {Data/HENR/Projected_Sensitivity/Results_method1/O\x.dat};
                      
            \noexpand \addplot[green, forget plot] fill between[of=nsig1 and psig1];
            \noexpand \addplot[yellow, forget plot] fill between[of=psig1 and psig2];
            
            \noexpand \addplot[black, name path = median] table[x=mass, y=median]
                      {Data/HENR/Projected_Sensitivity/Results_method1/O\x.dat};
            
            %\noexpand \addplot[black, dashed, name path = median] table[x=mass, y=cl]
            %          {Data/HENR/Projected_Sensitivity/Results/O\x.dat};
                     
        }
        \tmp 
        }
  \end{groupplot}
\end{tikzpicture}
\caption{LZ projected sensitivity at the 90\% CL for isoscalar WIMP-nucleon couplings for relativistic EFT operators $\Operator_1$,$\Operator_3$-$\Operator_9$.
         The solid black line is the projected sensitivity from this analysis. 
         The green and yellow bands are the 1$\sigma$ and 2$\sigma$ respectively.
         The blue line is the previous limits set by XENON100 \cite{xenon100_eft_ref}.
         The single point is the limit set by LUX \cite{LUX_RUN4_EFT_2021}.}
\label{fig:EFT_Result_Projected_Sensitivity_1}
\end{figure}



\begin{figure}[!htbp]%
\centering
\begin{tikzpicture}
\centering
  \begin{groupplot}[view={0}{90},
    group style = {group size = 2 by 3,
                   vertical sep=1.0cm,
                   horizontal sep=2.0cm}]
    
    \pgfplotsforeachungrouped \x in {10,11,12,13,14,15}{
     \edef\tmp{
        \noexpand \nextgroupplot[
                                xlabel={Mass [GeV/c$^2$]},
                                ylabel=$({c}^{s}_{\x}\times{m}^{2}_{w})^{2}$,
                                mark size=0pt,
                                width=0.45\textwidth,
                                height=5.5cm,
                                xmode=log,
                                ymode=log,
                                x label style={at={(axis description cs:0.75,-0.1)},anchor=near ticklabel},
                                y label style={at={(axis description cs:-0.13,.75)},anchor=near ticklabel},
                                ]
            
            \noexpand \addplot[blue, name path = xenon100] table[]
                      {Data/HENR/Xenon100/O\x.dat};
            
            \noexpand \addplot[only marks, mark size=1, error bars/.cd,
                               y dir=both, y explicit, error bar style={color=black}]
                               table[x=mass,y=median, y error plus index=3, y error minus index=2] {Data/HENR/Projected_Sensitivity/LUX/O\x.dat};
                        
            \noexpand \addplot[green, name path = nsig1] table[x=mass, y=nsig1]
                      {Data/HENR/Projected_Sensitivity/Results_method1/O\x.dat};
                      
            \noexpand \addplot[green, name path = psig1] table[x=mass, y=psig1]
                      {Data/HENR/Projected_Sensitivity/Results_method1/O\x.dat};
                      
            \noexpand \addplot[yellow, name path = psig2] table[x=mass, y=psig2]
                      {Data/HENR/Projected_Sensitivity/Results_method1/O\x.dat};
                      
            \noexpand \addplot[green, forget plot] fill between[of=nsig1 and psig1];
            \noexpand \addplot[yellow, forget plot] fill between[of=psig1 and psig2];
            
            \noexpand \addplot[black, name path = median] table[x=mass, y=median]
                      {Data/HENR/Projected_Sensitivity/Results_method1/O\x.dat};
            
            %\noexpand \addplot[black, dashed, name path = median] table[x=mass, y=cl]
            %          {Data/HENR/Projected_Sensitivity/Results/O\x.dat};
                     
        }
        \tmp 
        }
  \end{groupplot}
\end{tikzpicture}
\caption{LZ projected sensitivity at the 90\% CL for isoscalar WIMP-nucleon couplings for relativistic EFT operators $\Operator_{10}$-$\Operator_{15}$.
         The solid black line is the projected sensitivity from this analysis. 
         The green and yellow bands are the 1$\sigma$ and 2$\sigma$ respectively.
         The blue line is the previous limits set by XENON100 \cite{xenon100_eft_ref}.
         The single point is the limit set by LUX \cite{LUX_RUN4_EFT_2021}.}
\label{fig:EFT_Result_Projected_Sensitivity_2}
\end{figure}
