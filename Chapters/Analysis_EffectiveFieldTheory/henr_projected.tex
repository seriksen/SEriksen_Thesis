\section{Projected Sensitivity}
\par
In this section, a sensitivity study is performed to ascertain the physics capability of the LZ detector to higher energy dark matter recoils.

\subsection{Data Selection}
\par
Cuts

\paragraph{Single Scatters}
Exclude events which scatter more than once

\paragraph{Fiducial Volume}
Self shielding blah blah

\paragraph{Vetos}
Self shielding blah blah




\subsection{Backgrounds}
\par

\begin{table}[]
    \centering
    \begin{tabular}{c|c|c|c}
        Background Component                         & Type    & Rate (S1$_c$< 80 phd)    & Uncertainty  \\ \hline
        Detector + Surface + Environmental           & ER      & 54                       & 20\% \cite{LZ_projected_sensitivity_paper_ref}        \\
        pp + ${}^{7}$Be + ${}^{13}$N solar neutrinos & ER      & 221                      & 2\% \cite{pp_solar_neutrinos_rate_ref}       \\
        ${}^{222}$Rn                                 & ER      & 700                      & 10\%         \\
        ${}^{220}$Rn                                 & ER      & 115                      & 10\%         \\
        ${}^{136}$Xe 2$\mu\beta\beta$                & ER      & 83                       & 50\% \cite{double_beta_decay_rate_ref}        \\
        ${}^{85}$Kr                                  & ER      & 30                       & 20\%          \\
        ${}^{39}$Ar                                  & ER      & 3                        & 50\%         \\
        Detector + Surface + Environmental           & NR      & 0.53                     & 20\% \cite{LZ_projected_sensitivity_paper_ref}         \\
        ${}^{8}$B solar neutrinos                    & NR      & 38                       & 4\%         \\
        hep solar neutrinos                          & NR      & 0.31                     & 15\%         \\
        Diffuse supernova neutrinos                  & NR      & 0.14                     & 50\%         \\
        Atmospheric neutrons                         & NR      & 0.21                     & 25\%       
    \end{tabular}
    \caption{Primary backgrounds that need to be considered for an EFT search}
    \label{tab:projected_lz_backgrounds}
\end{table}

\par
In addition to these backgrounds it is important to note several which have been omitted: namely $\gamma$-X and ${}^{83m}$Kr.
Also missing accidental and wall based.

\subsection{$\gamma$-X}
It is possible for light-to-charge ration of an S1-S2 pulse pair to be attenuated, causing an ER event to appear NR-like.
For this S2-suppression, a $\gamma$-ray needs to scatter in two separate regions of the TPC, one where ionisation electrons are detected, and one where they are not.
In this case, a single S1 will be detected (due to how close in time the scatters are), and a single S2.
There are two significant locations where these events can originate: in the reverse field region (RFR) of the TPC and the walls of the TPC.
\par
These events have not been included here due to the need to include additional parameters in order to adequately model them: namely the position of the event needs to be taken into account.
It should be noted that it is also possible to cut out these events using a tighter fiducial cut, but this directly impacts exposure.
In future searches, LZ is planning on using a boosted decision tree machine learning technique that is based upon LUX Run-4 EFT analysis \cite{LUX_RUN4_EFT_2021}.

\par
The resultant backgrounds considered are shown in FIGURE XXX.



\subsection{Limits}
\par

\begin{figure}[!htbp]%
\centering
\begin{tikzpicture}
\centering
  \begin{groupplot}[view={0}{90},
    group style = {group size = 2 by 4,
                   vertical sep=1.0cm,
                   horizontal sep=2.0cm}]
    
    \pgfplotsforeachungrouped \x in {1,3,4,5,6,7,8,9}{
     \edef\tmp{
        \noexpand \nextgroupplot[
                                xlabel={Mass [GeV/c$^2$]},
                                ylabel=$({c}^{s}_{\x}\times{m}^{2}_{w})^{2}$,
                                mark size=0pt,
                                width=0.45\textwidth,
                                height=5.5cm,
                                xmode=log,
                                ymode=log,
                                yminorticks=true,
                                x label style={at={(axis description cs:0.75,-0.1)},anchor=near ticklabel},
                                y label style={at={(axis description cs:-0.13,.75)},anchor=near ticklabel},
                                ]
            
            \noexpand \addplot[blue, name path = xenon100] table[]
                      {Data/HENR/Xenon100/O\x.dat};
                        
            \noexpand \addplot[only marks, mark size=1, error bars/.cd,
                               y dir=both, y explicit, error bar style={color=black}]
                               table[x=mass,y=median, y error plus index=3, y error minus index=2] {Data/HENR/Projected_Sensitivity/LUX/O\x.dat};
            
            \noexpand \addplot[green, name path = nsig1] table[x=mass, y=nsig1]
                      {Data/HENR/Projected_Sensitivity/Results_method1/O\x.dat};
                      
            \noexpand \addplot[green, name path = psig1] table[x=mass, y=psig1]
                      {Data/HENR/Projected_Sensitivity/Results_method1/O\x.dat};
                      
            \noexpand \addplot[yellow, name path = psig2] table[x=mass, y=psig2]
                      {Data/HENR/Projected_Sensitivity/Results_method1/O\x.dat};
                      
            \noexpand \addplot[green, forget plot] fill between[of=nsig1 and psig1];
            \noexpand \addplot[yellow, forget plot] fill between[of=psig1 and psig2];
            
            \noexpand \addplot[black, name path = median] table[x=mass, y=median]
                      {Data/HENR/Projected_Sensitivity/Results_method1/O\x.dat};
            
            %\noexpand \addplot[black, dashed, name path = median] table[x=mass, y=cl]
            %          {Data/HENR/Projected_Sensitivity/Results/O\x.dat};
                     
        }
        \tmp 
        }
  \end{groupplot}
\end{tikzpicture}
\caption{LZ projected sensitivity at the 90\% CL for isoscalar WIMP-nucleon couplings for relativistic EFT operators $\Operator_1$,$\Operator_3$-$\Operator_9$.
         The solid black line is the projected sensitivity from this analysis. 
         The green and yellow bands are the 1$\sigma$ and 2$\sigma$ respectively.
         The blue line is the previous limits set by XENON100 \cite{xenon100_eft_ref}.
         The single point is the limit set by LUX \cite{LUX_RUN4_EFT_2021}.}
\label{fig:EFT_Result_Projected_Sensitivity_1}
\end{figure}



\begin{figure}[!htbp]%
\centering
\begin{tikzpicture}
\centering
  \begin{groupplot}[view={0}{90},
    group style = {group size = 2 by 3,
                   vertical sep=1.0cm,
                   horizontal sep=2.0cm}]
    
    \pgfplotsforeachungrouped \x in {10,11,12,13,14,15}{
     \edef\tmp{
        \noexpand \nextgroupplot[
                                xlabel={Mass [GeV/c$^2$]},
                                ylabel=$({c}^{s}_{\x}\times{m}^{2}_{w})^{2}$,
                                mark size=0pt,
                                width=0.45\textwidth,
                                height=5.5cm,
                                xmode=log,
                                ymode=log,
                                x label style={at={(axis description cs:0.75,-0.1)},anchor=near ticklabel},
                                y label style={at={(axis description cs:-0.13,.75)},anchor=near ticklabel},
                                ]
            
            \noexpand \addplot[blue, name path = xenon100] table[]
                      {Data/HENR/Xenon100/O\x.dat};
            
            \noexpand \addplot[only marks, mark size=1, error bars/.cd,
                               y dir=both, y explicit, error bar style={color=black}]
                               table[x=mass,y=median, y error plus index=3, y error minus index=2] {Data/HENR/Projected_Sensitivity/LUX/O\x.dat};
                        
            \noexpand \addplot[green, name path = nsig1] table[x=mass, y=nsig1]
                      {Data/HENR/Projected_Sensitivity/Results_method1/O\x.dat};
                      
            \noexpand \addplot[green, name path = psig1] table[x=mass, y=psig1]
                      {Data/HENR/Projected_Sensitivity/Results_method1/O\x.dat};
                      
            \noexpand \addplot[yellow, name path = psig2] table[x=mass, y=psig2]
                      {Data/HENR/Projected_Sensitivity/Results_method1/O\x.dat};
                      
            \noexpand \addplot[green, forget plot] fill between[of=nsig1 and psig1];
            \noexpand \addplot[yellow, forget plot] fill between[of=psig1 and psig2];
            
            \noexpand \addplot[black, name path = median] table[x=mass, y=median]
                      {Data/HENR/Projected_Sensitivity/Results_method1/O\x.dat};
            
            %\noexpand \addplot[black, dashed, name path = median] table[x=mass, y=cl]
            %          {Data/HENR/Projected_Sensitivity/Results/O\x.dat};
                     
        }
        \tmp 
        }
  \end{groupplot}
\end{tikzpicture}
\caption{LZ projected sensitivity at the 90\% CL for isoscalar WIMP-nucleon couplings for relativistic EFT operators $\Operator_{10}$-$\Operator_{15}$.
         The solid black line is the projected sensitivity from this analysis. 
         The green and yellow bands are the 1$\sigma$ and 2$\sigma$ respectively.
         The blue line is the previous limits set by XENON100 \cite{xenon100_eft_ref}.
         The single point is the limit set by LUX \cite{LUX_RUN4_EFT_2021}.}
\label{fig:EFT_Result_Projected_Sensitivity_2}
\end{figure}

\subsection{Discussion}