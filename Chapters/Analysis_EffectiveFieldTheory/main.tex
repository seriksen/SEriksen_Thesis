\chapter{Analysis of Effective Field Theory Interactions in LZ}
\label{chap:analysis_eft_work}
In this final analysis chapter, two studies are performed using the model independent dark matter approach from \autoref{chap:detection_theory}.
The first is a projected sensitivity of LZ to EFT signals.
The second is an analysis of the first science run (SR1).
\par
Both studies were performed on EFT operators in the isoscalar basis, setting limits (projected and real) by hypothesis testing with a profile likelihood ratio (PLR).
The same signal model generation framework was used for each.
These two features are explained prior to the individual studies.

%\par
%To re-emphasise what was said in \autoref{chapter:lz_outer_detector}, in order to perform any each for a rare events (in our case nuclear recoils) you need to understand the discrimination between ER and NR events, and you need a detailed understanding of backgrounds.
%In both studies a background discussion is included, explaining why each background was considered.


%

\begin{figure}[!htbp]%
\centering
\begin{tikzpicture}
\centering
  \begin{groupplot}[view={0}{90},
    group style = {group size = 2 by 4,
                   vertical sep=1.5cm,
                   horizontal sep=2.0cm}]
    
    \pgfplotsforeachungrouped \x in {1,3,4,5,6,7,8,9}{
     \edef\tmp{
        \noexpand \nextgroupplot[
                                xlabel=Mass (MeV),
                                ylabel=$({c}^{s}_{\x}\times{m}^{2}_{w})^{2}$,
                                mark size=0pt,
                                width=0.45\textwidth,
                                height=5.5cm,
                                xmode=log,
                                ymode=log,
                                x label style={at={(axis description cs:0.75,-0.1)},anchor=near ticklabel},
                                y label style={at={(axis description cs:-0.13,.75)},anchor=near ticklabel},
                                ]
            
            \noexpand \addplot[blue, name path = xenon100] table[]
                      {Data/HENR/Xenon100/O\x.dat};
                        
            
            \noexpand \addplot[green, opacity = 0.4, name path = ns1] table[x=mass, y=ns1]
                      {Data/HENR/Projected_Sensitivity/Results/O0\x.dat};
                      
            \noexpand \addplot[green, opacity = 0.4, name path = ps1] table[x=mass, y=ps1]
                      {Data/HENR/Projected_Sensitivity/Results/O0\x.dat};
                      
            \noexpand \addplot[yellow, opacity = 0.4, name path = ps2] table[x=mass, y=ps2]
                      {Data/HENR/Projected_Sensitivity/Results/O0\x.dat};
                      
            \noexpand \addplot[green, opacity = 0.4, forget plot] fill between[of=ns1 and ps1];
            \noexpand \addplot[yellow, opacity = 0.4, forget plot] fill between[of=ps1 and ps2];
            
            \noexpand \addplot[black, name path = exp] table[x=mass, y=exp]
                      {Data/HENR/Projected_Sensitivity/Results/O0\x.dat};
            
            %\noexpand \addplot[black, dashed, name path = exp] table[x=mass, y=cl]
            %          {Data/HENR/Projected_Sensitivity/Results/O0\x.dat};
                     
        }
        \tmp 
        }
  \end{groupplot}
\end{tikzpicture}
\caption{}
\label{fig:EFT_Result_Projected_Sensitivity_1}
\end{figure}



\begin{figure}[!htbp]%
\centering
\begin{tikzpicture}
\centering
  \begin{groupplot}[view={0}{90},
    group style = {group size = 2 by 3,
                   vertical sep=1.5cm,
                   horizontal sep=2.0cm}]
    
    \pgfplotsforeachungrouped \x in {10,11,12,13,14,15}{
     \edef\tmp{
        \noexpand \nextgroupplot[
                                xlabel=Mass (MeV),
                                ylabel=$({c}^{s}_{\x}\times{m}^{2}_{w})^{2}$,
                                mark size=0pt,
                                width=0.45\textwidth,
                                height=5.5cm,
                                xmode=log,
                                ymode=log,
                                x label style={at={(axis description cs:0.75,-0.1)},anchor=near ticklabel},
                                y label style={at={(axis description cs:-0.13,.75)},anchor=near ticklabel},
                                ]
            
            \noexpand \addplot[blue, name path = xenon100] table[]
                      {Data/HENR/Xenon100/O\x.dat};
                        
            \noexpand \addplot[green, opacity = 0.4, name path = ns1] table[x=mass, y=ns1]
                      {Data/HENR/Projected_Sensitivity/Results/O\x.dat};
                      
            \noexpand \addplot[green, opacity = 0.4, name path = ps1] table[x=mass, y=ps1]
                      {Data/HENR/Projected_Sensitivity/Results/O\x.dat};
                      
            \noexpand \addplot[yellow, opacity = 0.4, name path = ps2] table[x=mass, y=ps2]
                      {Data/HENR/Projected_Sensitivity/Results/O\x.dat};
                      
            \noexpand \addplot[green, opacity = 0.4, forget plot] fill between[of=ns1 and ps1];
            \noexpand \addplot[yellow, opacity = 0.4, forget plot] fill between[of=ps1 and ps2];
            
            \noexpand \addplot[black, name path = exp] table[x=mass, y=exp]
                      {Data/HENR/Projected_Sensitivity/Results/O\x.dat};
            
            %\noexpand \addplot[black, dashed, name path = exp] table[x=mass, y=cl]
            %          {Data/HENR/Projected_Sensitivity/Results/O\x.dat};
                     
        }
        \tmp 
        }
  \end{groupplot}
\end{tikzpicture}
\caption{}
\label{fig:EFT_Result_Projected_Sensitivity_2}
\end{figure}

\section{Statistical Method}
\par
In this section the Profile Likelihood Ratio (PLR) analysis used in 

\par
In direct dark matter searches, hypothesis testing is frequently used to determine whether an observation excludes a dark matter model or is significant enough to claim a discovery.
This is often performed using a frequentist approach where there are two well defined hypotheses, a null hypothesis ($H_0$) and an alternative hypothesis ($H_1$), which are tested to see which is the most compatible with the observation.
In general the null hypothesis is assumed to be true and is either rejected in favour of the alternative or is failed to be rejected, neither hypothesis is every accepted, where the hypotheses are models that could describe the observation.
The parameter of interest (POI), $\mu$, in $H_0$ is fixed to a defined value and is set to float in $H_1$ to all other possible values:
\begin{equation}
    \begin{split}
        H_0: \mu = \mu_0 \\
        H_1: \mu \neq \mu_0
    \end{split}
\end{equation}
where $\mu$ is just the number of WIMP-nucleon scatters that are expected in each model, and $\mu_0$ is the expected number of scatter in the dataset.

\par
By detecting the 
In the limit setting analysis in this chapter, $H_0$ is defined as the background plus signal model and $H_1$ as the background only model.
The test statistic for the hypothesis test


The of this section briefly covers how it was used for these studies.

In this chapter $H_0$ is defined by a background only model, and $H_1$ as a background plus signal model.
\par
The evaluation of 
A negative log-likelihood was used, $q = -2 \text{log}_{10}\lambda$, where $\lambda$ is the profile likelihood
\begin{equation}
    \lambda(\vec{x}) = \frac{\mathcal{L}(\mu_0, \textbf{\nu} | \vec{x})}{\mathcal{L}(\hat{\mu}, \hat{\nu} | \vec{x})}
\end{equation}


\par
Each model is defined in terms of 



A log likelihood method is 


where $\mu$ is simply the number of WIMP-nucleon scatters that are expected for each model.
$\mu_0$ is the number of WIMP-nucleon scatters expected in the observation dataset.



The outcome of the test is then either there not being enough evidence to reject a background only described what was observed or that the background model is rejected in favour of a background plus signal model.
There models are comprised of the expected number of event for the background (or background plus signal) and a PDF which 


$H_0$ is 

\par
In practice this means that both news and $\gamma$s


hypotheses are compared to what is observed.

used to either exclude a specific model of dark matter or where an observation is has sufficient significance to claim a discovery in an observation.

$H_0$ is assumed to be correct and then tests are done see if the
d

The parameter for this is the parameter of intersect, $\mu$, which is simply the number WIMP-nucleon scatters that are expected to be observed.




where $mu$ is the number of WIMP-nucleon scatters expected in the given model.
This boils.

\par
The actual test statistic for these hypothesis tests is the profile likelihood ratio (PLR) defined as: 





where $\lambda$ is the 

In order to access the the 

The observables used in both the SR1 study and the projected sensitively are just two dimensional: \{S1$_c$, log$_{10}$S2$_c$\} 

\par
In the projected study the test is summarised as:
"finding the value $\mu$ above which the background plus signal model is incompatible with a background only model.

Which has a dependency on the PDFs of both signal and backgrounds, $f_s(\textbf{x_e})$ and $f_b(\textbf{x_e})$.

For considering $N$ backgrounds in the model.
The log-likelihood for the analysis is given by:
\begin{equation}
\begin{split}
    -2 \text{log}\mathcal{L}(\mu_s(\sigma),\textbf{\nu}) =& 2\bigg(\mu_s + \sum^{N}_{b=1} \mu_b \bigg) \\
                                                          &-2\sum^{n_0}_{e=1} \bigg(\text{log}(\mu_s f_s(x_e)  + \sum^N_{b=1}\mu_b f_b(x_e)
                                                          \bigg) \\
                                                          &+\sum^N_{b=1} \frac{(\mu_b - a_b)^2}{s^2_b}
\end{split}
\end{equation}
where $e$ represented running over each event in the data set.
\par
The codebase used here which implements the above, \textit{LZStats}, was developed primarily by I. Olicina.
Details of the code can be found in his thesis \cite{LZ_Ibles_LZStats_Thesis_ref}.

\subsection{Model PDFs}
\par
In both the projected sensitivity study and the SR1 study the PDFs were created in the same fashion, albeit with different software package\footnote{The tool for generating the PDFs used in the collaboration changed between when these two studies were done}.



The approaches fall into two categories: those simulated through \textit{BACCARAT}, and those  simulated through the full-chain and not.

\par
Backgrounds which are unique to LZ as they are from the local environment were all simulated in \textit{BACCARAT} as energy deposit simulations.
In the case of the projected limits, analysis cuts were applied at this stage, so are on the simulation truth as a detector response for 
The energy deposits were when fed into a detector


\par

Contributions from neutrions were determined via DMCalc. 



The simulations used to produce the PDFs for the PLR originate from those used in \cite{LZ_projected_sensitivity_paper_ref} where an energy deposit only approach was adopted based upon a detector response model for which the author takes no credit and so only the essential information is detailed here.
A more complete description of the simulation framework and detector response can be found in the following Refs. \cite{lz_simulations_ref,theresafruth_thesis_ref}.
\par
In order to determine the detector response, a set of simulations were performed in the standard fashion; where an energy deposit in xenon is translated into scintillation photons and ionisation electrons using NEST \cite{nest_1_ref,nest_2_ref}.
These were then propagated until eventual detection (or not) at the PMT faces where the PMT response is then incorporated.
The resultant sum of these responses give us the S1 and S2 pulse sizes from which the detector response is made from, describing the size of the S1 and S2 for a given recoil.
Our detector is then described by a finite number of parameters given \autoref{tab:projected_sensitivity_detector_parameters} and the response in S1$_c$ and S2$_c$ for a NR events is shown in \autoref{fig:projected_detector_model_response_for_flat_nr}.
When it comes time to generate the PDFs required for our PLR we can simply use the differential scattering rate per recoil energy.

\section{Signal Model}
\par
The actual signal model is actually produced by taking a theoretical event rate spectrum (produced by a Mathematica package - DMFormFactor - developed by XXX) and applying the analysis acceptance and detector response.
Passing through LZLama or PdfMaker, this turns the above into event observable - S1 and LogS2).
\par
DMFormFactor takes performs the calculations described previously.
Each Xenon isotope is evaluated separately and is weighted by the Xenon abundance before being added together to produce an energy spectrum.
As mentioned previously, the energy spectrum is dependant upon the choice of coupling and the WIMP mass.
\par
The choice has been made to perform analysis in iso-scalar rather than proton-neutron. 
There is always an argument that one approach is more appropriate than the other
\par
In \autoref{tab:DMFormFactor_parameters} the parameters used for this are shown.
The majority of these values are to match those from previous studies and the Dark Matter community standards.

\begin{table}[]
    \centering
    \begin{tabular}{c|c}
        Parameter   & Value  \\ \hline
        $\nu_0$     & 220$km s^{-1}$ \\
        $\nu_{esc}$ & 544$km s^{-1}$ \\
        $\rho_{0}$     & 0.3 $GeV/cm^{3}$ \\
        $\nu_E$     & 245 $km s^{-1}$ 
    \end{tabular}
    \caption{DMFormFactor parameters used for the standard halo model. CITE XXX}
    \label{tab:DMFormFactor_parameters}
\end{table}


\par
A subset of the recoil spectra calculated are shown in \autoref{fig:HENR_Spin_Recoil_Spectrum} and \autoref{fig:HENR_NotSpin_Recoil_Spectrum}.
The masses for which the energy spectra were calculated were;
[5, 7, 10, 12, 14, 21, 33, 50, 100, 200, 400, 1000, 4000] GeV all of which can be seen in Annex XXX.
The reason for this is that the shape of the limit produced is fairly well understood, so only a subset of WIMP masses are needed.

\newcommand{\allrecoiloperators}{01s,03s,04s,05s,06s,07s,08s,09s,10s,11s,12s,13s,14s,15s}
\newcommand{\spinrecoiloperators}{01s,04s,06s,07s,09s,10s,11s,14s}
\newcommand{\roguerecoiloperators}{03s,05s,08s,12s,13s,15s}

\begin{figure}[!htbp]%
\centering
\begin{tikzpicture}
\centering
  \begin{groupplot}[view={0}{90},
    group style = {group size = 2 by 1}]
    \nextgroupplot[
    xlabel=Recoil Energy (keV),
    ylabel=Differential Rate (kg/day/keV),
    legend pos=north east,
    mark size=0pt,
    xmin=1, xmax=400, xmode=log,
    ymin=1e-14, ymax=1e3, ymode=log]
    \foreach \henrop in \spinrecoiloperators{
                \addplot 
                    table
                    %{Data/HENR/Signal/Recoils/dRdEr_EFT_O01s_m5GeV_0.txt};
                    {Data/HENR/Projected_Sensitivity/Signal/Recoils/dRdEr_EFT_O\henrop_m5GeV_0.dat};
            }
    \nextgroupplot[
    xlabel=Recoil Energy (keV),
    legend pos=north east,
    mark size=0pt,
    xmin=1, xmax=400, xmode=log,
    ymin=1e-14, ymax=1e3, ymode=log]
    \foreach \henrop in \spinrecoiloperators{
                \addplot
                    table
                    %{Data/HENR/Signal/Recoils/dRdEr_EFT_O01s_m5GeV_0.txt};
                    {Data/HENR/Projected_Sensitivity/Signal/Recoils/dRdEr_EFT_O\henrop_m21GeV_0.dat};
            }
  \end{groupplot}
\end{tikzpicture}
\caption{Literal Aids
\textbf{Left:} $Gd^{156}$ de-excitation path.
\textbf{Right:} $Gd^{158}$ de-excitation path.
}
\label{fig:HENR_Spin_Recoil_Spectrum}
\end{figure}

\begin{figure}[!htbp]%
\centering
\begin{tikzpicture}
\centering
  \begin{groupplot}[view={0}{90},
    group style = {group size = 2 by 1}]
    \nextgroupplot[
    xlabel=Recoil Energy (keV),
    ylabel=Differential Rate (kg/day/keV),
    legend pos=north east,
    mark size=0pt,
    xmin=1, xmax=400, xmode=log,
    ymin=1e-14, ymax=1e3, ymode=log]
    \foreach \henrop in \roguerecoiloperators{
                \addplot 
                    table
                    %{Data/HENR/Signal/Recoils/dRdEr_EFT_O01s_m5GeV_0.txt};
                    {Data/HENR/Projected_Sensitivity/Signal/Recoils/dRdEr_EFT_O\henrop_m5GeV_0.dat};
            }
    \nextgroupplot[
    xlabel=Recoil Energy (keV),
    legend pos=north east,
    mark size=0pt,
    xmin=1, xmax=400, xmode=log,
    ymin=1e-14, ymax=1e3, ymode=log]
    \foreach \henrop in \roguerecoiloperators{
                \addplot
                    table
                    %{Data/HENR/Signal/Recoils/dRdEr_EFT_O01s_m5GeV_0.txt};
                    {Data/HENR/Projected_Sensitivity/Signal/Recoils/dRdEr_EFT_O\henrop_m5GeV_0.dat};
            }
  \end{groupplot}
\end{tikzpicture}
\caption{Literal Aids
\textbf{Left:} $Gd^{156}$ de-excitation path.
\textbf{Right:} $Gd^{158}$ de-excitation path.
}
\label{fig:HENR_NotSpin_Recoil_Spectrum}
\end{figure}

\par
When performing this analysis, an important choice is the coupling choice.


\paragraph{Copied from Billy}
\par
To allow for ease of comparison to previous limit setting on the inelastic36
WIMP-nucleon EFT operators by the XENON collaboration [? ], this analysis was conducted in the isoscalar basis.37
In the isoscalar basis, the charge densities of the nucleons are effectively averaged such that the interaction becomes38
indiscriminate to the type of nucleon involved, even though both the isoscalar and proton-neutron basis provide insight.39
For this WIMP-nucleon EFT, the UV scale governing the physics is far higher than the energies that are probed in the40
experiment. At these lower energies, the UV interactions have to be reduced to effective ones, which is done at the u,41
d and s quark level. The assumption is made that the couplings to each quark at these scales are roughly equivalent.42
Therefore the mass differences of these quarks are negligible at the high-energy scale of the underlying physics, and the43
interaction would be isoscalar. By performing this analysis in the isoscalar basis, it is possible to test this assumption’s44
validity. Additionally, by using either an isoscalar or isovector basis, the target’s nuclear state can be considered as45
isospin symmetric; a property of the strong force that can aid in simplifying the analysis.

\begin{figure}
    \centering
    \includegraphics[width=0.5\textwidth]{Figures/Placeholder.png}
    \caption{Integrated rate for each operator for a 1000GeV DM particle}
    \label{fig:operator_integrated_rate}
\end{figure}

\begin{table}[]
    \centering
    \begin{tabular}{c|c}
        Parameter   & Value  \\ \hline
        $g_{1}$     & 0.119 \\
        $g_{2}$     & 79.1  
    \end{tabular}
    \caption{Key detector parameters for the LXe-TPC parameters as used in \cite{LZ_projected_sensitivity_paper_ref}}
    \label{tab:projected_sensitivity_detector_parameters}
\end{table}

\section{Full Exposure Sensitivity}
\par
In a typical WIMP search with liquid xenon, the recoil energy region of interest is limited to recoils below 30 keV \cite{LZ_TechnicalDesignReview_ref, LZ_projected_sensitivity_paper_ref, xenonnt_projected_sensitivty_ref}.
We saw this in \autoref{fig:si_recoil_and_form_factor} where the rate of recoils drops of rapidly as the recoil energy increases.
\autoref{fig:HENR_recoil_spectra_m50} and \autoref{fig:HENR_recoil_spectra_m1000} show that the response from a number of EFT operators peak at much higher energies and therefore motivate extending the search region.
In this section the sensitivity of LZ to these signals in an extended region of interest is determined for the planned full exposure of LZ; 1000 live-days with 5.6 tonne of xenon.
The approach adopted in this section is otherwise analogous to \cite{LZ_projected_sensitivity_paper_ref}.
\par
The assumed detector parameters from \cite{LZ_projected_sensitivity_paper_ref} are shown in \autoref{tab:projected_sensitivity_detector_parameters}.
The extended region of interest was set such that it was below the end point of any calibration of NEST ($\backsim$ 300 keV from AmBe) and therefore no extrapolation is needed \cite{nest_1_ref}.
The region of interest was defined as S1$_c$ [3, 500] phd.
This was determined from simulations of flat NR backgrounds between 0 and 300 keV$_{NR}$, from which S1$_c$ and S2$_c$ were determined (described in \autoref{sec:lz_detector_chapter}).
The mean of S1$_c$ and log$_{10}$(S2$_c$) per recoil energy are shown in \autoref{fig:projected_detector_model_response_for_flat_nr}.
An S1$_c$ of 500 phd will be from recoils around 250 keV, but the largest recoil (within 3 $\sigma$) that can produce S1$_c$ of 500 phd is from 278 keV$_{NR}$ recoils.

\begin{table}[]
    \centering
    \begin{tabular}{c|c}
        Parameter   & Value  \\ \hline
        $g_{1}$     & 0.119 \\
        $g_{2}$     & 79.1  \\
        Drift field & 310 Vcm$^{-1}$ \\
        electron lifetime & 850 $\mu$s
    \end{tabular}
    \caption{Key detector parameters for the LXe-TPC parameters in the full exposure case. 
             Values from \cite{LZ_projected_sensitivity_paper_ref}.}
    \label{tab:projected_sensitivity_detector_parameters}
\end{table}

\begin{figure}[!htbp]%
\centering
    \begin{tikzpicture}
    \centering
        \begin{groupplot}[view={0}{90},
            group style = {group size = 2 by 1,
            horizontal sep=0.6cm}]
            \nextgroupplot[
            width=0.48\textwidth, height=8cm,
            xlabel={Recoil Energy [keV$_{NR}$]},
            ylabel={S1$_{c}$ [phd]},
            mark size=0pt,
            xmin=0, xmax=300,
            ymin=0, ymax=600]

            \addplot[yellow, name path = psig2] table[x=energy, y=psig2]
                      {Data/HENR/Projected_Sensitivity/data_cuts/s1_vs_recoil.dat};
            \addplot[yellow, name path = nsig2] table[x=energy, y=nsig2]
                      {Data/HENR/Projected_Sensitivity/data_cuts/s1_vs_recoil.dat};
            
            \addplot[green, name path = psig1] table[x=energy, y=psig1]
                      {Data/HENR/Projected_Sensitivity/data_cuts/s1_vs_recoil.dat};
            \addplot[green, name path = nsig1] table[x=energy, y=nsig1]
                      {Data/HENR/Projected_Sensitivity/data_cuts/s1_vs_recoil.dat};
                      
            \addplot[yellow, forget plot] fill between[of=nsig2 and psig2]; 
            \addplot[green, forget plot] fill between[of=nsig1 and psig1];
            
            \addplot[black] table[x=energy, y=mean]
                    {Data/HENR/Projected_Sensitivity/data_cuts/s1_vs_recoil.dat};
            
            \addplot[blue, dashed] coordinates { (0,500)  (325,500)};
            \addplot[black, dashed] coordinates { (278,0)  (278,700)};
                  
            \nextgroupplot[
            width=0.48\textwidth, height=8cm,
            xlabel={Recoil Energy [keV$_{NR}$]},
            ylabel={log$_{10}$(S2$_{c}$ [phd])},
            yticklabel pos=right,
            mark size=0pt,
            xmin=0, xmax=300,
            ymin=2.0, ymax=5.0]
            
            \addplot[yellow, name path = psig2] table[x=energy, y=psig2]
                      {Data/HENR/Projected_Sensitivity/data_cuts/logs2_vs_recoil.dat};
            \addplot[yellow, name path = nsig2] table[x=energy, y=nsig2]
                      {Data/HENR/Projected_Sensitivity/data_cuts/logs2_vs_recoil.dat};
            \addplot[yellow, forget plot] fill between[of=nsig2 and psig2];          
            
            \addplot[green, name path = psig1] table[x=energy, y=psig1]
                      {Data/HENR/Projected_Sensitivity/data_cuts/logs2_vs_recoil.dat};
            \addplot[green, name path = nsig1] table[x=energy, y=nsig1]
                      {Data/HENR/Projected_Sensitivity/data_cuts/logs2_vs_recoil.dat};
            \addplot[green, forget plot] fill between[of=nsig1 and psig1];
            
            \addplot[black] table[x=energy, y=mean]
                    {Data/HENR/Projected_Sensitivity/data_cuts/logs2_vs_recoil.dat};
                    
            \addplot[black, dashed] coordinates { (278,2)  (278,5)};
        
        \end{groupplot}
    \end{tikzpicture}
    \caption{Detector response in S1 (\textbf{Left}) and S2 (\textbf{Right}) space for a given recoil in the LZ detector assuming the projected detector parameters.
             The values have been extrapolated from simulations of a flat NR spectrum.
    }
    \label{fig:projected_detector_model_response_for_flat_nr}
\end{figure}

\subsection{Analysis Cuts}
\par
Each background originating from the detector was simulated with energy deposit only simulations.
A series of cuts were then applied to simulate the full model.
As this is simulated, only the core cuts described in \autoref{sec:lz_analysis_cuts} were used.
They are described below:

\begin{enumerate}
    \item \textbf{SS}: Select events which have only scattered once, ``single scatter". An event is a single scatter in the TPC if the energy-weighted standard deviation of the deposits is less than the detector resolution. This is taken to be $\sigma_r <$ 3.0 cm and $\sigma_z <$ 0.2 cm.
    \item \textbf{ROI}: Select events where the recoil energy is in the range expected from a WIMP scatter. This cut is dependent upon which model of dark matter we are using. Here S1$_c$ must be less than 500 phd and have at least a 3-fold coincidence in the TPC PMTs. S2 must be greater than 415, the value required for at least 5 emitted electrons. This electron requirement is to ensure that the S2 size is large enough for position reconstruction.
    \item \textbf{FID}: The inner volume or fiducial volume of the TPC is taken, removing events near the edges. The FID is defined as a cylinder extending from the centre of the TPC to 4 cm from the TPC walls, 2 cm above the cathode grid, and 13 cm below the gate grid. This inner volume contains 5.6 tonnes of LXe, meaning that there are 1.4 tonnes of xenon used for self-shielding.
    \item \textbf{Veto}: TPC scatters where there is a time-coincident deposit in either of the veto detectors are removed. In the Skin detector the signal must be within 800 $\mu$s of the TPC scatter and be at least 3 phd in size. In the OD the deposit must be at least 200 keV in size, and within 500 $\mu$s of the TPC scatter. This OD selection was chosen to maintain consistency with \cite{LZ_projected_sensitivity_paper_ref} which used an older simulation framework.
\end{enumerate}
In addition to these, the detection efficiency is applied.
This was determined in the WIMP-SI projected sensitivity study detailed in \cite{LZ_projected_sensitivity_paper_ref}.
\par
This gives a differential rate per recoil energy for each background component.
Other sources of backgrounds which would only scatter once anyway are not simulated in this way.
Instead the differential rate is determined by the scattering cross section and particle flux.
The PDFs used in the PLR are generated from feeding the recoil spectra of each component into a detector response model.

\subsection{Backgrounds}
\par
The background model considered was made up of 11 components which represent the most significant contributors discussed in \autoref{sec:lz_backgrounds} and summarised below.
The groupings of certain backgrounds here were driven by the need to validate the simulations against previous low energy region of interest studies in \cite{LZ_projected_sensitivity_paper_ref,LZ_Ibles_LZStats_Thesis_ref}.
The differential rate per recoil energy for both ER and NR backgrounds considered is shown in \autoref{fig:sensitivity_paper_backgrounds}.

\par
Contributions from ``Detector components", ``Surface contamination", and ``Environmental" sources are summed together but kept separate as ER and NR components.
Within the ER contributions are the cavern-$\gamma$s and plate-out radon progeny.
The contributions from these are constrained by the radioassay performed by LZ prior to construction \cite{LZ_assay_ref} and $\gamma$-rate measurements \cite{LZ_Gamma_Ray_Background_ref}.
The largest contributor to NR events are ($\alpha,n$) neutrons, again this is constrained by the radioassay.
These are labelled as ``Detector contaminants" in both \autoref{fig:sensitivity_paper_backgrounds} and \autoref{tab:projected_lz_backgrounds}.
\par
The largest contributor to the ER rate is from an $^{136}$Xe.
This decays via the emission of two $\beta$s \cite{xenon136_ref} and so cannot be vetoed.
It will pass all analysis cuts as it will be evenly distributed within the TPC.
This background becomes significant above electron recoils of 20 keV$_{ee}$.
In the extended region, it becomes the dominant background as can be seen in \autoref{fig:sensitivity_paper_backgrounds}.
\par
${}^{222}$Rn progeny, ${}^{220}$Rn progeny and ${}^{85}$Kr are all dispersed within the TPC volume.
In the low energy region ${}^{222}$Rn and ${}^{220}$Rn can be considered as flat contributions as the dominant decay in each chain is $\beta$ from $^{214}$Pb and $^{216}$Pb respectively \cite{LZ_projected_sensitivity_paper_ref}.
At higher energies, the $\alpha$ decays of the decay chain become prevalent and the contribution is no longer flat.
${}^{85}$Kr is a $\beta$ decay with essentially a flat contribution out to 300 keV$_{ee}$ \cite{kr85_rate_ref}.
The contributions of these processes to the total background rate are driven primarily by the xenon circulation and purification system. 
\par
The final contribution to the background rate is from neutrinos.
The ER contribution stems from neutrino-electron scattering of solar neutrinos primarily from the $pp$ solar chain \cite{solar_neutrinos_ref}.
The NR contribution is split between low and high energy contributors. 
Low energy NR events are from solar neutrinos from the $^{8}$B \cite{b8_neutrino_rate_ref} and $hep$ \cite{solar_neutrinos_rate_ref}, both of which have a negligible impact on the rate above 5 keV$_{nr}$.
High energy NR events are from atmospheric \cite{atmospheric_neutrinos_rate_ref} and diffuse supernova neutrinos \cite{dissuse_supernova_neutrinos_rate_ref}.
\par
Several backgrounds mentioned in \autoref{sec:lz_backgrounds} were not included in this model.
Most notably $^{37}$Ar, $^{127}$Xe and $\gamma$-X.
$^{37}$Ar and $^{127}$Xe were excluded as this is a 1000-day projection, which requires a number of years of detector running.
Both $^{37}$Ar and $^{127}$Xe are created by xenon activation from cosmic rays and decay away once the xenon is inside the TPC underground \cite{lz_argon37_ref, lux_xenon_activation_ref}.
They are therefore only of significant concern for the first year of data taking.
$\gamma$-X events are more interesting as they are a background that is more significant at higher energies \cite{gregrischbieter_thesis_ref}.
It is expected that the contribution from these in LZ will be less significant than it has been for smaller detectors due to the increased amount of self-shielding \cite{LZ_TechnicalDesignReview_ref}.
These were excluded because in order to effectively model them the positional information of events need to be considered, and we have limited the observable parameters to \{S1$_c$,log$_{10}$(S2$_c$)\}.
Including addition dimensions is fairly trivial, but the PLR evaluation does not scale favourably to increased dimensionality.
When LUX included 5-dimensions into their PLR for RUN-4 analysis the resultant PLR required $\backsim$15,000 CPU-hours per mass point \cite{billyboxer_thesis_ref} and so it was desirable to avoid a similar situation here.
There are novel ways of overcoming this with GPU approaches detailed in \cite{flamenest_ref} and \cite{lux_ml_plr_ref}, but these would have required a significant effort to integrate into the LZ computing framework.
In future searches, LZ is planning on using a boosted decision tree machine learning technique to veto $\gamma$-X that is based upon LUX Run-4 EFT analysis \cite{LUX_RUN4_EFT_2021}.

\par
The number of events expected from each background in the region of interest is shown in \autoref{tab:projected_lz_backgrounds}.
Included as well in \autoref{tab:projected_lz_backgrounds} are the background rates for a WIMP-SI search to highlight the increase in the number of backgrounds.
Increasing the recoil energy window does not significantly increase the rate of the NR backgrounds.
Contributions from ${}^{8}$B and $hep$ neutrinos are completely unchanged as they are confined to low energy.
The contribution from the other NR sources decreases with recoil energy and so the additional contribution is minimal.
A simulated dataset of these backgrounds is shown in  \autoref{fig:projected_background_dataset}.

\begin{table}[]
    \centering
    \begin{tabular}{c|c|c|c}
        \multirow{2}{*}{Background}                  & \multicolumn{2}{c}{N}                            & \multirow{2}{*}{$\sigma$/N}  \\ 
                                                     &  (S1$_c <$ 80 phd)     & (S1$_c <$ 500 phd)      &              \\ \hline
        \textbf{ER contributions}                    &                        &                         &   \\
        Detector contaminants                        & 171                    & 1166                    & 20\% \cite{LZ_projected_sensitivity_paper_ref}        \\
        pp + ${}^{7}$Be + ${}^{13}$N solar neutrinos & 615                    & 2950                    & 2\% \cite{pp_solar_neutrinos_rate_ref}       \\
        ${}^{222}$Rn                                 & 1915                   & 12514                   & 10\% \cite{lz_predicted_radon_rate_ref}        \\
        ${}^{220}$Rn                                 & 316                    & 1902                    & 10\% \cite{lz_predicted_radon_rate_ref}        \\
        ${}^{136}$Xe 2$\nu\beta\beta$                & 495                    & 19183                   & 50\% \cite{double_beta_decay_rate_ref}        \\
        ${}^{85}$Kr                                  & 83                     & 557                     & 20\% \cite{kr85_rate_ref}         \\ \hline
        \textbf{NR contributions}                    &                        &                         &   \\
        Detector contaminants                        & 0.81                   & 1.51                    & 20\% \cite{LZ_projected_sensitivity_paper_ref}         \\
        ${}^{8}$B solar neutrinos                    & 36                     & 36                      & 4\%  \cite{b8_neutrino_rate_ref}       \\
        hep solar neutrinos                          & 0.9                    & 0.9                     & 15\% \cite{solar_neutrinos_rate_ref, pp_solar_neutrinos_rate_ref}        \\
        Diffuse supernova neutrinos                  & 0.15                   & 0.17                    & 50\% \cite{dissuse_supernova_neutrinos_rate_ref}        \\
        Atmospheric neutrons                         & 0.65                   & 0.85                    & 25\% \cite{atmospheric_neutrinos_rate_ref}      
    \end{tabular}
    \caption{Backgrounds considered in the PLR for a full exposure sensitivity. N is the number of events expected and $\sigma$/N is the error associated with the rate in the extended energy region.
    The values for the WIMP-SI search region values differ from previous studies \cite{LZ_projected_sensitivity_paper_ref,LZ_Ibles_LZStats_Thesis_ref}. This is due to an upgraded NEST package version used here.}
    \label{tab:projected_lz_backgrounds}
\end{table}

\begin{figure}[!htbp]%
\centering
    \begin{tikzpicture}
    \centering
        \begin{groupplot}[view={0}{90},
            group style = {group size = 2 by 1}]
            \nextgroupplot[
            width=0.5\textwidth, height=8cm,
            xlabel=Recoil Energy (keV),
            ylabel=Differential Rate (kg/day/keV),
            legend pos=north east,
            mark size=0pt,
            xmin=0, xmax=2700,
            ymode=log]
                \addplot
                    table[x=Energy,y=Rate]
                    {Data/HENR/Projected_Sensitivity/Background_Rates/detector_er.dat};
      %          \addlegendentry{Det. + Sur. + Env.};
      %         \addplot
      %             table[x=Energy,y=Rate]
      %             {Data/HENR/Projected_Sensitivity/Background_Rates/Xe136.dat};
      %          \addlegendentry{${}^{136}$Xe}
      %         \addplot
      %             table[x=Energy,y=Rate]
      %             {Data/HENR/Projected_Sensitivity/Background_Rates/Rn222.dat};
      %          \addlegendentry{${}^{222}$Rn};
      %         \addplot
      %             table[x=Energy,y=Rate]
      %             {Data/HENR/Projected_Sensitivity/Background_Rates/Rn220.dat};
      %          \addlegendentry{${}^{220}$Rn};
      %         \addplot
      %             table[x=Energy,y=Rate]
      %             {Data/HENR/Projected_Sensitivity/Background_Rates/Solar.dat};
      %          \addlegendentry{Solar $\nu$};
      %         \addplot
      %             table[x=Energy,y=Rate]
      %                {Data/HENR/Projected_Sensitivity/Background_Rates/Kr85.dat};
      %          \addlegendentry{${}^{85}$Kr};
                  
            \nextgroupplot[
            width=0.5\textwidth, height=8cm,
            xlabel=Recoil Energy (keV),
            yticklabel pos=right,
            legend pos=north east,
            mark size=0pt,
            xmin=0, xmax=250,
            ymin=1e-12, ymax=,
            ymode=log]
                \addplot
                    table[x=Energy,y=Rate]
                    {Data/HENR/Projected_Sensitivity/Background_Rates/detector_nr.dat};
                \addlegendentry{Det. + Sur. + Env.};
                \addplot
                    table[x=Energy,y=Rate]
                    {Data/HENR/Projected_Sensitivity/Background_Rates/atm.dat};
                \addlegendentry{Atm};
                \addplot
                    table[x=Energy,y=Rate]
                    {Data/HENR/Projected_Sensitivity/Background_Rates/DSN_DiffRate.dat};
                \addlegendentry{DSN};
                \addplot
                    table[x=Energy,y=Rate]
                    {Data/HENR/Projected_Sensitivity/Background_Rates/hep.dat};
           %     \addlegendentry{hep};
           %     \addplot
           %         table[x=Energy,y=Rate]
           %         {Data/HENR/Projected_Sensitivity/Background_Rates/B8.dat};
           %     \addlegendentry{${}^{8}$B}
        
        \end{groupplot}
    \end{tikzpicture}
    \caption{Backgrounds considered in the projected sensitivity}
    \label{fig:sensitivity_paper_backgrounds}
\end{figure}

\begin{figure}
    \centering
    \includegraphics[width=15cm]{Figures/EFT/Projected_backgrounds/projected_backgrounds_s1_s2.png}
    \caption{Simulated data set for a background-only 1000 live day run with a 5.6-tonne fiducial mass. The ER and NR bands are shown in blue and red, respectively; the solid lines are the mean, and the dashed are 10\% and 90\% quantiles.}
    \label{fig:projected_background_dataset}
\end{figure}

\subsection{Projected Sensitivity}
\par
Presented in \autoref{fig:EFT_Result_Projected_Sensitivity_1} and \autoref{fig:EFT_Result_Projected_Sensitivity_2} are the projected sensitivities from a one-sided PLR test statistic.
A one-sided PLR was chosen over using a two-sided test as the purpose is to determine the sensitivity of LZ to the couplings, $c^s_i$, not the discovery significance. 
This approach is also in line with that used for the SI and SD projected sensitivity studies \cite{LZ_projected_sensitivity_paper_ref} as well a other sensitivity studies within LZ \cite{LZ_Ibles_LZStats_Thesis_ref, umituktu_thesis_ref}.
The -2$\sigma$ is excluded from the plot as the limit will be power contained \cite{power_constrained_limits_ref}, again a reporting style adopted from \cite{LZ_projected_sensitivity_paper_ref}.
Included in the figures are the XENON100 \cite{xenon100_eft_ref} and LUX \cite{LUX_RUN4_EFT_2021} limits.
Only a single data point is shown from LUX as their analysis was originally done in an \{$neutron,proton$\} basis for elastic operators and only reported an $isoscalar$ limit for a single WIMP mass for each operator.
\par
The sensitivity projections are presented in terms of the dimensionless values $({c}^{s}_{i}\times{m}^{2}_{w})^{2}$ where $m_w$ is the Higg's vacuum expectation value.
${c}^{s}_{i}$ has a dimensionality of [mass]$^{-2}$, which originates from the decision of the authors of \textit{DMFormFactor} to normalise spinors to unity, to use dimensionless representations of the operators and to scale the couplings by $m^{-2}_w$.
Reporting results in this format mirrors the LUX and XENON100 approach, allowing for a direct comparison.
\par
The parameter space to which the LZ detector is sensitive represents a significant step forward compared to current limits, improving by typically 3-4 orders of magnitude.
This is largely driven by the increased exposure of LZ with 5.6 tonnes $\times$ 1000 live days, compared to 34 kg $\times$ 224.6 live days from XENON100 and 100 kg $\times$ 311.2 live days from LUX.
The level of $\gamma$-X events will be the primary hold back to any potential discovery in the EFT region, but assuming that they can be successfully modelled in a more complex PLR, LZ has a very high projected sensitivity to EFT signatures.

\begin{figure}[!htbp]%
\centering
\begin{tikzpicture}
\centering
  \begin{groupplot}[view={0}{90},
    group style = {group size = 2 by 4,
                   vertical sep=1.0cm,
                   horizontal sep=2.0cm}]
    
    \pgfplotsforeachungrouped \x in {1,3,4,5,6,7,8,9}{
     \edef\tmp{
        \noexpand \nextgroupplot[
                                xlabel={Mass [GeV/c$^2$]},
                                ylabel=$({c}^{s}_{\x}\times{m}^{2}_{w})^{2}$,
                                mark size=0pt,
                                width=0.45\textwidth,
                                height=5.5cm,
                                xmode=log,
                                ymode=log,
                                yminorticks=true,
                                x label style={at={(axis description cs:0.75,-0.1)},anchor=near ticklabel},
                                y label style={at={(axis description cs:-0.13,.75)},anchor=near ticklabel},
                                ]
            
            \noexpand \addplot[blue, name path = xenon100] table[]
                      {Data/HENR/Xenon100/O\x.dat};
                        
            \noexpand \addplot[only marks, mark size=1, error bars/.cd,
                               y dir=both, y explicit, error bar style={color=black}]
                               table[x=mass,y=median, y error plus index=3, y error minus index=2] {Data/HENR/Projected_Sensitivity/LUX/O\x.dat};
            
            \noexpand \addplot[green, name path = nsig1] table[x=mass, y=nsig1]
                      {Data/HENR/Projected_Sensitivity/Results_method1/O\x.dat};
                      
            \noexpand \addplot[green, name path = psig1] table[x=mass, y=psig1]
                      {Data/HENR/Projected_Sensitivity/Results_method1/O\x.dat};
                      
            \noexpand \addplot[yellow, name path = psig2] table[x=mass, y=psig2]
                      {Data/HENR/Projected_Sensitivity/Results_method1/O\x.dat};
                      
            \noexpand \addplot[green, forget plot] fill between[of=nsig1 and psig1];
            \noexpand \addplot[yellow, forget plot] fill between[of=psig1 and psig2];
            
            \noexpand \addplot[black, name path = median] table[x=mass, y=median]
                      {Data/HENR/Projected_Sensitivity/Results_method1/O\x.dat};
            
            %\noexpand \addplot[black, dashed, name path = median] table[x=mass, y=cl]
            %          {Data/HENR/Projected_Sensitivity/Results/O\x.dat};
                     
        }
        \tmp 
        }
  \end{groupplot}
\end{tikzpicture}
\caption{LZ projected sensitivity at the 90\% CL for isoscalar WIMP-nucleon couplings for relativistic EFT operators $\Operator_1$,$\Operator_3$-$\Operator_9$.
         The solid black line is the projected sensitivity from this analysis. 
         The green and yellow bands are the 1$\sigma$ and 2$\sigma$ respectively.
         The blue line is the previous limits set by XENON100 \cite{xenon100_eft_ref}.
         The single point is the limit set by LUX \cite{LUX_RUN4_EFT_2021}.}
\label{fig:EFT_Result_Projected_Sensitivity_1}
\end{figure}



\begin{figure}[!htbp]%
\centering
\begin{tikzpicture}
\centering
  \begin{groupplot}[view={0}{90},
    group style = {group size = 2 by 3,
                   vertical sep=1.0cm,
                   horizontal sep=2.0cm}]
    
    \pgfplotsforeachungrouped \x in {10,11,12,13,14,15}{
     \edef\tmp{
        \noexpand \nextgroupplot[
                                xlabel={Mass [GeV/c$^2$]},
                                ylabel=$({c}^{s}_{\x}\times{m}^{2}_{w})^{2}$,
                                mark size=0pt,
                                width=0.45\textwidth,
                                height=5.5cm,
                                xmode=log,
                                ymode=log,
                                x label style={at={(axis description cs:0.75,-0.1)},anchor=near ticklabel},
                                y label style={at={(axis description cs:-0.13,.75)},anchor=near ticklabel},
                                ]
            
            \noexpand \addplot[blue, name path = xenon100] table[]
                      {Data/HENR/Xenon100/O\x.dat};
            
            \noexpand \addplot[only marks, mark size=1, error bars/.cd,
                               y dir=both, y explicit, error bar style={color=black}]
                               table[x=mass,y=median, y error plus index=3, y error minus index=2] {Data/HENR/Projected_Sensitivity/LUX/O\x.dat};
                        
            \noexpand \addplot[green, name path = nsig1] table[x=mass, y=nsig1]
                      {Data/HENR/Projected_Sensitivity/Results_method1/O\x.dat};
                      
            \noexpand \addplot[green, name path = psig1] table[x=mass, y=psig1]
                      {Data/HENR/Projected_Sensitivity/Results_method1/O\x.dat};
                      
            \noexpand \addplot[yellow, name path = psig2] table[x=mass, y=psig2]
                      {Data/HENR/Projected_Sensitivity/Results_method1/O\x.dat};
                      
            \noexpand \addplot[green, forget plot] fill between[of=nsig1 and psig1];
            \noexpand \addplot[yellow, forget plot] fill between[of=psig1 and psig2];
            
            \noexpand \addplot[black, name path = median] table[x=mass, y=median]
                      {Data/HENR/Projected_Sensitivity/Results_method1/O\x.dat};
            
            %\noexpand \addplot[black, dashed, name path = median] table[x=mass, y=cl]
            %          {Data/HENR/Projected_Sensitivity/Results/O\x.dat};
                     
        }
        \tmp 
        }
  \end{groupplot}
\end{tikzpicture}
\caption{LZ projected sensitivity at the 90\% CL for isoscalar WIMP-nucleon couplings for relativistic EFT operators $\Operator_{10}$-$\Operator_{15}$.
         The solid black line is the projected sensitivity from this analysis. 
         The green and yellow bands are the 1$\sigma$ and 2$\sigma$ respectively.
         The blue line is the previous limits set by XENON100 \cite{xenon100_eft_ref}.
         The single point is the limit set by LUX \cite{LUX_RUN4_EFT_2021}.}
\label{fig:EFT_Result_Projected_Sensitivity_2}
\end{figure}


\iffalse
\subsection{Signal Model}
\par
As mentioned in \autoref{chap:detection_theory}, the parameters for the Standard Halo Model (SHM) used for generated signal models within direct dark matter experiments has been standardised since mid-2021 \cite{standard_halo_model_conventions_ref}.
However, this study began before that date and so used the parameters previously used within the LZ collaboration, in \cite{LZ_projected_sensitivity_paper_ref,LZ_TechnicalDesignReview_ref,LZ_Ibles_LZStats_Thesis_ref}.
The parameters used are shown in \autoref{tab:projected_DMFormFactor_parameters}.
One interesting benefit of this, however, is that the parameters are exactly the same as those used by both Xenon100 \cite{xenon100_eft_ref} LUX RUN-4 EFT analysis \cite{LUX_RUN4_EFT_2021}.

\begin{table}[]
    \centering
    \begin{tabular}{c|c}
        Parameter         & Value  \\ \hline
        $\nu_0$           & 220$km s^{-1}$ \\
        $\nu_{esc}$       & 544$km s^{-1}$ \\
        $\rho_{\chi}$     & 0.3 $GeV/cm^{3}$ \\
        $|\nu_E|$         & 245 $km s^{-1}$ 
    \end{tabular}
    \caption{Standard Halo Model parameters used for projected sensitivity study.}
    \label{tab:projected_DMFormFactor_parameters}
\end{table}

\par
With the projected detector parameters, the observable quantities \{$S1_c,log(S2_c)$\}, for a selection of operators at a mass of 100 GeV / c$^2$ are shown in \autoref{fig:projected_detector_model_signal_pdfs}.

\begin{figure}[!htbp]%
\centering
    \begin{tikzpicture}
    \centering
        \begin{groupplot}[view={0}{90},
            group style = {group size = 1 by 3,
            horizontal sep=0.6cm}]

        \nextgroupplot[
        width=15cm, height=8cm,
        xlabel={S1$_{c}$ [phd]},
        ylabel={log$_10$(S2$_{c}$ [phd])},
        mark size=0pt,
        xmin=0, xmax=500,
        ymin=2.5, ymax=5.5,
        colormap={blackwhite}{color=(white) color=(black)}]
        
        \addplot3[
              surf,
              shader=flat corner,
        	  mesh/cols=40,
        	  mesh/ordering=rowwise,
            ] file {Data/HENR/Projected_Sensitivity/Signal/pdf/o1_m100_pdf.dat};
            
        \addplot[blue, ]
            table [x=bin, y=mean]
            {Data/HENR/Projected_Sensitivity/Signal/pdf/er_band.dat};     
        \addplot[blue, dashed]
            table [x=bin, y=high]
            {Data/HENR/Projected_Sensitivity/Signal/pdf/er_band.dat};     
        \addplot[blue, dashed]
            table [x=bin, y=low]
            {Data/HENR/Projected_Sensitivity/Signal/pdf/er_band.dat};     

        \addplot[red, ]
            table [x=bin, y=mean]
            {Data/HENR/Projected_Sensitivity/Signal/pdf/nr_band.dat};    
        \addplot[red, dashed]
            table [x=bin, y=high]
            {Data/HENR/Projected_Sensitivity/Signal/pdf/nr_band.dat};     
        \addplot[red, dashed]
            table [x=bin, y=low]
            {Data/HENR/Projected_Sensitivity/Signal/pdf/nr_band.dat};  
            
        \nextgroupplot[
        width=15cm, height=8cm,
        xlabel={S1$_{c}$ [phd]},
        ylabel={log$_10$(S2$_{c}$ [phd])},
        mark size=0pt,
        xmin=0, xmax=500,
        ymin=2.5, ymax=5.5,
        colormap={blackwhite}{color=(white) color=(black)}]
        
        \addplot3[
              surf,
              shader=flat corner,
        	  mesh/cols=40,
        	  mesh/ordering=rowwise,
            ] file {Data/HENR/Projected_Sensitivity/Signal/pdf/o6_m100_pdf.dat};
            
        \addplot[blue, ]
            table [x=bin, y=mean]
            {Data/HENR/Projected_Sensitivity/Signal/pdf/er_band.dat};     
        \addplot[blue, dashed]
            table [x=bin, y=high]
            {Data/HENR/Projected_Sensitivity/Signal/pdf/er_band.dat};     
        \addplot[blue, dashed]
            table [x=bin, y=low]
            {Data/HENR/Projected_Sensitivity/Signal/pdf/er_band.dat};     

        \addplot[red, ]
            table [x=bin, y=mean]
            {Data/HENR/Projected_Sensitivity/Signal/pdf/nr_band.dat};    
        \addplot[red, dashed]
            table [x=bin, y=high]
            {Data/HENR/Projected_Sensitivity/Signal/pdf/nr_band.dat};     
        \addplot[red, dashed]
            table [x=bin, y=low]
            {Data/HENR/Projected_Sensitivity/Signal/pdf/nr_band.dat};  
            
        \nextgroupplot[
        width=15cm, height=8cm,
        xlabel={S1$_{c}$ [phd]},
        ylabel={log$_10$(S2$_{c}$ [phd])},
        mark size=0pt,
        xmin=0, xmax=500,
        ymin=2.5, ymax=5.5,
        colormap={blackwhite}{color=(white) color=(black)}]
        
        \addplot3[
              surf,
              shader=flat corner,
        	  mesh/cols=40,
        	  mesh/ordering=rowwise,
            ] file {Data/HENR/Projected_Sensitivity/Signal/pdf/o15_m100_pdf.dat};
            
        \addplot[blue, ]
            table [x=bin, y=mean]
            {Data/HENR/Projected_Sensitivity/Signal/pdf/er_band.dat};     
        \addplot[blue, dashed]
            table [x=bin, y=high]
            {Data/HENR/Projected_Sensitivity/Signal/pdf/er_band.dat};     
        \addplot[blue, dashed]
            table [x=bin, y=low]
            {Data/HENR/Projected_Sensitivity/Signal/pdf/er_band.dat};     

        \addplot[red, ]
            table [x=bin, y=mean]
            {Data/HENR/Projected_Sensitivity/Signal/pdf/nr_band.dat};    
        \addplot[red, dashed]
            table [x=bin, y=high]
            {Data/HENR/Projected_Sensitivity/Signal/pdf/nr_band.dat};     
        \addplot[red, dashed]
            table [x=bin, y=low]
            {Data/HENR/Projected_Sensitivity/Signal/pdf/nr_band.dat};  
        
        
        \end{groupplot}
    \end{tikzpicture}
    \caption{Detector response in S1 (\textbf{Left}) and S2 (\textbf{Right}) space for a given recoil in the LZ detector assuming the projected detector parameters.
             The values have been extrapolated from simulations of a flat NR spectrum 
    }
    \label{fig:projected_detector_model_signal_pdfs}
\end{figure}

\fi

\section{Results from First Science Run}
\par
In this section limits are placed on the elastic EFT operator couplings using data from the LZ's first science run (SR1).
The vanilla WIMP region of interest is considered here rather than an extended one as extending the region of interest would have resulted in a delay to this thesis.
An analysis for SI and SD dark matter interactions have already been performed in this recoil energy range, with null results \cite{lz_ws_sr1_ref}.
As this is the same dataset and cuts only a limited summary of the SR1 is presented here.
Additionally all plots shown in this section except the signal models and limits originate from collaborators from LZ and are not the authors own creation.

\subsection{Overview of SR1}
\par
SR1 ran for a total of 116 days from 23 December 2021 to 18 April 2022.
During this period, several breaks occurred for both calibration runs and system maintenance reducing the data set down to 89 live days.
For the entire period of SR1 the detector was in a stable state.
The length of time running was driven by a requirement to demonstrate physics capability and so only ran until the sensitivity to SI interactions was comparable to previous experiments.
In that sense, SR1 was more of an engineering run.
As such no blinding or salting of the data set was performed, as with any new detector there are unexpected features that are best understood by being able to see them.
In order to mitigate bias from this decision, analysis cuts were developed in side bands and calibration data.


\subsection{TPC Calibration}
\par
The spacial variation in $S1$ and $S2$ were corrected using primarily internal sources which are naturally dispersed throughout the detector volume.
${}^{83m}$Kr and ${}^{131m}$Xe were used for this along with injected tritium (injected as tritiated methane CH$_3$T).
Energy calibrations were performed with the internal sources as well and also used ${}^{129m}$Xe which is monoenergetic at higher energies outside of the region considered here.
$g_1$ was determined to be 0.114$\pm$0.002 phd/photon and $g_2$ as 47.1$\pm$1.1 phd/electron.
The size of a single electron was measured as 57.6$\pm$1.9 phd/electron which gives an electron extraction efficiency of 80.5$\pm$3.7\%.
For calibrating the detector model, NR events were produced produced using the DD neutrons and AmLi, whilst ER events were produced using tritium.
The detector response was tuned to these measurements which are shown in \autoref{fig:sr1_tpc_calibration} along with the ER and NR bands.
\begin{figure}
    \centering
    \includegraphics[width=10cm]{Figures/EFT/All_SR1_Plots/SR1WS_calOnly_0629_twoPanel.pdf}
    \caption{Calibration of ER and NR bands, shown in blue and red respectively.
             The solid line is the median and the dashed are the 10\% and 90\% quantiles.
             \textbf{Top:} ER events produced by $\beta$-decays from injected CH$_3$T.
             \textbf{Bottom:} NR events produced by DD neutrons.
             }
    \label{fig:sr1_tpc_calibration}
\end{figure}

\subsection{Analysis Cuts}
\par
In addition to the core-cuts that remove scatters that are inconsistent with dark matter that have been previously discussed, a multitude of other cuts were developed.
They fall broadly into three categories: live time, S2-based cuts and S1-based cuts.
These three, along with the core cuts are briefly discussed below.

\subsubsection{live time cuts}
The live time cuts are named as such as they have a large adverse affect on the amount of data that is uncharacteristic due to the high rate of pulses observed.
The two most significant cuts were an electron-train cut and a hot spot cut, together removing 35\% of the livetime.
After a large S2 event, a period of a sustained high rate is observed. 
This is due to electrons attaching to impurities and then being released at some later time, causing a delayed extraction and an elevated rate after the S2.
Single photons are also observed after the same event which is thought to be from delayed fluorescence.
The removal of time after an S2 until the detector rate settles corresponds to 29.8\% of live day loss.
Occasionally hot spots appeared in the TPC due to electron emission from the grids. 
These periods of time lasted approximately an hour at a time and reduced the live time by 6.6\%.
There were several other cuts in this category which combined livetime by a future 1\%.

\subsubsection{S2 cuts}
A suite of cuts were developed to remove S2s which did not look like an S2 should if it had originated from a LXe scatter.
These targeted accidental events.

\subsubsection{S1 cuts}


developed a hot spot where 

\subsubsection{Core cuts}
All of these core cuts focus on removing clean scatters in the TPC, but which do not match the properties of dark matter.
Only events with a single S1-S2 pulse pair are selected, multiple S1 or multiple S2 events are removed.
\par
The fiducial region was defined in $z$ by the drift time (time between the S1 and S2 pulses) of between [86,936.5] $\mu$s which corresponds to approximately 12.8 cm below the gate grid and 2.2 cm from the cathode grid.
$r$ was defined as either [4.0,5.0,5.2] cm from the TPC wall. The variation in the $r$ is to remove regions where the electric field is non-uniform, and effects from being close to the wall.
Within this region a the volume of xenon is 5.5$\pm$0.2 tonnes.
\autoref{fig:sr1_fiducial_cut} shows the fiducial volume definition.
\par
The region of interest was defined as 3 phd $<$ S1$_c <$ 80 phd, S2 $>$ 600 phd and S2$_c <$ 10$^5$ phd.
\par
The OD veto window was set to the value discussed in the previous chapter of 1200 $\mu$s, with an energy threshold of 200 keV.
The Skin veto window was set to 400 $\mu$s.
Additionally a prompt veto window was implemented to remove events where the S1 is within [-300,300] ns of a pulse in either the Skin or OD.

\par
The combined efficiency of all of these cuts is shown in \autoref{fig:sr1_nr_efficiency}.
The efficiency in the ROI was determined using AmLi and tritium data.
A total of 335 events remained after the application of all of these cuts.

\begin{figure}
    \centering
    \includegraphics[width=15cm]{Figures/EFT/All_SR1_Plots/fid.png}
    \caption{Distribution of events in the SR1 dataset in the ROI.
    The vertical green line which two steps is the fiducial region.
    The other green line shown is the wall definition.
    The events in grey are outside of the fiducial volume and so vetoed. 
    The events marked in cyan were removed by the OD and the events marked in purple by the Skin.
    The remaining events in the WIMP dataset are shown in black.}
    \label{fig:sr1_fiducial_cut}
\end{figure}

\begin{figure}
    \centering
    \includegraphics[width=10cm]{Figures/EFT/All_SR1_Plots/NR_efficiency.png}
    \caption{Signal efficiency as a function of NR energy for the trigger (blue), S1 threshold (orange), all analysis cuts expect ROI (green) and the ROI (black).
    The vertical dashed line is the the 50\% efficiency point.
    }
    \label{fig:sr1_nr_efficiency}
\end{figure}

\subsection{Background Model}
\par
In the background model, 9 components were included.
The grouping is based upon the distribution in the ROI and the level of uncertainty.

\par
The ${}^{222}$Rn chain and ${}^{220}$Rn chain both produce flat $\beta$ contribution.
The rate of these was determined via $\alpha$-peak fitting to the ER band outside of the ROI which can be seen in \autoref{fig:sr1_spectra}
The $\gamma$ spectrum from the detector contaminants is also flat in this region.
These contributions are treated as a single background labelled $\beta$ decays + Det. ER.
This is the largest contribution to the expected rate, with 218 $\pm$ 36 events.

The rate of detector 
\begin{figure}
    \centering
    \includegraphics[width=10cm]{Figures/EFT/All_SR1_Plots/ER_band_fit.png}
    \caption{SR1 vanilla WIMP-search data (black points) overlaid on the background model.
    The contours shown for each distribution are 1 and 2 $\sigma$.
    ${}^{8}$B is shown in green, ${}^{37}$Ar is shown in orange, the combined ER spectrum in grey.
    The red lines are the NR band with the medium, 10\% and 90\% quantiles.
    The distribution from a 30 GeV WIMP is also shown in purple. 
    }
    \label{fig:sr1_spectra}
\end{figure}


\par
Solar neutrinos which also produce a flat contribution are not included as the rate is more finely constrained \cite{pp_solar_neutrinos_rate_ref}.
\par
Decays from three Xenon isotopes were included.
${}^{127}$Xe and ${}^{124}$Xe contribute to the ER spectrum by double electron capture and double $\beta$ decay respectively.
The rates of these are constrained by know abundances \cite{xenon_isotopes_ref}.
\par
Two interesting backgrounds considered here is weren't considered in the projected case are ${}^{37}$Ar and ${}^{127}$Xe.
These are both short-lived isotopes, with half-lives of 35 days and 36.3 days respectively.
They appear due to cosmogenic activation of the xenon when the xenon is at the surface level\footnote{of the mine. Not the surface of the LXe.}.
Once underground these components decay away and so are not of concern in future data runs.
The rates are therefore constrained by the delivery schedule of the xenon into the Davis cavern \cite{lz_argon37_ref}.
There are expected to be $\approx$100 ${}^{37}$Ar events in this data set but there a very large uncertainty meaning that it could be has high at 300.

\par
The two NR contributions considered are from ${}^{8}$B solar neutrinos and neutrons from the detector.
There are expected to be 0 detector neutrons in this dataset, a value constrained by the radioassay of the detector components prior to construction \cite{LZ_assay_ref}.

\par
The expected contribution from each component in the model is \autoref{tab:sr1_ws_lz_backgrounds}.
The uncertainties on the values shown indicate the constraints placed on each component during the PLR.

\begin{table}[]
    \centering
    \begin{tabular}{c|c}
        Background Component     & Expected Number of Events  \\ \hline
        $\beta$ decays + Det. ER & 218 $\pm$ 36 \\
        $\nu$ ER                 & 27.3 $\pm$ 1.6 \\
        ${}^{127}$Xe             & 9.3 $\pm$ 0.8 \\
        ${}^{124}$Xe             & 5.0 $\pm$ 1.4 \\
        ${}^{136}$Xe             & 15.2 $\pm$ 2.4 \\
        ${}^{8}$B CE$\nu$NS      & 0.15 $\pm$ 0.01 \\
        Accidentals              & 1.2 $\pm$ 0.3 \\
        ${}^{37}$Ar              & [0, 291]      \\
        neutrons                 & 0.0${}^{+0.2}$
    \end{tabular}
    \caption{Expected number of events from each component in the background model.}
    \label{tab:sr1_ws_lz_backgrounds}
\end{table}

\par
The best-fit background model is shown in \autoref{fig:final_data_points} with the the data set overlaid.
The best-fit values were identical to the SI result, as such the plot is shown from the SI result \cite{lz_ws_sr1_ref}.
The WIMP contours (shown in purple) are for a 30 GeV/c${^2}$ which is analogous to $\Operator$1 interactions of the same mass.

\begin{figure}
    \centering
    \includegraphics[width=10cm]{Figures/EFT/All_SR1_Plots/final_data_points.png}
    \caption{SR1 vanilla WIMP-search data (black points) overlaid on the background model.
    The contours shown for each distribution are 1 and 2 $\sigma$.
    ${}^{8}$B is shown in green, ${}^{37}$Ar is shown in orange, the combined ER spectrum in grey.
    The red lines are the NR band with the medium, 10\% and 90\% quantiles.
    The distribution from a 30 GeV WIMP is also shown in purple. 
    }
    \label{fig:final_data_points}
\end{figure}

\subsection{Limits}
\par
Consistent with the SI and SD search, a null result was found, setting limits on the operator couplings.
These are shown in \autoref{fig:EFT_Result_Projected_Sensitivity} in the same dimensionless scale as the projected limits.
\par
World-leading sensitivity has been achieved for all operators at at least some masses.
In several cases, such as $\Operator$6 the sensitivity of LZ is less than that of both LUX and XENON100. 
This is due to the recoil distributions of these operators peaking outside of the ROI used here for high WIMP masses.
Despite this reduced parameter space, the huge increase in exposure has made LZ very competitive.

\begin{figure}[!htbp]%
\centering
\begin{tikzpicture}
\centering
    \begin{axis}[
            ylabel={log${}_{10}$(S2$_c$ [phd])},
            xlabel={S1 [phd]},
            width=15cm,
            height=8cm,
            ymin=2.75, ymax=4.5,
            xmin=0, xmax=80,
            ]
        \addplot[black, only marks,]
            table [x=S1c, y=log_10S2c]
            {Data/HENR/sr1_ws/ws_data.dat};
    \end{axis}
\end{tikzpicture}
    \caption{SR1 search data after all cuts.}
    \label{fig:henr_ws_sr1_events}
\end{figure}

For SR1 fewer parameters were taken into consideration with the PLR working in [S1,logS2] space.
Increasing the set of parameters will allow for improved discrimination and a potential discovery but runs into significant computational limitations as discovered in both \cite{nicolelarsen_thesis_ref, shaunalsum_thesis_ref, billyboxer_thesis_ref}, a different approach such as that described in \cite{flamenest_ref} and \cite{lux_ml_plr_ref} offer some hope for this.

\input{Chapters/Analysis_EffectiveFieldTheory/Figures/WIMP_Search/ws_limits}

%\section{LZ MDC3}
\par
Mock-Data Challenge 3 was the third and final simulation campaign prior to the availability of real data.
This campaign was designed to test the entirety of analysis framework to ensure data-readiness.

\par
As this simulation dataset was performed before the availability of real-data naturally some of the detector parameters used in the simulation are wrong.

\par
Something about PdfMaker being used

\subsection{Backgrounds}
\par
For this, PdfMaker was used for both backgrounds and signal.
Additionally, fewer backgrounds were taken into consideration and some simplifications were made.
Billy has more details on this though...

\subsection{Limits}
\par


\subsection{Discussion}

%\section{Results from First Science Run}
\par
In this section limits are placed on the elastic EFT operator couplings using data from the LZ's first science run (SR1).
The vanilla WIMP region of interest is considered here rather than an extended one as extending the region of interest would have resulted in a delay to this thesis.
An analysis for SI and SD dark matter interactions have already been performed in this recoil energy range, with null results \cite{lz_ws_sr1_ref}.
As this is the same dataset and cuts only a limited summary of the SR1 is presented here.
Additionally all plots shown in this section except the signal models and limits originate from collaborators from LZ and are not the authors own creation.

\subsection{Overview of SR1}
\par
SR1 ran for a total of 116 days from 23 December 2021 to 18 April 2022.
During this period, several breaks occurred for both calibration runs and system maintenance reducing the data set down to 89 live days.
For the entire period of SR1 the detector was in a stable state.
The length of time running was driven by a requirement to demonstrate physics capability and so only ran until the sensitivity to SI interactions was comparable to previous experiments.
In that sense, SR1 was more of an engineering run.
As such no blinding or salting of the data set was performed, as with any new detector there are unexpected features that are best understood by being able to see them.
In order to mitigate bias from this decision, analysis cuts were developed in side bands and calibration data.


\subsection{TPC Calibration}
\par
The spacial variation in $S1$ and $S2$ were corrected using primarily internal sources which are naturally dispersed throughout the detector volume.
${}^{83m}$Kr and ${}^{131m}$Xe were used for this along with injected tritium (injected as tritiated methane CH$_3$T).
Energy calibrations were performed with the internal sources as well and also used ${}^{129m}$Xe which is monoenergetic at higher energies outside of the region considered here.
$g_1$ was determined to be 0.114$\pm$0.002 phd/photon and $g_2$ as 47.1$\pm$1.1 phd/electron.
The size of a single electron was measured as 57.6$\pm$1.9 phd/electron which gives an electron extraction efficiency of 80.5$\pm$3.7\%.
For calibrating the detector model, NR events were produced produced using the DD neutrons and AmLi, whilst ER events were produced using tritium.
The detector response was tuned to these measurements which are shown in \autoref{fig:sr1_tpc_calibration} along with the ER and NR bands.
\begin{figure}
    \centering
    \includegraphics[width=10cm]{Figures/EFT/All_SR1_Plots/SR1WS_calOnly_0629_twoPanel.pdf}
    \caption{Calibration of ER and NR bands, shown in blue and red respectively.
             The solid line is the median and the dashed are the 10\% and 90\% quantiles.
             \textbf{Top:} ER events produced by $\beta$-decays from injected CH$_3$T.
             \textbf{Bottom:} NR events produced by DD neutrons.
             }
    \label{fig:sr1_tpc_calibration}
\end{figure}

\subsection{Analysis Cuts}
\par
In addition to the core-cuts that remove scatters that are inconsistent with dark matter that have been previously discussed, a multitude of other cuts were developed.
They fall broadly into three categories: live time, S2-based cuts and S1-based cuts.
These three, along with the core cuts are briefly discussed below.

\subsubsection{live time cuts}
The live time cuts are named as such as they have a large adverse affect on the amount of data that is uncharacteristic due to the high rate of pulses observed.
The two most significant cuts were an electron-train cut and a hot spot cut, together removing 35\% of the livetime.
After a large S2 event, a period of a sustained high rate is observed. 
This is due to electrons attaching to impurities and then being released at some later time, causing a delayed extraction and an elevated rate after the S2.
Single photons are also observed after the same event which is thought to be from delayed fluorescence.
The removal of time after an S2 until the detector rate settles corresponds to 29.8\% of live day loss.
Occasionally hot spots appeared in the TPC due to electron emission from the grids. 
These periods of time lasted approximately an hour at a time and reduced the live time by 6.6\%.
There were several other cuts in this category which combined livetime by a future 1\%.

\subsubsection{S2 cuts}
A suite of cuts were developed to remove S2s which did not look like an S2 should if it had originated from a LXe scatter.
These targeted accidental events.

\subsubsection{S1 cuts}


developed a hot spot where 

\subsubsection{Core cuts}
All of these core cuts focus on removing clean scatters in the TPC, but which do not match the properties of dark matter.
Only events with a single S1-S2 pulse pair are selected, multiple S1 or multiple S2 events are removed.
\par
The fiducial region was defined in $z$ by the drift time (time between the S1 and S2 pulses) of between [86,936.5] $\mu$s which corresponds to approximately 12.8 cm below the gate grid and 2.2 cm from the cathode grid.
$r$ was defined as either [4.0,5.0,5.2] cm from the TPC wall. The variation in the $r$ is to remove regions where the electric field is non-uniform, and effects from being close to the wall.
Within this region a the volume of xenon is 5.5$\pm$0.2 tonnes.
\autoref{fig:sr1_fiducial_cut} shows the fiducial volume definition.
\par
The region of interest was defined as 3 phd $<$ S1$_c <$ 80 phd, S2 $>$ 600 phd and S2$_c <$ 10$^5$ phd.
\par
The OD veto window was set to the value discussed in the previous chapter of 1200 $\mu$s, with an energy threshold of 200 keV.
The Skin veto window was set to 400 $\mu$s.
Additionally a prompt veto window was implemented to remove events where the S1 is within [-300,300] ns of a pulse in either the Skin or OD.

\par
The combined efficiency of all of these cuts is shown in \autoref{fig:sr1_nr_efficiency}.
The efficiency in the ROI was determined using AmLi and tritium data.
A total of 335 events remained after the application of all of these cuts.

\begin{figure}
    \centering
    \includegraphics[width=15cm]{Figures/EFT/All_SR1_Plots/fid.png}
    \caption{Distribution of events in the SR1 dataset in the ROI.
    The vertical green line which two steps is the fiducial region.
    The other green line shown is the wall definition.
    The events in grey are outside of the fiducial volume and so vetoed. 
    The events marked in cyan were removed by the OD and the events marked in purple by the Skin.
    The remaining events in the WIMP dataset are shown in black.}
    \label{fig:sr1_fiducial_cut}
\end{figure}

\begin{figure}
    \centering
    \includegraphics[width=10cm]{Figures/EFT/All_SR1_Plots/NR_efficiency.png}
    \caption{Signal efficiency as a function of NR energy for the trigger (blue), S1 threshold (orange), all analysis cuts expect ROI (green) and the ROI (black).
    The vertical dashed line is the the 50\% efficiency point.
    }
    \label{fig:sr1_nr_efficiency}
\end{figure}

\subsection{Background Model}
\par
In the background model, 9 components were included.
The grouping is based upon the distribution in the ROI and the level of uncertainty.

\par
The ${}^{222}$Rn chain and ${}^{220}$Rn chain both produce flat $\beta$ contribution.
The rate of these was determined via $\alpha$-peak fitting to the ER band outside of the ROI which can be seen in \autoref{fig:sr1_spectra}
The $\gamma$ spectrum from the detector contaminants is also flat in this region.
These contributions are treated as a single background labelled $\beta$ decays + Det. ER.
This is the largest contribution to the expected rate, with 218 $\pm$ 36 events.

The rate of detector 
\begin{figure}
    \centering
    \includegraphics[width=10cm]{Figures/EFT/All_SR1_Plots/ER_band_fit.png}
    \caption{SR1 vanilla WIMP-search data (black points) overlaid on the background model.
    The contours shown for each distribution are 1 and 2 $\sigma$.
    ${}^{8}$B is shown in green, ${}^{37}$Ar is shown in orange, the combined ER spectrum in grey.
    The red lines are the NR band with the medium, 10\% and 90\% quantiles.
    The distribution from a 30 GeV WIMP is also shown in purple. 
    }
    \label{fig:sr1_spectra}
\end{figure}


\par
Solar neutrinos which also produce a flat contribution are not included as the rate is more finely constrained \cite{pp_solar_neutrinos_rate_ref}.
\par
Decays from three Xenon isotopes were included.
${}^{127}$Xe and ${}^{124}$Xe contribute to the ER spectrum by double electron capture and double $\beta$ decay respectively.
The rates of these are constrained by know abundances \cite{xenon_isotopes_ref}.
\par
Two interesting backgrounds considered here is weren't considered in the projected case are ${}^{37}$Ar and ${}^{127}$Xe.
These are both short-lived isotopes, with half-lives of 35 days and 36.3 days respectively.
They appear due to cosmogenic activation of the xenon when the xenon is at the surface level\footnote{of the mine. Not the surface of the LXe.}.
Once underground these components decay away and so are not of concern in future data runs.
The rates are therefore constrained by the delivery schedule of the xenon into the Davis cavern \cite{lz_argon37_ref}.
There are expected to be $\approx$100 ${}^{37}$Ar events in this data set but there a very large uncertainty meaning that it could be has high at 300.

\par
The two NR contributions considered are from ${}^{8}$B solar neutrinos and neutrons from the detector.
There are expected to be 0 detector neutrons in this dataset, a value constrained by the radioassay of the detector components prior to construction \cite{LZ_assay_ref}.

\par
The expected contribution from each component in the model is \autoref{tab:sr1_ws_lz_backgrounds}.
The uncertainties on the values shown indicate the constraints placed on each component during the PLR.

\begin{table}[]
    \centering
    \begin{tabular}{c|c}
        Background Component     & Expected Number of Events  \\ \hline
        $\beta$ decays + Det. ER & 218 $\pm$ 36 \\
        $\nu$ ER                 & 27.3 $\pm$ 1.6 \\
        ${}^{127}$Xe             & 9.3 $\pm$ 0.8 \\
        ${}^{124}$Xe             & 5.0 $\pm$ 1.4 \\
        ${}^{136}$Xe             & 15.2 $\pm$ 2.4 \\
        ${}^{8}$B CE$\nu$NS      & 0.15 $\pm$ 0.01 \\
        Accidentals              & 1.2 $\pm$ 0.3 \\
        ${}^{37}$Ar              & [0, 291]      \\
        neutrons                 & 0.0${}^{+0.2}$
    \end{tabular}
    \caption{Expected number of events from each component in the background model.}
    \label{tab:sr1_ws_lz_backgrounds}
\end{table}

\par
The best-fit background model is shown in \autoref{fig:final_data_points} with the the data set overlaid.
The best-fit values were identical to the SI result, as such the plot is shown from the SI result \cite{lz_ws_sr1_ref}.
The WIMP contours (shown in purple) are for a 30 GeV/c${^2}$ which is analogous to $\Operator$1 interactions of the same mass.

\begin{figure}
    \centering
    \includegraphics[width=10cm]{Figures/EFT/All_SR1_Plots/final_data_points.png}
    \caption{SR1 vanilla WIMP-search data (black points) overlaid on the background model.
    The contours shown for each distribution are 1 and 2 $\sigma$.
    ${}^{8}$B is shown in green, ${}^{37}$Ar is shown in orange, the combined ER spectrum in grey.
    The red lines are the NR band with the medium, 10\% and 90\% quantiles.
    The distribution from a 30 GeV WIMP is also shown in purple. 
    }
    \label{fig:final_data_points}
\end{figure}

\subsection{Limits}
\par
Consistent with the SI and SD search, a null result was found, setting limits on the operator couplings.
These are shown in \autoref{fig:EFT_Result_Projected_Sensitivity} in the same dimensionless scale as the projected limits.
\par
World-leading sensitivity has been achieved for all operators at at least some masses.
In several cases, such as $\Operator$6 the sensitivity of LZ is less than that of both LUX and XENON100. 
This is due to the recoil distributions of these operators peaking outside of the ROI used here for high WIMP masses.
Despite this reduced parameter space, the huge increase in exposure has made LZ very competitive.

\begin{figure}[!htbp]%
\centering
\begin{tikzpicture}
\centering
    \begin{axis}[
            ylabel={log${}_{10}$(S2$_c$ [phd])},
            xlabel={S1 [phd]},
            width=15cm,
            height=8cm,
            ymin=2.75, ymax=4.5,
            xmin=0, xmax=80,
            ]
        \addplot[black, only marks,]
            table [x=S1c, y=log_10S2c]
            {Data/HENR/sr1_ws/ws_data.dat};
    \end{axis}
\end{tikzpicture}
    \caption{SR1 search data after all cuts.}
    \label{fig:henr_ws_sr1_events}
\end{figure}

For SR1 fewer parameters were taken into consideration with the PLR working in [S1,logS2] space.
Increasing the set of parameters will allow for improved discrimination and a potential discovery but runs into significant computational limitations as discovered in both \cite{nicolelarsen_thesis_ref, shaunalsum_thesis_ref, billyboxer_thesis_ref}, a different approach such as that described in \cite{flamenest_ref} and \cite{lux_ml_plr_ref} offer some hope for this.

\input{Chapters/Analysis_EffectiveFieldTheory/Figures/WIMP_Search/ws_limits}

