\section{Signal Model}
\par
The actual signal model is actually produced by taking a theoretical event rate spectrum (produced by a Mathematica package - DMFormFactor - developed by XXX) and applying the analysis acceptance and detector response.
Passing through LZLama or PdfMaker, this turns the above into event observable - S1 and LogS2).
\par
DMFormFactor takes performs the calculations described previously.
Each Xenon isotope is evaluated separately and is weighted by the Xenon abundance before being added together to produce an energy spectrum.
As mentioned previously, the energy spectrum is dependant upon the choice of coupling and the WIMP mass.
\par
The choice has been made to perform analysis in iso-scalar rather than proton-neutron. 
There is always an argument that one approach is more appropriate than the other
\par
In \autoref{tab:DMFormFactor_parameters} the parameters used for this are shown.
The majority of these values are to match those from previous studies and the Dark Matter community standards.

\begin{table}[]
    \centering
    \begin{tabular}{c|c}
        Parameter   & Value  \\ \hline
        $\nu_0$     & 220$km s^{-1}$ \\
        $\nu_{esc}$ & 544$km s^{-1}$ \\
        $\rho_{0}$     & 0.3 $GeV/cm^{3}$ \\
        $\nu_E$     & 245 $km s^{-1}$ 
    \end{tabular}
    \caption{DMFormFactor parameters used for the standard halo model. CITE XXX}
    \label{tab:DMFormFactor_parameters}
\end{table}


\par
A subset of the recoil spectra calculated are shown in \autoref{fig:HENR_Spin_Recoil_Spectrum} and \autoref{fig:HENR_NotSpin_Recoil_Spectrum}.
The masses for which the energy spectra were calculated were;
[5, 7, 10, 12, 14, 21, 33, 50, 100, 200, 400, 1000, 4000] GeV all of which can be seen in Annex XXX.
The reason for this is that the shape of the limit produced is fairly well understood, so only a subset of WIMP masses are needed.

\newcommand{\allrecoiloperators}{01s,03s,04s,05s,06s,07s,08s,09s,10s,11s,12s,13s,14s,15s}
\newcommand{\spinrecoiloperators}{01s,04s,06s,07s,09s,10s,11s,14s}
\newcommand{\roguerecoiloperators}{03s,05s,08s,12s,13s,15s}

\begin{figure}[!htbp]%
\centering
\begin{tikzpicture}
\centering
  \begin{groupplot}[view={0}{90},
    group style = {group size = 2 by 1}]
    \nextgroupplot[
    xlabel=Recoil Energy (keV),
    ylabel=Differential Rate (kg/day/keV),
    legend pos=north east,
    mark size=0pt,
    xmin=1, xmax=400, xmode=log,
    ymin=1e-14, ymax=1e3, ymode=log]
    \foreach \henrop in \spinrecoiloperators{
                \addplot 
                    table
                    %{Data/HENR/Signal/Recoils/dRdEr_EFT_O01s_m5GeV_0.txt};
                    {Data/HENR/Projected_Sensitivity/Signal/Recoils/dRdEr_EFT_O\henrop_m5GeV_0.dat};
            }
    \nextgroupplot[
    xlabel=Recoil Energy (keV),
    legend pos=north east,
    mark size=0pt,
    xmin=1, xmax=400, xmode=log,
    ymin=1e-14, ymax=1e3, ymode=log]
    \foreach \henrop in \spinrecoiloperators{
                \addplot
                    table
                    %{Data/HENR/Signal/Recoils/dRdEr_EFT_O01s_m5GeV_0.txt};
                    {Data/HENR/Projected_Sensitivity/Signal/Recoils/dRdEr_EFT_O\henrop_m21GeV_0.dat};
            }
  \end{groupplot}
\end{tikzpicture}
\caption{Literal Aids
\textbf{Left:} $Gd^{156}$ de-excitation path.
\textbf{Right:} $Gd^{158}$ de-excitation path.
}
\label{fig:HENR_Spin_Recoil_Spectrum}
\end{figure}

\begin{figure}[!htbp]%
\centering
\begin{tikzpicture}
\centering
  \begin{groupplot}[view={0}{90},
    group style = {group size = 2 by 1}]
    \nextgroupplot[
    xlabel=Recoil Energy (keV),
    ylabel=Differential Rate (kg/day/keV),
    legend pos=north east,
    mark size=0pt,
    xmin=1, xmax=400, xmode=log,
    ymin=1e-14, ymax=1e3, ymode=log]
    \foreach \henrop in \roguerecoiloperators{
                \addplot 
                    table
                    %{Data/HENR/Signal/Recoils/dRdEr_EFT_O01s_m5GeV_0.txt};
                    {Data/HENR/Projected_Sensitivity/Signal/Recoils/dRdEr_EFT_O\henrop_m5GeV_0.dat};
            }
    \nextgroupplot[
    xlabel=Recoil Energy (keV),
    legend pos=north east,
    mark size=0pt,
    xmin=1, xmax=400, xmode=log,
    ymin=1e-14, ymax=1e3, ymode=log]
    \foreach \henrop in \roguerecoiloperators{
                \addplot
                    table
                    %{Data/HENR/Signal/Recoils/dRdEr_EFT_O01s_m5GeV_0.txt};
                    {Data/HENR/Projected_Sensitivity/Signal/Recoils/dRdEr_EFT_O\henrop_m5GeV_0.dat};
            }
  \end{groupplot}
\end{tikzpicture}
\caption{Literal Aids
\textbf{Left:} $Gd^{156}$ de-excitation path.
\textbf{Right:} $Gd^{158}$ de-excitation path.
}
\label{fig:HENR_NotSpin_Recoil_Spectrum}
\end{figure}

\par
When performing this analysis, an important choice is the coupling choice.


\paragraph{Copied from Billy}
\par
To allow for ease of comparison to previous limit setting on the inelastic36
WIMP-nucleon EFT operators by the XENON collaboration [? ], this analysis was conducted in the isoscalar basis.37
In the isoscalar basis, the charge densities of the nucleons are effectively averaged such that the interaction becomes38
indiscriminate to the type of nucleon involved, even though both the isoscalar and proton-neutron basis provide insight.39
For this WIMP-nucleon EFT, the UV scale governing the physics is far higher than the energies that are probed in the40
experiment. At these lower energies, the UV interactions have to be reduced to effective ones, which is done at the u,41
d and s quark level. The assumption is made that the couplings to each quark at these scales are roughly equivalent.42
Therefore the mass differences of these quarks are negligible at the high-energy scale of the underlying physics, and the43
interaction would be isoscalar. By performing this analysis in the isoscalar basis, it is possible to test this assumption’s44
validity. Additionally, by using either an isoscalar or isovector basis, the target’s nuclear state can be considered as45
isospin symmetric; a property of the strong force that can aid in simplifying the analysis.

\begin{figure}
    \centering
    \includegraphics[width=0.5\textwidth]{Figures/Placeholder.png}
    \caption{Integrated rate for each operator for a 1000GeV DM particle}
    \label{fig:operator_integrated_rate}
\end{figure}

\begin{table}[]
    \centering
    \begin{tabular}{c|c}
        Parameter   & Value  \\ \hline
        $g_{1}$     & 0.119 \\
        $g_{2}$     & 79.1  
    \end{tabular}
    \caption{Key detector parameters for the LXe-TPC parameters as used in \cite{LZ_projected_sensitivity_paper_ref}}
    \label{tab:projected_sensitivity_detector_parameters}
\end{table}