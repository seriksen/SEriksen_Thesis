\section{Signal Model}
\label{sec:eft_signal_models}
\par
Both of the analyses described in this chapter were performed in the \{$isoscalar,isovector$\} basis.
This was chosen over \{$neutron,proton$\} as it allows for comparison to both LUX and XENON collaboration results and in future, Panda-X results\footnote{Panda-X have published Lagrangian results for an EFT analysis in the \{$isoscalar,isovector$\} basis \cite{pandax_2_eft_ref}}.


\par
The assumption taken was that a single operator dominates the interaction and that there is no mixing or interference between operators.
As such only one operator was considered at a time, with its coupling to nucleons set to 1 and all others to 0.
In this case, the differential recoil rate scales linearly with $c^{a^2}_i$, becoming:
\begin{equation}
    \frac{dR}{dE_R} = \frac{c^{a^2}_i \rho_\chi}{32 \pi m^3_\chi m^2_N} \int_{v_{min}} \frac{f(\vec{v})}{v} F^{a,a}_{i,i} (v^2, q^2) d^3 v
\end{equation}


\par
The differential recoil spectra were generated using \textit{DMFormFactor}, a \textit{Mathematica} package \cite{dmformfactor_ref}.
This was chosen over other software solutions such as \textit{WimpyDD} (written in python) \cite{wimpydd_ref} and \textit{DMFortFactor} (written in fortran) \cite{dmfortfactor_ref} as it is the same as was used 
This was chosen over the other software options mentioned in \autoref{sec:eft_theory} as it is the same as is used by the LUX, Panda-X and XENON collaborations.
The limits produced by XENON100 \cite{xenon100_eft_ref} and LUX \cite{LUX_RUN4_EFT_2021} used \textit{DMFormFactor} v6.0.
The recent publication by Panda-X \cite{pandax_2_eft_ref} used an unpublished version of the codebase which uses different nuclear form factors\footnote{unpublished at time of writing, however it is expected to be released in early 2023}.
In order to maintain consistency with operator limit analyses, \textit{DMFormFactor} v6.0 was used here.
Within \textit{DMFormFactor}, the natural abundance of Xe isotopes is folded in.
%The recent publication on $\mathcal{L}$ by Panda-X used this version \cite{pandax_2_eft_ref}.
%For a selection of operators the difference between the resultant form factors is shown in XXX.
%The change in the resultant Bessel functions alter the shape of the recoils.


\par
Between the sensitivity study and SR1, the parameters for the Standard Halo Model were altered.
These are shown in \autoref{tab:DMFormFactor_parameters}.
The complete analysis framework also changed between the two.


To improve the realism of the recoil and to maintain consistency with previous experiments the natural abundance of Xe isotopes were folded in.
Each operator was generated out to 1000 keV and in the isoscalar basis and a selection of the resultant differential rates are shown in \autoref{fig:HENR_Spin_Recoil_Spectrum}.
Each Operator was performed for a WIMP mass of [5, 7, 10, 12, 14, 21, 33, 50, 100, 200, 400, 1000, 4000] GeV.


\par
The operators recoil spectra for a mass of m = 50 GeV/c$^2$ and m = 1000 GeV/c$^2$ are shown in \autoref{fig:HENR_recoil_spectra_m50} and \autoref{fig:HENR_recoil_spectra_m1000}.

\begin{figure}[!htbp]%
\centering
\begin{tikzpicture}
\centering
  \begin{groupplot}[view={0}{90},
    group style = {group size = 1 by 3}]
    \iffalse
    \nextgroupplot[
    xlabel=Recoil Energy (keV),
    ylabel=Differential Rate (kg/day/keV),
    legend pos=north east,
    mark size=0pt,
    xmin=1, xmax=400, xmode=log,
    ymin=1e-14, ymax=1e3, ymode=log]
    \addplot table {Data/HENR/Projected_Sensitivity/Signal/Recoils/dRdEr_EFT_O01s_m50GeV_0.dat};
    \addplot table {Data/HENR/Projected_Sensitivity/Signal/Recoils/dRdEr_EFT_O11s_m50GeV_0.dat};       
    \legend{$\Operator$1s, $\Operator$11s};
    \fi
    \nextgroupplot[
    xlabel=Recoil Energy (keV),
    ylabel=Differential Rate (kg/day/keV),
    legend pos=north east,
    mark size=0pt,
    xmin=1, xmax=400, xmode=log,
    ymin=1e-14, ymax=1e3, ymode=log,
    cycle list name=color list]
    \addplot table {Data/HENR/Projected_Sensitivity/Signal/Recoils/dRdEr_EFT_O04s_m50GeV_0.dat};
    \addplot table {Data/HENR/Projected_Sensitivity/Signal/Recoils/dRdEr_EFT_O06s_m50GeV_0.dat};     
    \addplot table {Data/HENR/Projected_Sensitivity/Signal/Recoils/dRdEr_EFT_O07s_m50GeV_0.dat};
    \addplot table {Data/HENR/Projected_Sensitivity/Signal/Recoils/dRdEr_EFT_O09s_m50GeV_0.dat};
    \addplot table {Data/HENR/Projected_Sensitivity/Signal/Recoils/dRdEr_EFT_O10s_m50GeV_0.dat};
    \addplot table {Data/HENR/Projected_Sensitivity/Signal/Recoils/dRdEr_EFT_O14s_m50GeV_0.dat};
    \legend{$\Operator$4s, $\Operator$6s, $\Operator$7s, $\Operator$9s, $\Operator$10s, $\Operator$14s};
    \iffalse
    \nextgroupplot[
    xlabel=Recoil Energy (keV),
    ylabel=Differential Rate (kg/day/keV),
    legend pos=north east,
    mark size=0pt,
    xmin=1, xmax=400, xmode=log,
    ymin=1e-14, ymax=1e3, ymode=log]
    \addplot table {Data/HENR/Projected_Sensitivity/Signal/Recoils/dRdEr_EFT_O03s_m50GeV_0.dat};
    \addplot table {Data/HENR/Projected_Sensitivity/Signal/Recoils/dRdEr_EFT_O05s_m50GeV_0.dat};
    \addplot table {Data/HENR/Projected_Sensitivity/Signal/Recoils/dRdEr_EFT_O08s_m50GeV_0.dat};
    \addplot table {Data/HENR/Projected_Sensitivity/Signal/Recoils/dRdEr_EFT_O12s_m50GeV_0.dat};
    \addplot table {Data/HENR/Projected_Sensitivity/Signal/Recoils/dRdEr_EFT_O13s_m50GeV_0.dat};
    \addplot table {Data/HENR/Projected_Sensitivity/Signal/Recoils/dRdEr_EFT_O15s_m50GeV_0.dat};       
    \legend{$\Operator$3s, $\Operator$5s, $\Operator$8s, $\Operator$12s, $\Operator$13s, $\Operator$15s};
    \fi
  \end{groupplot}
\end{tikzpicture}
\caption{Differential recoil spectra for the fourteen non-relativistic EFT WIMP-nucleon operators with a WIMP mass of 50 GeV/c$^2$.
         \textbf{Top:} spin-independent \textbf{Middle:} spin-dependent \textbf{Bottom:} novel }
\label{fig:HENR_recoil_spectra_m50}
\end{figure}


\begin{figure}[!htbp]%
\centering
\begin{tikzpicture}
\centering
  \begin{groupplot}[view={0}{90},
    group style = {group size = 1 by 3}]
    
    \nextgroupplot[
    xlabel=Recoil Energy (keV),
    ylabel=Differential Rate (kg/day/keV),
    legend pos=north east,
    mark size=0pt,
    xmin=1, xmax=400, xmode=log,
    ymin=1e-14, ymax=1e3, ymode=log]
    \addplot table {Data/HENR/Projected_Sensitivity/Signal/Recoils/dRdEr_EFT_O01s_m33GeV_0.dat};
    \addplot table {Data/HENR/Projected_Sensitivity/Signal/Recoils/dRdEr_EFT_O11s_m33GeV_0.dat};       
    \legend{$\Operator$1s, $\Operator$11s};
    \iffalse
    \nextgroupplot[
    xlabel=Recoil Energy (keV),
    ylabel=Differential Rate (kg/day/keV),
    legend pos=north east,
    mark size=0pt,
    xmin=1, xmax=400, xmode=log,
    ymin=1e-14, ymax=1e3, ymode=log,
    cycle list name=color list]
    \addplot table {Data/HENR/Projected_Sensitivity/Signal/Recoils/dRdEr_EFT_O04s_m1000GeV_0.dat};
    \addplot table {Data/HENR/Projected_Sensitivity/Signal/Recoils/dRdEr_EFT_O06s_m1000GeV_0.dat};     
    \addplot table {Data/HENR/Projected_Sensitivity/Signal/Recoils/dRdEr_EFT_O07s_m1000GeV_0.dat};
    \addplot table {Data/HENR/Projected_Sensitivity/Signal/Recoils/dRdEr_EFT_O09s_m1000GeV_0.dat};
    \addplot table {Data/HENR/Projected_Sensitivity/Signal/Recoils/dRdEr_EFT_O10s_m1000GeV_0.dat};
    \addplot table {Data/HENR/Projected_Sensitivity/Signal/Recoils/dRdEr_EFT_O14s_m1000GeV_0.dat};
    \legend{$\Operator$4s, $\Operator$6s, $\Operator$7s, $\Operator$9s, $\Operator$10s, $\Operator$14s};
    
    \nextgroupplot[
    xlabel=Recoil Energy (keV),
    ylabel=Differential Rate (kg/day/keV),
    legend pos=north east,
    mark size=0pt,
    xmin=1, xmax=400, xmode=log,
    ymin=1e-14, ymax=1e3, ymode=log]
    \addplot table {Data/HENR/Projected_Sensitivity/Signal/Recoils/dRdEr_EFT_O03s_m1000GeV_0.dat};
    \addplot table {Data/HENR/Projected_Sensitivity/Signal/Recoils/dRdEr_EFT_O05s_m1000GeV_0.dat};
    \addplot table {Data/HENR/Projected_Sensitivity/Signal/Recoils/dRdEr_EFT_O08s_m1000GeV_0.dat};
    \addplot table {Data/HENR/Projected_Sensitivity/Signal/Recoils/dRdEr_EFT_O12s_m1000GeV_0.dat};
    \addplot table {Data/HENR/Projected_Sensitivity/Signal/Recoils/dRdEr_EFT_O13s_m1000GeV_0.dat};
    \addplot table {Data/HENR/Projected_Sensitivity/Signal/Recoils/dRdEr_EFT_O15s_m1000GeV_0.dat};       
    \legend{$\Operator$3s, $\Operator$5s, $\Operator$8s, $\Operator$12s, $\Operator$13s, $\Operator$15s};
    \fi
  \end{groupplot}
\end{tikzpicture}
\caption{Differential recoil spectra for the fourteen non-relativistic EFT WIMP-nucleon operators with a WIMP mass of 1000 GeV/c$^2$.
         \textbf{Top:} spin-independent \textbf{Middle:} spin-dependent \textbf{Bottom:} novel }
\label{fig:HENR_recoil_spectra_m1000}
\end{figure}

\iffalse

\begin{figure}[!htbp]%
\centering
\begin{tikzpicture}
\centering
  \begin{groupplot}[view={0}{90},
    group style = {group size = 3 by 1}]
    \nextgroupplot[
    xlabel=Recoil Energy (keV),
    ylabel=Differential Rate (kg/day/keV),
    legend pos=north east,
    mark size=0pt,
    xmin=1, xmax=400, xmode=log,
    ymin=1e-14, ymax=1e3, ymode=log]
    \foreach \henrop in \roguerecoiloperators{
                \addplot 
                    table
                    %{Data/HENR/Signal/Recoils/dRdEr_EFT_O01s_m5GeV_0.txt};
                    {Data/HENR/Projected_Sensitivity/Signal/Recoils/dRdEr_EFT_O\henrop_m5GeV_0.dat};
            }
    \nextgroupplot[
    xlabel=Recoil Energy (keV),
    legend pos=north east,
    mark size=0pt,
    xmin=1, xmax=400, xmode=log,
    ymin=1e-14, ymax=1e3, ymode=log]
    \foreach \henrop in \roguerecoiloperators{
                \addplot
                    table
                    %{Data/HENR/Signal/Recoils/dRdEr_EFT_O01s_m5GeV_0.txt};
                    {Data/HENR/Projected_Sensitivity/Signal/Recoils/dRdEr_EFT_O\henrop_m5GeV_0.dat};
            }
  \end{groupplot}
\end{tikzpicture}
\caption{Literal Aids
\textbf{Left:} $Gd^{156}$ de-excitation path.
\textbf{Right:} $Gd^{158}$ de-excitation path.
}
\label{fig:HENR_NotSpin_Recoil_Spectrum}
\end{figure}


\begin{figure}[!htbp]%
\centering
\begin{tikzpicture}
\centering
  \begin{groupplot}[view={0}{90},
    group style = {group size = 2 by 1}]
    \nextgroupplot[
    xlabel=Recoil Energy (keV),
    ylabel=Differential Rate (kg/day/keV),
    legend pos=north east,
    mark size=0pt,
    xmin=1, xmax=400, xmode=log,
    ymin=1e-14, ymax=1e3, ymode=log]
    \foreach \henrop in \spinrecoiloperators{
                \addplot 
                    table
                    %{Data/HENR/Signal/Recoils/dRdEr_EFT_O01s_m5GeV_0.txt};
                    {Data/HENR/Projected_Sensitivity/Signal/Recoils/dRdEr_EFT_O\henrop_m5GeV_0.dat};
            }
    \nextgroupplot[
    xlabel=Recoil Energy (keV),
    legend pos=north east,
    mark size=0pt,
    xmin=1, xmax=400, xmode=log,
    ymin=1e-14, ymax=1e3, ymode=log]
    \foreach \henrop in \spinrecoiloperators{
                \addplot
                    table
                    %{Data/HENR/Signal/Recoils/dRdEr_EFT_O01s_m5GeV_0.txt};
                    {Data/HENR/Projected_Sensitivity/Signal/Recoils/dRdEr_EFT_O\henrop_m21GeV_0.dat};
            }
  \end{groupplot}
\end{tikzpicture}
\caption{Literal Aids
\textbf{Left:} $Gd^{156}$ de-excitation path.
\textbf{Right:} $Gd^{158}$ de-excitation path.
}
\label{fig:HENR_Spin_Recoil_Spectrum}
\end{figure}

\begin{figure}[!htbp]%
\centering
\begin{tikzpicture}
\centering
  \begin{groupplot}[view={0}{90},
    group style = {group size = 2 by 1}]
    \nextgroupplot[
    xlabel=Recoil Energy (keV),
    ylabel=Differential Rate (kg/day/keV),
    legend pos=north east,
    mark size=0pt,
    xmin=1, xmax=400, xmode=log,
    ymin=1e-14, ymax=1e3, ymode=log]
    \foreach \henrop in \roguerecoiloperators{
                \addplot 
                    table
                    %{Data/HENR/Signal/Recoils/dRdEr_EFT_O01s_m5GeV_0.txt};
                    {Data/HENR/Projected_Sensitivity/Signal/Recoils/dRdEr_EFT_O\henrop_m5GeV_0.dat};
            }
    \nextgroupplot[
    xlabel=Recoil Energy (keV),
    legend pos=north east,
    mark size=0pt,
    xmin=1, xmax=400, xmode=log,
    ymin=1e-14, ymax=1e3, ymode=log]
    \foreach \henrop in \roguerecoiloperators{
                \addplot
                    table
                    %{Data/HENR/Signal/Recoils/dRdEr_EFT_O01s_m5GeV_0.txt};
                    {Data/HENR/Projected_Sensitivity/Signal/Recoils/dRdEr_EFT_O\henrop_m5GeV_0.dat};
            }
  \end{groupplot}
\end{tikzpicture}
\caption{Literal Aids
\textbf{Left:} $Gd^{156}$ de-excitation path.
\textbf{Right:} $Gd^{158}$ de-excitation path.
}
\label{fig:HENR_NotSpin_Recoil_Spectrum}
\end{figure}

\fi

\iffalse
\par
Let us first discuss the signal model which we expect.
The differential rate we saw in CHAPTER XXX though difficult to calculate is fairly easy for the end user, with a suite of tools written to determine this.
Some of the original authors of the theory produced one such program, DMFormFactor (written in Mathematica) \cite{dmformfactor_ref}.
Other software solutions such as WimpyDD (python) and DMFortFactor (fortran) also exist.
None of these produce exactly the same rate which in large part is due to differing nuclear form factors and therefore different.

\par
The package used here is DMFormFactor v6.0 as was the choice for both LUX and XENON collaboration results\footnote{At the time of writing this is the newest public release. However these is a newer version which uses updated form factors which was used in \cite{pandax_2_eft_ref}.}.
The Standard Halo Model parameters used to describe the dark matter are summarised in \autoref{tab:DMFormFactor_parameters}.
\begin{table}[]
    \centering
    \begin{tabular}{c|c}
        Parameter         & Value  \\ \hline
        $\nu_0$           & 220$km s^{-1}$ \\
        $\nu_{esc}$       & 544$km s^{-1}$ \\
        $\rho_{\chi}$     & 0.3 $GeV/cm^{3}$ \\
        $|\nu_E|$         & 245 $km s^{-1}$ 
    \end{tabular}
    \caption{DMFormFactor parameters used for the standard halo model. CITE XXX}
    \label{tab:DMFormFactor_parameters}
\end{table}
To improve the realism of the recoil and to maintain consistency with previous experiments the natural abundance of Xe isotopes were folded in.
Each operator was generated out to 1000 keV and in the isoscalar basis and a selection of the resultant differential rates are shown in \autoref{fig:HENR_Spin_Recoil_Spectrum}.
Each Operator was performed for a WIMP mass of [5, 7, 10, 12, 14, 21, 33, 50, 100, 200, 400, 1000, 4000] GeV.

\begin{figure}[!htbp]%
\centering
\begin{tikzpicture}
\centering
  \begin{groupplot}[view={0}{90},
    group style = {group size = 1 by 3}]
    \iffalse
    \nextgroupplot[
    xlabel=Recoil Energy (keV),
    ylabel=Differential Rate (kg/day/keV),
    legend pos=north east,
    mark size=0pt,
    xmin=1, xmax=400, xmode=log,
    ymin=1e-14, ymax=1e3, ymode=log]
    \addplot table {Data/HENR/Projected_Sensitivity/Signal/Recoils/dRdEr_EFT_O01s_m50GeV_0.dat};
    \addplot table {Data/HENR/Projected_Sensitivity/Signal/Recoils/dRdEr_EFT_O11s_m50GeV_0.dat};       
    \legend{$\Operator$1s, $\Operator$11s};
    \fi
    \nextgroupplot[
    xlabel=Recoil Energy (keV),
    ylabel=Differential Rate (kg/day/keV),
    legend pos=north east,
    mark size=0pt,
    xmin=1, xmax=400, xmode=log,
    ymin=1e-14, ymax=1e3, ymode=log,
    cycle list name=color list]
    \addplot table {Data/HENR/Projected_Sensitivity/Signal/Recoils/dRdEr_EFT_O04s_m50GeV_0.dat};
    \addplot table {Data/HENR/Projected_Sensitivity/Signal/Recoils/dRdEr_EFT_O06s_m50GeV_0.dat};     
    \addplot table {Data/HENR/Projected_Sensitivity/Signal/Recoils/dRdEr_EFT_O07s_m50GeV_0.dat};
    \addplot table {Data/HENR/Projected_Sensitivity/Signal/Recoils/dRdEr_EFT_O09s_m50GeV_0.dat};
    \addplot table {Data/HENR/Projected_Sensitivity/Signal/Recoils/dRdEr_EFT_O10s_m50GeV_0.dat};
    \addplot table {Data/HENR/Projected_Sensitivity/Signal/Recoils/dRdEr_EFT_O14s_m50GeV_0.dat};
    \legend{$\Operator$4s, $\Operator$6s, $\Operator$7s, $\Operator$9s, $\Operator$10s, $\Operator$14s};
    \iffalse
    \nextgroupplot[
    xlabel=Recoil Energy (keV),
    ylabel=Differential Rate (kg/day/keV),
    legend pos=north east,
    mark size=0pt,
    xmin=1, xmax=400, xmode=log,
    ymin=1e-14, ymax=1e3, ymode=log]
    \addplot table {Data/HENR/Projected_Sensitivity/Signal/Recoils/dRdEr_EFT_O03s_m50GeV_0.dat};
    \addplot table {Data/HENR/Projected_Sensitivity/Signal/Recoils/dRdEr_EFT_O05s_m50GeV_0.dat};
    \addplot table {Data/HENR/Projected_Sensitivity/Signal/Recoils/dRdEr_EFT_O08s_m50GeV_0.dat};
    \addplot table {Data/HENR/Projected_Sensitivity/Signal/Recoils/dRdEr_EFT_O12s_m50GeV_0.dat};
    \addplot table {Data/HENR/Projected_Sensitivity/Signal/Recoils/dRdEr_EFT_O13s_m50GeV_0.dat};
    \addplot table {Data/HENR/Projected_Sensitivity/Signal/Recoils/dRdEr_EFT_O15s_m50GeV_0.dat};       
    \legend{$\Operator$3s, $\Operator$5s, $\Operator$8s, $\Operator$12s, $\Operator$13s, $\Operator$15s};
    \fi
  \end{groupplot}
\end{tikzpicture}
\caption{Differential recoil spectra for the fourteen non-relativistic EFT WIMP-nucleon operators with a WIMP mass of 50 GeV/c$^2$.
         \textbf{Top:} spin-independent \textbf{Middle:} spin-dependent \textbf{Bottom:} novel }
\label{fig:HENR_recoil_spectra_m50}
\end{figure}


\begin{figure}[!htbp]%
\centering
\begin{tikzpicture}
\centering
  \begin{groupplot}[view={0}{90},
    group style = {group size = 1 by 3}]
    
    \nextgroupplot[
    xlabel=Recoil Energy (keV),
    ylabel=Differential Rate (kg/day/keV),
    legend pos=north east,
    mark size=0pt,
    xmin=1, xmax=400, xmode=log,
    ymin=1e-14, ymax=1e3, ymode=log]
    \addplot table {Data/HENR/Projected_Sensitivity/Signal/Recoils/dRdEr_EFT_O01s_m33GeV_0.dat};
    \addplot table {Data/HENR/Projected_Sensitivity/Signal/Recoils/dRdEr_EFT_O11s_m33GeV_0.dat};       
    \legend{$\Operator$1s, $\Operator$11s};
    \iffalse
    \nextgroupplot[
    xlabel=Recoil Energy (keV),
    ylabel=Differential Rate (kg/day/keV),
    legend pos=north east,
    mark size=0pt,
    xmin=1, xmax=400, xmode=log,
    ymin=1e-14, ymax=1e3, ymode=log,
    cycle list name=color list]
    \addplot table {Data/HENR/Projected_Sensitivity/Signal/Recoils/dRdEr_EFT_O04s_m1000GeV_0.dat};
    \addplot table {Data/HENR/Projected_Sensitivity/Signal/Recoils/dRdEr_EFT_O06s_m1000GeV_0.dat};     
    \addplot table {Data/HENR/Projected_Sensitivity/Signal/Recoils/dRdEr_EFT_O07s_m1000GeV_0.dat};
    \addplot table {Data/HENR/Projected_Sensitivity/Signal/Recoils/dRdEr_EFT_O09s_m1000GeV_0.dat};
    \addplot table {Data/HENR/Projected_Sensitivity/Signal/Recoils/dRdEr_EFT_O10s_m1000GeV_0.dat};
    \addplot table {Data/HENR/Projected_Sensitivity/Signal/Recoils/dRdEr_EFT_O14s_m1000GeV_0.dat};
    \legend{$\Operator$4s, $\Operator$6s, $\Operator$7s, $\Operator$9s, $\Operator$10s, $\Operator$14s};
    
    \nextgroupplot[
    xlabel=Recoil Energy (keV),
    ylabel=Differential Rate (kg/day/keV),
    legend pos=north east,
    mark size=0pt,
    xmin=1, xmax=400, xmode=log,
    ymin=1e-14, ymax=1e3, ymode=log]
    \addplot table {Data/HENR/Projected_Sensitivity/Signal/Recoils/dRdEr_EFT_O03s_m1000GeV_0.dat};
    \addplot table {Data/HENR/Projected_Sensitivity/Signal/Recoils/dRdEr_EFT_O05s_m1000GeV_0.dat};
    \addplot table {Data/HENR/Projected_Sensitivity/Signal/Recoils/dRdEr_EFT_O08s_m1000GeV_0.dat};
    \addplot table {Data/HENR/Projected_Sensitivity/Signal/Recoils/dRdEr_EFT_O12s_m1000GeV_0.dat};
    \addplot table {Data/HENR/Projected_Sensitivity/Signal/Recoils/dRdEr_EFT_O13s_m1000GeV_0.dat};
    \addplot table {Data/HENR/Projected_Sensitivity/Signal/Recoils/dRdEr_EFT_O15s_m1000GeV_0.dat};       
    \legend{$\Operator$3s, $\Operator$5s, $\Operator$8s, $\Operator$12s, $\Operator$13s, $\Operator$15s};
    \fi
  \end{groupplot}
\end{tikzpicture}
\caption{Differential recoil spectra for the fourteen non-relativistic EFT WIMP-nucleon operators with a WIMP mass of 1000 GeV/c$^2$.
         \textbf{Top:} spin-independent \textbf{Middle:} spin-dependent \textbf{Bottom:} novel }
\label{fig:HENR_recoil_spectra_m1000}
\end{figure}

\iffalse

\begin{figure}[!htbp]%
\centering
\begin{tikzpicture}
\centering
  \begin{groupplot}[view={0}{90},
    group style = {group size = 3 by 1}]
    \nextgroupplot[
    xlabel=Recoil Energy (keV),
    ylabel=Differential Rate (kg/day/keV),
    legend pos=north east,
    mark size=0pt,
    xmin=1, xmax=400, xmode=log,
    ymin=1e-14, ymax=1e3, ymode=log]
    \foreach \henrop in \roguerecoiloperators{
                \addplot 
                    table
                    %{Data/HENR/Signal/Recoils/dRdEr_EFT_O01s_m5GeV_0.txt};
                    {Data/HENR/Projected_Sensitivity/Signal/Recoils/dRdEr_EFT_O\henrop_m5GeV_0.dat};
            }
    \nextgroupplot[
    xlabel=Recoil Energy (keV),
    legend pos=north east,
    mark size=0pt,
    xmin=1, xmax=400, xmode=log,
    ymin=1e-14, ymax=1e3, ymode=log]
    \foreach \henrop in \roguerecoiloperators{
                \addplot
                    table
                    %{Data/HENR/Signal/Recoils/dRdEr_EFT_O01s_m5GeV_0.txt};
                    {Data/HENR/Projected_Sensitivity/Signal/Recoils/dRdEr_EFT_O\henrop_m5GeV_0.dat};
            }
  \end{groupplot}
\end{tikzpicture}
\caption{Literal Aids
\textbf{Left:} $Gd^{156}$ de-excitation path.
\textbf{Right:} $Gd^{158}$ de-excitation path.
}
\label{fig:HENR_NotSpin_Recoil_Spectrum}
\end{figure}


\begin{figure}[!htbp]%
\centering
\begin{tikzpicture}
\centering
  \begin{groupplot}[view={0}{90},
    group style = {group size = 2 by 1}]
    \nextgroupplot[
    xlabel=Recoil Energy (keV),
    ylabel=Differential Rate (kg/day/keV),
    legend pos=north east,
    mark size=0pt,
    xmin=1, xmax=400, xmode=log,
    ymin=1e-14, ymax=1e3, ymode=log]
    \foreach \henrop in \spinrecoiloperators{
                \addplot 
                    table
                    %{Data/HENR/Signal/Recoils/dRdEr_EFT_O01s_m5GeV_0.txt};
                    {Data/HENR/Projected_Sensitivity/Signal/Recoils/dRdEr_EFT_O\henrop_m5GeV_0.dat};
            }
    \nextgroupplot[
    xlabel=Recoil Energy (keV),
    legend pos=north east,
    mark size=0pt,
    xmin=1, xmax=400, xmode=log,
    ymin=1e-14, ymax=1e3, ymode=log]
    \foreach \henrop in \spinrecoiloperators{
                \addplot
                    table
                    %{Data/HENR/Signal/Recoils/dRdEr_EFT_O01s_m5GeV_0.txt};
                    {Data/HENR/Projected_Sensitivity/Signal/Recoils/dRdEr_EFT_O\henrop_m21GeV_0.dat};
            }
  \end{groupplot}
\end{tikzpicture}
\caption{Literal Aids
\textbf{Left:} $Gd^{156}$ de-excitation path.
\textbf{Right:} $Gd^{158}$ de-excitation path.
}
\label{fig:HENR_Spin_Recoil_Spectrum}
\end{figure}

\begin{figure}[!htbp]%
\centering
\begin{tikzpicture}
\centering
  \begin{groupplot}[view={0}{90},
    group style = {group size = 2 by 1}]
    \nextgroupplot[
    xlabel=Recoil Energy (keV),
    ylabel=Differential Rate (kg/day/keV),
    legend pos=north east,
    mark size=0pt,
    xmin=1, xmax=400, xmode=log,
    ymin=1e-14, ymax=1e3, ymode=log]
    \foreach \henrop in \roguerecoiloperators{
                \addplot 
                    table
                    %{Data/HENR/Signal/Recoils/dRdEr_EFT_O01s_m5GeV_0.txt};
                    {Data/HENR/Projected_Sensitivity/Signal/Recoils/dRdEr_EFT_O\henrop_m5GeV_0.dat};
            }
    \nextgroupplot[
    xlabel=Recoil Energy (keV),
    legend pos=north east,
    mark size=0pt,
    xmin=1, xmax=400, xmode=log,
    ymin=1e-14, ymax=1e3, ymode=log]
    \foreach \henrop in \roguerecoiloperators{
                \addplot
                    table
                    %{Data/HENR/Signal/Recoils/dRdEr_EFT_O01s_m5GeV_0.txt};
                    {Data/HENR/Projected_Sensitivity/Signal/Recoils/dRdEr_EFT_O\henrop_m5GeV_0.dat};
            }
  \end{groupplot}
\end{tikzpicture}
\caption{Literal Aids
\textbf{Left:} $Gd^{156}$ de-excitation path.
\textbf{Right:} $Gd^{158}$ de-excitation path.
}
\label{fig:HENR_NotSpin_Recoil_Spectrum}
\end{figure}

\fi







\par
The actual signal model is actually produced by taking a theoretical event rate spectrum (produced by a Mathematica package - DMFormFactor - developed by XXX) and applying the analysis acceptance and detector response.
Passing through LZLama or PdfMaker, this turns the above into event observable - S1 and LogS2).
\par
DMFormFactor takes performs the calculations described previously.
Each Xenon isotope is evaluated separately and is weighted by the Xenon abundance before being added together to produce an energy spectrum.
As mentioned previously, the energy spectrum is dependant upon the choice of coupling and the WIMP mass.


\par
A subset of the recoil spectra calculated are shown in \autoref{fig:HENR_Spin_Recoil_Spectrum} and \autoref{fig:HENR_NotSpin_Recoil_Spectrum}.
The masses for which the energy spectra were calculated were;
all of which can be seen in Annex XXX.
The reason for this is that the shape of the limit produced is fairly well understood, so only a subset of WIMP masses are needed.

\begin{figure}[!htbp]%
\centering
\begin{tikzpicture}
\centering
  \begin{groupplot}[view={0}{90},
    group style = {group size = 1 by 3}]
    \iffalse
    \nextgroupplot[
    xlabel=Recoil Energy (keV),
    ylabel=Differential Rate (kg/day/keV),
    legend pos=north east,
    mark size=0pt,
    xmin=1, xmax=400, xmode=log,
    ymin=1e-14, ymax=1e3, ymode=log]
    \addplot table {Data/HENR/Projected_Sensitivity/Signal/Recoils/dRdEr_EFT_O01s_m50GeV_0.dat};
    \addplot table {Data/HENR/Projected_Sensitivity/Signal/Recoils/dRdEr_EFT_O11s_m50GeV_0.dat};       
    \legend{$\Operator$1s, $\Operator$11s};
    \fi
    \nextgroupplot[
    xlabel=Recoil Energy (keV),
    ylabel=Differential Rate (kg/day/keV),
    legend pos=north east,
    mark size=0pt,
    xmin=1, xmax=400, xmode=log,
    ymin=1e-14, ymax=1e3, ymode=log,
    cycle list name=color list]
    \addplot table {Data/HENR/Projected_Sensitivity/Signal/Recoils/dRdEr_EFT_O04s_m50GeV_0.dat};
    \addplot table {Data/HENR/Projected_Sensitivity/Signal/Recoils/dRdEr_EFT_O06s_m50GeV_0.dat};     
    \addplot table {Data/HENR/Projected_Sensitivity/Signal/Recoils/dRdEr_EFT_O07s_m50GeV_0.dat};
    \addplot table {Data/HENR/Projected_Sensitivity/Signal/Recoils/dRdEr_EFT_O09s_m50GeV_0.dat};
    \addplot table {Data/HENR/Projected_Sensitivity/Signal/Recoils/dRdEr_EFT_O10s_m50GeV_0.dat};
    \addplot table {Data/HENR/Projected_Sensitivity/Signal/Recoils/dRdEr_EFT_O14s_m50GeV_0.dat};
    \legend{$\Operator$4s, $\Operator$6s, $\Operator$7s, $\Operator$9s, $\Operator$10s, $\Operator$14s};
    \iffalse
    \nextgroupplot[
    xlabel=Recoil Energy (keV),
    ylabel=Differential Rate (kg/day/keV),
    legend pos=north east,
    mark size=0pt,
    xmin=1, xmax=400, xmode=log,
    ymin=1e-14, ymax=1e3, ymode=log]
    \addplot table {Data/HENR/Projected_Sensitivity/Signal/Recoils/dRdEr_EFT_O03s_m50GeV_0.dat};
    \addplot table {Data/HENR/Projected_Sensitivity/Signal/Recoils/dRdEr_EFT_O05s_m50GeV_0.dat};
    \addplot table {Data/HENR/Projected_Sensitivity/Signal/Recoils/dRdEr_EFT_O08s_m50GeV_0.dat};
    \addplot table {Data/HENR/Projected_Sensitivity/Signal/Recoils/dRdEr_EFT_O12s_m50GeV_0.dat};
    \addplot table {Data/HENR/Projected_Sensitivity/Signal/Recoils/dRdEr_EFT_O13s_m50GeV_0.dat};
    \addplot table {Data/HENR/Projected_Sensitivity/Signal/Recoils/dRdEr_EFT_O15s_m50GeV_0.dat};       
    \legend{$\Operator$3s, $\Operator$5s, $\Operator$8s, $\Operator$12s, $\Operator$13s, $\Operator$15s};
    \fi
  \end{groupplot}
\end{tikzpicture}
\caption{Differential recoil spectra for the fourteen non-relativistic EFT WIMP-nucleon operators with a WIMP mass of 50 GeV/c$^2$.
         \textbf{Top:} spin-independent \textbf{Middle:} spin-dependent \textbf{Bottom:} novel }
\label{fig:HENR_recoil_spectra_m50}
\end{figure}


\begin{figure}[!htbp]%
\centering
\begin{tikzpicture}
\centering
  \begin{groupplot}[view={0}{90},
    group style = {group size = 1 by 3}]
    
    \nextgroupplot[
    xlabel=Recoil Energy (keV),
    ylabel=Differential Rate (kg/day/keV),
    legend pos=north east,
    mark size=0pt,
    xmin=1, xmax=400, xmode=log,
    ymin=1e-14, ymax=1e3, ymode=log]
    \addplot table {Data/HENR/Projected_Sensitivity/Signal/Recoils/dRdEr_EFT_O01s_m33GeV_0.dat};
    \addplot table {Data/HENR/Projected_Sensitivity/Signal/Recoils/dRdEr_EFT_O11s_m33GeV_0.dat};       
    \legend{$\Operator$1s, $\Operator$11s};
    \iffalse
    \nextgroupplot[
    xlabel=Recoil Energy (keV),
    ylabel=Differential Rate (kg/day/keV),
    legend pos=north east,
    mark size=0pt,
    xmin=1, xmax=400, xmode=log,
    ymin=1e-14, ymax=1e3, ymode=log,
    cycle list name=color list]
    \addplot table {Data/HENR/Projected_Sensitivity/Signal/Recoils/dRdEr_EFT_O04s_m1000GeV_0.dat};
    \addplot table {Data/HENR/Projected_Sensitivity/Signal/Recoils/dRdEr_EFT_O06s_m1000GeV_0.dat};     
    \addplot table {Data/HENR/Projected_Sensitivity/Signal/Recoils/dRdEr_EFT_O07s_m1000GeV_0.dat};
    \addplot table {Data/HENR/Projected_Sensitivity/Signal/Recoils/dRdEr_EFT_O09s_m1000GeV_0.dat};
    \addplot table {Data/HENR/Projected_Sensitivity/Signal/Recoils/dRdEr_EFT_O10s_m1000GeV_0.dat};
    \addplot table {Data/HENR/Projected_Sensitivity/Signal/Recoils/dRdEr_EFT_O14s_m1000GeV_0.dat};
    \legend{$\Operator$4s, $\Operator$6s, $\Operator$7s, $\Operator$9s, $\Operator$10s, $\Operator$14s};
    
    \nextgroupplot[
    xlabel=Recoil Energy (keV),
    ylabel=Differential Rate (kg/day/keV),
    legend pos=north east,
    mark size=0pt,
    xmin=1, xmax=400, xmode=log,
    ymin=1e-14, ymax=1e3, ymode=log]
    \addplot table {Data/HENR/Projected_Sensitivity/Signal/Recoils/dRdEr_EFT_O03s_m1000GeV_0.dat};
    \addplot table {Data/HENR/Projected_Sensitivity/Signal/Recoils/dRdEr_EFT_O05s_m1000GeV_0.dat};
    \addplot table {Data/HENR/Projected_Sensitivity/Signal/Recoils/dRdEr_EFT_O08s_m1000GeV_0.dat};
    \addplot table {Data/HENR/Projected_Sensitivity/Signal/Recoils/dRdEr_EFT_O12s_m1000GeV_0.dat};
    \addplot table {Data/HENR/Projected_Sensitivity/Signal/Recoils/dRdEr_EFT_O13s_m1000GeV_0.dat};
    \addplot table {Data/HENR/Projected_Sensitivity/Signal/Recoils/dRdEr_EFT_O15s_m1000GeV_0.dat};       
    \legend{$\Operator$3s, $\Operator$5s, $\Operator$8s, $\Operator$12s, $\Operator$13s, $\Operator$15s};
    \fi
  \end{groupplot}
\end{tikzpicture}
\caption{Differential recoil spectra for the fourteen non-relativistic EFT WIMP-nucleon operators with a WIMP mass of 1000 GeV/c$^2$.
         \textbf{Top:} spin-independent \textbf{Middle:} spin-dependent \textbf{Bottom:} novel }
\label{fig:HENR_recoil_spectra_m1000}
\end{figure}

\iffalse

\begin{figure}[!htbp]%
\centering
\begin{tikzpicture}
\centering
  \begin{groupplot}[view={0}{90},
    group style = {group size = 3 by 1}]
    \nextgroupplot[
    xlabel=Recoil Energy (keV),
    ylabel=Differential Rate (kg/day/keV),
    legend pos=north east,
    mark size=0pt,
    xmin=1, xmax=400, xmode=log,
    ymin=1e-14, ymax=1e3, ymode=log]
    \foreach \henrop in \roguerecoiloperators{
                \addplot 
                    table
                    %{Data/HENR/Signal/Recoils/dRdEr_EFT_O01s_m5GeV_0.txt};
                    {Data/HENR/Projected_Sensitivity/Signal/Recoils/dRdEr_EFT_O\henrop_m5GeV_0.dat};
            }
    \nextgroupplot[
    xlabel=Recoil Energy (keV),
    legend pos=north east,
    mark size=0pt,
    xmin=1, xmax=400, xmode=log,
    ymin=1e-14, ymax=1e3, ymode=log]
    \foreach \henrop in \roguerecoiloperators{
                \addplot
                    table
                    %{Data/HENR/Signal/Recoils/dRdEr_EFT_O01s_m5GeV_0.txt};
                    {Data/HENR/Projected_Sensitivity/Signal/Recoils/dRdEr_EFT_O\henrop_m5GeV_0.dat};
            }
  \end{groupplot}
\end{tikzpicture}
\caption{Literal Aids
\textbf{Left:} $Gd^{156}$ de-excitation path.
\textbf{Right:} $Gd^{158}$ de-excitation path.
}
\label{fig:HENR_NotSpin_Recoil_Spectrum}
\end{figure}


\begin{figure}[!htbp]%
\centering
\begin{tikzpicture}
\centering
  \begin{groupplot}[view={0}{90},
    group style = {group size = 2 by 1}]
    \nextgroupplot[
    xlabel=Recoil Energy (keV),
    ylabel=Differential Rate (kg/day/keV),
    legend pos=north east,
    mark size=0pt,
    xmin=1, xmax=400, xmode=log,
    ymin=1e-14, ymax=1e3, ymode=log]
    \foreach \henrop in \spinrecoiloperators{
                \addplot 
                    table
                    %{Data/HENR/Signal/Recoils/dRdEr_EFT_O01s_m5GeV_0.txt};
                    {Data/HENR/Projected_Sensitivity/Signal/Recoils/dRdEr_EFT_O\henrop_m5GeV_0.dat};
            }
    \nextgroupplot[
    xlabel=Recoil Energy (keV),
    legend pos=north east,
    mark size=0pt,
    xmin=1, xmax=400, xmode=log,
    ymin=1e-14, ymax=1e3, ymode=log]
    \foreach \henrop in \spinrecoiloperators{
                \addplot
                    table
                    %{Data/HENR/Signal/Recoils/dRdEr_EFT_O01s_m5GeV_0.txt};
                    {Data/HENR/Projected_Sensitivity/Signal/Recoils/dRdEr_EFT_O\henrop_m21GeV_0.dat};
            }
  \end{groupplot}
\end{tikzpicture}
\caption{Literal Aids
\textbf{Left:} $Gd^{156}$ de-excitation path.
\textbf{Right:} $Gd^{158}$ de-excitation path.
}
\label{fig:HENR_Spin_Recoil_Spectrum}
\end{figure}

\begin{figure}[!htbp]%
\centering
\begin{tikzpicture}
\centering
  \begin{groupplot}[view={0}{90},
    group style = {group size = 2 by 1}]
    \nextgroupplot[
    xlabel=Recoil Energy (keV),
    ylabel=Differential Rate (kg/day/keV),
    legend pos=north east,
    mark size=0pt,
    xmin=1, xmax=400, xmode=log,
    ymin=1e-14, ymax=1e3, ymode=log]
    \foreach \henrop in \roguerecoiloperators{
                \addplot 
                    table
                    %{Data/HENR/Signal/Recoils/dRdEr_EFT_O01s_m5GeV_0.txt};
                    {Data/HENR/Projected_Sensitivity/Signal/Recoils/dRdEr_EFT_O\henrop_m5GeV_0.dat};
            }
    \nextgroupplot[
    xlabel=Recoil Energy (keV),
    legend pos=north east,
    mark size=0pt,
    xmin=1, xmax=400, xmode=log,
    ymin=1e-14, ymax=1e3, ymode=log]
    \foreach \henrop in \roguerecoiloperators{
                \addplot
                    table
                    %{Data/HENR/Signal/Recoils/dRdEr_EFT_O01s_m5GeV_0.txt};
                    {Data/HENR/Projected_Sensitivity/Signal/Recoils/dRdEr_EFT_O\henrop_m5GeV_0.dat};
            }
  \end{groupplot}
\end{tikzpicture}
\caption{Literal Aids
\textbf{Left:} $Gd^{156}$ de-excitation path.
\textbf{Right:} $Gd^{158}$ de-excitation path.
}
\label{fig:HENR_NotSpin_Recoil_Spectrum}
\end{figure}

\fi

\par
When performing this analysis, an important choice is the coupling choice.


\paragraph{Copied from Billy}
\par
To allow for ease of comparison to previous limit setting on the inelastic36
WIMP-nucleon EFT operators by the XENON collaboration [? ], this analysis was conducted in the isoscalar basis.37
In the isoscalar basis, the charge densities of the nucleons are effectively averaged such that the interaction becomes38
indiscriminate to the type of nucleon involved, even though both the isoscalar and proton-neutron basis provide insight.39
For this WIMP-nucleon EFT, the UV scale governing the physics is far higher than the energies that are probed in the40
experiment. At these lower energies, the UV interactions have to be reduced to effective ones, which is done at the u,41
d and s quark level. The assumption is made that the couplings to each quark at these scales are roughly equivalent.42
Therefore the mass differences of these quarks are negligible at the high-energy scale of the underlying physics, and the43
interaction would be isoscalar. By performing this analysis in the isoscalar basis, it is possible to test this assumption’s44
validity. Additionally, by using either an isoscalar or isovector basis, the target’s nuclear state can be considered as45
isospin symmetric; a property of the strong force that can aid in simplifying the analysis.

\begin{figure}
    \centering
    \includegraphics[width=0.5\textwidth]{Figures/Placeholder.png}
    \caption{Integrated rate for each operator for a 1000GeV DM particle}
    \label{fig:operator_integrated_rate}
\end{figure}

\begin{table}[]
    \centering
    \begin{tabular}{c|c}
        Parameter   & Value  \\ \hline
        $g_{1}$     & 0.119 \\
        $g_{2}$     & 79.1  
    \end{tabular}
    \caption{Key detector parameters for the LXe-TPC parameters as used in \cite{LZ_projected_sensitivity_paper_ref}}
    \label{tab:projected_sensitivity_detector_parameters}
\end{table}

\fi