\begin{figure}[]%
\centering
\begin{tikzpicture}
\centering
    \begin{axis}[
            ylabel=Rate (Hz),
            xlabel=Energy (MeV),
            width=15cm,
            height=8cm,
            grid=major,
            xmin=0.2, xmax=2.5,
            ymin=1e-4, ymode=log,]
            
        \addplot[red, only marks, mark size=1, error bar legend,
                 error bars/.cd, error bar style={color=black},
                 y dir=both, y explicit, ]
            table [x=energy,y=rate,y error plus index=2, y error minus index=3]
            {Data/OD_Backgrounds/background_constraints/od_bipo_first_alpha.dat};
            
        \addplot[blue, only marks, mark size=1, error bar legend,
                 error bars/.cd, error bar style={color=black},
                 y dir=both, y explicit, ]
            table [x=energy,y=rate,y error plus index=2, y error minus index=3]
            {Data/OD_Backgrounds/background_constraints/od_bipo_second_alpha.dat};
        
        \addplot[green, only marks, mark size=1, error bar legend,
                 error bars/.cd, error bar style={color=black},
                 y dir=both, y explicit, ]
            table [x=energy,y=rate,y error plus index=2, y error minus index=3]
            {Data/OD_Backgrounds/background_constraints/od_bipo_third_alpha.dat};
        
        \legend{${}^{219}$Rn, ${}^{215}$Po, ${}^{212}$Po}
    \end{axis}
            
\end{tikzpicture}
    \caption{Distributions of close in time and position decays where there is an $\alpha$. Each peak is an $\alpha$ decay. The pulse areas have been converted into energy using the conversion shown in \autoref{fig:od_energy_scale}.}
    \label{fig:od_extracted_alphas}
\end{figure}