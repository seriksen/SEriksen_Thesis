\begin{figure}[!htbp]%
\centering
\begin{tikzpicture}
\centering
  \begin{groupplot}[%view={0}{90},
    group style = {group size = 1 by 2,vertical sep=1.5cm,
                   horizontal sep=1.5cm},
                   height=6cm, width=0.5\textwidth]

	\nextgroupplot[height=10cm, width=10cm,
    view={0}{90},
    ylabel={Second Pulse (phd)},
    xlabel={First Pulse (phd)},
    colorbar,
    colorbar style={ylabel={Rate (Hz)},ymode=log,},
    ]
    \addplot3[
      surf,
      shader=flat corner,
	  mesh/cols=40,
	  mesh/ordering=rowwise,
	  point meta = {z<0.0000001 ? nan : z}
    ] file {Data/OD_Backgrounds/background_constraints/rnpo_rate_2d.dat};
    
    
    \nextgroupplot[height=10cm, width=10cm,
    view={0}{90},
    ylabel={Second Pulse (phd)},
    xlabel={First Pulse (phd)},
    colorbar,
    colorbar style={ylabel={Rate (Hz)},ymode=log,},
    ]
    \addplot3[
      surf,
      shader=flat corner,
	  mesh/cols=40,
	  mesh/ordering=rowwise,
	  point meta = {z<0.0000001 ? nan : z}
    ] file {Data/OD_Backgrounds/background_constraints/bipo_rate_2d.dat};
   
  \end{groupplot}
\end{tikzpicture}
\caption{Relationship of pulses reconstructed in close proximity to each other. Only pulses above 200keV have been included.
         \textbf{Top:} All pulses in an event. The population between 150 and 200 phd in both the first and second pulse correspond to the $\alpha$'s of ${}^{219}Rn$ (first) and ${}^{215}Po$ (second).
         \textbf{Bottom:} Pulses within 1000$\mu$s of each other. The population above 150 phd in the second pulse are the $\beta$ from ${}^{214}Bi$ (first) and $\alpha$ from ${}^{214}Po$ (second).}
\label{fig:od_bipo_pulses_2d}
\end{figure}

