\begin{figure}[]%
\centering
\begin{tikzpicture}
\centering
  \begin{axis}[
    height=10cm, width=10cm,
    view={0}{90},
    ylabel={Second Pulse (phd)},
    xlabel={First Pulse (phd)},
    colorbar,
    colorbar style={ylabel={Rate (Hz)},ymode=log,},
    ]
    \addplot3[
      surf,
      shader=flat corner,
	  mesh/cols=40,
	  mesh/ordering=rowwise,
	  point meta = {z<0.0000001 ? nan : z}
    ] file {Data/OD_Backgrounds/background_constraints/rnpo_rate_2d.dat};

  \end{axis}
\end{tikzpicture}
\caption{Relationship of pulses reconstructed in close proximity to each other. Only pulses above 200 keV have been included.}
\label{fig:od_all_pulse_pairs_2d}
\end{figure}

\begin{figure}[]%
\centering
\begin{tikzpicture}
\centering
  \begin{axis}[
    height=10cm, width=10cm,
    view={0}{90},
    ylabel={Second Pulse (phd)},
    xlabel={First Pulse (phd)},
    colorbar,
    colorbar style={ylabel={Rate (Hz)},ymode=log,},
    ]
    \addplot3[
      surf,
      shader=flat corner,
	  mesh/cols=40,
	  mesh/ordering=rowwise,
	  point meta = {z<0.0000001 ? nan : z}
    ] file {Data/OD_Backgrounds/background_constraints/bipo_rate_2d.dat};

  \end{axis}
    
\end{tikzpicture}
\caption{Relationship of pulses reconstructed in close proximity and time to each other. Only pulses above 200 keV have been included.
         }
\label{fig:od_time_dependent_pulses_2d}
\end{figure}

