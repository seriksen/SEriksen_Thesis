\begin{figure}[]%
\centering
\begin{tikzpicture}
\centering
    \begin{axis}[
            ylabel=Rate (Hz),
            xlabel=Pulse Area (phd),
            width=15cm,
            height=8cm,
            grid=major,
            xmin=0, xmax=700,
            ymin=1e-4, ymode=log,]
            
        \addplot[only marks, mark size=1.0pt, error bar legend] 
            plot[error bars/.cd, x dir=both, x explicit, y dir=both, y explicit]
            table[x=pulsearea,y=weight,x error=xerror, y error=yerror]
            {Data/OD_Backgrounds/background_constraints/od_data.dat};
            
        \addplot[red, const plot]
            table [x=centre,y=rate]
            {Data/OD_Backgrounds/background_fit/scaling_options/total_improved_scaling_4.37_phd.dat};
        \addplot[green, const plot]
            table [x=centre,y=rate]
            {Data/OD_Backgrounds/background_fit/scaling_options/total_improved_scaling_4.54_phd.dat};
        \addplot[blue, const plot]
            table [x=centre,y=rate]
            {Data/OD_Backgrounds/background_fit/scaling_options/total_improved_scaling_4.77_phd.dat};
        \addplot[purple, const plot]
            table [x=centre,y=rate]
            {Data/OD_Backgrounds/background_fit/scaling_options/total_improved_scaling_4.97_phd.dat};
            
        \legend{Data, 4.37 scaling, 4.54 scaling, 4.77 scaling, 4.97 scaling};
                
    \end{axis}
            
\end{tikzpicture}
    \caption{Comparison of observed and simulated pulse area distributions. 
             The simulated pulse areas have been scaled by a range of scaling values.
             The simulation distribution is from the expected rates described in \autoref{sec:simulated_od_backgrounds}.
             Both simulation and data are normalised to the same exposure.}
    \label{fig:od_sim_vs_data_scaling_options}
\end{figure}