\begin{figure}[]%
\centering
\begin{tikzpicture}
\centering
    \begin{axis}[
            ylabel=log$_{10}$(S2$_c$) (phd),
            xlabel=S1$_c$ (phd),
            width=15cm,
            height=8cm,
            xmin=0, xmax=150,
            ymin=2.75, ymax=4.5,
            %ymin=45, ymax=100,
            %minor y tick num=1,
            ]
        \addplot[blue, ]
            table [x=s1c, y=mean]
            {Data/tpc/sr1_er_band.dat};     
        \addplot[blue, dashed]
            table [x=s1c, y=nsig1]
            {Data/tpc/sr1_er_band.dat};     
        \addplot[blue, dashed]
            table [x=s1c, y=psig1]
            {Data/tpc/sr1_er_band.dat};     

        \addplot[red, ]
            table [x=s1c, y=mean]
            {Data/tpc/sr1_nr_band.dat};     
        \addplot[red, dashed]
            table [x=s1c, y=nsig1]
            {Data/tpc/sr1_nr_band.dat};     
        \addplot[red, dashed]
            table [x=s1c, y=psig1]
            {Data/tpc/sr1_nr_band.dat};     
            
        \addplot[black, only marks, mark size=0.25pt]
            table []
            {Data/Neutron_Efficiency/AmLi_Commissioning/amli_s1logs2.dat};
            
    \end{axis}
            
\end{tikzpicture}
    \caption{TPC events which passed all vetoes to attain a high purification of neutron recoils.
             The data points are shown in black which are from all tree AmLi calibration positions.
             The blue and red lines are the electron recoil and nuclear recoil bands respectively. The solid lines are the means whilst the dashed are the 10\% and 90\% quantiles.}
    \label{fig:amli_tpc_events}
\end{figure}