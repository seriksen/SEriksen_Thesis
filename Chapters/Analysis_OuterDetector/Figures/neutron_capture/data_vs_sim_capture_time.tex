\begin{figure}[!htbp]%
\centering
\begin{tikzpicture}
\centering
    \begin{axis}[
            ylabel=Count (Arb.),
            xlabel=Capture Time ($\mu$s),
            width=15cm,
            height=8cm,
            grid=major,
            ymode=log,
            xmin=-10,xmax=800,
            ymin=1e-3, ymax=5,
            %minor y tick num=1,
            ]
        \addplot[black, only marks, mark size=0.5,
                 error bars/.cd, error bar style={color=black},
                 y dir=both, y explicit, 
                 ]
            table [x=time,y=weight, y error=yerror]
            {Data/cf252/cf252_gd_capture_time_normed.dat}; 
            
        \addplot[red, const plot]
            table [x=time,y=weight]
            {Data/cf252/amli_0mm_default.dat}; 
            
        \addplot[green, const plot]
            table [x=time,y=weight]
            {Data/cf252/amli_0mm_water_saturated.dat}; 

        \addplot[blue, const plot]
            table [x=time,y=weight]
            {Data/cf252/amli_0mm_water.dat}; 
            
    \end{axis}
            
\end{tikzpicture}
    \caption{Neutron capture time on Gadolinium in the OD. The observed capture time from ${}^{252}$Cf is shown in black. In red is the expected capture time from simulation. In green is the capture time if the foam has been saturated with water up to 5\% of its volume. In blue is the capture time if all foam around the OCV has been replaced with water. Each distribution has been normalised the maximum value.}
    \label{fig:data_vs_sim_gd_capture_time}
\end{figure}