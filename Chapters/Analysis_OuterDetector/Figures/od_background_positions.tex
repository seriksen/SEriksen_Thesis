\begin{figure}[!htbp]%
\centering
\begin{tikzpicture}
\centering
  \begin{groupplot}[%view={0}{90},
    group style = {group size = 2 by 3,vertical sep=1.5cm,
                   horizontal sep=1.5cm},
                   height=8cm, width=0.5\textwidth]
    \nextgroupplot[
            title=Pulse Area,
            ylabel=Rate (Hz),
            xlabel=Pulse Area,
            width=\textwidth,
            height=6cm,
            %xshift=0.5\textwidth,
            xmin=0, xmax=2000,
            ]
            \addplot [blue] {rnd};
            
    \nextgroupplot[group/empty plot]

    \nextgroupplot[title=All pulses,
            xshift=-0.25\textwidth]
            \addplot [blue] {rnd};

    \nextgroupplot[title=Region 1,
                  xshift=-0.5\textwidth,
                  yshift=1cm,]
            \addplot [blue] {rnd};
    
    \nextgroupplot[title=Region 2]
            \addplot [blue] {rnd};
                   
    \nextgroupplot[title=Region 3]
            \addplot [blue] {rnd};
   
  \end{groupplot}
\end{tikzpicture}
\caption{Position Reconstruction of pulses from various regions in pulse area space defined in the top plot. 
         Each pulse has had the noise cut applied and the position reconstructed using \autoref{eq:OD_xy_position}.}
\label{fig:od_backgrounds_position_reconstruction}
\end{figure}