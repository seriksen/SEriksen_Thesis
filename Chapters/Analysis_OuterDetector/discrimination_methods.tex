\section{Discrimination Methods}

\par
There are countless different methods to discriminate different particles


\par
An important point to note is that the ability of any method to discriminate different event types is how the input variables are selected.
Ideally, each of the input variables would not be correlated with each other.
This is not the case, particularly as we have limited ourselves to using reduced quantities rather than raw waveforms.
So rather than doing pulse-shape discrimination, we are looking at some pre-extracted quantities of a given pulse.

\subsection{Maximum Likelihood Method}
\par
In shorterned form. 
For a given event, the likelihood for being of signal type is obtained by multiplying the signal probability densities of all input variables which are assumed to be independent and normalising this by the sum of the signal and background likelihoods.
Because correlations among the variables are ignored, this PDE approach is also called "naive Bayes estimator" \cite{TMVA_ref}.

Following the style of \cite{TMVA_ref}, the likelihood ratio $y_{\Lagrangian}$ of any event is defined as;
\begin{equation}
    y_{\Lagrangian} = \frac{\Lagrangian_{S}}{\Lagrangian_{S} + \Lagrangian_{B}}
\end{equation}
where
\begin{equation}
    \Lagrangian_{S,(B)} = \prod_k p_{S,(B),k}(x_k)
\end{equation}
where $p_{S,(B),k}$ is the signal (background) PDF for the kth input variable $x_k$.