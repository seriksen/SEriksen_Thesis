\chapter{Analysis of the Outer Detector} \label{chap:analysis_of_the_od}

\par
In this chapter, the calibration of the OD is described.
This forms the foundations for assessing if the requirements set out in \autoref{tab:veto_requirements} are achieved.
These requirements are: that the neutron veto efficiency is at least 95\% and that no more than 5\% of the WIMP search time is vetoed.
For ascertaining an appropriate veto window, a study on the OD backgrounds is performed. 
This is followed by the results from a neutron calibration.
%\par
%Finally a particle discrimination method is suggested, which in future may allow for the OD to be used as a more intelligent veto.


\section{OD Energy Scale} \label{sec:od_energy_scale}
\par
After construction and filling of the OD, the single photoelectron response for each PMT was determined using the OD optical calibration system (OCS) \cite{lz_ocs_system_ref}.
A detailed description of the PMT monitoring during commissioning and SR1 can be found in the works of E. Fraser \cite{ewanfraser_thesis_ref}. 
During this period, an energy calibration of the OD was also performed.
Again, details of the methodology are the work of E. Fraser \cite{ewanfraser_thesis_ref} but summarised below, as the result is used throughout this chapter.
\par
For the energy calibration, 7 $\gamma$ sources were used which are summarised in \autoref{tab:od_energy_calibration_sources}.
Each were lowered into the CSD's to the 700mm level (see \autoref{fig:CSD1_Geometry}).
The signals from these $\gamma$s were picked out giving 7 data points for a conversion between signal size and $\gamma$ energy.
As discussed in \autoref{sec:od_physics}, there is a non-linear energy response in the scintillator, and therefore a non-linear energy scale below 300 keV.
The empirical formula for the non-linearity developed by DayaBay \cite{dayabay_antineutrino_oscillation_ref, ls_nonlinear_energy_response_ref}, shown in \autoref{eq:ls_light_response}, was used to fit to these calibration points:
\begin{equation}
    \frac{E_{vis}}{E_{true}} = \frac{p_0  + p_3 \times E_{true}}{1 + p_1 \times e^{p_2 \times E_{true}}}
    \label{eq:ls_light_response}
\end{equation}
This equation was then used to fit the 7 data points which is shown in \autoref{fig:od_energy_scale}.
This then gives a translation from signal size to energy for any signal, and importantly for the OD veto, a value in phd for 100 keV and 200 keV events.
The amount of light observed by the LZ OD is in line with other LAB-based experiments as can be seen in \autoref{tab:od_phe_per_mev_comparison}.
\begin{table}[!htbp]%
    \centering
    \begin{tabular}{c|c|c}
        Source      & Specific observable         &  $\gamma$ energy (keV) \\ \hline
        ${}^{57}Co$ & direct $\gamma$             & 122                        \\
        ${}^{22}Na$ & positron annihilation       & 511               \\
        ${}^{54}Mn$ & direct $\gamma$             & 835                        \\
        ${}^{22}Na$ & direct $\gamma$             & 1275               \\
        ${}^{252}Cf$ & neutron capture on H       & 2223            \\
        ${}^{228}Th$ & ${}^{208}Tl$ $\beta-$decay & 2615            \\
        ${}^{252}Cf$ & neutron capture on Gd      & 8000            
        
    \end{tabular}
    \caption{Calibration sources used to determine the OD energy scale.}
    \label{tab:od_energy_calibration_sources}
\end{table}

\begin{figure}[!htbp]
    \centering
   \begin{tikzpicture}
    \begin{groupplot}[%view={0}{90},
    group style = {group size = 1 by 2,vertical sep=0.5cm},
                   height=8cm, width=\textwidth]
    \nextgroupplot[
            ylabel=LS energy response ($\frac{E_{vis}}{E_{true}}$),
            %xlabel=,
            xticklabels={,,}
            width=\textwidth,
            xmin=0, xmax=10,
            grid=major,
            ]
            \addplot[black]
                    table [x=Energy,y=Value]
                    {Data/OD_Energy_Scale/emperical_fit.dat};

    \nextgroupplot[
            ylabel=N. photons detected,
            xlabel=Energy (MeV),
            width=\textwidth,
            xmin=0, xmax=10,
            ymin=0, 
            grid=major,
            ]
            \addplot[black]
                    table [x=Energy,y=Value]
                    {Data/OD_Energy_Scale/phd.dat};
            
   
  \end{groupplot}
    \end{tikzpicture}
    \caption{OD Energy Scale and the relation between observed number of photons and the energy deposited.}
    \label{fig:od_energy_scale}
\end{figure}

\begin{table}[!htbp]%
    \centering
    \begin{tabular}{c|c}
        Detector & phe/MeV \\ \hline
        RENO     & 150 \cite{reno_phe_per_mev_ref} \\
        Borexino & 438 \cite{pablo_mosteiro_thesis_ref} \\
        Daya Bay & 162 \cite{dayabay_phe_per_mev_ref} \\
        Kamland  & 200 \cite{kamland_phe_per_mev_ref} \\
        SNO+     & 300 \cite{snoplus_phe_per_mev_ref} \\
        LZ       & 230 
    \end{tabular}
    \caption{LZ OD photons detected per electron equivalent MeV deposited compared to other large LAB experiments.}
    \label{tab:od_phe_per_mev_comparison}
\end{table}

\subsection{Energy Uncertainty}
\par
One key largest drawback of the energy calibration approach used here is that only a single location was used, in detector $z$-axis.
Therefore the variation in light collection efficiency in the OD is not taken into account (see \autoref{fig:od_lce}), and so energy deposits in other locations in the GdLS would be reconstructed with incorrect energies.
This is tested in the following sections and appears to be insignificant as other factors are more important.


\subsection{PMT Noise}
\par
The waveforms from PMTs in the OD were occasionally observed erratic behaviour such as that shown in \autoref{fig:noise_od_waveform}.
This was tracked down to a grounding issue, which occasionally caused high energy pulses with large PMT multiplicities.
During this high noise period no physical OD event can be made out from the noise.
Included as well is a waveform from during a quiet period of data taking showing expected behaviour.

\begin{figure}[!htbp]
\begin{subfigure}{\textwidth}
  \centering
  \includegraphics[width=\linewidth]{Figures/OD_Backgrounds/noise_pulse.png}
  \caption{Noise pulse from poor grounding.}
  \label{fig:noise_od_waveform}
  \end{subfigure}
  \begin{subfigure}{\textwidth}
  \centering
  \includegraphics[width=\linewidth]{Figures/OD_Backgrounds/regular_pulse.png}
  \caption{Regular pulse}
  \label{fig:regular_od_waveform}
  \end{subfigure}
\caption{OD summed waveforms.}
\label{fig:od_noise_cut_waveforms}
\end{figure}

\par
Within the pulse parameters it is fairly easy to pick out these noisy moments as can be seen in \autoref{fig:od_noise_cut}.
The central distribution with a pulse amplitude to pulse area of 0.01 correspond to a regular PMT response and are the physics signals.
There are then two other distributions.
The population with a amplitude to area pulse ratio of 0.05 are the noise events from grounding.
The other population begin above a ratio of 0.15 but as the area increases the ratio drops, until at 400 phd the ratio is zero and can originate from the same behaviour.
In order to remove these pulses from any analysis, a noise cut was developed to remove this based upon the pulse shape, requiring that a large portion of the pulse integral be within 100 ns of the pulse starting.
Combined with a PMT multiplicity requirement of 5, any other PMT effect was also removed.
This combined selection is used throughout this chapter being referred to as the "noise cut".

%%%%%%%%%%%%%%
\subsection{Comparison to Simulations}
\par

When data and simulations were first compared, it became obvious that the light modelling in the OD was significantly different to reality.
In \autoref{fig:od_sim_vs_data_raw} a comparison between the expected OD backgrounds and the observed background spectra are shown for two parameters: pulse area and PMT multiplicity.
In the pulse area, the simulated pulse area is significantly lower than is observed in data.
Around 50\% of this difference can be accounted for by a mismatch between the PMT gains simulated against those in the reconstruction database.
This results in smaller simulated pulses but it is just a linear scaling.
However when we look at the pulse multiplicity the is also a mis-match.
This is more significant as it indicates that the light propagation within the simulation is incorrect.
In fact it would indicate that the LCE is twice what is currently simulated (\autoref{fig:od_lce}).

\begin{figure}[]%
\centering
\begin{tikzpicture}
\centering
    \begin{groupplot}[
    group style = {group size = 1 by 2,vertical sep=2.0cm}
    ]
    \nextgroupplot[
            ylabel=Rate (Hz/5phd),
            xlabel=Pulse Area (phd),
            width=15cm,
            height=8cm,
            grid=major,
            xmin=0, xmax=700,
            ymin=1e-2, ymax=100,
            ymode=log,
            ]
        \addplot[only marks, mark size=1.0pt] 
            plot[error bars/.cd, x dir=both, x explicit]
            table[x=pulsearea,y=weight,x error=xerror, y error=yerror]
            {Data/OD_Backgrounds/background_constraints/od_data.dat};
        \addplot[red, const plot]
            table [x=centre,y=rate]
            {Data/OD_Backgrounds/background_fit/no_scaling/total_improved_bg_phd.dat};
        \legend{Data, Simulation};
        
            
    \nextgroupplot[
            ylabel=Rate (Hz),
            xlabel=Pulse Coincidence (PMT multiplicity),
            width=15cm,
            height=8cm,
            grid=major,
            xmin=0, xmax=120,
            %ymin=1e-2, ymax=50,
            ymode=log,
            ]
        \addplot[black, only marks, mark size=0.5, 
                 error bars/.cd, error bar style={color=black},
                 y dir=both, y explicit, 
                 x dir=both, x explicit,
                 ]
            table [x=centre,y=rate,
             y error minus index=4, 
             x error minus index=6, 
            ]
            {Data/OD_Backgrounds/background_fit/no_scaling/data_bg_coincidence.dat};
        \addplot[red, const plot]
            table [x=centre,y=rate]
            {Data/OD_Backgrounds/background_fit/no_scaling/total_improved_bg_coincidence.dat};
        \legend{Data, Simulation};
    
    \end{groupplot}
\end{tikzpicture}
    \caption{Comparison of two analysis quantities; pulse area and pulse multiplicity.
             Data rates are from a single week of Random Trigger data taken during SR1.
             The simulated rates include all detector components and cavern-$\gamma$'s.}
    \label{fig:od_sim_vs_data_raw}
\end{figure}

\par
Despite this difference there are a number of ways in which data and simulations can be compared.
The simplest approach is to assume that all physics is correct in the simulation and the discrepancy can be accounted for entirely by incorrect linear scaling in the simulated PMT response.
An alternative approach to comparing simulations to data would be to translate phd into energy for both simulations and data, though generally the less manipulation that is performed on data the better.
Both of these approach allow for the signal sizes to be compared whilst ignoring the underlying difference in light propagation.

\par
In order to determine the factor needed to scale the simulations to data, two data points were compared.
The first was from ${}^{228}$Th.
In the decay chain (\autoref{fig:decay_chains}) there is ${}^{208}$Tl, which has a prominent 2.6 MeV $\gamma$ which is what was used for the OD energy scale.
The second was from ${}^{152}$Gd, a 2.2 MeV $\alpha$-decay.
As this decay is internal to the GdLS it is spread throughout the acrylic tanks and is always present in the OD background spectrum.
As $\alpha$s are so heavily quenched in the GdLS, the visible energy is in the non-linear region of the energy scale.

\par
Full-propagation simulations were performed of ${}^{228}$Th in CSD-1 at 700 mm (\autoref{fig:CSD1_Geometry}).
The 2.6 MeV $\gamma$ is the dominate decay which is able to escape the CSD and enter the OD, making it easy to pick out which feature corresponds to pulse area.
Similarly, simulations of ${}^{152}$Gd were performed with the spread throughout the GdLS.
As only a single source was simulated a single peak is present in the pulse area distribution.
In data as the distribution is so low in energy is appears as a shoulder.
A comparison between the simulated and observed spectra for each source is shown in \autoref{fig:od_scaling_points}.
The distributions have been normalised such that the quantity of interest is at 1.

\par
The scaling factors were determined by fitting Gaussians to the pulse area distributions and dividing the means.
Both methods provided similar, yet slightly different scaling values of 4.54 from ${}^{152}$Gd and 4.77 from ${}^{208}$Tl.
There are a number of possible reasons for this difference.
In the case of ${}^{152}$Gd, there is another $\alpha$-decaying contaminant in the GdLS: ${}^{147}$Sm \cite{scotthaselschwardt_thesis_ref}.
${}^{147}$Sm decays emitting an $\alpha$ of of 2.3 MeV, however once quenching has been taken into account the difference in the observed energy will be negligible.
The ${}^{152}$Gd is most likely being heavily influenced by its proximity to the high rate of very low energy events.
\par
For a full background model, a range of scaling factors were compared against the observation.
These are shown in \cite{fig:od_sim_vs_data_scaling_options}.
Of the four which were tried, the scaling factor of 4.77 proved to be the closes match visually via features and so was taken as the value to use.
This approach obviously has some significant uncertainty, which will become apparent later in the next section where the backgrounds are studied.


\begin{figure}[!htbp]%
\centering
\begin{tikzpicture}
\centering
    \begin{groupplot}[
    group style = {group size = 2 by 2,vertical sep=3cm,
                   horizontal sep=1.5cm},
                   height=6cm, width=0.5\textwidth
    ]
    \nextgroupplot[
            ylabel=Arbitrary,
            grid=major,
            xmin=0, xmax=50,
            ymin=1e-2, ymax=100,
            ymode=log,
            ]
        \addplot[black, only marks, mark size=0.5, 
                 error bars/.cd, error bar style={color=black},
                 y dir=both, y explicit, 
                 x dir=both, x explicit,
                 ]
            table [x=centre,y=rate, y error minus index=4, y error minus index=5, x error minus index=6, x error minus index=6, ]
            {Data/OD_Backgrounds/background_fit/no_scaling/data_bg_phd.dat};
    
    \nextgroupplot[
            grid=major,
            xmin=0, xmax=50,
            ymin=1e-2, ymax=100,
            ymode=log,
            ]
        \addplot[only marks, mark size=0.5pt] 
            plot[error bars/.cd, x dir=both, x explicit]
            table[x=pulsearea,y=rate,x error=x_error, y error=y_error]
            {Data/OD_Backgrounds/background_constraints/od_data_pulsearea_middle_tank_binwidth_5.dat};
            
    \nextgroupplot[
            ylabel=Rate (Hz/5phd),
            xlabel=Pulse Area (phd),
            grid=major,
            xmin=50, xmax=100,
            ymin=1e-2, ymax=100,
            ymode=log,
            ]
        \addplot[black, only marks, mark size=0.5, 
                 error bars/.cd, error bar style={color=black},
                 y dir=both, y explicit, 
                 x dir=both, x explicit,
                 ]
            table [x=centre,y=rate, y error minus index=4, y error minus index=5, x error minus index=6, x error minus index=6, ]
            {Data/OD_Backgrounds/background_fit/no_scaling/data_bg_phd.dat};
            
    \nextgroupplot[
            xlabel=Pulse Area (phd),
            grid=major,
            xmin=50, xmax=100,
            ymin=1e-2, ymax=50,
            ymode=log,
            ]
        \addplot[black, only marks, mark size=0.5, 
                 error bars/.cd, error bar style={color=black},
                 y dir=both, y explicit, 
                 x dir=both, x explicit,
                 ]
            table [x=centre,y=rate,
            y error minus index=4, y error minus index=5, 
            x error minus index=6, x error minus index=6, 
            ]
            {Data/OD_Backgrounds/background_fit/no_scaling/data_bg_coincidence.dat};
    
    \end{groupplot}
\end{tikzpicture}
    \caption{Scaling points to compare simulations to observations in pulse area.}
    \label{fig:od_scaling_points}
\end{figure}

\begin{figure}[!htbp]%
\centering
\begin{tikzpicture}
\centering
    \begin{axis}[
            ylabel=Rate (Hz/5phd),
            xlabel=Pulse Area (phd),
            width=15cm,
            height=8cm,
            grid=major,
            xmin=0, xmax=700,
            ymin=1e-4, ymode=log,]
            
        \addplot[only marks, mark size=1.0pt] 
            plot[error bars/.cd, x dir=both, x explicit]
            table[x=pulsearea,y=weight,x error=xerror, y error=yerror]
            {Data/OD_Backgrounds/background_constraints/od_data.dat};
            
        \addplot[red, const plot]
            table [x=centre,y=rate]
            {Data/OD_Backgrounds/background_fit/scaling_options/total_improved_scaling_4.37_phd.dat};
        \addplot[green, const plot]
            table [x=centre,y=rate]
            {Data/OD_Backgrounds/background_fit/scaling_options/total_improved_scaling_4.5_phd.dat};
        \addplot[blue, const plot]
            table [x=centre,y=rate]
            {Data/OD_Backgrounds/background_fit/scaling_options/total_improved_scaling_4.77_phd.dat};
        \addplot[purple, const plot]
            table [x=centre,y=rate]
            {Data/OD_Backgrounds/background_fit/scaling_options/total_improved_scaling_4.97_phd.dat};
            
        \legend{Data, 4.37 scaling, 4.5 scaling, 4.77 scaling, 4.97 scaling};
                
    \end{axis}
            
\end{tikzpicture}
    \caption{Possibly values to scale simulation pulse area to observed pulse area.
             The simulation distribution is from the expected rates described in \autoref{sec:simulated_od_backgrounds}}
    \label{fig:od_sim_vs_data_scaling_options}
\end{figure}


\clearpage
\section{Backgrounds}
Things to mention and explain;
\begin{itemize}
    \item noise-cut
    \item energy scale
\end{itemize}

\par
Using the Random Trigger, the rate seen in the OD was observed for a number of months.
The result of which is shown in Figure \ref{fig:OD_SR1_Rate} along with the date range used for SR1.
\par
As time progresses the rate of decreases as the OD stabilises.
The various fluctuations in the noise-cut has been linked to the chiller system being turned on, suggesting a grounding failure.
However, as these can be removed with a noise-cut they are not of significant concern.
\begin{figure}[!htbp]
    \centering
    \begin{tikzpicture}
        \begin{axis}[
            title=TODO: Replace with dates and errors,
            xlabel=Data taking week,
            ylabel=Rate (Hz),
            width=15cm,
            height=6cm,
            xmin=-2,
            xmax=14,
            legend style = {column sep = 10pt, legend columns = -1,}]
            \addplot[red, only marks]
                    table [x=Week,y=Rate]
                    {Data/OD_Backgrounds/od_sr1_rate_noise.dat};
            \addlegendentry{Noise Cut};
            \addplot[blue, only marks]
                    table [x=Week,y=Rate]
                    {Data/OD_Backgrounds/od_sr1_rate_100.dat};
            \addlegendentry{100keV};
            \addplot[green, only marks]
                    table [x=Week,y=Rate]
                    {Data/OD_Backgrounds/od_sr1_rate_200.dat};
            \addlegendentry{200keV};
        \end{axis}
    \end{tikzpicture}
    \caption{Rate in OD during and before SR1 data taking on a week-by-week basis using the Random Trigger.
    Week -1 corresponds to the month prior to SR1 when the OD PMT gains were higher.}
    \label{fig:OD_SR1_Rate}
\end{figure}
\par
During the SR1 data taking period, the rate-per-phe is shown in Figure \ref{fig:od_sr1_rate_vs_threshold}.
Overlaid is the expected rate of backgrounds from Table \ref{tab:od_expected_rates}.
Interestingly the observed rate is below what was anticipated.
This unexpected result means that the veto threshold could be reduced to 100keV, which should increase the veto tagging efficiency.
Alternatively, the threshold could be left at 200keV, but the window extended, also increasing the tagging efficiency.
\begin{figure}[!htbp]
    \centering
    \begin{tikzpicture}
        \begin{axis}[
            xlabel=OD Threshold (phe),
            ylabel=Rate (Hz),
            width=15cm, height=8cm,
            xmin=-1, xmax=55,
            ymin=0, ymax=350,
            legend pos=north east,
            grid=major]
             \addplot+[black, smooth, mark=none]
                    table [x=Threshold,y=Rate]
                    {Data/OD_Backgrounds/od_sr1_rate_vs_threshold_smooth_line.dat};
            \addplot[black, only marks, 
                     error bar legend,
                     error bars/.cd,
                     x dir=both, x explicit, error bar style={color=black}]
                    table [x=Threshold,y=Rate, x error=XError]
                    {Data/OD_Backgrounds/od_sr1_rate_vs_threshold_error_bars.dat};
             \addplot[dashed, mark=none, red] coordinates {(0,100) (60,100)};
             \addplot[dashed, mark=none, blue] coordinates {(17.6,0) (17.6,350)};
             \addplot[dashed, mark=none, green] coordinates {(37.5,0) (37.5,350)};
             
             \addplot[orange, only marks, 
                      error bar legend,
                      error bars/.cd,
                      y dir=both, y explicit, error bar style={color=orange}]
                      table [x=Threshold,y=Rate, y error=YError]
                      {Data/OD_Backgrounds/od_sr1_rate_expected.dat};
             
             \legend{,SR1 Data,$<$100Hz Requirement,100 keV (17.6 phe),200 keV (37.5 phe),Expected}                
        \end{axis}
    \end{tikzpicture}
    \caption{Rate of OD backgrounds during SR1 using the Random Trigger. The noise cut has been applied. 100Hz Requirement is for a 500$\mu$s veto window as proposed in \cite{LZ_TechnicalDesignReview_ref}. Expected values are from Table \ref{tab:od_expected_rates}}
    \label{fig:od_sr1_rate_vs_threshold}
\end{figure}
\par
Due to this unexpectedly lower rate, a study was performed in order to understand what backgrounds are (and aren't) being seen.
This is described in the remainder of this section.


\begin{tcolorbox}[colback=red!5!white, colframe=red!50!black, title=Key Plots]
\begin{enumerate}
    \item Simulation Expected Rate of top Components
    \item Scotts Measured GdLS vs Data GdLS
    \item Scaling in X
    \item Scaling justification
    \item Fitted Result
    \item Table of Measured Rates
    \item LCE map?
    \item OD rate over time
    \item R vs Z in background data -> get rock gamma positions    
\end{enumerate}
\end{tcolorbox}


\subsection{Energy Scaling}
\par
When data and simulations were first compared, as in Figure XXX, it became obvious that the light modelling in the OD was significantly different to reality.
Around 50\% of this difference can be accounted for by a mismatch between the PMT gains simulated against those in the LZap reconstruction database.
This effect simply results in smaller pulses and so smaller pulse areas.
However, the remaining difference is not clear where it is coming from, as it would indicate that the LCE is twice what is currently simulated (Figure \ref{fig:od_lce}).
In the pulse area parameter, this should remain just a linear scaling, though in other parameters (such as number of PMTs receiving light) it may not be linear.

\par
In order to calculate the energy scaling, Th228 data calibration was used. 
In the decay chain (Figure XXX), Tl208 has a prominent 2.6MeV $\gamma$. 
There are two simulated sources of this, firstly, from a calibration source, and secondly from the ${}^{232}Th$ cavern-$\gamma$'s as shown in Figure \ref{fig:cavern_gamma_energy_distribution}.
For cavern-$\gamma$ simulations, the 2.6MeV was extracted using truth information and the resultant phe plotted.


\begin{figure}
    \centering
    \includegraphics[width=0.5\textwidth]{Figures/Placeholder.png}
    \caption{Number of PMTs contributing to a pulse against the phe of the pulse. Or some similar plot like that}
    \label{fig:OD_coincidence_difference}
\end{figure}


\subsection{Constraints in Data}
\par
Though the exact rate of GdLS is not known, it is possible to constrain some of the components observed.
Notably, decays within the U and Th decay chains where there the second decay has a short enough half-life such that it is within the LZ event window.
This leaves 3-possible decays, summarised in Table \ref{tab:od_constrainable_decays_in_data}.
However, in reality only  ${}^{214}Bi \to {}^{214}Po$ and ${}^{219}Rn \to {}^{215}Po$ can be searched for as the interactions from ${}^{212}Bi \to {}^{212}Po$ are close enough together that they will be merged into one pulse.

\begin{table}[!htbp]
    \centering
    \begin{tabular}{c|c|c|c|c|c}
        \multirow{2}{*}{Decay Pair (chain)}                    & \multicolumn{2}{c|}{First Decay}   & \multicolumn{3}{c}{Second Decay}    \\ 
                                                               & Decay    & Energy (MeV) & Decay    & Energy (MeV) & half-life ($\mu$s) \\ \hline
        ${}^{214}Bi \to {}^{214}Po$ (${}^{238}U_{m}$)          & $\beta$  & 3.3          & $\alpha$ & 7.69         & 160   \\ 
        ${}^{219}Rn \to {}^{215}Po$ (${}^{235}U_{l}$)          & $\alpha$ & 6.76         & $\alpha$ & 7.39         & 1800  \\ 
        ${}^{212}Bi \to {}^{212}Po$ (${}^{232}Th_{l}$)         & $\beta$  & 2.1          & $\alpha$ & 8.78         & 0.3
    \end{tabular}
    \caption{Th and U decay chain pairs with half-lives within the LZ event window of 4.5ms}
    \label{tab:od_constrainable_decays_in_data}
\end{table}

\par
The decay pairs can be searched for by looking for pulses which are reconstructed to be close to each other.

These can be searched for by requiring that two pulses be close in time and close in position. 
The time requirement is dictated by the half-life of the second decay.
The position requirement is to reduce the impact of coincident interactions and decays in other areas of the OD.
As the second decay is an $\alpha$ it does not have significant penetrating power so the interaction will be close to where the initial decay was.
The cuts selected were; the noise cut previously discussed, a maximum position separation as defined by the reconstructed position in Equation \ref{eq:OD_xy_position}.
The time selection

\iffalse
\begin{figure}[!htbp]%
\centering
\begin{tikzpicture}
\centering
  \begin{axis}[%point meta max=150,
    %point meta min=0.0,
    view={0}{90},
    ylabel={Kinetic Energy (MeV)},
    xlabel={Number of Interactions},
    colorbar,
    colorbar style={ylabel={Count}},
    ]
    \addplot3[
      surf,
      shader=flat corner,
      mesh/cols=94,
      mesh/ordering=rowwise,
      point meta = {z<1 ? nan : z}
    ] file {Data/OD_Backgrounds/od_bipo_rate_200kev_2d.dat};
\end{axis}
\end{tikzpicture}
\caption{The relationship between the number of interactions a neutron has in the scintillator and the neutrons kinetic energy when it enters the liquid scintillator volume for neutrons which are 'captured' in the scintillator volume.
}
\label{fig:OD_BiPo_RnPo_rates}
\end{figure}
\fi

\begin{figure}
    \centering
    \includegraphics[width=0.5\textwidth]{Figures/Placeholder.png}
    \caption{Rate of BiPo and RnPo decays during SR1.}
    \label{fig:OD_BiPo_Rate}
\end{figure}

\par
One of the difficulties in constraining the rates is that the ~8MeV $\alpha$'s in the chains lie in the region where the OD-trigger becomes active.


\subsection{Position Reconstruction}
\par
For any pulse it is possible to reconstruct the location of the interaction that caused the pulse by a weighted average.
This was performed using Equation \ref{eq:OD_xy_position}.

\begin{equation}
    \begin{array}{lcl}
        x(y) = \frac{\sum{\text{phe}_\text{Ch} * \text{x(y)}_\text{Ch}}}{\sum{\text{phe}_\text{Ch}}} \\
        z = XXX 
    \end{array}
    \label{eq:OD_xy_position}
\end{equation}


\begin{figure}[!htbp]%
\centering
\begin{tikzpicture}
\centering
  \begin{groupplot}[%view={0}{90},
    group style = {group size = 2 by 3,vertical sep=1.5cm,
                   horizontal sep=1.5cm},
                   height=8cm, width=0.5\textwidth]
    \nextgroupplot[
            title=Pulse Area,
            ylabel=Rate (Hz),
            xlabel=Pulse Area,
            width=\textwidth,
            height=6cm,
            %xshift=0.5\textwidth,
            xmin=0, xmax=2000,
            ]
            \addplot [blue] {rnd};
            
    \nextgroupplot[group/empty plot]

    \nextgroupplot[title=All pulses,
            xshift=-0.25\textwidth]
            \addplot [blue] {rnd};

    \nextgroupplot[title=Region 1,
                  xshift=-0.5\textwidth,
                  yshift=1cm,]
            \addplot [blue] {rnd};
    
    \nextgroupplot[title=Region 2]
            \addplot [blue] {rnd};
                   
    \nextgroupplot[title=Region 3]
            \addplot [blue] {rnd};
   
  \end{groupplot}
\end{tikzpicture}
\caption{Position Reconstruction of pulses from various regions in pulse area space defined in the top plot. 
         Each pulse has had the noise cut applied and the position reconstructed using Equation \ref{eq:OD_xy_position}.}
\label{fig:od_backgrounds_position_reconstruction}
\end{figure}


\par
As can be seen in Figure \ref{fig:od_backgrounds_position_reconstruction}, events reconstructed to be at the top and bottom of the OD dominate everywhere.
This supports both Table XXX (Section XXX) and Figure XXX (Section XXX), where Cavern Gammas are expected to be the largest contributor with the largest contribution coming from the bottom.


\par
As a way of suppressing cavern-$\gamma$'s, a slice of events reconstructed to be in the middle of the side tanks was taken.
The resultant pulse spectrum is shown in Figure \ref{fig:od_data_pulsearea_middle_tank}.
What is interesting is that two additional features have appeared.


\begin{figure}[!htbp]
    \centering
    \begin{tikzpicture}
    
    \begin{axis}[
        xlabel=Pulse Area,
        ylabel=Rate (Hz/5phe),
        width=15cm, height=10cm,
        xmin=0, xmax=800,
        ymin=1e-4, ymode=log,
        legend pos=north east,
        grid=major]
            
        \addplot[only marks, mark size=0.5pt] 
            plot[error bars/.cd, x dir=both, x explicit]
            table[x=pulsearea,y=rate,x error=x_error, y error=y_error]
            {Data/OD_Backgrounds/od_data_pulsearea_middle_tank_binwidth_5.dat};
            
        \end{axis}
    \end{tikzpicture}
    \caption{OD pulse area spectrum from pulses reconstructed to the middle of the OD side tanks.}
    \label{fig:od_data_pulsearea_middle_tank}
\end{figure}


\subsection{Fit}
\par
Given the mitigation processes described in Section \ref{od_construction_sec}, the contamination is likely to be minimal everywhere. 
Additionally, given what can be seen in Figure XXX, the improved GdLS purification appears to have worked.
As the GdLS came from the same batch produced at Brookhaven as used in DayaBay it is reassuring that it matches closely what was seen there

\par
Instead, what is likely to have happened is that \cite{KamLAND_LS_contaminants_ref}.


\subsection{Cavern-$\gamma$'s}
\par
As discussed in Section \ref{sec:cavern_gamma_generator}, the $\gamma$'s from decays within the rock are of significant concern.
blah blah blah...
Over the continued course of the experiment it will be interesting see if an annual modulation is observed such as that in \cite{cavern_gamma_annual_modulation_CoGeNT_ref, cavern_gammas_in_Soudan_mine_ref} as it may be a more significant background for LZ than previously considered.
\clearpage
\section{Neutron Inefficiency}
\par
As discussed in Section XXX, neutrons are a particularly important background with the signal produced in the TPC being indistinguishable from a DM interaction.
The design motivation for the Outer Detector was to actively veto events which contain neutrons by having a very high tagging efficiency to neutrons.
When combined with the Skin Detector, it was shown in the Technical Design Report for LZ that a greater than <5\% inefficiency to neutrons that single scatter in the WIMP region could be achieved with a veto window of XXX.
However, as discussed in Section XXX, several design changes were made during construction which will affect that result.
In this section, the neutron inefficiency is re-calculated with a more accurate geometry following the method described in \cite{sallyshaw_thesis_ref}.
The calculation is taken a step further here with the detector LCE taken into consideration, and finally applying to data. 



\begin{tcolorbox}[colback=red!5!white, colframe=red!50!black, title=Key Plots]
\begin{enumerate}
    \item Simulation Neutron Capture Time; Gd isotopes, H isotopes, break up by AmLi energy?
    \item Data Neutron Capture Time
    \item Simulation Inefficiency
    \item Simulation AmLi Inefficiency
    \item Rate of neutrons entering the GdLS (from simulations)
    \item Data AmLi Inefficiency
    \item Equation of energy lost from a scatter (why neutron loses hardly any energy so can assume it's already single scattered)
    \item Neutron ratio in each capture time (sims and data); 3 capture constants
\end{enumerate}
\end{tcolorbox}

\par
The largest light source resulting from a neutron, that will be seen by the OD PMTs are the neutron capture.
Therefore this is a valid quantity to measure.

\subsection{Neutron Captures}
\par
Another source that can be used to study the neutron capture time is the ${}^{252}{Cf}$ calibration source.
Spontaneous fission of ${}^{252}{Cf}$ typically results in a number of $\gamma$'s with a combined energy of up to 10MeV which are accompanied by a number of neutrons.
An example event from the LZ calibration run pre-SR1 is shown in Figure \ref{fig:cf252_event_viewer}.
This allows for neutrons to be tagged by tagging the fission by coincident $\gamma$'s in a each detector, and assuming later pulses are dominated by neutrons.

\begin{figure}[!htbp]
\includegraphics[width=10cm]{Figures/NeutronCaptureTime/cf252_eventviewer_5916.png}
\centering
\caption{Example event from ${}^{252}{Cf}$ calibration run showing the $\gamma$'s causing coincident pulses in each detector followed by two neutrons being captured in the Outer Detector}
\label{fig:cf252_event_viewer}
\end{figure}

\par
Although Figure \ref{fig:cf252_event_viewer} indicates that prompt $\gamma$'s can be used for tagging it does come with a significant caveat; namely that expressed in Figure \ref{fig:fission_fragments_time}.
There is a non-insignificant probability that $\gamma$'s will be emitted at the same time as a $\gamma$ from a neutron capture is released.
Additionally the majority of study into prompt $\gamma$'s from fission focus on the first 100ns post-fission and the energy and multiplicity beyond that time is not well theorised. 
However, 


\begin{figure}[!htbp]
\includegraphics[width=10cm]{Figures/NeutronCaptureTime/fission_fragment_times.png}
\centering
\caption{Fission Components time blah blah, Adapted from \cite{cf252_fission_ref}}
\label{fig:fission_fragments_time}
\end{figure}



\par
For the other neutron calibration sources, DD and AmLi, the same neutron tagging as described in Section XXX was used, where the NR band is isolated.
The result of this search is shown in figure XXX - overlaid is the neutron capture sim in the simulations.
What is apparent is that there is more than one exponential decays in data.
The simplest explanation is that a portion of the neutrons are being captured on H in either the water or acrylic that is between the source and GdLS.
In this case, the capture constant would be 220$\mu$s.
Initially this was excluded by selecting pulses above the H-peak, selecting events above 500phd.
However, as can be seen in Figure XXX, at least two exponents remain.
A fit was performed on each of the calibration sources separately, to equation XXX, and the results are summarised in Table \ref{tab:neutron_capture_times}.
\begin{table}[!htbp]
    \centering
    \begin{tabular}{c|c}
        Source            &  \\ \hline
        AmLi (CSD1 700mm) & \\ 
        DD                & \\
        ${}^{252}{Cf}$    &
    \end{tabular}
    \caption{Neutron capture time on Gadolinium isotopes in GdLS}
    \label{tab:neutron_capture_times}
\end{table}

Given that these neutrons are captured releasing in excess of 3MeV, they will have been captured by a Gd-isotope. 
Therefore what is being observed is most likely that the neutron bounces around in the acrylic, which would likely capture after 220$\mu$s before entering the GdLS to be captured.
This effect was observed in TODO CITE XXX so is reasonable that it would be seen here as well.
It could be validated deploying the calibration sources in a number of different locations where the thickness of the acrylic differs, this study may be possible pre-SR2.


\par
The GdLS cocktail used by LZ is similar to that of the DayaBay experiment, and as such, the neutron capture properties should be similar.
Using Equation XXX which is an extended version of that from DayaBay \cite{Dayabay_neutron_capture_fit_ref} was used to fit to both the thermalisation peak and the subsequent captures.

\begin{equation}
    N = Y
\end{equation}



%\section{Particle Discrimination}
\par
As previously mentioned, the OD has been designed to act as an ``active but dumb" veto.
Given that is has been shown that BiPo and RnPo events can be identified based upon their proximity to each other with relative ease, these events would be un-vetoed, increasing the number of candidate events as well as the live-time and ultimately sensitivity of LZ to a dark matter interaction.
In this section a different method of discriminating between events is introduced, using a likelihood approach.
This is then tested on some simulated datasets.

\subsection{Maximum Likelihood Method}
\par
The basic concept of a maximum likelihood approach is to have a signal model and background model that is described in terms of a number of variables.
The likelihood of an event being a signal is then given simply by:
\begin{equation}
    y_{\Lagrangian} = \frac{\Lagrangian_{S}}{\Lagrangian_{S} + \Lagrangian_{B}}
\end{equation}
where 
\begin{equation}
    \Lagrangian_{S,(B)} = \prod_k p_{S,(B),k}(x_k)
\end{equation}
where $p_{S,(B),k}$ is the signal (background) PDF for the $k^{th}$ input variable $x_k$.
In this approach any correlation between variables is ignored, and is often referred to as a ``naive Bayes estimator" \cite{TMVA_ref}.


\subsection{Variables}
\par
Three variables were selected in the application of this method to LZ simulations which are described below.



\paragraph{coincidence}
Sometimes referred to as pulse multiplicity, the coincidence value of a pulse is the number of PMTs which produced a signal that contributed to the pulse.
As the more photons produced, the more will be detected, and so the more likely is it that any given PMT will see this light.
It is a very robust and simple variable, and therefore ideal to start with.
A neutron capture will likely have a higher coincidence as the multiple $\gamma$'s can scatter in different areas of the OD.


\paragraph{pulse area}
The number of photons detected is directly proportional to the energy of the interaction that occurred.
Combined with the coincidence, it has the potential to isolate unwanted pulses such as those caused by PMT after-pulsing.


\paragraph{$\frac{\text{width}}{\text{pulse area}}$}
This variable is the most complicated of the selection, which gives a pulse-shape.
The width of a pulse is the time.
Due to pulse finding limitations, the full reconstructed pulse width was not used.
Instead, the pulse start time was defined as; the time where 5\% of the pulse integral was reached, and the end time as 75\% of the pulse integral.

However, the quantity can be improved by combining it with a second variable; the pulse area.
This has the advantage of removing the pulse area as an influencing parameter in the value and making the variable more stable.
In the simplest case, the quantity gives is how fast or slow the pulse decreases after the peak.
An example of different interactions with this variable can be found in Figure XXX.

\begin{figure}[!htbp]
    \centering
    \includegraphics[width=0.5\textwidth]{Figures/Placeholder.png}
    \caption{Variables used for particle discrimination}
    \label{fig:discrimination_variables}
\end{figure}

\begin{figure}[]%
\centering
\begin{tikzpicture}
\centering
    \begin{groupplot}[%view={0}{90},
    group style = {group size = 1 by 3,vertical sep=1.5cm}]
    \nextgroupplot[
            xlabel=Pulse Area,
            ylabel=Proportion,
            width=15cm, height=6cm,
            xmin=0, xmax=1000,
            ymin=0,% ymax=100,
            %minor y tick num=4,
            grid=major,
            legend style = { column sep = 10pt, legend columns = -1, legend to name = Simulated_Likelihood_Variables_CommonLegend,}]
            \addplot[green, mark=none]
                    table [x=Centre,y=Normed]
                    {Data/OD_Likelihood/gd_capture_pulseArea_phd.dat};
            \addplot[blue, mark=none]
                    table [x=Centre,y=Normed]
                    {Data/OD_Likelihood/h_capture_pulseArea_phd.dat};
            \addplot[red, mark=none]
                    table [x=Centre,y=Normed]
                    {Data/OD_Likelihood/rockgamma_pulseArea_phd.dat};
            %\node[] at (axis cs: 800,87) {\large 0 mm};
            \legend{Gd-capture,H-capture,Cavern $\gamma$}
        
        \nextgroupplot[
            xlabel=N. PMTs,
            ylabel=Proportion,
            width=15cm, height=6cm,
            xmin=0, xmax=120,
            ymin=0, %ymax=100,
            %minor y tick num=4,
            grid=major,]
            \addplot[green, mark=none]
                    table [x=Centre,y=Normed]
                    {Data/OD_Likelihood/gd_capture_coincidence.dat};
            \addplot[blue, mark=none]
                    table [x=Centre,y=Normed]
                    {Data/OD_Likelihood/h_capture_coincidence.dat};
            \addplot[red, mark=none]
                    table [x=Centre,y=Normed]
                    {Data/OD_Likelihood/rockgamma_coincidence.dat};
            %\node[] at (axis cs: 800,87) {\large 1400 mm};
        \nextgroupplot[
            xlabel=Pulse Time / pulse Area,
            ylabel=Proportion,
            width=15cm, height=6cm,
            xmin=0, xmax=2,
            ymin=0, %ymax=100,
            %minor y tick num=4,
            grid=major,]
            \addplot[green, mark=none]
                    table [x=Centre,y=Normed]
            {Data/OD_Likelihood/gd_capture_areaFractionTime75_ns_vs_pulseArea_phd.dat};
            \addplot[blue, mark=none]
                   table [x=Centre,y=Normed]
              {Data/OD_Likelihood/h_capture_areaFractionTime75_ns_vs_pulseArea_phd.dat};
            \addplot[red, mark=none]
                    table [x=Centre,y=Normed]
                    {Data/OD_Likelihood/rockgamma_areaFractionTime75_ns_vs_pulseArea_phd.dat};
            %\node[] at (axis cs: 800,87) {\large 1400 mm};
             
    \end{groupplot}
    \node at ($(group c1r1) + (-0.5cm, 3.0cm)$) {\ref{Simulated_Likelihood_Variables_CommonLegend}};
\end{tikzpicture}
    \caption{Parameter-space population for various interactions.}
    \label{fig:simulated_likelihood_variables}
\end{figure}

\subsection{Performance}
\par
For this study, the signal was selected to be neutron captures and the background as cavern-$\gamma$'s
The neutron captures were simulated by starting thermalised neutrons in the GdLS.
This meant that only the energy deposits in the OD as a result post-capture were detectable.
The 
\par
The likelihood was performed against a set of simulations which contained a mixture of both neutron captures and cavern-$\gamma$'s in quantities expected \cite{LZ_assay_ref}.
The neutrons were, as above, thermalised so the only interaction of note are from the result of neutron captures.






\begin{figure}[!htbp]
    \centering
    \includegraphics[width=0.5\textwidth]{Figures/Placeholder.png}
    \caption{ROC curve of}
    \label{fig:discrimination_performance}
\end{figure}

\par
Although it is possible to use this method with data, it has been impractical given the differences mentioned earlier in this Chapter.
In order to perform this, a data-driven approach is required and simulation corrections implemented.
Both of which have not been performed due to time constraints.





\par
Interaction discrimination based upon a likelihood has the potential to improve the OD effectiveness; reducing the dead-time as as result of false vetoes.
It can also reduce the uncertainties on backgrounds - particularly if combined with detector coincidences on Gd-captures as was done in Section XXX.
There is scope for additional interesting studies such as seasonal modulation in rates of interactions such as ($\alpha$,$\gamma$) \cite{cavern_gammas_in_Soudan_mine_ref}.
All of these together can directly improve the sensitivity of LZ to a potential dark matter detection.

\par
Once data is better understood, it may be possible to implement higher dimensional variables to discriminate interactions.
Though all of these are subject to an eventual match-up in simulation and data.


\subsubsection*{Summary}
\par
In this chapter, the backgrounds in the OD were measured and fitted.
Most backgrounds were within error of what was expected but there was an elevated rate of ${}^{210}$Po which has been linked to cavern air entering the acrylic tanks during construction.
The neutron veto efficiency was measured as 84.6\%, which is lower than the 91.0\% expected from simulations and the 95\% required by the design requirements.
The most likely cause of this was determined to be the neutrons taking a longer time to reach the GdLS.
The dominant factor appears to be water ingress into the foam which surrounds the OCV.
Neutrons thermalise in this water, causing them to take longer to reach the GdLS and can also be captured there.