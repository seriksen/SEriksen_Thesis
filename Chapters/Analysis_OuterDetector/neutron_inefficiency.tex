\section{Neutron Efficiency}
\par
With a measure of the background rate, the focus now shifts towards determining the veto time window and veto energy threshold that is needed to meet the requirements set out in \autoref{tab:veto_requirements}.
The structure of this section is analogous to \autoref{sec:od_simulation_efficiency}, where an AmLi calibration source is used to determine the veto efficiency.

\subsection{AmLi Veto Efficiency}
\par
For determining the neutron tagging efficiency of the OD, the AmLi calibration source was used.
Three AmLi sources were used simultaneously to improve the neutron flux, each of which were lowered into a separate calibration source tube.
The activity and locations of each of the sources is summarised in \autoref{tab:amli_source_activities}.

\begin{table}[!htbp]
    \centering
    \begin{tabular}{c|c|c}
        CSD & AmLi Source No. & Activity (n/s) \\ \hline
        1   & Source-2        & 13.8           \\
        2   & Source-1        & 9.3            \\ 
        3   & Source-3        & 11.9                
    \end{tabular}
    \caption{AmLi source activities in each calibration source deployment tube (CSD).
             The activity of the sources are from \cite{LZ_TechnicalDesignReview_ref}.}
    \label{tab:amli_source_activities}
\end{table}

\par
Data was taken for a number of hours, with the goal of having a minimum of 1000 NR events in the TPC from the neutrons.
Three separate sets of data were taken at varying heights in the CSD tubes: 0 mm, 700 mm, and 1400 mm.
These are the same was were presented in \autoref{sec:od_simulation_efficiency}.
During data taking the S2 Trigger (see \autoref{sec:lz_detector}) was used, requiring that a large pulse is seen in the TPC.

\par
On the recorded events a number of analysis cuts were applied to determine which TPC events were significant.
In addition to the core cuts; of single scatters (\textbf{SS}), region of interest (\textbf{ROI}) but adapted slightly for to isolate nuclear recoils (\textbf{NR}) and a fiducial cut (\textbf{FID}), two other subgroup of cuts were applied.
These were based on the S1 pulse (\textbf{S1Cuts}) and the S2 pulse (\textbf{S2Cuts}).
These are a subset of the same cuts which were developed for the SR1 WIMP search \cite{lz_ws_sr1_ref} and are summarised below:
\begin{enumerate}
    \item \textbf{SS}: Single Scatter cut as determined by an interaction finder algorithm. This was tuned on the same dataset used here.
    \item \textbf{S1Cuts}: This cut focused on only ensuring that the S1 pulse was not causing by any other process. An example of this is the removal of S1 pulses where the contribution is primarily from a single PMT, indicating that it is actually an accidental. S1 pulses which did not have a 3-fold coincidence are also removed.
    \item \textbf{S2Cuts}: This primarily removed events which originated in or near to the GXe. If the scatter occurs very close to the GXe then the time separation between the S1 and S2 would be sufficiently low that the pulse finding algorithm could not separate them. Similarly, GXe scatters would have merged S1 and S2s. Events with a poor position reconstruction (which uses the S2 hit pattern) are also removed this way so as to not influence the \textbf{FID} selection. In combination with the \textbf{S1Cuts}, nonphysical events are removed by the time difference between the pulses.
    \item \textbf{FID}: The inner volume of the TPC defined in terms of \{$r,z$\}. $r$ is set to a distance [4.0 5.2] cm from the TPC wall depending on $z$. $z$ was defined by electron drift time of [86, 936.5] $\mu$s.
    \item \textbf{NR}: Only events which fell within the $\pm 1\sigma$ of the nuclear recoil expectation. The NR band used is from DD neutron calibrations.
\end{enumerate}
A summary of the number of events available for this analysis are presented in \cite{tab:amli_calibration_summary}, and the resulting events shown in Figure XXX.

\begin{table}[!htbp]
    \centering
    \begin{tabular}{c|c|c|c}
        \multirow{2}{*}{z position (mm)} & \multirow{2}{*}{Run IDs}  & \multicolumn2{c}{Number of events}  \\ 
                                         &                           & Acquired    & After all cuts     \\ \hline
        0                                & 8350-8369                 & 5,504,700  & 2252               \\
        700                              & 8304-8317                 & 3,688,200  & 1464               \\ 
        1400                             & 8319-8348                 & 7,041,500  & 3293                
    \end{tabular}
    \caption{Summary of AmLi source deployment during post SR1 calibrations}
    \label{tab:amli_calibration_summary}
\end{table}

\par
In a similar fashion to the previous chapter the efficiency for vetoing a neutron is defined as:
\begin{equation}
    \epsilon = 1 - \frac{\text{events passing}\mathbf{SS, S1Cuts, S2Cuts, NR, FID, Veto}}{\text{events passing}\mathbf{SS, S1Cuts, S2Cuts, NR, FID}}
    \label{eq:data_neutron_efficiency}
\end{equation}
OD pulses were subject to the same requirements described in the previous section, with the noise-cut being applied.
The efficiency vs the veto time window is shown in \autoref{fig:commissioning_amli_efficiency_with_bg_rate}.
Overlaid are the rate of backgrounds above 200 keV in the OD.
The red dashed line indicates the when the a 5\% false veto rate is reached, which can be considered the same as the impact on WIMP search livetime.
The point where these two lines cross is the value of the veto window when 5\% of live time will be vetoed.
For a 200 keV threshold this is just above 1200 $\mu$s.
In order to maximise the veto efficiency, 1200 $\mu$s was selected as the time window for SR1.
Complimentary, to the efficiency vs time, \autoref{fig:commissioning_amli_efficiency_per_phd} shows the efficiency vs threshold.

\begin{figure}[]%
\centering
\begin{tikzpicture}
\centering
    \begin{axis}[
            ylabel=Efficiency (\%),
            xlabel=Photons Detected (phd),
            width=15cm,
            height=8cm,
            grid=major,
            xmin=0, xmax=100,
            %ymin=45, ymax=100,
            %minor y tick num=1,
            ]
        \addplot[red,
                 error bars/.cd, error bar style={color=black},
                 y dir=both, y explicit, 
                 x dir=both, x explicit,
                 ]
            table [x=bin,y=value, y error minus index=4, y error plus index=5, x error minus index=2, x error plus index=3,]
            {Data/Neutron_Efficiency/AmLi_Commissioning/pos_0_efficiency_per_phd.dat};    
        \addplot[green, 
                 error bars/.cd, error bar style={color=black},
                 y dir=both, y explicit, 
                 x dir=both, x explicit,
                 ]
            table [x=bin,y=value, y error minus index=4, y error plus index=5, x error minus index=2, x error plus index=3,]
            {Data/Neutron_Efficiency/AmLi_Commissioning/pos_700_efficiency_per_phd.dat};             
        \addplot[blue,
                 error bars/.cd, error bar style={color=black},
                 y dir=both, y explicit, 
                 x dir=both, x explicit,
                 ]
            table [x=bin,y=value, y error minus index=4, y error plus index=5, x error minus index=2, x error plus index=3,]
            {Data/Neutron_Efficiency/AmLi_Commissioning/pos_1400_efficiency_per_phd.dat};    
        \legend{0mm,700mm,1400mm}  
            
    \end{axis}
            
\end{tikzpicture}
    \caption{Efficiency as a function of pulse area}
    \label{fig:commissioning_amli_efficiency_per_phd}
\end{figure}

\begin{figure}[!htbp]%
\centering
\begin{tikzpicture}
\centering
    \begin{axis}[
            ylabel=Efficiency (\%),
            xlabel=Time ($\mu$s),
            width=15cm,
            height=8cm,
            grid=major,
            axis y line*=left,
            xmin=0, xmax=1500,
            ymin=45, ymax=100,
            minor y tick num=1,
            legend style = { column sep = 10pt, legend columns = -1,}
            ]
        \addplot[red,
                 mark size=0.5, mark=*, mark options={solid},
                 error bar legend,
                 error bars/.cd, error bar style={color=red},
                 y dir=both, y explicit, 
                 x dir=both, x explicit,
                 ]
            table [x=bin,y=value, y error minus index=4, y error plus index=5, x error minus index=2, x error plus index=3,]
            {Data/Neutron_Efficiency/AmLi_Commissioning/pos_0_total_200kev_efficiency.dat};    
        \addplot[green, 
                 mark size=0.5, mark=*, mark options={solid},
                 error bar legend,
                 error bars/.cd, error bar style={color=green},
                 y dir=both, y explicit, 
                 x dir=both, x explicit,
                 ]
            table [x=bin,y=value, y error minus index=4, y error plus index=5, x error minus index=2, x error plus index=3,]
            {Data/Neutron_Efficiency/AmLi_Commissioning/pos_700_total_200kev_efficiency.dat};             
        \addplot[blue, 
                 mark size=0.5, mark=*, mark options={solid},
                 error bar legend,
                 error bars/.cd, error bar style={color=blue},
                 y dir=both, y explicit, 
                 x dir=both, x explicit,
                 ]
            table [x=bin,y=value, y error minus index=4, y error plus index=5, x error minus index=2, x error plus index=3,]
            {Data/Neutron_Efficiency/AmLi_Commissioning/pos_1400_total_200kev_efficiency.dat};    
        \legend{0mm,700mm,1400mm}  
            
    \end{axis}
    \begin{axis}[
            ylabel=False Veto (\%),
            yticklabel pos=right,
            axis y line*=right,
            axis x line=none,
            width=15cm,
            height=8cm,
            %grid=major,
            xmin=0, xmax=1500,
            ymin=0, ymax=50,
            minor y tick num=9,]
        \addplot[domain=0:1500,
            samples=3,
            ]
            {x * 42.5 * 100 / 1000000};
        \addplot[domain=0:1500,
            samples=3,
            ]
            {x * 95.8 * 100 / 1000000};        
        \addplot[domain=0:1500,
            samples=3,
            ]
            {x * 276.8 * 100 / 1000000};  
         \addplot[dashed, mark=none, red] coordinates {(0,5) (1500,5)};
         
         \node[rotate=29] at (axis cs: 1000,26) {276.8Hz};
         \node[rotate=13] at (axis cs: 1400,15) {95.8Hz};
         \node[rotate=5] at (axis cs: 1400,8) {42.5Hz};
    \end{axis}
            
\end{tikzpicture}
    \caption{Neutron tagging efficiency from AmLi at each height for the 200keV phe threshold.
    The horizontal dashed line is the 5\% impact on livetime requirement.}
    \label{fig:commissioning_amli_efficiency_with_bg_rate}
\end{figure}

\par
In SR1 the impact on livetime in the \textbf{ROI} was 4.99\%.
In \autoref{fig:sr1_vetoed_events} the events which passed all other cuts expect the OD veto as well as events which were vetoed.
Overlaid are the ER and NR bands.
No vetoed event is actually in the NR band, but interestingly there are a number of events there.
These events are not neutrons though and are consistent with ER leakage into the NR band.
Every event which was vetoed is an ER event from one of the many backgrounds in the TPC.
The veto is being triggered by coincident internal decays.

\begin{figure}[!htbp]%
\centering
\begin{tikzpicture}
\centering
    \begin{axis}[
            ylabel=Log10(S2) (phd),
            xlabel=S1 (phd),
            width=15cm,
            height=8cm,
            %ymin=45, ymax=100,
            %minor y tick num=1,
            ]
        %\addplot[black, only marks,]
        %    table []
        %    {Data/Neutron_Efficiency/AmLi_Commissioning/s1s2_passed_od_veto_in_fid.dat};    
        \addplot[red, only marks,]
            table []
            {Data/Neutron_Efficiency/AmLi_Commissioning/s1s2_failed_od_veto_in_fid.dat};
            
    \end{axis}
            
\end{tikzpicture}
    \caption{TPC events vetoed by the OD. In Red are events which passed all cuts including the OD veto. In Blue are events which passed all cuts except the OD veto.}
    \label{fig:sr1_vetoed_events}
\end{figure}

\par
Turning back to the efficiency, when the shape of the curve is compared to the simulation result in the previous chapter the curve shape appears to be slightly different.
This prompted the study in the next section into neutron propagation.

\subsection{Neutron Propagation}
\par
In order to study the neutron propagation difference between the simulation and data, neutron captures were used.
In order to do this, a high purity of neutrons are needed, and ideally high statistics.
${}^{252}$Cf calibrations were deemed the most appropriate as the neutron rate is 430 per second.
It is deployed in the calibration source tube in the same manor as the AmLi so neutrons will travel from the OCV towards the OD.
\par
${}^{252}$Cf decays via spontaneous fission releasing on average 3.7 neutrons each time \cite{californium252_ref}.
A number of $\gamma$s with a combined energy of up to 10 MeV are also released.
The process is shown in \autoref{fig:fission_fragments_time}.
The vast majority of these $\gamma$s are released within the first 100 ns of the fission event \cite{cf252_fission_ref,californium_spectra_ref}.
%\begin{figure}[!htbp]
%\includegraphics[width=13cm]{Figures/NeutronCaptureTime/fission_fragment_times.png}
%\centering
%\caption{Emission time-frame of fission components. Adapted from \cite{cf252_fission_ref}}
%\label{fig:fission_fragments_time}
%\end{figure}
Given that the $\gamma$s are released in close in time to when the neutrons are released they can be used to tag the neutron creation time, $t_0$.
An example event from a ${}^{252}$Cf is shown in \autoref{fig:cf252_event_viewer}.
There is a signal in all three detectors at the same time which are from the $\gamma$s.
The later pulses in the OD are then from neutron capture, two captures in this case.

\begin{figure}[!htbp]
\includegraphics[width=\textwidth]{Figures/NeutronCaptureTime/cf252_eventviewer_5916.png}
\centering
\caption{Example event from ${}^{252}{Cf}$ calibration run showing the $\gamma$'s causing coincident pulses in each detector followed by two neutrons being captured in the Outer Detector}
\label{fig:cf252_event_viewer}
\end{figure}

\par
The ${}^{252}$Cf datasets selected for this are from runs which had a single source in CSD-1 at the 700 mm level.
An event was only considered if a triple coincidence was seen, defined as both the Skin and TPC detectors seeing light within [-200, 200] ns of the pulse seen in the OD.
Additionally, to avoid backgrounds contributing to this, an energy threshold in the OD of 250 keV was used.
The spectra of OD pulses which make up the $t_0$ event are shown in \autoref{fig:cf252_pulse_selection}.
The OD pulses both before and after $t_0$ are also shown.
The shapes are quite different between each selection.
The peak around 450 phd in the before and after $t_0$ are the 2.2 MeV from n-H captures and it is reassuring that it is missing from the $t_0$ distribution.
The number of events after $t_0$ is consistent with neutrons being capture some time after the $\gamma$.
As there will still be pulses after $t_0$ which are OD background events, only the largest pulse in an event was used.
This distribution is shown in \autoref{cf252_gd152_captures}.
Finally, only pulses above 600 phd were considered.
These are n-Gd captures.
This is to ensure that captures in other H-rich mediums were not included (such as in the acrylic tanks and foam).

\begin{figure}[]%
\centering
\begin{tikzpicture}
\centering
    \begin{axis}[
            ylabel=Count,
            xlabel=Pulse Area (phd),
            width=15cm,
            height=8cm,
            grid=major,
            xmin=0, xmax=2500,
            ymode=log,
            %ymin=45, ymax=100,
            %minor y tick num=1,
            ]
        \addplot[red, only marks, mark size=1.0,
                 error bar legend,
                 error bars/.cd, error bar style={color=black},
                 y dir=both, y explicit, 
                 x dir=both, x explicit,
                 ]
            table [x=pulsearea,y=weight, x error=xerror, y error=yerror]
            {Data/cf252/cf252_od_pulses_before400ns.dat};    
        \addplot[green, only marks, mark size=1.0,
                 error bar legend,
                 error bars/.cd, error bar style={color=black},
                 y dir=both, y explicit, 
                 x dir=both, x explicit,
                 ]
            table [x=pulsearea,y=weight, x error=xerror, y error=yerror]
            {Data/cf252/cf252_od_pulses_within400ns.dat};           
        \addplot[blue, only marks, mark size=1.0,
                 error bar legend,
                 error bars/.cd, error bar style={color=black},
                 y dir=both, y explicit, 
                 x dir=both, x explicit,
                 ]
            table [x=pulsearea,y=weight, x error=xerror, y error=yerror]
            {Data/cf252/cf252_od_pulses_after400ns.dat};
        \legend{Before 400ns,Within 400ns,After 400ns}  
            
    \end{axis}
            
\end{tikzpicture}
    \caption{OD pulses from ${}^{252}$Cf where there was a triple detector coincidence pulse.
             `Within 400ns' are OD pulses with a triple detector coincidence defined as when the Skin and TPC saw a significant pulse within [-200,200] ns of the OD.
             `Before 400ns' are all OD pulses before the triple coincidence pulse.
             `After 400ns' are all OD pulses after the triple coincidence pulse.}
    \label{fig:cf252_pulse_selection}
\end{figure}

\begin{figure}[]%
\centering
\begin{tikzpicture}
\centering
    \begin{axis}[
            ylabel=Count,
            xlabel=Pulse Area (phd),
            width=15cm,
            height=8cm,
            grid=major,
            xmin=0, xmax=2500,
            ymin=0, ymax=500,
            %minor y tick num=1,
            ]
        \addplot[black, only marks, mark size=1.0,
                 error bars/.cd, error bar style={color=black},
                 y dir=both, y explicit, 
                 x dir=both, x explicit,
                 ]
            table [x=pulsearea,y=weight, x error=xerror, y error=yerror]
            {Data/cf252/cf252_od_largest_after400ns.dat}; 

         \addplot[red, dashed]
            coordinates{(600, 0)(600,500)};
            
    \end{axis}
            
\end{tikzpicture}
    \caption{Largest OD pulse in an event window 400 ns after the largest triple detector coincidence pulse.
             The peak at 450~phd are n-H captures.
             Pulses with a pulse area to the right of the red dashed line are caused by n-Gd captures.}
    \label{fig:cf252_gd152_captures}
\end{figure}

\par
The time difference between the capture pulse and $t_0$ is shown in \autoref{fig:cf252_gd_capture_time}.
The observed neutrons are taking longer to reach the GdLS.
A fit (following \autoref{eq:neutron_capture_time}) was performed and found that two exponential were best, with $t_0 = 31.39 \pm 0.4\mu$s and $t_1 = 124.5 \pm 1.7\mu$s, with a $\chi^2=0.821$.
Included as well is the result from simulation of neutrons from the same starting position.
$\tau_0$ is in good agreement with the simulations, indicating that once the neutrons reach the GdLS they are captured quite fast.
However, neutrons are clearly be held up in other regions of the detector, somewhere between the calibration tube and the GdLS that isn't simulated.
The most likely reason is that water has gotten in between the OCV and acrylic tanks.

\begin{figure}[!htbp]%
\centering
\begin{tikzpicture}
\centering
    \begin{axis}[
            ylabel=Count,
            xlabel=Capture Time ($\mu$s),
            width=15cm,
            height=8cm,
            grid=major,
            ymode=log,
            xmin=-10,xmax=1000,
            %ymin=45, ymax=100,
            %minor y tick num=1,
            ]
        \addplot[black, only marks, mark size=0.5,
                 error bars/.cd, error bar style={color=black},
                 y dir=both, y explicit, 
                 x dir=both, x explicit,
                 ]
            table [x=time,y=weight, x error=xerror, y error=yerror]
            {Data/cf252/cf252_gd_capture_time.dat}; 
            
        \addplot[red,
                 domain=15:1500,
                 ]
            {2305 * exp(-x/30.5) + 342 * exp(-x/125.2) + 0.9663}; 
            
    \end{axis}
            
\end{tikzpicture}
    \caption{Pulse Distribution depending upon when the tagging pulse was.
             In red is the best fit from 2 exponential decays.}
    \label{fig:cf252_gd_capture_time}
\end{figure}

\par
Water ingress into foam has been well studied at least in part, with the industrial measures of water absorption defined as via diffusion and by submersion \cite{foam_with_water_ref}.
The effectiveness of any foam type to keep water out is based on a number of factors including density and most importantly if the structure is closed or open\footnote{in terms of whether the cells which make up the foam structure. Open cell foams are softer and more flexible as each cell is open to the atmosphere. Closed cell foams are firmer with the majority of cells being closed to the atmosphere.} \cite{mechanical_properties_of_foam_ref}.
An example of water ingress is shown in \autoref{fig:foam_mri_water_ingress}.


\begin{figure}[!tbhp]
\includegraphics[width=0.5\textwidth]{Figures/NeutronCaptureTime/foam_water_absorption.png}
\centering
\caption{Magnetic resonance images of 30 mm cube samples of polyurethane foam for two types of different foam of the same density, 0.3 g m${}^{-3}$: a closed cell foam (top) and an open cell foam (bottom).
(a) closed cell foam after 8 hours. (b) closed cell foam after 63 hours.
(c) open cell foam after 8 hours. (d) open cell foam after 72 hours.
Images from \cite{foam_mri_data_ref}.
}
\label{fig:foam_mri_water_ingress}
\end{figure}


\par
The foam installed around the OCV (green Styrodur foam in \autoref{sec:od_construction_sec}) is a closed cell foam expanded using Co$_2$, so each cell is filled with Co$_2$; foam 3000 CS in \cite{styrodur_water_ingress_ref}.
As such the foam is expected to experience a 0.7\% volume ingress of water due to long term submersion and 5\% volume ingress of water from diffusion.
Long-term in this case refers to a 28-day test as is the industry standard\footnote{see BS EN ISO 16535 (previously BS EN 12087) and BS EN ISO 16536 (previously BS EN 12088)}.
These values do not necessarily map obviously onto how it was used in this application, particularly given that water pressure is not considered in the and the submersion length is orders of magnitude less than the LZ foam has had already.
The other two foams, HandiFoam\textsuperscript{\textregistered} and white polyethylene foam are also closed-cell and have slightly different submersion results
\cite{handifoam_water_ingress_ref, white_foam_ref}. 
\par
In order to probing this, a test was conducted to observe what happens to this foam (and what happens to the water) during the submersion.
An example of each piece of foam used, was placed in de-ionised water and nitrogen was bubbling through the water. 
The water resistivity was measured every few hours as a monitor of the waters purity.
The result over the first week of running is shown in \autoref{fig:od_foam_degredation}.
What is observed during this period is the water becoming contaminated with ions from the foam from out gassing.
Over a longer period of study it will be possible to measure the water saturation, particularly with a test using a piece of foam shaped to the same dimensions as those used in the installation as the rate at which water ingress occurs will depend upon the surface area.
\par
In relation to LZ, this has two impacts, firstly these ions can circulate throughout the OD water and contribute to the low energy rate.
Secondly, this will impact on the time it takes for a neutron to reach the GdLS.
In \autoref{fig:data_vs_sim_gd_capture_time} are the simulated neutron capture time in GdLS assuming the extreme case of all of the water being directly replaced with water, and the more realistic scenario of the water being filled 5\% (by volume) with water.
There 5\% saturation case provides a close match to the observed time, though there needs to be some tuning as it is not perfect.

\begin{figure}[!htbp]%
\centering
\begin{tikzpicture}
\centering
    \begin{axis}[
            xlabel=Time (hours),
            ylabel=Resistivity (mS/cm),
            width=15cm,
            height=8cm,
            grid=major,
            xmin=-1, xmax=250,
            legend style={at={(0.95,0.5)},anchor=east},
            %ymin=45, ymax=100,
            %minor y tick num=1,
            ]
        \addplot[green, only marks,
                 error bars/.cd, error bar style={color=black},
                 y dir=both, y explicit, 
                 ]
            table [x=time,y=di,y error=error]
            {Data/cf252/foam_in_di_water.dat}; 
        \addplot[blue, only marks,
                 error bars/.cd, error bar style={color=black},
                 y dir=both, y explicit, 
                 ]
            table [x=time,y=foam,y error=error]
            {Data/cf252/foam_in_di_water.dat}; 
        \addplot[red, dashed,
                 domain=-10:750,
                 samples=3]
                {1};
            
        \legend{Control, Foam,};
    \end{axis}
            
\end{tikzpicture}
    \caption{Measure of water purity from foam used in the OCV.
             Both the Control (just water) and the Foam (water with foam pieces) were under $N_2$ purge for the duration of the data taking.
             The dashed red line indicated the Type 2 DI water limit.}
    \label{fig:od_foam_degredation}
\end{figure}

\begin{figure}[!htbp]%
\centering
\begin{tikzpicture}
\centering
    \begin{axis}[
            ylabel=Count (Arb.),
            xlabel=Capture Time ($\mu$s),
            width=15cm,
            height=8cm,
            grid=major,
            ymode=log,
            xmin=-10,xmax=800,
            ymin=1e-3, ymax=5,
            %minor y tick num=1,
            ]
        \addplot[black, only marks, mark size=0.5,
                 error bars/.cd, error bar style={color=black},
                 y dir=both, y explicit, 
                 ]
            table [x=time,y=weight, y error=yerror]
            {Data/cf252/cf252_gd_capture_time_normed.dat}; 
            
        \addplot[red, const plot]
            table [x=time,y=weight]
            {Data/cf252/amli_0mm_default.dat}; 

        \addplot[blue, const plot]
            table [x=time,y=weight]
            {Data/cf252/amli_0mm_water.dat}; 
            
    \end{axis}
            
\end{tikzpicture}
    \caption{Neutron capture time on Gadolinium in the OD. The observed capture time from ${}^{252}Cf$ is shown in black. In red is the expected capture time from simulation. In blue is the capture time if all foam around the OCV has been replaced with water. Each distribution has been normalised the maximum value.}
    \label{fig:data_vs_sim_gd_capture_time}
\end{figure}
