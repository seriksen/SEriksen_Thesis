\section{Neutron Inefficiency}
\par
As discussed in Section XXX, neutrons are a particularly important background with the TPC signal produced very similar to the 



\begin{tcolorbox}[colback=red!5!white, colframe=red!50!black, title=Key Plots]
\begin{enumerate}
    \item Simulation Neutron Capture Time
    \item Data Neutron Capture Time
    \item Simulation Inefficiency
    \item Simulation AmLi Inefficiency
    \item Data AmLi Inefficiency
\end{enumerate}
\end{tcolorbox}

\par
The largest light source resulting from a neutron, that will be seen by the OD PMTs are the neutron capture.
Therefore this is a valid quantity to measure.

\subsection{Neutron Captures}
\par
The ${}^{252}{Cf}$ calibration source used provides a very good indication of the neutron capture times. 
Spontaneous fission of ${}^{252}{Cf}$ typically results in a number of $\gamma$'s with a combined energy of up to 10MeV which are accompanied by a number of neutrons.
An example event from the LZ calibration run pre-SR1 is shown in Figure \ref{fig:cf252_event_viewer}.
This allows for neutrons to be tagged by tagging the fission by coincident $\gamma$'s in a each detector, and assuming later pulses are dominated by neutrons.

\begin{figure}[!htbp]
\includegraphics[width=10cm]{Figures/NeutronCaptureTime/cf252_eventviewer_5916.png}
\centering
\caption{Example event from ${}^{252}{Cf}$ calibration run showing the $\gamma$'s causing coincident pulses in each detector followed by two neutrons being captured in the Outer Detector}
\label{fig:cf252_event_viewer}
\end{figure}

\par
For the other neutron calibration sources, DD and AmLi, the same neutron tagging as described in Section XXX was used, where the NR band is isolated.
The result of this search is shown in figure XXX - overlaid is the neutron capture sim in the simulations.
What is apparent is that there is more than one exponential decays in data.
The simplest explanation is that a portion of the neutrons are being captured on H in either the water or acrylic that is between the source and GdLS.
In this case, the capture constant would be 220$\mu$s.
Initially this was excluded by selecting pulses above the H-peak, selecting events above 500phd.
However, as can be seen in Figure XXX, at least two exponents remain.
A fit was performed on each of the calibration sources separately, to equation XXX, and the results are summarised in Table \ref{tab:neutron_capture_times}.
\begin{table}[!htbp]
    \centering
    \begin{tabular}{c|c}
        Source            &  \\ \hline
        AmLi (CSD1 700mm) & \\ 
        DD                & \\
        ${}^{252}{Cf}$    &
    \end{tabular}
    \caption{Neutron capture time on Gadolinium isotopes in GdLS}
    \label{tab:neutron_capture_times}
\end{table}

Given that these neutrons are captured releasing in excess of 3MeV, they will have been captured by a Gd-isotope. 
Therefore what is being observed is most likely that the neutron bounces around in the acrylic, which would likely capture after 220$\mu$s before entering the GdLS to be captured.
This effect was observed in TODO CITE XXX so is reasonable that it would be seen here as well.
It could be validated deploying the calibration sources in a number of different locations where the thickness of the acrylic differs, this study may be possible pre-SR2.

