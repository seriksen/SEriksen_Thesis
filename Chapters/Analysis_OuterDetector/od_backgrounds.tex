\section{Backgrounds}
\par
As mentioned in Section XXX (LZ chapter bit), an assay was performed to establish the backgorund rates of all components.
In addition to the assay, a separate cavern gamma study was performed along with the GdLS.

\begin{tcolorbox}[colback=red!5!white, colframe=red!50!black, title=Key Plots]
\begin{enumerate}
    \item Simulation Expected Rate of top Components
    \item Scotts Measured GdLS vs Data GdLS
    \item Scaling in X
    \item Scaling justification
    \item Fitted Result
    \item Table of Measured Rates
    \item LCE map?
\end{enumerate}
\end{tcolorbox}


\begin{table}[!htbp]
    \centering
    \begin{tabular}{c|c|c}
        Isotope or Subchain  &  Initial Purification & Improved Purification \\ \hline
        ${}^{238}U_{e}$      &  $< 1.04$             & $< 0.017$             \\ 
        ${}^{238}U_{m}$      &  $0.092\pm0.02$       & $0.010\pm0.004$       \\
        ${}^{235}U_{e}$      &  $0.011$              & $< 0.018$             \\
        ${}^{235}U_{l}$      &  $0.10\pm0.04$        & $< 0.012$             \\
        ${}^{232}Th_{e}$     &  $< 0.027$            & $< 0.0036$            \\
        ${}^{232}Th_{l}$     &  $< 0.020$            & $< 0.0030$            \\
        ${}^{40}K$           &  $< 0.22$             & $< 0.0092$            \\
        ${}^{138}La$         &  $< 0.0055$           & $< 0.0017$            \\
        ${}^{176}Lu$         &  $0.30\pm0.07$        & $0.0081\pm0.0018$     \\
        ${}^{152}{Gd}$       &  $1.61\pm0.08$        & $1.61\pm0.08$         \\
        ${}^{152}{Sm}$       &  $1.02\pm0.05$        & $1.02\pm0.05$         \\
        ${}^{14}{C14}$       &  $4.77\pm0.098$       & $4.77\pm0.098$ 
    \end{tabular}
    \caption{Measured activities of GdLS components during LS Screener. Values from Table 4.9 and 6.11 of \cite{scotthaselschwardt_thesis_ref}}
    \label{tab:gdls_assay_rates}
\end{table}

\subsection{Energy Scaling}
\par
When data and simulations were first compared, as in Figure XXX, it became obvious that the light modelling in the OD significantly different to reality.
Around 50\% of this difference can be accounted for by a mismatch between the PMT gains simulated against those in the LZap reconstruction database.
This effect will simply result in smaller pulses and so smaller pulse areas.
However, the remaining difference is not clear where it is coming from, as it would indicate that the LCE is twice what is currently simulated (Figure \ref{fig:od_lce}).
In the pulse area parameter, this should remain just a linear scaling, though in other parameters (such as number of PMTs receiving light) it may not be linear.

\par
In order to calculate the energy scaling, Th228 data calibration was used. 
In the decay chain (Figure XXX), Tl208 has a prominent 2.6MeV $\gamma$. 


Figure XXX shows the background data from random trigger data with the higher simulated rate overlaid.
Two things are strikingly obvious, firstly, the x-axis is different, with the data going out to significantly higher values.
