\section{Backgrounds}
\label{sec:od_analysis_backgrounds}
\par
Using the aforementioned energy scale, noise cut, and scaling factor, in this section the background rate in the OD is measured and attempts are made to understand what is seen.
\par
In \autoref{fig:od_random_trigger} a comparison between the observed events in the OD during the entirety of SR1 and those expected are shown from the Random Trigger.
The expected rates are those described in \autoref{sec:simulated_od_requirements} and were simulated using the ``full-propagation" chain with the result scaled by the factor determined in the previous section.
Both data and simulations were handled by the same analysis tools, with the noise cut applied to both.
Included as well in \autoref{fig:od_random_trigger} is the expected rate in the OD if the GdLS had not undergo an improve purification.
Neither case fits what is observed particularly well.
More worryingly there is a peak in the data at 100 phd which does is not in the expected rate.


\begin{figure}[]
    \centering
    \begin{tikzpicture}
    
    \begin{axis}[
        xlabel=Pulse Area,
        ylabel=Rate (Hz/5phe),
        width=15cm, height=10cm,
        xmin=0, xmax=1000,
        %ymax=1e-7, 
        ymode=log,
        legend pos=north east,
        grid=major]
            
        \addplot[only marks, mark size=0.5pt,
                 error bar legend,] 
            plot[error bars/.cd, x dir=both, x explicit]
            table[x=pulsearea,y=weight,x error=xerror, y error=yerror]
            {Data/OD_Backgrounds/background_constraints/od_data.dat};
        
        \addplot[red, const plot]
            table [x=pulsearea,y=weight]
            {Data/OD_Backgrounds/background_fit/starting_point/backgrounds_improved_purification.dat};
            
        \addplot[green, const plot]
            table [x=pulsearea,y=weight]
            {Data/OD_Backgrounds/background_fit/starting_point/backgrounds_original_purification.dat};
        
        \legend{Data, Original purification, Improved purification};
        \end{axis}
    \end{tikzpicture}
    \caption{OD pulse area spectrum from using the Random Trigger in the region. 
    Only the noise cut has been applied to the data.
    Overlaid are the expected rates from all backgrounds with the improved and original GdLS internal rates.}
    \label{fig:od_random_trigger}
\end{figure}

\par
In the remainder of this section, an attempt is made to understand what is observed and why it is different to what is expected.

\subsection{Rate Stability}

\par
To begin in the journey of understanding what is in the data, the first thing that was done was observe if the rate and distribution were stable over time.
Events from a month before the beginning of SR1 until the end were monitored every week using the Random Trigger.
The rate of events above the noise-cut, 100 keV and 200 keV were measured and are shown in \autoref{fig:OD_SR1_Rate}.
The noise-cut was applied as a base-cut, so the 100 keV is made up of the the noise-cut plus a phd cut, and similarly for 200 keV.
The three gaps in the data are when a calibrations were being performed.
This occurred three times in the region shown: just before SR1, 4 weeks into SR1, and straight after SR1.
\par
There are minor fluctuations in the noise-cut rate, but these are linked a xenon chiller and is consistent with a grounding failure, thus the behaviour is not mirrored in the 100 keV or 200 keV rates.
Importantly, over this period the OD rate remains stable, with no features in the observed distribution changing during that time.
Therefore what was shown in \autoref{fig:od_random_trigger} is the truly representative of the backgrounds in the OD.

\begin{figure}[]
    \centering
   \begin{tikzpicture}
        \begin{axis}[
        date coordinates in=x,
        %xtick=data,
        xtick={2021-10-01 12:00, 2021-11-01 12:00, 2021-12-01 12:00, 2022-01-01 12:00, 2022-02-01 12:00, 2022-03-01 12:00, 2022-04-01 12:00, 2022-05-01 12:00, 2022-06-01 12:00},
        xticklabel style=
        {rotate=45,anchor=near xticklabel},
        xticklabel=\month.\year,
        xlabel={Date},
        %ymin=247, ymax=250,
        y tick label style={/pgf/number format/1000 sep=},
        extra y tick style={grid=major, tick label style={xshift=-1cm}},
        ylabel={Rate (Hz)},
        %date ZERO=2009-08-18,% <- improves precision!
        width=15cm,
        height=8cm,
        ymin=0, ymax=320,
        legend style = {column sep = 10pt, legend columns = -1,},
        ]
        \addplot[red, smooth, forget plot] table[x=date,y=noise] {Data/OD_Backgrounds/background_rates/random_trig_rates.txt};
        \addplot[black, only marks, mark size=1.0pt, error bar legend,
                 error bars/.cd, error bar style={color=black},
                 y dir=both, y explicit,] table[x=date,y=noise, y error=noise_error] {Data/OD_Backgrounds/background_rates/random_trig_rates.txt};
        \addlegendentry{Noise Cut};
        \addplot[blue, smooth, forget plot] table[x=date,y=noise] {Data/OD_Backgrounds/background_rates/random_trig_rates.txt};
        \addplot[black, only marks, mark size=1.0pt, error bar legend,
                 error bars/.cd, error bar style={color=black},
                 y dir=both, y explicit,] table[x=date,y=100kev, y error=200kev_error] {Data/OD_Backgrounds/background_rates/random_trig_rates.txt};
        \addlegendentry{100 keV};
        \addplot[green, smooth, forget plot] table[x=date,y=noise] {Data/OD_Backgrounds/background_rates/random_trig_rates.txt};
        \addplot[black, only marks, mark size=1.0pt, error bar legend,
                 error bars/.cd, error bar style={color=black},
                 y dir=both, y explicit,] table[x=date,y=200kev, y error=200kev_error] {Data/OD_Backgrounds/background_rates/random_trig_rates.txt};
        \addlegendentry{200 keV};         
         \addplot[black, dashed] coordinates { (2021-12-23,-10)  (2021-12-23,350)};
         \node [rotate=90] at (axis cs:2021-12-26,150) {SR1 start};
         \addplot[black, dashed] coordinates { (2022-04-18,-10)  (2022-04-18,350)};
         \node [rotate=90] at (axis cs:2022-04-15,150) {SR1 end};
        
        \end{axis}
    \end{tikzpicture}
    \caption{Rate in OD during and before SR1 data taking on a week-by-week basis using the Random Trigger.
             The SR1 boundary is also shown which ran from 23-12-2021 to 18-04-2022. The gap in data during SR1 was due to a calibration campaign during which different trigger configurations were used, which did not include the Random Trigger.}
    \label{fig:OD_SR1_Rate}
\end{figure}

\iffalse

\begin{figure}[!htbp]
    \centering
   \begin{tikzpicture}
        \begin{axis}[
        date coordinates in=x,
        %xtick=data,
        xticklabel style=
        {rotate=90,anchor=near xticklabel},
        xticklabel=\day.\month.\year,
        xlabel={Date},
        %ymin=247, ymax=250,
        y tick label style={/pgf/number format/1000 sep=},
        extra y tick style={grid=major, tick label style={xshift=-1cm}},
        ylabel={Rate (Hz)},
        date ZERO=2009-08-18,% <- improves precision!
        width=15cm,
        height=6cm,
        ]
        \addplot[smooth, error bar legend,
                 error bars/.cd,
                 y dir=both, y explicit, error bar style={color=orange}] table[x=date,y=noise, y error=noise_error] {Data/OD_Backgrounds/background_rates/random_trig_rates.txt};
                 
        \addplot[smooth, error bar legend,
                 error bars/.cd,
                 y dir=both, y explicit, error bar style={color=orange}] table[x=date,y=100kev, y error=200kev_error] {Data/OD_Backgrounds/background_rates/random_trig_rates.txt};
        
        \addplot[smooth, error bar legend,
                 error bars/.cd,
                 y dir=both, y explicit, error bar style={color=orange}] table[x=date,y=200kev, y error=200kev_error] {Data/OD_Backgrounds/background_rates/random_trig_rates.txt};
                 
        \end{axis}
    \end{tikzpicture}
    \caption{Rate in OD during and before SR1 data taking on a week-by-week basis using the Random Trigger.
    Week -1 corresponds to the month prior to SR1 when the OD PMT gains were higher.}
    \label{fig:OD_SR1_Rate_spare}
\end{figure}
%\par


\begin{figure}[!htbp]
    \centering
    \begin{tikzpicture}
        \begin{axis}[
            title=TODO: Replace with dates and errors,
            xlabel=Data taking week,
            ylabel=Rate (Hz),
            width=15cm,
            height=6cm,
            xmin=-2,
            xmax=14,
            legend style = {column sep = 10pt, legend columns = -1,}]
            \addplot[red, only marks]
                    table [x=Week,y=Rate]
                    {Data/OD_Backgrounds/background_rates/od_sr1_rate_noise.dat};
            \addlegendentry{Noise Cut};
            \addplot[blue, only marks]
                    table [x=Week,y=Rate]
                    {Data/OD_Backgrounds/background_rates/od_sr1_rate_100.dat};
            \addlegendentry{100keV};
            \addplot[green, only marks]
                    table [x=Week,y=Rate]
                    {Data/OD_Backgrounds/background_rates/od_sr1_rate_200.dat};
            \addlegendentry{200keV};
        \end{axis}
    \end{tikzpicture}
    \caption{Rate in OD during and before SR1 data taking on a week-by-week basis using the Random Trigger.
    Week -1 corresponds to the month prior to SR1 when the OD PMT gains were higher.}
    \label{fig:OD_SR1_Rate}
\end{figure}
\fi

\par
Viewing the rate in a slightly different way, the rate-per-phd for the SR1 period is shown in \autoref{fig:od_sr1_rate_vs_threshold}.
Overlaid is the expected rate of backgrounds from \autoref{tab:od_expected_rates} for 100 and 200 keV.
Interestingly the rate above 100 keV is in fairly good agreement with what was predicted in \autoref{sec:simulated_od_backgrounds}.
This is consistent with being able to set the veto energy threshold to 100 keV, assuming that achieves an appropriate veto efficiency.
However differences arise at the 200 keV level, where the expected is 62.7$\pm$5.3 Hz were as the observed is 42.5$\pm$2.1 Hz, a fairly significant difference.

\begin{figure}[]
    \centering
    \begin{tikzpicture}
        \begin{axis}[
            xlabel=OD Threshold (phd),
            ylabel=Rate (Hz),
            width=15cm, height=8cm,
            xmin=-1, xmax=55,
            ymin=0, ymax=350,
            legend pos=north east,
            grid=major]
             \addplot+[black, smooth, mark=none]
                    table [x=Threshold,y=Rate]
                    {Data/OD_Backgrounds/background_rates/od_sr1_rate_vs_threshold_smooth_line.dat};
            \addplot[black, only marks, 
                     error bar legend,
                     error bars/.cd,
                     x dir=both, x explicit, error bar style={color=black}]
                    table [x=Threshold,y=Rate, x error=XError]
                    {Data/OD_Backgrounds/background_rates/od_sr1_rate_vs_threshold_error_bars.dat};
             \addplot[dashed, mark=none, red] coordinates {(0,100) (60,100)};
             \addplot[dashed, mark=none, blue] coordinates {(17.6,0) (17.6,350)};
             \addplot[dashed, mark=none, green] coordinates {(37.5,0) (37.5,350)};
             
             \addplot[orange, only marks, 
                      error bar legend,
                      error bars/.cd,
                      y dir=both, y explicit, error bar style={color=orange}]
                      table [x=Threshold,y=Rate, y error=YError]
                      {Data/OD_Backgrounds/background_rates/od_sr1_rate_expected.dat};
             
             \legend{,SR1 Data,$<$100 Hz Requirement,100 keV (17.6 phd),200 keV (37.5 phd),Expected}                
        \end{axis}
    \end{tikzpicture}
    \caption{Rate of OD backgrounds during SR1 using the Random Trigger. The noise cut has been applied. 100~Hz Requirement is for a 500~$\mu$s veto window as proposed in \cite{LZ_TechnicalDesignReview_ref}. Expected values are from \autoref{tab:od_expected_rates}.}
    \label{fig:od_sr1_rate_vs_threshold}
\end{figure}

%%%%%%%%%%%%
\subsection{Position Reconstruction}
\par
Next we can look at the spacial distribution of events.
For any pulse it is possible to reconstruct the location of the interaction that caused the pulse by a weighted average such as:
\begin{equation}
    x = \frac{\sum{\text{Ch}_{\text{phd}} * \text{Ch}_\text{x}}}{\sum{\text{Ch}_\text{phd}}} 
\label{eq:OD_xy_position}
\end{equation}
where Ch$_{phd}$ is the phd of a PMT channel and Ch$_{x}$ is the position of the PMT.
Due to scheduling constraints associated with SR1, there was an insufficient variety of calibration sources were used at varying \{$x,y,z$\} positions in order to adequately determine the resolution of this approach, but it is something a future calibration campaign may be able to tackle. 
Additionally, this approach does not take into account the OCV in the centre of the detector, so reconstructed pulses will have an incorrect position, but the correct shape.
The OCV can be taken into account by converting coordinate system, but has explicitly not been done here due to the lack of knowledge in the actual resolution of this approach.
Regardless however, this approach does provide an insight in a way not thought possible based upon optical simulations.

\par
This approach was performed on slices in phd-space, the result of which can be seen in \autoref{fig:od_backgrounds_position_reconstruction}.
The first region focuses on the peak at 100 phd ($\backsim$ 0.5 MeV).
The second region focuses on the area above 2 MeV, where cavern-$\gamma$s should dominate.

\begin{figure}[]%
\centering
\begin{tikzpicture}
\centering
  \begin{groupplot}[%view={0}{90},
    group style = {group size = 2 by 3,vertical sep=3cm,
                   horizontal sep=1.5cm},
                   height=6cm, width=0.5\textwidth]
    \nextgroupplot[
            ylabel=Rate (Hz),
            xlabel=Pulse Area (phd),
            width=0.95\textwidth,
            height=6cm,
            %xshift=0.5\textwidth,
            xmin=0, xmax=800,
            ymin=1e-4, ymax=1e3,
            ymode=log,
            ]
            \addplot[only marks, mark size=1.0pt] 
            plot[error bars/.cd, x dir=both, x explicit]
            table[x=pulsearea,y=weight,x error=xerror, y error=yerror]
            {Data/OD_Backgrounds/background_constraints/od_data.dat};
            
            \addplot[dashed, mark=none, name path=A,blue] coordinates {(75,0.00001) (75,10000)};
            \addplot[dashed, mark=none, name path=B,blue] coordinates {(125,0.00001) (125,10000)};
            \addplot[dashed, mark=none, name path=C,green] coordinates {(500,0.00001) (500,10000)};
            \addplot[dashed, mark=none, name path=D,green] coordinates {(1000,0.00001) (1000,10000)};

            \addplot[blue!50] fill between[of=A and B];
            \addplot[green!50] fill between[of=C and D];
            
    \nextgroupplot[group/empty plot]

    \nextgroupplot[colorbar, 
    colorbar style={title=Rate (Hz),ymode=log,},
    width=0.4\textwidth, view={0}{90},
    xshift=-0.3\textwidth,
    ylabel=Z (cm),
	xlabel=R (cm),
	y label style={at={(axis description cs:-0.13,0.5)},anchor=near ticklabel},]
    \addplot3[
		surf,
		shader=flat corner,
		mesh/cols=50,
		mesh/ordering=rowwise,
		point meta = {z>1 ? nan : z}
		] file {Data/playground/alpha_peak_r_z.csv};
	\node [rotate=90] at (axis cs:0,1050) {Region 1};
	\nextgroupplot[colorbar, 
	colorbar style={title=Rate (Hz),ymode=log,},
	width=0.4\textwidth, view={0}{90},
    xshift=-0.5\textwidth, %yshift=1.5cm,
    ylabel=Y (cm),
	xlabel=X (cm),
	y label style={at={(axis description cs:-0.13,0.5)},anchor=near ticklabel},]
    \addplot3[
		surf,
		shader=flat corner,
		mesh/cols=54,
		mesh/ordering=rowwise,
		point meta = {z>1 ? nan : z}
		] file {Data/playground/alpha_peak_x_y.csv};

    \nextgroupplot[colorbar, 
    colorbar style={title=Rate (Hz),ymode=log,},
    width=0.4\textwidth, view={0}{90},
    ylabel=Z (cm),
	xlabel=R (cm),
	y label style={at={(axis description cs:-0.13,0.5)},anchor=near ticklabel},]
    \addplot3[
		surf,
		shader=flat corner,
		mesh/cols=50,
		mesh/ordering=rowwise,
		point meta = {z>1 ? nan : z}
		] file {Data/playground/rg_th232_r_z.csv};
		
	\nextgroupplot[colorbar, 
	colorbar style={title=Rate (Hz),ymode=log,},
	width=0.4\textwidth, view={0}{90},
	ylabel=Y (cm),
	xlabel=X (cm),
    y label style={at={(axis description cs:-0.13,0.5)},anchor=near ticklabel},]
    \addplot3[
		surf,
		shader=flat corner,
		mesh/cols=54,
		mesh/ordering=rowwise,
		point meta = {z>1 ? nan : z}
		] file {Data/playground/rg_th232_x_y.csv};
   
  \end{groupplot}
  
  \node at ($(group c1r2) + (-1.0cm, 3.5cm)$) {\textbf{Region 1 (blue)}};
  \node at ($(group c1r3) + (-1.0cm, 3.5cm)$) {\textbf{Region 2 (green)}};
  
\end{tikzpicture}
\caption{Position Reconstruction of pulses from various regions in pulse area space defined in the top plot. 
         Each pulse has had the noise cut applied and the position reconstructed using \autoref{eq:OD_xy_position}.}
\label{fig:od_backgrounds_position_reconstruction}
\end{figure}

\par
There are two useful observations in both regions.
Firstly, in $r-z$ there is a clear bias to events at the bottom of the detector.
This is consistent with cavern-$\gamma$s distribution shown in \autoref{fig:cavern_gamma_position_distribution}.
This indication that the cavern-$\gamma$s is the most significant contributor, again in agreement in the prediction (\autoref{tab:od_expected_rates}).
Secondly, in $x-y$ there is an elevated rate of events in \{$+x,+y$\}.
This can be understood easiest by labelling the SATs with letters A-D starting from \{$+x,+y$\} and going around clockwise.
Using this labelling, over the period of SR1, SAT A has a rate in excess of 8\% higher than any of the other tanks, with SAT B seeing the next highest rate (3\% higher than the mean).
Both SATs C and D saw an equivalent rate.
In \autoref{fig:OD_conduit_geometry}, the SAT placement are shown along with the conduits.
SAT A and SAT B are the only tanks which are obstructed by a single conduit, additionally both tanks cover an entire BAT and TAT.
The CSD-ports and the OCV legs block the other SATs from the TATs and BATs.
This results in the light having a more direct path to more PMTs and therefore higher probability of detection.

\begin{figure}[!htbp]
\includegraphics[width=\textwidth]{Figures/Geometry/geometry_with_conduits.png}
\centering
\caption{LZ geometry schematic. The OD geometry excluding the BATs and TATs is shown in black. The BATs and bottom on OCV are shown in green. The TATs, CSD ports and PMT conduits in blue. The DD calibration conduits and the High-Voltage feed through for the TPC are shown in red.}
\label{fig:OD_conduit_geometry}
\end{figure}


\par
As a way of suppressing cavern-$\gamma$'s is to take a slice in $z$, taking events reconstructed to be in the middle of the side tanks.
The resultant pulse area spectrum is shown in \autoref{fig:od_data_pulsearea_middle_tank}.
When compared to \autoref{fig:od_backgrounds_position_reconstruction} additional features appear around 300 phd.
These are consistent with ${}^{60}Co$ which is present in the OCV.
In future it may be possible to accurately measure the rate of this using this volume cut, but that will require a more dedicated calibration campaign, as the true quantity of the SAT selected is not clear.

\begin{figure}[]
    \centering
    \begin{tikzpicture}
    
    \begin{axis}[
        xlabel=Pulse Area (phd),
        ylabel=Rate (Hz),
        width=15cm, height=8cm,
        xmin=0, xmax=800,
        ymin=1e-4, ymode=log,
        legend pos=north east,
        grid=major]
            
        \addplot[only marks, mark size=0.5pt] 
            plot[error bars/.cd, x dir=both, x explicit]
            table[x=pulsearea,y=rate,x error=x_error, y error=y_error]
            {Data/OD_Backgrounds/background_constraints/od_data_pulsearea_middle_tank_binwidth_5.dat};
        \end{axis}
    \end{tikzpicture}
    \caption{OD pulse area spectrum from pulses reconstructed to the middle of the OD side tanks, suppressing the rate from Cavern-$\gamma$'s.
             The peak at 280 phd is from ${}^{60}$Co which originates from the OCV.}
    \label{fig:od_data_pulsearea_middle_tank}
\end{figure}


\subsection{$\gamma$ constraints}
\par
Another area to look at is high energy $\gamma$'s, from ($\alpha$,$\gamma$) reactions, which were discussed in \autoref{sec:cavern_gamma_generator}.
In \autoref{fig:od_high_energy} a the expected and observed rate per pulse area above 400 phd is shown.
The rate of events expected is significantly greater than that observed.
The findings here support the discussion in \autoref{sec:cavern_gamma_generator}, that the statistical model used does not extend well to ${}^{17}$O.
However, there are still features seen in data between 1000 and 1800 phd that are not accounted for in simulations.
These features have attributed to neutron captures on the Fe of the water tank (which is made of steel) producing high energy $\gamma$'s \autoref{iron_neutrons_ref}.
These neutrons originate primarily from the ${}^{252}$Cf calibration source, which was stored in a movable safe underground during SR1.
It was moved location within the cavern during SR1 which matches with a change in the reconstructed position of these events.
Backgrounds of this type were not previously considered within LZ, though are of great importance in fusion experiments \cite{iter_neutrons_ref}.
Primarily due to the source safe placement during SR1, it was not possible to determine if there was any fluctuation in the high-energy $\gamma$-rate during SR1.

\begin{figure}[]
    \centering
    \begin{tikzpicture}
    
    \begin{axis}[
        xlabel=Pulse Area (phd),
        ylabel=Rate (Hz),
        width=15cm, height=10cm,
        xmin=400, xmax=3000,
        %ymax=1e-7, 
        ymode=log,
        legend pos=north east,
        grid=major]
            
        \addplot[only marks, mark size=0.5pt,
                 error bar legend,] 
            plot[error bars/.cd, x dir=both, x explicit]
            table[x=pulsearea,y=weight,x error=xerror, y error=yerror]
            {Data/OD_Backgrounds/background_constraints/od_data.dat};
            
        \addplot[red, const plot]
            table [x=pulsearea,y=weight]
            {Data/OD_Backgrounds/background_fit/starting_point/backgrounds_original_purification.dat};
            
        \legend{Data, Expectation};
        \end{axis}
    \end{tikzpicture}
    \caption{OD pulse area spectrum for high energy $\gamma$ events compared to the expected rate.
             The events observed in data between 1000 and 1800 phd are high energy $\gamma$-rays from neutron capture the iron.}
    \label{fig:od_high_energy}
\end{figure}


%%%%%%%%%%%%
\subsection{$\alpha$ constraints}
\par
As was discussed in \autoref{sec:simulated_od_backgrounds} the exact rate of backgrounds from GdLS internals components is not known.
Fortunately we are able to constrain the some of the components. 
The decays within the U and Th decay chains all have an isotope in them with a half-life shorter than the LZ event window, of 4.5 ms.
It will therefore be possible to observe two decays from the same decay chain, which will appear as a pulse-pair (a pulse from each decay).
There are 3 possible decays, one from decay chain ${}^{238}$U, ${}^{235}$U and ${}^{232}$Th.
These are summarised in \autoref{tab:od_constrainable_decays_in_data}.
In reality though, only ${}^{214}$Bi$ \to {}^{214}$Po and ${}^{219}$Rn $\to {}^{215}$Po can be searched for as the interactions from ${}^{212}$Bi $\to {}^{212}$Po are close enough together that they will be merged into one pulse.

\begin{table}[!htbp]
    \centering
    \begin{tabular}{c|c|c|c|c|c}
        \multirow{2}{*}{Decay Pair (chain)}                    & \multicolumn{2}{c|}{First Decay}   & \multicolumn{3}{c}{Second Decay}    \\ 
                                                               & Decay    & Energy (MeV) & Decay    & Energy (MeV) & half-life ($\mu$s) \\ \hline
        ${}^{214}$Bi $\to {}^{214}$Po (${}^{238}$U$_{m}$)          & $\beta$  & 3.27         & $\alpha$ & 7.83         & 160   \\ 
        ${}^{219}$Rn $\to {}^{215}$Po (${}^{235}$U$_{l}$)          & $\alpha$ & 6.95         & $\alpha$ & 7.53         & 1800  \\ 
        ${}^{212}$Bi $\to {}^{212}$Po (${}^{232}$Th$_{l}$)         & $\beta$  & 2.25         & $\alpha$ & 8.95         & 0.3
    \end{tabular}
    \caption{Th and U decay chain pairs with half-lives within the LZ event window of 4.5 ms. 
             Decay information from \cite{radon_chains_ref}.
             See \autoref{fig:decay_chains} for the complete Decay Chains.}
    \label{tab:od_constrainable_decays_in_data}
\end{table}

\par
The decay pairs were searched for by looking for pulses which were reconstructed in be in close proximity to each other.
As the GdLS is not circulated and the decays particles ($\beta$ and $\alpha$) have do not have a significant penetrating power, the site of the two interactions with be very close to each other.
This position requirement acts to reduce the impact of coincident signals in other areas of the OD being mistakenly identified as part of the pulse pair.
The data selection criteria used was simply requiring that the reconstructed $z$ and $\theta$ be within some range of each other: $z_{\text{diff}} < 100$ and $\theta_{\text{diff}} < 1.0$, where diff refers to the difference between pulses.
It was also limited to pulses passing the noise-cut, primarily the PMT multiplicity (coincidence) requirements so that enough PMTs saw light to make the reconstruction meaningful.
This cut is fairly loose to account for potential variable inefficiencies in the position reconstruction with $z$ and $\theta$ based on pulse size.
The resultant pulse-pairs are shown in \autoref{fig:od_all_pulse_pairs_2d}.
Only pulses above 200keV in visible energy are included in the 2D histograms as the features are easier to see.

\begin{figure}[!htbp]%
\centering
\begin{tikzpicture}
\centering
  \begin{groupplot}[%view={0}{90},
    group style = {group size = 1 by 2,vertical sep=1.5cm,
                   horizontal sep=1.5cm},
                   height=6cm, width=0.5\textwidth]

	\nextgroupplot[height=10cm, width=10cm,
    view={0}{90},
    ylabel={Second Pulse (phd)},
    xlabel={First Pulse (phd)},
    colorbar,
    colorbar style={ylabel={Rate (Hz)},ymode=log,},
    ]
    \addplot3[
      surf,
      shader=flat corner,
	  mesh/cols=40,
	  mesh/ordering=rowwise,
	  point meta = {z<0.0000001 ? nan : z}
    ] file {Data/OD_Backgrounds/background_constraints/rnpo_rate_2d.dat};
    
    
    \nextgroupplot[height=10cm, width=10cm,
    view={0}{90},
    ylabel={Second Pulse (phd)},
    xlabel={First Pulse (phd)},
    colorbar,
    colorbar style={ylabel={Rate (Hz)},ymode=log,},
    ]
    \addplot3[
      surf,
      shader=flat corner,
	  mesh/cols=40,
	  mesh/ordering=rowwise,
	  point meta = {z<0.0000001 ? nan : z}
    ] file {Data/OD_Backgrounds/background_constraints/bipo_rate_2d.dat};
   
  \end{groupplot}
\end{tikzpicture}
\caption{Relationship of pulses reconstructed in close proximity to each other. Only pulses above 200keV have been included.
         \textbf{Top:} All pulses in an event. The population between 150 and 200 phd in both the first and second pulse correspond to the $\alpha$'s of ${}^{219}Rn$ (first) and ${}^{215}Po$ (second).
         \textbf{Bottom:} Pulses within 1000$\mu$s of each other. The population above 150 phd in the second pulse are the $\beta$ from ${}^{214}Bi$ (first) and $\alpha$ from ${}^{214}Po$ (second).}
\label{fig:od_bipo_pulses_2d}
\end{figure}



\par
In \autoref{fig:od_bipo_pulses_2d} there is a distribution where both the first and second pulse have a signal size around 170 phd.
These correspond to the double $\alpha$-decay in the ${}^{235}U_{l}$ chain.
The first pulse is slightly smaller than the second, which follows the pattern expected given the $\alpha$ energies shown in \autoref{tab:od_constrainable_decays_in_data}.
The second pulse-pair (from ${}^{238}U_{m}$) has a $\alpha$-decay in the same energy region, $\backsim$170 phd.
The first decay is a $\beta$-decay with the resultant electron have a spectrum of energies up the the end point of 3.27 MeV ($\backsim$ 400 phd).
As such even though the second pulse will be within a small range, there is no clear feature in \autoref{fig:od_bipo_pulses_2d} as the first pulse will be a wide range of values.

\par
It is still possible to isolate ${}^{214}$Bi $\to {}^{214}$Po but taking advantage of the much shorter decay half-life of ${}^{214}$Po vs ${}^{215}$Po.
This was done on the same same data as before with using the same position requirement, except now the time difference between the two pulses is considered.
The of time separation of the pulses has to be less than 800 $\mu$s.
This was selected as it is 5 full half-lives, in which 97\% of all ${}^{214}$Po will have decayed, and is less than half the half-life of ${}^{215}$Po.
The result of this is shown in \autoref{fig:od_time_dependent_pulses_2d}.
Included for completeness as well in \autoref{fig:od_time_dependent_pulses_2d} is the opposite case.
A new distribution has appeared is a second pulse size means of 240 phd, corresponding to the $\alpha$-decay of ${}^{214}$Po.

\par
The distributions in \autoref{fig:od_time_dependent_pulses_2d} allow for the rates of both ${}^{214}$Bi $\to {}^{214}$Po and ${}^{219}$Rn $\to {}^{215}$Po to be constrained.
This gives a rate of 98$\pm$2.5 mHz for ${}^{214}Bi \to {}^{214}Po$ and 7.7$\pm$1.2 mHz for ${}^{219}Rn \to {}^{215}Po$.
Both are rates are lower than those from expected from even the `improved purification' case (discussed in \autoref{sec:simulated_od_requirements}).


\par
Importantly, these constraints indicate that the $\alpha$-peak seen in \autoref{fig:od_random_trigger} is not from those subchains, as the rate would be too low.
This opens up the possibility that it is from ${}^{210}$Po from the late chain of ${}^{238}$U. 
This claim appears at odds with the relatively low internal rate of ${}^{238}$U.
However, it is known and measured that there is significant Rn in the cavern air, and as previously discussed (\autoref{sec:od_construction_sec}) the acrylic tanks were open to the air for long periods prior to installation.
There was opportunity for decay daughters to plate-out on the inside of the acrylic tanks, a feature that is of great concern within the TPC \cite{radon_plateout_ref}.
${}^{210}$Pb is the only long-lived particle in the Radon chain, with a half-life in excess of 22-years. 
The remainder of the isotopes would drastically reduce in abundance after a few weeks, leaving only ${}^{210}$Pb daughters visible, at a sustained rate.
This allows the acrylic holding the scintillator to act as a reservoir, providing a constant supply of $\alpha$ decays from ${}^{210}$Po.
LZ is not alone in having experienced this, with KamLAND measuring a higher than expected rate of ${}^{238}$U$_l$ decays \cite{KamLAND_LS_contaminants_ref}.

\par
The spectra for each of the three visible $\alpha$s are shown in \autoref{fig:od_extracted_alphas}.
The signal size has been converted into visible energy according by the energy scale previously discussed (\autoref{sec:od_energy_scale}).
Although it is unexpected to see a four $\alpha$'s, it becomes possible to perform an $\alpha$ energy calibration, and will be continuously measurable during all Science runs, without the need for Science data taking to be stopped.
An additional advantage is that PMT gain-drift will be more easily detectable.
The observed energy of each of the $\alpha$'s verse the observed energy are shown in \autoref{fig:od_alpha_quenching}, the Birk's law fit is also provided using the parameters in \autoref{tab:Birks_law_parameters}.
It can be seen that the parameters taken from pure LAB remain in good agreement with Gd-doped LAB used here.
This also isolates the difference between observed data and simulations to light propagation.

\begin{figure}[!htbp]%
\centering
\begin{tikzpicture}
\centering
    \begin{axis}[
            ylabel=Rate (Hz/2keV),
            xlabel=Energy (MeV),
            width=15cm,
            height=8cm,
            grid=major,
            xmin=0, xmax=3,
            ymin=1e-4, ymode=log,]
            
        \addplot[red, only marks, mark size=1, 
                 error bars/.cd,
                 y dir=both, y explicit, error bar style={color=black}]
            table [x=energy,y=rate,y error plus index=2, y error minus index=3]
            {Data/OD_Backgrounds/background_constraints/od_bipo_first_alpha.dat};
            
        \addplot[blue, only marks, mark size=1, 
                 error bars/.cd,
                 y dir=both, y explicit, error bar style={color=black}]
            table [x=energy,y=rate,y error plus index=2, y error minus index=3]
            {Data/OD_Backgrounds/background_constraints/od_bipo_second_alpha.dat};
        
        \addplot[green, only marks, mark size=1, 
                 error bars/.cd,
                 y dir=both, y explicit, error bar style={color=black}]
            table [x=energy,y=rate,y error plus index=2, y error minus index=3]
            {Data/OD_Backgrounds/background_constraints/od_bipo_third_alpha.dat};
        
        \legend{${}^{219}$Rn, ${}^{215}$Po, ${}^{214}$Po}
    \end{axis}
            
\end{tikzpicture}
    \caption{$\alpha$ energies in from BiPo and RnPo.}
    \label{fig:od_extracted_alphas}
\end{figure}

\begin{figure}[!htbp]%
\centering
\begin{tikzpicture}
\centering
    \begin{axis}[
            ylabel=Visible Energy (MeV),
            xlabel=Particle Energy (MeV),
            width=15cm,
            height=8cm,
            grid=major,
            xmin=0, xmax=10,
            ymin=0]
            
        % Birks Fit
        \addplot[red]
            table [x=Energy,y=Quenched]
            {Data/GdLS_Physics/Quenching/alpha.dat};
        \addplot[black, only marks, mark size=1, 
                 error bars/.cd,
                 y dir=both, y explicit, error bar style={color=black}]
            table [x=energy,y=observed,y error plus index=2, y error minus index=3]
            {Data/OD_Backgrounds/background_constraints/od_alpha_energies.dat};
        \legend{Birk's, Measured};
    \end{axis}
            
\end{tikzpicture}
    \caption{Observed energy vs particle energy from $\alpha$ particles.
             The expected quenching from Birk's Law using the parameters in \autoref{tab:Birks_law_parameters} is shown to be in good agreement with what is observed.}
    \label{fig:od_alpha_quenching}
\end{figure}


\subsection{${}^{152}$Gd}
\par
The final background that can be constrained is ${}^{152}$Gd, which produces a 2.2 MeV $\alpha$ decay ($\approx$ 120 keV visible energy).
The GdLS is loaded with 0.1\% (by mass) of natural Gadolinium.
With a total mass of GdLS used in the detector this corresponds to 16000kg, and ${}^{152}Gd$ has a natural abundance of 0.2\%, this gives a maximum possible rate of 27Hz, where there is a 5\% uncertainty on the doping by mass.
\par
Unfortunately this does not constrain does this region entirely.
Of particular note are ${}^{14}$C, a $\beta$ decay with an end-point of 156 keV, and ${}^{147}$Sm which decays via the emission of a 2.3 MeV $\alpha$.
All three were previously measured in the LS screener campaign, with ${}^{14}$C being the dominant contributor (>100 Hz) \cite{scotthaselschwardt_thesis_ref}.


\subsection{Background Fitting}
\par
Using all of the information of the OD backgrounds above, we finish this section is a fit of the expected background to the observed.
Simulations of all of the backgrounds identified in \autoref{sec:simulated_od_backgrounds}, but full light propagation and processed through the same analysis software suite as the observed data.
During this process the simulation signal sizes were scaled by 4.77.
Rather than fitting to the full observable space, a sub range was defined between 20 phd and 600 phd.
The lower limit was set so as to be far enough away from the noise-cut that any uncertainty in scaling factor would not remove these events.
The upper limit was set predominately by the rate of cavern-$\gamma$s.
In this region the expected background rate is 78.43$\pm$2.51 Hz.
This is slightly below the observed rate of 85.26$\pm$0.22 Hz.
\par
In order to keep the number of fit parameters down a number of steps were taken.
Firstly, all detector components were in combined. 
Secondly, contributions from ${}^{152}$Gd and ${}^{147}$Sm were combined together as the energy resolution is not sufficient to separate them.
The cavern-$\gamma$'s were allowed to float within the uncertainties they were measured in \cite{LZ_Gamma_Ray_Background_ref}.
${}^{238}$U$_l$ was allowed to fit between a range of [0, 15] Hz, where 15 is roughly double the size of the integrated rate of the peak at 100 phd above the background rest of the spectra.
${}^{238}$U$_{m}$ and ${}^{235}$U$_{l}$ were constrained by the rates previously reported.
All other backgrounds were left to allowed to float up to the maximum value measured in the LS screener campaign \cite{scotthaselschwardt_thesis_ref}.
\par
The result of the fit is shown in \autoref{fig:od_background_fit_to_data}.
Only components which contributed a more than 0.5 Hz to the OD rate have been included in \autoref{fig:od_background_fit_to_data}, but the fitted line should is from the all contributions.
The individual contributions of each component is shown in \autoref{tab:od_constrainable_decays_in_data}.

\par
${}^{238}$U$_l$ can successfully account for the peak observed at 100 phd, contributing in excess of 8 Hz to the OD rate.
The rate of cavern-$\gamma$s remain the background with the largest uncertainty for the backgrounds in the OD.
There remain some differences between the observed and fitted result, most notably at 50 phd where is a noticeable gap in a background that can fill in this region.
This can be incorporated as an uncertainty in the scaling factor.
The flat scaling of 4.77 did not take into account any position variation, and given that the light collection efficiency observed is significant greater than simulated, it is logical to suggest that the distribution in the LCE map (shown in \autoref{fig:od_lce}) is different as well.
This would effectively make contributions in lower light collection regions appear smaller than they are observed to be, which may go so way to explain what is seen.


\begin{figure}[]%
\centering
\begin{tikzpicture}
\centering
    \begin{groupplot}[
    group style = {group size = 1 by 2,vertical sep=1.0cm}
    ]
    \nextgroupplot[
            ylabel=Rate (Hz/5phd),
            xlabel=,
            xmajorticks=false,
            width=15cm,
            height=10cm,
            grid=major,
            xmin=20, xmax=600,
            ymin=1e-2, ymax=100,
            ymode=log,
            legend style = { column sep = 10pt, legend columns = 3, cells={line width=1.5pt}}
            ]
        \addplot[black, only marks, mark size=1.0pt, error bar legend,
                 error bars/.cd, error bar style={color=black},
                 y dir=both, y explicit, 
                 x dir=both, x explicit,] 
            table[x=pulsearea,y=weight,x error=xerror, y error=yerror]
            {Data/OD_Backgrounds/background_constraints/od_data.dat};
        \addlegendentry{data};
            
        \addplot[const plot, teal]
            table [x=bin_low, y=weight]
            {Data/OD_Backgrounds/background_fit/fit_components/rg_k40.dat};
        \addlegendentry{Cavern ${}^{40}$K};
        \addplot[const plot, cyan]
            table [x=bin_low, y=weight]
            {Data/OD_Backgrounds/background_fit/fit_components/rg_u238.dat};
        \addlegendentry{Cavern ${}^{238}$U};
        \addplot[const plot, green]
            table [x=bin_low, y=weight]
            {Data/OD_Backgrounds/background_fit/fit_components/rg_th232.dat};
        \addlegendentry{Cavern ${}^{232}$Th};
        \addplot[const plot, lime]
            table [x=bin_low, y=weight]
            {Data/OD_Backgrounds/background_fit/fit_components/ocv.dat};
        \addlegendentry{Detector};
        \addplot[const plot, yellow]
            table [x=bin_low, y=weight]
            {Data/OD_Backgrounds/background_fit/fit_components/Internal_improved_u238_early.dat};
        \addlegendentry{${}^{238}$U early};
        \addplot[const plot, pink]
            table [x=bin_low, y=weight]
            {Data/OD_Backgrounds/background_fit/fit_components/Internal_improved_u238_late.dat};
        \addlegendentry{${}^{238}$U late};    
        \addplot[const plot, olive]
            table [x=bin_low, y=weight]
            {Data/OD_Backgrounds/background_fit/fit_components/Internal_improved_th232_early.dat};
        \addlegendentry{${}^{232}$Th early};
        \addplot[const plot, brown]
            table [x=bin_low, y=weight]
            {Data/OD_Backgrounds/background_fit/fit_components/Internal_improved_C14.dat};
        \addlegendentry{${}^{14}$C};
        \addplot[const plot, orange]
            table [x=bin_low, y=weight]
            {Data/OD_Backgrounds/background_fit/fit_components/Internal_improved_Gd152.dat};
        \addlegendentry{${}^{152}$Gd + ${}^{147}$Sm};
        \addplot[const plot, violet]
            table [x=bin_low, y=weight]
            {Data/OD_Backgrounds/background_fit/fit_components/total_fit.dat};
        \addlegendentry{fit};
            
            
    \nextgroupplot[
            ylabel=Ratio (data/fit),
            xlabel=Pulse Area (phd),
            width=15cm,
            height=5cm,
            grid=major,
            xmin=20, xmax=600,
            ymin=0, ymax=2,
            ]
        \addplot[only marks, mark size=1.0pt, black, error bar legend] 
            plot[error bars/.cd, y dir=both, y explicit]
            table[x=bin,y=value, y error=error]
            {Data/OD_Backgrounds/background_fit/fit_components/ratio.dat};
        
        %\addplot[black, mark=none, dashed] coordinates {(20,-1) (20,3)};
        %\addplot[black, mark=none, dashed] coordinates {(600,-1) (600,3)};
        
    \end{groupplot}
\end{tikzpicture}
    \caption{Result of fit to data.
             \textbf{Top:} Contribution from each source. Only components with a contribution greater than 1 Hz have been included.
             \textbf{Bottom:} Ratio of fit to data. 
             In generally there is good agreement.}
    \label{fig:od_background_fit_to_data}
\end{figure}


\begin{table}[!htbp]
    \centering
    \begin{tabular}{c|c|c}
        Source               &  Decay Type(s)               & Fitted Rate (Hz) \\ \hline
        \multicolumn{3}{l}{\textbf{Externals}} \\
        Cavern ${}^{238}$U   & $\gamma$                     & 7.65$\pm$1.65                    \\ 
        Cavern ${}^{232}$Th  & $\gamma$                     & 12.94$\pm$3.04                    \\ 
        Cavern ${}^{40}$K    & $\gamma$                     & 12.50$\pm$5.77                    \\ 
        Detector             & $\gamma$,$\beta$             & 3.63$\pm$0.67              \\ \hline
        \multicolumn{3}{l}{\textbf{Internals}} \\
        ${}^{238}$U$_{e}$     & $\gamma$,$\alpha$,$\beta$   & 0.98$\pm$0.21                    \\ 
        ${}^{238}$U$_{m}$     & $\gamma$,$\alpha$,$\beta$   & 0.03$\pm$0.10                   \\
        ${}^{238}$U$_{l}$     & $\gamma$,$\alpha$,$\beta$   & 8.74$\pm$2.34                \\
        ${}^{235}$U$_{e}$     & $\gamma$,$\alpha$,$\beta$   & 0.39$\pm$0.23                    \\
        ${}^{235}$U$_{l}$     & $\gamma$,$\alpha$,$\beta$   & 0.44$\pm$0.12                    \\
        ${}^{232}$Th$_{e}$    & $\gamma$,$\alpha$,$\beta$   & 0.18$\pm$0.04                    \\
        ${}^{232}$Th$_{l}$    & $\gamma$,$\alpha$,$\beta$   & 0.51$\pm$0.08                  \\
        ${}^{40}$K          & $\gamma$,$\beta$              & 0.05$\pm$0.01                 \\
        ${}^{14}$C          & $\beta$                       & 26.97$\pm$7.70           \\
        ${}^{176}$Lu        & $\gamma$,$\beta$              & 0.14$\pm$0.05               \\
        ${}^{147}$Sm and ${}^{152}$Gd    & $\alpha$         & 10.32$\pm$7.24                     
        
    \end{tabular}
    \caption{Fitting contributions of OD backgrounds to the observed data. The reported errors are statistical only.}
    \label{tab:od_constrainable_decays_in_data}
\end{table}

\subsection{Conclusion}
\par
To summarise this section, the OD rate for a number of months.
During this time the OD backgrounds were stable, with the rate above the noise-cut of 276.8 Hz, above 100 keV of 95.8 Hz and above 200 keV as 42.5 Hz.
A veto energy threshold of 200 keV was selected as the probability of a false veto is less than half what it would be for a 100 keV threshold, and it allows for a longer time window.
\par
Several background contributions have been measured in the OD and have been fitted against.
An un-expected contribution from ${}^{238}$U$_{l}$ is present which is believed to be from the radon emanation.
This entered all of the tanks via the frequent opening and closing the valves discussed in \autoref{sec:od_construction_sec}.
