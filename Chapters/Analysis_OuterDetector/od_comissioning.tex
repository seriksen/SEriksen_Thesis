\section{OD Energy Scale} \label{sec:od_energy_scale}
\par
After construction and the filling of the OD, the SPE for each PMT was checked using the OD optical calibration system (OCS) described in \cite{lz_ocs_system_ref}.
A detailed description of the PMT monitoring during the commissioning and SR1 can be found in the works of E. Fraser \cite{ewanfraser_thesis_ref}. 
\par
During this period, an energy calibration of the OD was performed.
Again, details of the methodology are described in \cite{ewanfraser_thesis_ref} but summarised below.
As mentioned in Sec. \ref{sec:lz_detector_chapter}, a variety of calibration sources were available with output a range of particles at different energies.
$\gamma$ sources were lowered into the CSD's to the 700mm level (see Fig. \ref{fig:CSD1_Geometry}) and the number of photo electrons measured.
The empirical formula for the non-linear energy response of the scintillator from DayaBay \cite{dayabay_antineutrino_oscillation_ref, ls_nonlinear_energy_response_ref}, shown in Eq. \ref{eq:ls_light_response}, was used to fit to these points.
\begin{equation}
    \frac{E_{vis}}{E_{true}} = \frac{p_0  + p_3 \times E_true}{1 + p_1 \times e^{p_2 \times E_{true}}}
    \label{eq:ls_light_response}
\end{equation}
The fit and the translation from photons detected to energy are shown in Fig. \ref{fig:od_energy_scale}.

\begin{figure}[]
    \centering
   \begin{tikzpicture}
    \begin{groupplot}[%view={0}{90},
    group style = {group size = 1 by 2,vertical sep=0.5cm},
                   width=0.98\textwidth]
    \nextgroupplot[
            ylabel=LS energy response ($\frac{E_{vis}}{E_{true}}$),
            %xlabel=,
            xticklabels={,,}
            height=10cm, width=\textwidth,
            xmin=0, xmax=10,
            grid=major,
            ]
            %\addplot[black, smooth, domain=0:10] 
                    %{(0.00535 + 0.0286*x) / (1-0.9918*exp(-0.03018*x))};
                    %{((0.001894 + 0.03614*x) / (1-1*exp(-0.001453*x)))};
            \addplot[black, smooth]
                table[x=Energy,y=Value]
                {Data/OD_Energy_Scale/emperical_fit.dat};
                    
            \addplot[only marks,
                 error bars/.cd,
                 y dir=both, y explicit, error bar style={color=black}] table[x=Energy,y=light_response, y error=light_response_Error] {Data/OD_Energy_Scale/phd_energy.dat};

    \nextgroupplot[
            ylabel=N. photons detected,
            xlabel=Energy (MeV),
            height=6cm, width=\textwidth,
            xmin=0, xmax=10,
            ymin=0, 
            grid=major,
            ]
            \addplot[only marks,
                 error bars/.cd,
                 y dir=both, y explicit, error bar style={color=black}] table[x=Energy,y=pulseArea, y error=pulseArea_Error] {Data/OD_Energy_Scale/phd_energy.dat};
            \addplot[black, smooth, domain=0:10] 
                    {-33.98 + 0.2425*x*1000};
            
  \end{groupplot}
    \end{tikzpicture}
    \caption{OD Energy Scale and the relation between observed number of photons and the energy deposited. Analysis by E. Fraser \cite{ewanfraser_thesis_ref}.}
    \label{fig:od_energy_scale}
\end{figure}

%    \nextgroupplot[
%            ylabel=N. photons detected,
%            xlabel=Energy (MeV),
%            height=6cm, width=\textwidth,
%            xmin=0, xmax=10,
%            ymin=0, 
%            grid=major,
%            ]
%            \addplot[black]
%                    table [x=Energy,y=Value]
%                    {Data/OD_Energy_Scale/phd.dat};

\par
The fit 
Although this provides a good result for SR1, as the calibrations were only performed at a single position, the varying light collection efficiency around the OD was not taken into account (see Fig. \ref{fig:od_lce}).

