\section{Particle Discrimination}
\par
As previously mentioned, the OD has been designed to act as an ``active but dumb" veto.
Given that is has been shown that BiPo and RnPo events can be identified based upon their proximity to each other with relative ease, these events would be un-vetoed, increasing the number of candidate events as well as the live-time and ultimately sensitivity of LZ to a dark matter interaction.
In this section a different method of discriminating between events is introduced, using a likelihood approach.
This is then tested on some simulated datasets.

\subsection{Maximum Likelihood Method}
\par
The basic concept of a maximum likelihood approach is to have a signal model and background model that is described in terms of a number of variables.
The likelihood of an event being a signal is then given simply by:
\begin{equation}
    y_{\Lagrangian} = \frac{\Lagrangian_{S}}{\Lagrangian_{S} + \Lagrangian_{B}}
\end{equation}
where 
\begin{equation}
    \Lagrangian_{S,(B)} = \prod_k p_{S,(B),k}(x_k)
\end{equation}
where $p_{S,(B),k}$ is the signal (background) PDF for the $k^{th}$ input variable $x_k$.
In this approach any correlation between variables is ignored, and is often referred to as a ``naive Bayes estimator" \cite{TMVA_ref}.


\subsection{Variables}
\par
Three variables were selected in the application of this method to LZ simulations which are described below.



\paragraph{coincidence}
Sometimes referred to as pulse multiplicity, the coincidence value of a pulse is the number of PMTs which produced a signal that contributed to the pulse.
As the more photons produced, the more will be detected, and so the more likely is it that any given PMT will see this light.
It is a very robust and simple variable, and therefore ideal to start with.
A neutron capture will likely have a higher coincidence as the multiple $\gamma$'s can scatter in different areas of the OD.


\paragraph{pulse area}
The number of photons detected is directly proportional to the energy of the interaction that occurred.
Combined with the coincidence, it has the potential to isolate unwanted pulses such as those caused by PMT after-pulsing.


\paragraph{$\frac{\text{width}}{\text{pulse area}}$}
This variable is the most complicated of the selection, which gives a pulse-shape.
The width of a pulse is the time.
Due to pulse finding limitations, the full reconstructed pulse width was not used.
Instead, the pulse start time was defined as; the time where 5\% of the pulse integral was reached, and the end time as 75\% of the pulse integral.

However, the quantity can be improved by combining it with a second variable; the pulse area.
This has the advantage of removing the pulse area as an influencing parameter in the value and making the variable more stable.
In the simplest case, the quantity gives is how fast or slow the pulse decreases after the peak.
An example of different interactions with this variable can be found in Figure XXX.

\begin{figure}[!htbp]
    \centering
    \includegraphics[width=0.5\textwidth]{Figures/Placeholder.png}
    \caption{Variables used for particle discrimination}
    \label{fig:discrimination_variables}
\end{figure}

\begin{figure}[]%
\centering
\begin{tikzpicture}
\centering
    \begin{groupplot}[%view={0}{90},
    group style = {group size = 1 by 3,vertical sep=1.5cm}]
    \nextgroupplot[
            xlabel=Pulse Area,
            ylabel=Proportion,
            width=15cm, height=6cm,
            xmin=0, xmax=1000,
            ymin=0,% ymax=100,
            %minor y tick num=4,
            grid=major,
            legend style = { column sep = 10pt, legend columns = -1, legend to name = Simulated_Likelihood_Variables_CommonLegend,}]
            \addplot[green, mark=none]
                    table [x=Centre,y=Normed]
                    {Data/OD_Likelihood/gd_capture_pulseArea_phd.dat};
            \addplot[blue, mark=none]
                    table [x=Centre,y=Normed]
                    {Data/OD_Likelihood/h_capture_pulseArea_phd.dat};
            \addplot[red, mark=none]
                    table [x=Centre,y=Normed]
                    {Data/OD_Likelihood/rockgamma_pulseArea_phd.dat};
            %\node[] at (axis cs: 800,87) {\large 0 mm};
            \legend{Gd-capture,H-capture,Cavern $\gamma$}
        
        \nextgroupplot[
            xlabel=N. PMTs,
            ylabel=Proportion,
            width=15cm, height=6cm,
            xmin=0, xmax=120,
            ymin=0, %ymax=100,
            %minor y tick num=4,
            grid=major,]
            \addplot[green, mark=none]
                    table [x=Centre,y=Normed]
                    {Data/OD_Likelihood/gd_capture_coincidence.dat};
            \addplot[blue, mark=none]
                    table [x=Centre,y=Normed]
                    {Data/OD_Likelihood/h_capture_coincidence.dat};
            \addplot[red, mark=none]
                    table [x=Centre,y=Normed]
                    {Data/OD_Likelihood/rockgamma_coincidence.dat};
            %\node[] at (axis cs: 800,87) {\large 1400 mm};
        \nextgroupplot[
            xlabel=Pulse Time / pulse Area,
            ylabel=Proportion,
            width=15cm, height=6cm,
            xmin=0, xmax=2,
            ymin=0, %ymax=100,
            %minor y tick num=4,
            grid=major,]
            \addplot[green, mark=none]
                    table [x=Centre,y=Normed]
            {Data/OD_Likelihood/gd_capture_areaFractionTime75_ns_vs_pulseArea_phd.dat};
            \addplot[blue, mark=none]
                   table [x=Centre,y=Normed]
              {Data/OD_Likelihood/h_capture_areaFractionTime75_ns_vs_pulseArea_phd.dat};
            \addplot[red, mark=none]
                    table [x=Centre,y=Normed]
                    {Data/OD_Likelihood/rockgamma_areaFractionTime75_ns_vs_pulseArea_phd.dat};
            %\node[] at (axis cs: 800,87) {\large 1400 mm};
             
    \end{groupplot}
    \node at ($(group c1r1) + (-0.5cm, 3.0cm)$) {\ref{Simulated_Likelihood_Variables_CommonLegend}};
\end{tikzpicture}
    \caption{Parameter-space population for various interactions.}
    \label{fig:simulated_likelihood_variables}
\end{figure}

\subsection{Performance}
\par
For this study, the signal was selected to be neutron captures and the background as cavern-$\gamma$'s
The neutron captures were simulated by starting thermalised neutrons in the GdLS.
This meant that only the energy deposits in the OD as a result post-capture were detectable.
The 
\par
The likelihood was performed against a set of simulations which contained a mixture of both neutron captures and cavern-$\gamma$'s in quantities expected \cite{LZ_assay_ref}.
The neutrons were, as above, thermalised so the only interaction of note are from the result of neutron captures.






\begin{figure}[!htbp]
    \centering
    \includegraphics[width=0.5\textwidth]{Figures/Placeholder.png}
    \caption{ROC curve of}
    \label{fig:discrimination_performance}
\end{figure}

\par
Although it is possible to use this method with data, it has been impractical given the differences mentioned earlier in this Chapter.
In order to perform this, a data-driven approach is required and simulation corrections implemented.
Both of which have not been performed due to time constraints.





\par
Interaction discrimination based upon a likelihood has the potential to improve the OD effectiveness; reducing the dead-time as as result of false vetoes.
It can also reduce the uncertainties on backgrounds - particularly if combined with detector coincidences on Gd-captures as was done in Section XXX.
There is scope for additional interesting studies such as seasonal modulation in rates of interactions such as ($\alpha$,$\gamma$) \cite{cavern_gammas_in_Soudan_mine_ref}.
All of these together can directly improve the sensitivity of LZ to a potential dark matter detection.

\par
Once data is better understood, it may be possible to implement higher dimensional variables to discriminate interactions.
Though all of these are subject to an eventual match-up in simulation and data.
