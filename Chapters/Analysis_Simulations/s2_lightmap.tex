\section{S2 Light Map}
\par
Suppose that a single free electron was to be simulated in the liquid Xenon for it's response.
Due to the electric field, this electron will accelerate upwards towards the gas layer.
As it travels through the medium, it may interact with the Xenon, exciting numerous other electrons into being free.
%A single free electron in the liquid phase can excite numerous other electrons into being free (depending upon energies) as it travels upwards in the electric field.
Each of these electrons could enter the gas-phase and produce optical photons.
In this phase, the mean photon yield is $\approx$ 825.
This number is discussed in more detail in Section \ref{S2RadialVariation}.
Therefore an initially simple simulation requires several thousand S2 photons to be simulated.
For more complex simulations (such as are needed for scale tests of the analysis chain), this becomes a bottleneck and an impractical approach.

\par
In theory, each of these photons can be traced independently but current software and computing infrastructure cannot cope with such parallelism. 
Instead, a probabilistic map is used to determine the photon hit pattern and arrival time on the PMTs.
For the S2 signal, this is broken up into two sets of probability density functions (PDF).
The first, is the likelihood of a photon hitting a given PMT from a given position.
This is essentially the number of photons which hit this PMT divided by the number of photons which hit any PMT.
It is a set of 3D histograms (one for each PMT).
The second is the likely path taken by the photon, as this effects the time it takes for the photon to hit the PMT, and thus the signal.
This kind of map can only be created by performing simulations of photons inside the gas-phase; where they would be produced during electroluminescence. 
The information needed is the photon start position, the end position, and the time of flight.
These maps are created such that they contain the cumulative probabilities.

\par
For every simulation after the map has been generated, there will be a significant speed increase as the photons themselves no longer need to be simulated.
However, some processing is still required.
When a subsequent simulation is performed, the electron paths are stopped once they reach the liquid surface.
The position of the electron is used to determine the path it would have taken and points along this path are taken to be the photon origin. % to the anode.
The maps are then used along with random numbers and binary searches to determine which PMT the photon hit (if any) as well as how long it took.

\par
A notable downside of this approach is that for different scenarios one may wish to investigate (such as using a different purity of Xenon or the temperature and therefore pressure) then a new light map should be used as these factors would affect the S2 photon production.



\subsubsection{S2 Radial Variation} \label{S2RadialVariation}
\par
The electric field originates from three electrodes; a cathode grid just above the bottom PMTs, a gate grid just below the surface of the liquid Xenon, and an anode grid in the gaseous Xenon.
In previous simulations it was assumed that the anode and gate were both rigid structures and so had a fixed distance, and electric field between them.
In this case, the number of photons produced in the gas gap by an electron is given by Equation \ref{NoPhotons} and is equal no matter the radius from the centre. 
\begin{equation}
    N_{photons/e} = eep \times (\alpha \times E_{gas} \times 1000 - \beta \times P - \gamma) \times d_{gas}
    \label{NoPhotons}
\end{equation}
Where $\alpha$, $\beta$ and $\gamma$ are constants, $E_{gas}$ is the electric field in the gas, P is the pressure of the liquid, and $d_{gas}$ is the distance from the top of the liquid to the anode \cite{NoPhotonsPerElectron}.
$eep$ is the electron emission probability at the surface.
%is taken from PiXeY data and fitted to give $eta = -0.03754 \times (eliq^{2}) + 0.5266 \times eliq - 0.84645$, where $eliq$ is the electric field in the liquid Xenon \cite{ElectronExtractionEfficiency}.
The emission probability is included as we want to use the average number of photons per electron rather than for one just one electron.
This was known to be incorrect as a stronger signal (more photons) should be achieved in the detector centre.
Tests conducted at SLAC indicated that there is bending of the wires according to Equation \ref{WireDeflection} due to the electric field.
\begin{equation}
    x(r,E) = A \sqrt{ \bigg( 1 + \cos{ \Big( \frac{r \pi}{R} } \Big) \bigg) \bigg( B^2 + r^2 \bigg) } \bigg( e^{(r-C)} + 1 \bigg)^{-1} 
    \label{WireDeflection}
\end{equation}
Where R is the radius of the wire, r is the point on the wire and A, B and C are all of the form: $A(E) = \sum_{j=0}^{M} a_{j} E^{j}$. E is the electric field and $a_{j}$ are constants.
As such, the electric field is not constant with radius and thus the number of photons produced by electroluminescence is changed.
I implemented this bending and the electric field change into the simulation package.
However, as the bending changes the electric field, and the electric field changes the bending, what I implemented remains an approximation.
