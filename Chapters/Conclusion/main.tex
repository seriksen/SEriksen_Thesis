\chapter{Conclusion}
\label{chap:conclusion}
\par
The source of the unknown mass of the universe remains a mystery.
The particle interpretation remains the most favoured, with WIMPs at the forefront.
\par
The LZ collaboration has constructed the most sensitive direct detection dark matter experiment, setting world-leading constraints on spin-independent and spin-dependent cross-sections.
This has been achieved using a xenon time projection chamber (TPC), two veto detectors and a strong understanding of the backgrounds in each detector.

\par
The simulation challenge of being able to properly simulate all backgrounds completely remains a difficult task.
A solution to optical simulations was presented in \autoref{chap:lz_simulations}, using GPUs, showing an improvement of in excess of 700-times that is achievable by CPU simulations.
In future, the method described can be extended to all parts of the simulation chain and particles one may wish to simulate.

\par
The Outer Detector (OD), which surrounds the TPC, acts to veto neutron events.
Measured in simulation in \autoref{chapter:lz_outer_detector} and in data in \autoref{chap:analysis_of_the_od}, it was found that it has not met the performance requirements it was designed for.
The neutron veto efficiency was measured during the commissioning phase as 84.6$\pm$1.2\%.
The decrease compared to simulations has been linked to water saturation of foam, causing fewer neutrons to reach the Outer Detector.
\par
The backgrounds expected in the OD were also examined in both simulation and data.
The observed rate above 200 keV was lower than expected.
An unexpectedly high rate of ${}^{210}$Po was observed which was linked to the acrylic tanks of the OD being open to the cavern air underground for prolonged periods of time, causing plate-out.
A fit to the background spectra was performed with the largest contribution from cavern-$\gamma$ events.

\par
Two analyses of non-relativistic EFT operator WIMP-nucleon couplings were presented in \autoref{chap:analysis_eft_work}.
A projected sensitivity of LZ from a 1000 live-day exposure to these signal models was performed using an extended region of interest.
It was shown that LZ will be able to probe 4 orders of magnitude lower than previous limits.
A search was performed on the first science run data of LZ in a limited region of interest.
This set world-leading exclusion constraints on all EFT operator couplings at almost all WIMP masses probed.
With an extended region of interest, LZ will be able to increase sensitivity to all WIMP masses.

\par
LZ is still in its infancy, having completed a 60 live-day WIMP search of a planned 1000 live-day search.
This is an increase in exposure of nearly 17-times.
The possibility of detecting a WIMP with LZ remains.
These are exciting times for direct dark matter detection.
