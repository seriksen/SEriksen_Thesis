\section{Candidates}
\par
With there being a plethora of evidence that "something is there", we can turn our attention to what it might be.
From the observation, we can say that an potential candidate for dark matter must be;
\begin{itemize}
    \item Weakly, or not interacting via electromagnetism 
    \item Weakly, or not interacting via the strong force
    \item Have some kind of clustering; either by being slow moving (cold) or some other way
    \item Not baryonic matter
    \item Stable over the timescale of the Universe.
\end{itemize}
Additionally, it may be beneficial to start with the Standard Model of particle physics, and look for extensions which satisfy the above requirements.
In this section a brief description of proposed candidates are described briefly, followed by possible the most favoured candidate; the weakly interacting massive particle (WIMP)
It should be noted that many of these particles.

\subsection{MACHOs and primordial black holes}
\par
The first set of candidates seems like a very good candidate on the surface.
Massive astrophysical compact halo objects (MACHOs) are objects such as star remnants, very faint stars and 
and primordial black holes are non-luminous.
These have been some of the earliest possible explanations and have been favourable as they don't require new matter.

\par
Microlening surveys are able to observe objects with XYZ properties, but are limited to XYZ masses.
If such 
Unfortunately microlening searching have shown that MACHOs cannot make up more than 25\% of galactic halo masses.

\par
Black holes are also a contributor. 
These are a favoured solution as well as they are comprised of understood matter, and will the LIGO collaboration having measuring the merger of black holes, and the first image of a black hole being taken in XXX, our understanding of them as a component is increasing.
However, black holes such as that at the centre of the Milky Way are not believed to have been around in the minutes after the Big Bang but rather from the collapsing of some of the earliest super massive stars.
Instead, primordial black holes must have existed, formed from a different process in the early minutes after the Big Bang.
This is supported by some of the measurements from LIGO such as the merger of black holes of around 30 solar masses.
A black hole of this mass is not as likely to be produced from the regular stellar evolution.
Assuming that there are indeed primordial black holes, then they would contribute at least in part to dark matter.
If these primordial black holes were to be the dominant explanation for dark matter, then the model for the formation of the early universe needs addressing.
Our understanding on this topic is still evolving and the amount that this contributes to dark matter will be pinned down.

\subsection{Modified Gravity}
\par

\subsection{Neutrinos}
\par
One of the first candidates suggested as dark matter were neutrinos: the stable, long-lived and weakly-interacting particles in the standard model.
Unfortunately, N-body simulations shown that given the relativistic velocities are inconsistent with the structural formations that we observe \cite{neutrinos_and_galaxy_clustering_ref}. 
This has not ruled out neutrinos entirely, with a new species suggested which interacts only gravitational \cite{sterile_neutrinos_ref}.
Postulated to have a small mixing angle with standard model neutrinos, this would allow for limited interaction between this dark matter and SM matter.
\par
The most recent excitement around this is an unidentified 3.55keV line in X-ray spectra from several galactic clusters with high dark matter content \cite{sterile_neutrino_xray_decay_ref}, with one possible interpretation is the decay of sterile neutrinos, though there is much debate around this \cite{xray_from_sterile_neutrons_2_ref, xray_from_sterile_neutrons_3_ref}.
Separately, results by MicroBooNE have indicated neutrino mixing that is inline with the standard model, going against previous results from MiniBooNE where an excess in neutrino oscillations \cite{miniboone_and_microboone_sterile_neutrino_ref}.
The BEST experiment has also pointed towards sterile neutrinos to account for the deficit observed in germanium isotope production \cite{best_sterile_neutrino_result_ref,best_sterile_neutrino_2_ref}.
With the correct properties, the sterile neutron could contribute to dark matter, though as the understanding of the neutrinos we already know about contains significant gaps (such as the mass hierarchy), more work is needed \cite{sterile_neutrino_as_dm_ref, sterile_neutrinos_dm_ref}.

\subsection{Axions}
\par
Hypothesised in 1977, axions were postulated to solve the strong CP problem of the standard model \cite{axion_origins_ref}.
Axions are a pseudo-Goldstone boson generated by spontaneous symmetry breaking at some energy scale, $f_a$, with the mass of this particle predicted to follow:
\begin{equation}
    m_a \approx (0.6eV)\frac{10^7 GeV}{f_a}
\end{equation}
The value of $f_a$ is constrained to be greater than that of the electroweak symmetry-breaking scale, given exciting constraints on experiments.
This leaves a particle with a mass between 10${}^{-6}$ and 10${}^{-2}$eV for which they could account for dark matter \cite{axions_ref}.
\par
There are multiple proposed production mechanisms for axions such as via thermal production in the early universe.
However, axions produced in this fashion would contribute to hot dark matter the constraints are very limitting \cite{hot_axions_ref}.
Separately, slower, non-relativistic axions can be created through other processes such as "re-alignment mechanism" \cite{cold_axion_ref}.
\par
Axions may couple to photons, allowing for a conversion to microwaves.
These can be searched for in a microwave cavity with a strong magnetic field applied to it.
In this situation it is expected that the axion will convert into monochromatic microwave photons.
The ADMX experiment has adopted this approach and is currently the most sensitive experiment to this \cite{admx_experiment_ref}.
\par
Another search of axions is where an axion is absorbed and an atomic electron is ejected.
This electron is then detectable as an electronic recoil.
These searches are typically performed by large underground direct dark matter experiments, with the tightest constraint on axio-electric coupling coming from XENONnT \cite{xenonnt_sr1_er_ref}.
\par
A third approach is via laser beams whereby a beam is propagated between two superconduction magnets that are optically separated.
If axions are coupling to photons, then the initial beam will transform into an axion and then later convert back, allowing the light to be seen through the optical barrier.
Though not as sensitive as the other axion-photon conversion approaches, it does not have the same uncertainties associated with astrophysics and cosmology.
Two notable experiments using this approach are ALPS \cite{alps_axion_result_ref} and OSQAR \cite{osqar_axion_result_ref}. 
\par
Other search methods are also being explored such as axion induced nuclear electric dipole moment (see CASPEr \cite{casper_experiment_ref}), axion to X-ray conversion in the presence of a magnetic field (see IAXO \cite{iaxo_experiment_ref}), and energy-loss in galactic observables such as supernova explosions due to axion-electron coupling \cite{axions_from_supernova_ref}.

\subsection{WIMPS}
\label{sec:wimp_as_a_candidate}
\par
Another candidate, are arguably the most popular is a set of new particles that interact via the weak force, but have a lot of mass.
Namely, Weakly Interacting Massive Particles (WIMPs)\footnote{this differs from the original meaning where the interaction probability was low}. 
\par
In this case, dark matter is formed of stable particles from the early universe.
At this time, WIMPs were in thermal equilibrium with the rest of the universe: so the self-annihilation rate was equal to that of the creation rate from interactions between light particles.
As the universe expanded, the density of WIMPs decreased until the self-annihilation rate dropped.
Referred to as "thermal freeze-out", the remaining WIMPs appear as a relic abundance. 

If the dark matter in our own Milky Way galaxy is comprised of WIMPs, then there should be a continuous WIMP flux crossing the Earth.
WIMP-nucleon interactions are predicted to be rare, the flux should be large enough to, in principle, be detectable via Earth-based detectors, by elastic scattering off a nucleus \cite{wimp_nucleon_interactions_first_suggestion_ref,supersymmetric_dark_matter_ref}.

As WIMPs travel at relative non-relativistic speeds, the recoil energy of the nucleon resulting from an elastic scatter is by only the centre of mass scattering angle, $\theta$ \cite{direct_detection_of_wimps_ref};
\begin{equation}
    E_{R} = \frac{{\mu}_{N}^{2}\nu_{\chi}^2}{m_{N}}(1-\cos(\theta))
\end{equation}


\par
Known as the "WIMP miracle", this connection between cosmology and particle physics has helped give WIMPs the prevalence it has.


\par
Additionally, there are certain extensions to the standard model which are able to solve problems including the hierarchy problem and gauge coupling unification which also produce a WIMP candidate.


\par
In addition to supyersymmetry, WIMP-like dark matter particles do appear from other theories.
The typical search strategy for WIMPs by large underground detectors such as LUX-ZEPLIN \cite{LZ_TechnicalDesignReview_ref}, XENONnT \cite{xenonnt_projected_sensitivty_ref} and PandaX-4T \cite{pandax_4t_ref}.