\section{Candidates}
\par
With there being a plethora of evidence that "something is there", we can turn our attention to what it might be.
From the observation, we can say that an potential candidate for dark matter must be;
\begin{itemize}
    \item Weakly, or not interacting via electromagnetism 
    \item Weakly, or not interacting via the strong force
    \item Have some kind of clustering; either by being slow moving (cold) or some other way
    \item Not baryonic matter
    \item Stable over the timescale of the Universe.
\end{itemize}
Additionally, it may be beneficial to start with the Standard Model of particle physics, and look for extensions which satisfy the above requirements.
In this section a brief description of proposed candidates are described briefly, followed by possible the most favoured candidate; the weakly interacting massive particle (WIMP)\footnote{The Weak in WIMP refers to the particle interacting via the weak-force, this differs from the original meaning where the interaction proability was low}.
It should be noted that many of these particles.

\subsection{MACHOs and primordial black holes}
\par
The first set of candidates seems like a very good candidate on the surface.
Massive astrophysical compact halo objects (MACHOs) are objects such as star remnants, very faint stars and 
and primordial black holes are non-luminous.
These have been some of the earliest possible explanations and have been favourable as they don't require new matter.

\par
Microlening surveys are able to observe objects with XYZ properties, but are limited to XYZ masses.
If such 
Unfortunately microlening searching have shown that MACHOs cannot make up more than 25\% of galactic halo masses.

\par
Black holes are also a contributor. 
These are a favoured solution as well as they are comprised of understood matter, and will the LIGO collaboration having measuring the merger of black holes, and the first image of a black hole being taken in XXX, our understanding of them as a component is increasing.
However, black holes such as that at the centre of the Milky Way are not believed to have been around in the minutes after the Big Bang but rather from the collapsing of some of the earliest super massive stars.
Instead, primordial black holes must have existed, formed from a different process in the early minutes after the Big Bang.
This is supported by some of the measurements from LIGO such as the merger of black holes of around 30 solar masses.
A black hole of this mass is not as likely to be produced from the regular stellar evolution.
Assuming that there are indeed primordial black holes, then they would contribute at least in part to dark matter.
If these primordial black holes were to be the dominant explanation for dark matter, then the model for the formation of the early universe needs addressing.
Our understanding on this topic is still evolving and the amount that this contributes to dark matter will be pinned down.

\subsection{Modified Gravity}
\par

\subsection{neutrinos}
\par
One property of neutrinos is the mixing angle.
A small mixing angle between sterile neutrinos and standard model neutrinos would allow for the limited interaction between this dark matter and SM matter.
Results on this are XXX

\subsection{Axions}
\par

\subsection{WIMPS}
\par
Another candidate, are arguably the most popular is a set of new particles that interact via the weak force, but have a lot of mass.
Namely, Weakly Interacting Massive Particles (WIMPs). 