\section{Effective Field Theory Searches} \label{eft_theory}

\par
Although the LZ detector has been built with vanilla dark-matter searches at the forefront of the design, this does not mean that other searches cannot be performed. 

\par
The assumption that the dark-matter interaction with nucleons overlooks the possible sensitivity to different scenarios such as dark-matter with different spins, and interactions with higher nuclear recoil energies.
A wide variation in sensitivity to interactions based on detection medium.

\par
Given the limitations within the standard model, it is natural to ask if there is a possible adaptation to the standard model which would allow to the existence of dark matter - and to explain why we have yet to detect it.
This naturally leads to the introduction of effect theories - an approximation of the complete underlying theory.
The primary benefits of this are that it allows for model-dependent analyses.

\par
Possible the most clear example of an effect field theory in physics is that of Newtonian gravity?
I actually need to research this bit.

\par
In the more generalised form, this is General Relatively.
However, this is expected to be the low energy effective field theory as it is only valid up to Planck scales.


\par
In the case of dark-matter, the underlying theory is not known or is too complex to compute, so instead we look at some approximation.

\par
An EFT looks at physics at a particular scale in a manner that allows for corrections to arbitrary precision.


\par
The steps for creating an EFT can be boiled down into 4-steps;
\begin{enumerate}
    \item Determine the degrees of freedom - not all degrees of freedom are relevant in all cases. For example, a low energy theory does not need to high mass (and so high energy) particles
    \item Symmetries - this is guided by the degrees of freedom
    \item Expansion parameters - this is just power counting ie, to what precision are terms needed
    \item Lagrangian - the simplest Lagrangian can then be developed from the above choices
\end{enumerate}


\section{Dark-Matter EFT}
\par
In this section we follow the approach described in XXX where we treat WIMP-nucleon scattering as a four-field interaction comprising of the WIMP field $\chi$, the nucleon field $N$ and the effective operators $\Operator$;

\begin{equation}
    \Lagrangian_{int} = \chi_{1} \Operator_{\chi} \chi_{2} N_{1} \Operator_{N} N_{2}
\end{equation}
Here, $1$ and $2$ indicate incoming and outgoing particles respectively.

\par
Now, as this theory is non-relativistic, the interaction must be Galilean invariant.
This leaves us with 4 quantities; the change in momentum $\vec{q}$, the relative velocity of the WIMP $\vec{v}=\vec{v}_{\chi,in} - \vec{v}_{N,in}$, the spin of the WIMP $\vec{S}_{\chi}$ and the spin of the nucleon $\vec{S}_{N}$.


% I have these written down as part of it... not clear where I got it from... I think papers
% spin-dependent, spin-independent, angular-momentum dependent and spin and angular-momentum dependent. 


\par
Galilean invariance momentum and velocity;
$\Vec{v} = \Vec{v}_{\chi,in} - \Vec{v}_{N,in}$

\par
Finally we have our set of Galilean-invariant and Hermitian-invariant quantities;
\begin{enumerate}
    \item momentum transfer: $i\Vec{q}$
    \item energy conservation: $\Vec{v}^{\bot}$
    \item nucleon spin: $\Vec{S}_{N}$
    \item dark matter spin: $\Vec{S}_{\chi}$
\end{enumerate}

\par
The invariant quantities can then be combined to form operators;

\begin{equation}
\begin{array}{lcl}
\Operator_{1}=1_{\chi}1_N \\ 
\Operator_{2}=\left(v^{\perp}\right)^{2} \\
\Operator_{3}=i \vec{S}_{N} \cdot\left(\frac{\vec{q}}{m_N} \times \vec{v}^{\perp}\right) \\ 
\Operator_{4}=\vec{S}_{\chi} \cdot \vec{S}_{N} \\ 
\Operator_{5}=i \vec{S}_{\chi} \cdot(\frac{\vec{q}}{m_N} \times \vec{v}^{\perp}) \\ 
\Operator_{6}=\left(\vec{S}_{\chi} \cdot \frac{\vec{q}}{m_N}\right)\left(\vec{S}_{N} \cdot \frac{\vec{q}}{m_N}\right) \\
\Operator_{7}=\vec{S}_{N} \cdot \vec{v}^{\perp} \\
\Operator_{8}=\vec{S}_{\chi} \cdot \vec{v}^{\perp} \\ 
\Operator_{9}=i \vec{S}_{\chi} \cdot\left(\vec{S}_{N} \times \frac{\vec{q}}{m_N}\right) \\
\Operator_{10}=i \vec{S}_{N} \cdot \frac{\vec{q}}{m_N} \\ 
\Operator_{11}=i \vec{S}_{\chi} \cdot \frac{\vec{q}}{m_N} \\
\Operator_{12} =\vec{S}_{\chi} \cdot\left(\vec{S}_{N} \times \vec{v}^{\perp}\right) \\
\Operator_{13} =i\left(\vec{S}_{\chi} \cdot \vec{v}^{\perp}\right)\left(\vec{S}_{N} \cdot \frac{\vec{q}}{m_N}\right) \\ 
\Operator_{14} =i\left(\vec{S}_{\chi} \cdot \frac{\vec{q}}{m_N}\right)\left(\vec{S}_{N} \cdot \vec{v}^{\perp}\right) \\ 
\Operator_{15} =-\left(\vec{S}_{\chi} \cdot \frac{\vec{q}}{m_N}\right)\left(\left(\vec{S}_{N} \times \vec{v}^{\perp}\right) \cdot \frac{\vec{q}}{m_N}\right)
\end{array}
\label{eq:EFT_Operators}
\end{equation}

\par
When it actually comes to using these operators, the $\mathcal{O}_{2}$ is not considered as it is relativistic. 
Additionally $\mathcal{O}_{16}$ that is mentioned in XXX is a linear combination of $\mathcal{O}_{15}$ and $\mathcal{O}_{12}$ so does not need to be included.

\par
The final piece of the puzzle is how the WIMPs interactions with isoscalars and isovectors could in fact be different.
This means that the complete interaction Lagrangian is;

\begin{equation}
    \Lagrangian_{int} = 
    \label{eq:HENR_Interaction_Lagrangian}
\end{equation}

