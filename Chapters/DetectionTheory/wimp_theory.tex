\section{WIMP-nucleon interactions} \label{sec:wimp_nucleus_interactions}
\par
If the dark matter in our own Milky Way galaxy is comprised of WIMPs, then there should be a continuous WIMP flux crossing the Earth.
WIMP-nucleon interactions are predicted to be rare, the flux should be large enough to, in principle, be detectable via Earth-based detectors, by elastic scattering off a nucleus \cite{wimp_nucleon_interactions_first_suggestion_ref,supersymmetric_dark_matter_ref}.

As WIMPs travel at relative non-relativistic speeds, the recoil energy of the nucleon resulting from an elastic scatter is by only the centre of mass scattering angle, $\theta$ \cite{direct_detection_of_wimps_ref};
\begin{equation}
    E_{R} = \frac{{\mu}_{N}^{2}\nu_{\chi}^2}{m_{N}}(1-\cos(\theta))
\end{equation}
where the $\mu_N$ is the WIMP-nucleon reduced mass $\frac{m_\chi m_N}{(m_\chi + m_N)}$ and $\nu_\chi$ is the incoming dark matter velocity.

\par
The detectability of this kind of interaction
\par
The differential rate of these nuclear recoils can be expressed as energy per day per kilogram \cite{direct_detection_of_wimps_ref};
\begin{equation}
    \frac{dR}{dE_{R}} = \frac{\rho}{m_{\chi}m_{A}} \int_{\nu_{min}}^{\infty} \nu f(\nu) \frac{d\sigma_{A}}{dE_{R}} d\nu
\end{equation}


\begin{equation}
    E_{R} = \frac{q^{2}}{2m_{N}} = \frac{\mu^2\nu_{\chi}^2}{m_{N}}(1-\cos(\theta))
\end{equation}

The recoil energy of the recoiling nucleus, $E_{R}$




\par
The differential event-rate thus becomes;

\begin{equation}
    \frac{dR}{dE_{R}} = \frac{\rho}{m_{\chi}m_{A}} \int_{\nu_{min}}^{\infty} \nu f(\nu) \frac{d\sigma_{A}}{dE_{R}} d\nu
\end{equation}
where $q=\sqrt{2{m}_{A}E_{R}}$ is the exchanged momentum.


\section{Local Dark Matter Properties}
\par
What can be seen from Equation XXX, is that the 



\par
Density and velocity


\begin{equation}
    f(\Vec{v}) = \frac{1}{\sqrt{2\pi}\sigma} \exp{-\frac{|v|^2}{e\sigma^2}} 
\end{equation}

\par
Historically the parameters used for determining the event-rate have varied between experiments, making comparisons between results impractical.
However, resent collaboration between dark matter experiments have now agreed upon parameters \cite{standard_halo_model_conventions_ref}, these are summarised in \autoref{tab:standard_parameters_for_dm}.

\begin{table}[!htbp]
    \centering
    \begin{tabular}{c|c|c|c}
        Parameter                               & Description               & Value             & Reference \\ \hline
        $\rho_\chi$                             & Local dark matter density & $0.3GeV/c^2/cm^2$ &            \\
        $\nu_{esc}$                             & Galactic escape velocity  & $544km/s$ &            \\
        $\langle|\Vec{v}_{\oplus}|\rangle$      & Average Galactic Earth velocity & $29.8km/s$ &            \\
        $\Vec{v}_{\circledast}$ & Solar peculiar velocity & (11.1, 12.2, 7.3)km/s &            \\
        $\Vec{v}_0$ & Local standard of rest velocity & $(0,238,0)km/s$ &            
         
    \end{tabular}
    \caption{Suggested Standard Halo Model parameters from \cite{standard_halo_model_conventions_ref}. Vectors are given as $(\nu_r,\nu_\phi,\nu_\theta)$.}
    \label{tab:standard_parameters_for_dm}
\end{table}



\section{WIMP-nucleus cross-section}
\par
Nuclear Model info \cite{wimp_nuclear_model_ref}.


\par
\begin{equation}
    (\frac{d\sigma_A}{dE_R})_{SI} = \frac{m_A}{2\mu_A^2 \nu^2}F_{SI}^2 (E_R) \sigma_A
\end{equation}

\begin{equation}
    \sigma_A = (\frac{\mu_A}{\mu_p})^2 A^2 \sigma_N^{SI}
\end{equation}


\par
There is also a spin-dependent scattering which arises from axial-vector interactions between WIMPs and quarks.
In this case, the cross section depends on the total nuclear spin, J, and the spin structure inside the nucleus, which are described by structure functions, $S_A(q)$.
For a complete derivation of the spin-dependent scattering, the reader should consult XXX, but the final result is;
\begin{equation}
    \frac{dR}{dE_R}_{SD} = \sum_{i=1}^{N_{iso}} \frac{(f_A N_T)_i 2\pi \rho_0 \sigma_{n,p}^{SD}}{3m_\chi \mu_{n,p}^2 (2J_i + 1)} (S_{n,p} (E_R))_i \zeta (E_R,t)
\end{equation}

