\section{WIMP-nucleon interactions} \label{sec:wimp_nucleus_interactions}
\par
If the dark matter in our own Milky Way galaxy is comprised of WIMPs, then there should be a continuous WIMP flux crossing the Earth.
WIMP-nucleon interactions are predicted to be rare, the WIMP flux is large enough to, in principle, be detectable via Earth-based detectors, by elastic scattering off a nucleus \cite{wimp_nucleon_interactions_first_suggestion_ref,supersymmetric_dark_matter_ref}.
Taking this assumption, then a non-relativistic WIMP of mass $m_{\chi}$ and velocity $\nu$ would  elastically scatter off a nucleus of mass $m_{A}$ with a recoil energy, $E_{R}$, which is dependent upon the scatter angle $\theta$;
\begin{equation}
    E_{R} = \frac{q^{2}}{2m_{N}} = \frac{\mu^2\nu_{\chi}^2}{m_{N}}(1-\cos(\theta))
\end{equation}

\par
The differential rate of these nuclear recoils can be expressed as energy per day per kilogram \cite{direct_detection_of_wimps_ref};
\begin{equation}
    \frac{dR}{dE_{R}} = \frac{\rho}{m_{\chi}m_{A}} \int_{\nu_{min}}^{\infty} \nu f(\nu) \frac{d\sigma_{A}}{dE_{R}} d\nu
\end{equation}


\begin{equation}
    E_{R} = \frac{q^{2}}{2m_{N}} = \frac{\mu^2\nu_{\chi}^2}{m_{N}}(1-\cos(\theta))
\end{equation}

The recoil energy of the recoiling nucleus, $E_{R}$




\par
The process which is trying to be observed is the elastic scattering of a WIMP off a target nucleus of masses $m_{\chi}$ and $m_{N}$ respectively.
From the transfer of momentum $q$, the nucleus recoils with energy $E_{R}$ given by


where $\nu_{\chi}$ is the incoming WIMP velocity, $\theta$ is the scattering angle in the center-of-mass frame, and $\mu=\frac{m_{\chi}m_{N}}{m_{\chi} + m_{N}}$


\section{Local Dark Matter Properties}
\par
Density and velocity




\section{WIMP-nucleus cross-section}
\par
