\chapter{Introduction}
\par
For thousands of years, humans have tried to understand the universe that we inhabit.
The majestic beauty of the sky has been one of the most intriguing things - with our knowledge developing from believing we are the centre of the universe to realising that we are a tiny part of it.
We are still learning about the universe, and there is much that we do not understand.
\par
Since the earliest days of astronomical photography, discrepancies between what is expected and what is observed have arisen.
For nearly a century, cosmological observations have given results that do not fit within the standard model of particle physics (SM), which though can predict most experimental results, cannot account for gravity and conflicts with general relativity. 
There is now a significant body of evidence for the existence of some kind of dark matter (DM), which contributes five times more mass to the universe than baryonic matter.
\par
Many candidate particles have been proposed as dark matter, but so far, no evidence for any of them has been found.
One of the favoured candidates are Weakly Interacting Massive Particles (WIMPs).
These particles interact via gravity and a force of similar or weaker strength than the weak force.
There is a probability of seeing WIMP scattering in Earth-based detectors, and numerous experiments have tried to observe it.
One experiment which is searching for WIMP-nucleon interactions is the LUX-ZEPLIN (LZ) experiment, housed at the Sanford Underground Research Facility.
LZ utilises a dual-phase xenon time projection chamber (TPC) to search for these rare interactions.
LZ has recently set the world-leading limits of spin-independent scattering \cite{lz_ws_sr1_ref}.
This thesis details the author's contributions to the LZ collaboration.

\par
\autoref{chap:dark_matter_evidence} contains an overview of the astronomical and cosmological evidence for DM.
This is followed by a description of the leading candidates and the current search strategies adopted to find them.

\par
In \autoref{chap:detection_theory} the theory of how WIMPs can be detected via nuclear recoils on Earth-based detectors is shown.
This chapter concludes with a description of non-relativistic Effective Field Theory (EFT), which is the basis of the analysis in the final chapter.

\par
\autoref{sec:lz_detector_chapter} introduces how xenon can be used to search for WIMPs in dual-phase time projection chamber experiments.
This chapter then describes the LZ experiment, the expected backgrounds, calibration sources, the dark matter search strategy and an overview of the simulation framework.

\par
In \autoref{chap:lz_simulations} developments to the LZ simulation capabilities are described, with a focus on the migration of CPU-based simulations to a GPU implementation.

\par
In \autoref{chapter:lz_outer_detector} a detailed description of the Outer Detector and its subsystems is given.
This is followed by details of the final assembly, along with design changes that occurred.
Finally, two performance requirements are evaluated using simulations.

\par
In \autoref{chap:analysis_of_the_od} the same performance requirements introduced in the previous chapter are evaluated on the completed detector during the commissioning phase.
This is taken further with a fit of the background spectrum and a comparison between neutron propagation in simulations and what is observed in data.
%Differences between the expectation from simulations and what is observed are linked to installation and assembly processes.
The OD veto cut used in the first science run by LZ is defined by the work in this chapter.

\par
The final analysis presented in this thesis, presented in \autoref{chap:analysis_eft_work}, is of non-relativistic elastic WIMP-nucleon EFT coupling described in \autoref{chap:detection_theory}.
In this chapter, the statistical method and signal models are described.
This is followed by a projected sensitivity study from a 1000 live-day exposure.
This chapter concludes with a description of the first science run of LZ and its sensitivity to isoscalar coupling strengths of EFT operators.

\par
The overall findings of this work are summarised in \autoref{chap:conclusion}.