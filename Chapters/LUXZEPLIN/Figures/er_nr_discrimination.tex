\begin{figure}[]%
\centering
\begin{tikzpicture}
\centering
    \begin{axis}[
            xlabel={S1${}_c$ [phd]},
            ylabel={log${}_{10}$ (S2${}_{c}$ [phd])},
            width=15cm, height=10cm,
            xmin=0, xmax=90,
            %ymin=0, %ymax=100,
            %minor y tick num=4,
            grid=major,]
            
            % ER Band
            \addplot[blue, opacity = 0.4, name path = er_high] table[x=x, y=high_er]
                {Data/tpc/bands.dat};
            \addplot[blue, opacity = 0.4, name path = er_low] table[x=x, y=low_er]
                {Data/tpc/bands.dat};
            \addplot[blue, opacity = 0.4] fill between[of=er_high and er_low];
            \addplot[blue, opacity = 0.8, name path = er_mid] table[x=x, y=mid_er]
                {Data/tpc/bands.dat};
            
            % NR Band
            \addplot[red, opacity = 0.4, name path = nr_high] table[x=x, y=high_nr]
                {Data/tpc/bands.dat};
            \addplot[red, opacity = 0.4, name path = nr_low] table[x=x, y=low_nr]
                {Data/tpc/bands.dat};
            \addplot[red, opacity = 0.4] fill between[of=nr_high and nr_low];
            \addplot[red, opacity = 0.8, name path = nr_mid] table[x=x, y=mid_nr]
                {Data/tpc/bands.dat};
            
            \addplot[blue, only marks,]
                   table [x=x,y=y]
              {Data/tpc/er_points.dat};
            \addplot[red, only marks,]
                   table [x=x,y=y]
              {Data/tpc/nr_points.dat};

    \end{axis}
\end{tikzpicture}
    \caption[Discrimination between electronic recoil and nuclear recoil events by the light-to-charge ratio]{Discrimination between electronic recoil (Blue) events and nuclear recoil (Red) events by the light-to-charge ratio.
    The solid lines are the median response whilst the shading regions are the 10-90\% quantiles.  
    Simulations were performed by G. Rischbieter and additional details can be found in \cite{gregrischbieter_thesis_ref}}
    \label{fig:er_nr_discrimination}
\end{figure}