\section{Dark Matter Search Strategy}
\par
As previously discussed, the primary direct dark matter search strategy is to search for an excess in NR events.
In order to do this, an event needs to be defined and to ultimately be recorded, and the pulses to be reconstructed.
The planned approach is detailed in \cite{LZ_TechnicalDesignReview_ref}, with the approach adapted for the first data run in \cite{lz_ws_sr1_ref}.

\par
An event defined simply as a window of time in which the read out from all PMTs were saved, typically for LZ this is a window of 4ms in length.
LZ utilises a number of different triggers to determine when data should be recorded, utilising the logic based upon \cite{lux_trigger_logic_ref}, with application to LZ detailed in \cite{nicolasangelides_thesis_ref}.
The three of the triggers are briefly discussed below.

\paragraph{S2 Trigger}
\par
This is the primary dark matter search trigger.
Tuned to trigger of a large pulse in the TPC that is line with what is expected from an S2 pulse.

\paragraph{OD Trigger}
\par
Triggered by large events in the Outer Detector, focusing on 2.2MeV events from neutron captures on Hydrogen.

\paragraph{Random Trigger}
Also often referred to as the heartbeat trigger, data is recorded for a fixed time.
It can act as a measurement of backgrounds as it is a fixed amount of data recorded periodically.

\par
From the trigger activation, a period of time before and after the trigger is recorded.
This pre-trigger time is required as that is where the S1 pulse will be in the S2 trigger.

\par
\subsection{Data Cuts}
\par
The core cuts are listed below;
\begin{itemize}
    \item \textbf{SS}: Select events which have only scattered once, "single scatter". This refers to the particle which set of the cascade seen in \autoref{fig:er_nr_tracks}. Particles which scatter more than once are not compatible with dark matter as their likelihood to scatter twice is too low. This cut is defined as events which have a single S1 pulse and a single S2 pulse in the event.
    \item \textbf{FID}: Removal of events where reconstruction of energy is poor and to incorporate LXe self-shielding, reducing the background rate. This cut is applied on the reconstructed \{$x,y,z$\} of a scatter to select events in an inner region of the TPC LXe.
    \item \textbf{ROI}: Select events where the recoil energy is in the range expected from a WIMP scatter. Typically this is between 1.65-6.5 keV ER events and 6-30keV NR events. For other dark matter models, such as EFT searches this requirement changes. This cut is applied on the S1 pulse size.
    \item \textbf{Veto}: Remove events which are inconsistent with dark matter due to scattering more than once. This removed both neutrons and $\gamma$'s. The veto detectors work as anti-coincident detectors so an event is vetoed if the Skin or OD saw a pulse within some time window of the S1 in the TPC.
\end{itemize}
In data taking, additional cuts are often required to account for experimental features not anticipated in simulations.
As such the above list should be considered as a "core" selection, with a requirement of data-driven cuts potentially required later.

\par
On the data which passes the cuts above, a statistical analysis is performed.
A multidimensional Profile Likelihood Ratio (PLR) fit is used to prob how well a background only model vs a background and WIMP signal model fit the data.
In both projected sensitivity and first science run, LZ has limited this to two dimensions: {$S1_c$, $S2_c$}, however, future analysis will extend this as has been done in previous TPC experiments \cite{LUX_RUN1_EFT_2021,LUX_RUN4_EFT_2021,shaunalsum_thesis_ref}.
A full description of the PLR used for the WIMP sensitivity projections and SR1 can be found in \cite{LZ_Ibles_LZStats_Thesis_ref}. 
A detailed description of the application to LZ monte-carlo data between the sensitivity projection and SR1 is described in \cite{jonathannikoleyczik_thesis_ref}.