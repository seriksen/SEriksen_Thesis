\section{LZ Dark Matter Search Strategy}
\par
LZ will search for NR events.
The core cuts are listed below;
\begin{itemize}
    \item \textbf{SS}: DM will only scatter once
    \item \textbf{FID}: to remove wall events
    \item \textbf{ROI}: between 1.65-6.5 keV ER events and 6-30keV NR events, for EFT searches this requirement will inevitably change.
    \item \textbf{Veto}: remove neutrons and $\gamma$'s 
\end{itemize}

\par
LZ will initial run a two dimensional Profile Likelihood Ratio (PLR) fit to distinguish between NR and ER events.
A full description of the PLR used for the WIMP sensitivity projections and SR1 can be found in \cite{LZ_Ibles_LZStats_Thesis_ref}. 
A detailed application to monte-carlo data is described in \cite{jonathannikoleyczik_thesis_ref}.




\subsection{Triggers}
\par
LZ utilises a number of different triggers to determine when data should be recorded, utilising the logic based upon \cite{lux_trigger_logic_ref}, with application to LZ detailed in \cite{nicolasangelides_thesis_ref}.
The three of the triggers are briefly discussed below.


\paragraph{S2 Trigger}
\par
Triggering off the S2 signal in the TPC. 
This is the dark matter search trigger.

\paragraph{OD Trigger}
\par
Triggered by large events in the Outer Detector, focusing on 2.2MeV events from neutron captures on Hydrogen.

\paragraph{Random Trigger}
Also often referred to as the heartbeat trigger, data is recorded for a fixed time.
It can act as a measurement of backgrounds as it is a fixed amount of data recorded periodically.


\par
There are two important features to note;
Firstly the logic is hierarchical, meaning that the Random Trigger will only record when none of the other triggers have switched.
Secondly, the DAQ captures an event when it receives a trigger signal unless there is already an event being captured, or the trigger signal arrives during the hold off time of the previous event.
In practical terms, it means that the Random Trigger can be biased to low energy events if another trigger is active.
For example, in the Outer Detector, if the OD Trigger is active then the Random Trigger will be biased to low energy events as the higher energy events are likely to be encompassed by the OD Trigger.

