\section{PLR}
\par
For example, when we set a limit on spin-independent WIMP-nucleon cross section σSI at 90\% confidence level for a given WIMP mass (call the limit σ90CL), we are saying:
“The model described by background plus signal (where σSI  = σ90CL) generates datasets more inconsistent with that model than our observed data 10\% of the time” (i.e. we exclude that model with Type I Error 10%).






\par
In this chapter the statistical statistical technique used by the LZ collaboration is explained.
A detailed explanation of the coding framework can be found in \cite{LZ_Ibles_LZStats_Thesis_ref}.

\section{Profile Likelihood Ratio Tests}
\par
As highlighted, the purpose of the LZ experiment is to discover dark matter by a WIMP interaction.
In order to actually achieve this, one must determine that the observed data cannot be explained by a background-only scenario.

\par
As all other direct-dark matter experiments, the LZ collaboration relies upon the Profile Likelihood Ratio (PLR) statistical technique.


\subsection{Hypothesis Testing}
\par
In direct dark matter searching hypothesis testing is used to discover or exclude a dark matter model in a given observation.



\section{Model Building}
\par
The above has given some recoil spectra that would be expected within natural Xenon (ie, that used in the LUX-ZEPLIN TPC).

\par
Within LZ, the experimental data will have a selection criteria placed upon it, as described in XXX.
This leaves the selected events which are parameterised by the observed quantities, namely RQs.
These observables, $x_{e}$.
The probability of obtaining a particular event is given by the event probability model which is a parametric family of probability density functions; $f(x_{e}|\theta)$ where $\theta$ is the model parameters.

\section{Statistical Framework}
With the model described previously, we now turn our attention to how to perform a statistical analysis with this.

\cite{LZ_Ibles_LZStats_Thesis_ref}

Hypothesis testing is used to either to;
\begin{enumerate}
    \item determine discovery significance of a model
    \item exclude a model
\end{enumerate}
In practice, models are rarely excluded, rather they are constrained.
Essentially, which region of parameter space is excluded by that observation.

\par
A frequentist hypothesis test a comparison of two well-defined hypotheses with the purpose of determining which one is more compatible with the observation.
The hypotheses generally are refered to as a \emph{null} ($H_{0}$) and \emph{alternative} ($H_{1}$).
The hypotheses describe a set of processes that could have produced the observation, namely models such as those described in XXX.

\par
In the exclusion-limit use, $H_{0}$ may describe the signal and background where as $H_{1}$ may describe just the background processes.
$H_{0}$ is tested with increasing values of the parameter of interest until a value is found above which the hypothesis can be excluded in favour of $H_{1}$.


\subsection{Application to LZ quantities}
\par
Within LZ, the experimental data will have a selection criteria placed upon it, as described in XXX.
This leaves the selected events which are parameterised by the observed quantities, namely RQs.

\subsection{Analysis framework}
\par
Add flow diagram of how it works for simulation and for real data?
\subsubsection{Signal}
\par
This may not be needed here as it could be done higher up... when the signal model is introduced?
\subsubsection{Backgrounds}
\subsubsection{PdfMaker}
\subsubsection{LZLama}
\subsubsection{LZStats}
\par
LZStats uses a test static defined by the -2log(lambda(mu)) where $\lambda_{\mu}$ is the profile likeliihood ratio for some mu. 
mu is the poisson mean of the number of events observed in some timescale.

\par
Given an observation of data - which we assume to be background data only - how big does the signal need to be in order to reject the hypothesis that it is just a background at the 90\% confidence level.
LZStats \cite{LZ_Ibles_LZStats_Thesis_ref}

