\chapter{The LUX-ZEPLIN Experiment}
\label{sec:lz_detector_chapter}
\par
As discussed in the previous chapter, there is a reasonable expectation that if dark matter is comprised at least in part of WIMPs, then WIMP-nucleon scattering should be detectable.
The question then becomes, what is the optimal detector to detect this scattering?
As the WIMP-nucleon scattering is a rare event, a low background detector design is required.
Regardless of what the detector is made from, there will always be some backgrounds, most commonly $\beta$s and $\gamma$s that need to be categorised and removed.
\par
In this chapter, the concepts of xenon-based dark matter detectors are introduced.
This is then followed by the presentation of the LUX-ZEPLIN dark matter experiment upon which the analyses in this thesis are based.
This is not the only type of direct dark matter detector, and for a description of other detector designs, the reader is directed to \cite{direct_detector_designs_ref}.

\section{WIMP identification in Xenon}
\label{sec:wimps_with_xenon}

\par
The choice to build a detector using xenon is motivated by the A$^2$ enhancement in the scattering rate (\autoref{eq:wimp_si_differential_rate}).
Not only does the high atomic mass give good sensitivity to SI interactions, but two naturally occurring xenon isotopes, ${}^{129}$Xe and ${}^{131}$Xe have a non-zero nuclear spin which makes them sensitive to SD interactions.
\par
The target medium (xenon) will have to be held in place by some other material.
This casing will contribute some amount of background due to radioactive impurities, which will be observed as interactions in the target medium.
Xenon in a liquid phase (LXe) is particularly good to counter this due to its high density (approximately 3g/cm${}^{3}$), and twinned with the large atomic mass has a high stopping power to both $\beta$ and $\gamma$ radioactive decays\footnote{for reference, aluminium has a density of 2.7 g/cm${}^{3}$}.
The inner volume of the LXe medium will contain a very low background from the detector components.

%Though the sensitivity to WIMP-proton spin-dependent interactions is less obvious as it requires neutron-proton mixing and is very dependent upon the nuclear model used.
%A further benefit of xenon is that it does not naturally contain any long-lived radioactive isotopes, or activation products.
%This means that the background rate is drive by 
%There are however several isotopes than can be produced by cosmogenically: and two metastable isotopes ${}^{129m}Xe$ and ${}^{131m}Xe$.


\subsection{Energy deposits in Xenon}
\par
The interaction of a particle within the target medium will result in either an interaction with the atomic nucleus, causing a nuclear recoil (NR) or within the electron cloud, resulting in an electron recoil (ER).
In both cases, the recoiling particle scatters with nearby atomic electrons and nuclei, transferring its kinetic energy: causing both excitation and ionisation of these atoms as a cascade of secondary recoils.
Regardless of which type of recoil occurs, an ionisation track is produced, which culminates in scintillation light.
The reader is directed towards both \cite{xenon_physics_ref}, and \cite{carldahl_thesis_ref} for thorough reviews of these processes, which are summarised below.
\par
An excited atom of xenon can bond to another xenon atom to form an excited molecule, or excimer, denoted as Xe$_2^{*\nu}$, which is both electronically and vibrationally excited.
This excimer will eventually vibrationally relax via collisions with other xenon atoms and will decay to the ground state via the emission of a photon.
This process of excition luminescence and is shown below for an electronic recoil:
\begin{align*}
    e^- + Xe &\rightarrow Xe^* + e^+  &\text{impact excitation} \\ 
    Xe^* + Xe &\rightarrow Xe_2^{*,\nu} &\text{excimer formation} \\
    Xe_2^{*,\nu} + Xe &\rightarrow Xe_2^* + Xe &\text{relaxation} \\
    Xe_2^* &\rightarrow Xe + Xe + \gamma &\text{VUV emission} 
\end{align*}
An ionised atom can also produce scintillation light, though the path to it is somewhat more complex.
The ionised atom can form diatomic molecules with nearby atoms.
Some of the electrons which were liberated during the cascade will recombine with the molecule, causing it to split, leaving one of the atoms in a highly excited state (Xe$^{**}$).
The excited atom will then relax down, allowing for photon emission as before.
This process is referred to as recombination luminescence and is shown below for an electronic recoil:
\begin{align*}
    e^- + Xe &\rightarrow Xe^+ + 2e^+- &\text{ionisation} \\ 
    Xe^+ + Xe + Xe &\rightarrow Xe_2^{+} + Xe &\text{diatomic molecule formation} \\
    e^- + Xe_2^+ + Xe&\rightarrow Xe^{**} + Xe &\text{recombination} \\
    Xe^{**} + Xe &\rightarrow Xe^{*} + Xe + heat &\text{relaxation} \\
    Xe^{*} + Xe  &\rightarrow Xe_2^{*,\nu} &\text{excimer formation} \\
    Xe^{*} + Xe + Xe &\rightarrow Xe_2^{*} + Xe + heat &\text{relaxation} \\
    Xe_2^* &\rightarrow Xe + Xe + \gamma &\text{VUV emission} 
\end{align*}
\par
The difference in how these interactions appear between electronic recoils and nuclear recoils is shown in \autoref{fig:er_nr_tracks}.
From a nuclear recoil, the initial atom has a number of scatters, but these are typically below the ionisation level.
On the other hand, an electron recoil will result in a number of photons.
Additionally, as the electron is less stopped by the LXe, the track length is significantly greater.
If the interaction occurs in an electric field, not all of the electrons will recombine, allowing for potentially a second scintillation sighting.

\begin{figure}
    \centering
    \includegraphics[width=\textwidth]{Figures/LZ/er_nr_tracks.png}
    \caption[Depiction of a 20 keV recoil on liquid xenon]{Depiction of 20 keV recoil of electronic recoil (top) and nuclear recoil (bottom) in liquid xenon.
    Figure is from RIVAL simulations by C. Dahl \cite{carldahl_thesis_ref}, with adaptations by C. Fahan \cite{carlosfahan_thesis_ref}.}
    \label{fig:er_nr_tracks}
\end{figure}

\par
The energy deposited in LXe, is split between the atoms and liberated electrons, and so the energy of the recoil, E, can be written as:
\begin{equation}
    E = \frac{W}{L}(N_i + N_{ex})
\end{equation}
Here, W is the W-factor for LXe, generally taken to be 13.7 $\pm$ 0.2 eV \cite{light_and_charge_of_xenon_ref}, though there are more recent measurements indicating it could be as low as 11.5 eV \cite{electron_excitation_energy_of_xenon_ref}. 
N$_i$ and N$_{ex}$ are the number of ionised atoms and number of excited atoms, respectively.
L corresponds to the ``Lindhard factor" or ``quenching" which accounts for the reduced light and charge that is lost to heat.
It is generally taken that L=1 for electron recoils as the heat loss does not vary with energy, allowing the impact this has to be taken into W.
\par
The ratio between the excitation atoms to ionised atoms allows for ER and NR events to be distinguished, with the resultant number of photons observed differing between the two.
Experimentally, N$_{ex}$/N$_i$ has been measured as $\approx$ 1 for nuclear recoils, and 0.06 for electron recoils \cite{ionisation_to_excitation_ratio_xenon_ref}.

\subsection{Dual-phase Time Projection Chamber}
\par
One such detector design that can utilise the charge-to-light ratio between ER and NR signals are dual-phase time projection chambers (TPCs).
Modern variants of these detectors contain a single element in a majority liquid state with a region of gas above.
\par
An interaction event in a TPC is characterised by two signals.
Firstly a prompt scintillation signal in the liquid region - referred to as the S1.
This is caused by excimers and recombination previously discussed.
Secondly, there is a delayed scintillation signal or S2.
This is observed in the gas phase via electroluminescence.
Both of these signals produce VUV light ($\approx$175 nm), which can be detected by photomultiplier tubes (PMTs) or silicon photomultipliers (SiPMs).
\par
In a typical TPC design, as shown in \autoref{fig:TPC_theory}, an electric field is applied.
This causes electrons that are freed as a result of the interaction to drift upwards towards the gas phase, where there is a probability of being extracted into the gas phase and producing an S2 signal.
\par
With PMTs or SiPMs placed at both the top and bottom of the TPC, the size of the interaction signals can be measured.
The preferred unit of this is ``photons detected" (phd)\footnote{Historically, the signal was measured in photoelectrons (phe) which can be traced back to the first TPC designs of the 1970s \cite{tpc_origins_ref}. However, as there is a non-negligible probability of two photoelectrons being emitted from a single VUV photon in a PMT \cite{pmts_in_xenon_ref}, phd has become more favourable as this can be taken into account.}.

\begin{figure}
    \centering
    \includegraphics[width=0.6\textwidth]{Figures/LZ/tpc_theory.png}
    \caption{An illustration of the LUX dual-phase TPC \cite{lux_ref}.}
    \label{fig:TPC_theory}
\end{figure}

\par
The position of an interaction in the TPC can be reconstructed using only the S1 and S2 signals.
Firstly the $x-y$ position of an interaction can be reconstructed by the hit-pattern of photons on the gas phase PMTs. 
As the electrons will not deviate significantly from the interaction site while drifting, the hit-pattern is a good approximation of this location, with $\backsim$5 mm resolution demonstrated \cite{lux_position_reconstruction_ref}.
The $z$-position can be directly measured by the time difference between the two signals.
This is possible as the electrons reach a terminal drift velocity rapidly within the LXe, allowing for a resolution $\backsim$5 mm \cite{LZ_TechnicalDesignReview_ref}.
This high-resolution position reconstruction allows for the self-shielding of Xenon to be utilised by only selecting regions of LXe away from the TPC edge, which will experience a significantly lower background rate.

\par
The size of the two pulses in terms of photons detected, which will now be referred to as S1 and S2, can be related to the underlying quanta which produced them via:
\begin{align}
    n_\gamma = N_{ex} + N_i r && n_e = N_i (1-r) \\
    S1 = n_\gamma G_1(x,y,z) && S2 = n_e G_2(x,y,z)
\end{align}
where both $G_1(x,y,z)$ and $G_2(x,y,z)$ account for the variable light collection efficiency.
It is useful to note that $G_1$ varies between 0 and 1 as the photons will not produce additional photons and so the maximum S1 can be is the number of photons produced.
On the other hand, $G_2$ can be greater than 1, as an extracted electron will produce more photons in the GXe.
In fact, $G_2$ is actually dependent upon the length of time an electron spends in the LXe, $t$, the efficiency the electron has to be extracted on the liquid surface, $\epsilon$, the scintillation yield in the gas phase, $Y_e$, and the light collection efficiency (LCE) in the gas $G_1^{gas}$.
The full expression of $G_2$ is given by:
\begin{equation}
    G_2 = e^{-t/\tau} \epsilon Y_e G_1^{gas}(x,y)
\end{equation}
In a typical analysis, the S1 and S2 signals are manipulated to account for the change in LCE with position.
For this, two new quantities are introduced as the corrected values: S1$_c$ and S2$_c$.
Both the S1 and S2 are adjusted to the centre of the TPC in \{$x,y$\}.
S1 is also adjusted to a $z$-position, typically taken to be the centre of the LXe.
\begin{align}
    S1_c = S1 \frac{G_1(0,0,z_c)}{G_1(x,y,z)} && S2_c = S2 \frac{G_1^{gas}(0,0)}{G_1^{gas}(x,y)} e^{t/\tau}
    \label{eq:s1c_and_s2c_full}
\end{align}
S1$_c$ and S2$_c$ are then just both linear relationships to the original quanta, related by a gain factor, $g$, which can be written as:
\begin{align}
    S1_c = g_1 n_\gamma && S2_c = g_2 n_e
\end{align}
With this linear relationship, the reconstructed recoil energy from the event of two pulses (and S1 and S2) becomes:
\begin{equation}
    E_R = \frac{W}{L}(\frac{S1_c}{g_1} + \frac{S2_c}{g_2})
\end{equation}
For NR events, L has an energy dependence.
\par
An example of the possible discrimination between ER and NR from a xenon TPC is shown in \autoref{fig:er_nr_discrimination} for an arbitrary TPC detector.
The discrimination between ER and NR events is not perfect, with a non-negligible amount of leakage between bands.
The viability of using TPCs for direct dark matter searches can be summarised by two requirements: understanding backgrounds and understanding ER and NR bands.

\begin{figure}[]%
\centering
\begin{tikzpicture}
\centering
    \begin{axis}[
            xlabel={S1${}_c$ [phd]},
            ylabel={log${}_{10}$ (S2${}_{c}$ [phd])},
            width=15cm, height=10cm,
            xmin=0, xmax=90,
            %ymin=0, %ymax=100,
            %minor y tick num=4,
            legend pos=south east,
            grid=major,]
            
            % ER Band
            \addplot[blue, opacity = 0.4, name path = er_high, forget plot] table[x=x, y=high_er]
                {Data/tpc/bands.dat};
            \addplot[blue, opacity = 0.4, name path = er_low, forget plot] table[x=x, y=low_er]
                {Data/tpc/bands.dat};
            \addplot[blue, opacity = 0.4, forget plot] fill between[of=er_high and er_low];
            \addplot[blue, opacity = 0.8, name path = er_mid, forget plot] table[x=x, y=mid_er]
                {Data/tpc/bands.dat};
            
            % NR Band
            \addplot[red, opacity = 0.4, name path = nr_high, forget plot] table[x=x, y=high_nr]
                {Data/tpc/bands.dat};
            \addplot[red, opacity = 0.4, name path = nr_low, forget plot] table[x=x, y=low_nr]
                {Data/tpc/bands.dat};
            \addplot[red, opacity = 0.4, forget plot] fill between[of=nr_high and nr_low];
            \addplot[red, opacity = 0.8, name path = nr_mid, forget plot] table[x=x, y=mid_nr]
                {Data/tpc/bands.dat};
            
            \addplot[blue, only marks,]
                   table [x=x,y=y]
              {Data/tpc/er_points.dat};
            \addlegendentry{Electronic Recoil};
            \addplot[red, only marks,]
                   table [x=x,y=y]
              {Data/tpc/nr_points.dat};
            \addlegendentry{Nuclear Recoil};    
            

    \end{axis}
\end{tikzpicture}
    \caption[Discrimination between electronic recoil and nuclear recoil events by the light-to-charge ratio]{Discrimination between electronic recoil (blue) events and nuclear recoil (red) events by the light-to-charge ratio.
    The solid lines are the median response whilst the shading regions are the 10-90\% quantiles.  
    Simulations were performed by G. Rischbieter and additional details can be found in \cite{gregrischbieter_thesis_ref}.}
    \label{fig:er_nr_discrimination}
\end{figure}

\par
Modelling of detector responses from S1 and S2 signals have undergone a huge amount of development in recent years, with the Noble Element Simulation Technique (NEST) library leading this.
Both Argon and Xenon detectors are modelled for energy deposits of sub-keV to several MeV \cite{nest_1_ref}.
Additional details of response modelling can be found in \cite{gregrischbieter_thesis_ref, flamenest_ref}.


\iffalse
\par


There are numerous different detector designs to choose from, and the reader is directed towards \cite{direct_detector_designs_ref} for an indepth comparison.


Fortunately, the majority of these will interact in the electron cloud surrounding the nucleus and result in an electronic recoil (ER).
This differs from the nuclear recoil (NR) which would be observed from a WIMP-nucleon interaction.
Therefore to detect a WIMP we are searching for an excess in NR events compared to out background.

\par
Therefore in order to achieve this, a target with minimal isotopes that will decay and cause an excess to be hidden need to be avoided.
Additionally, any detector design should allow for particle discrimination - or more explicit ER and NR discrimination. 
\fi

\clearpage
\section{The LUX-ZEPLIN Detector}
\label{sec:lz_detector}
\par
The LUX-ZEPLIN (LZ) experiment is a second-generation direct detection dark matter experiment, named from it's predecessor LUX \cite{lux_ref}, and the ZEPLIN series of experiments which pioneered xenon phase detectors \cite{zeplin_3_ref}.
The LZ experiment is hosted at the Sanford Underground Research Facility (South Dakota, USA), at the 4850" level underground.


\par
The LZ experiment is a multi-detector system, which work together to increase the sensitivity to dark matter candidates.
LZ is comprised of three discrete detectors; a TPC, and two active veto-detectors.
A diagram of the detector systems is shown in \autoref{fig:LZ_Cut_CAD}.

\begin{figure}
    \centering
    \includegraphics[width=\textwidth]{Figures/LZ/LZ_CAD_with_interactions.png}
    \caption{A schematic of the LZ detector systems as described in the Technical Design Review \cite{LZ_TechnicalDesignReview_ref}.
             Figure from \cite{LZ_TechnicalDesignReview_ref} using adaptation from \cite{LZ_Ibles_LZStats_Thesis_ref}.}
    \label{fig:LZ_Cut_CAD}
\end{figure}

\par
In the remainder of this chapter, the design and construction of the LZ experiment are detailed with particular emphasis on the TPC, Skin and OD which are relevant for the remainder of this thesis. 
It should be noted that the design is detailed in significant more detail in the Technical Design Review \cite{LZ_TechnicalDesignReview_ref}.

\subsection{Time Projection Chamber}
\label{sec:lz_tpc}
\par
At the core of LZ is a dual-phase (liquid and gas) Xenon TPC.
The schematic of the LZ TPC is shown in XXX.
Photons which are emitted inside of the TPC are detected by 494 Hamamatsu R11410-22 3-inch diameter PMTs, arranged into two arrays.
The first array contains 253 PMTs and is located at the top of the TPC in the GXe, looking down.
The second array, containing 241 PMTs and located at the bottom is immersed in the LXe looking up.
These PMTs were selected as they were developed with low levels of radioactivity and a high quantum efficiency at wavelengths around 175nm, the characteristic of LXe scintillation.
The arrangement of these PMT arrays is not identical, with the bottom array optimised for light collection of S1 signals, and the top array optimised for position reconstruction of S2 signals.
\par
The TPC contains three electrodes: a cathode at the bottom, a gate below the liquid surface and an anode in the gas phase, each of which are woven mesh grids \cite{lz_grids_ref}.
The walls of the TPC are enclosed by titanium rings encased in polytetrafluoroethylene (PTFE) panels - a highly reflective material (in excess of 97.3\% in LXe \cite{ptfe_lxe_reflectivity_ref}).
The titanium rings are connected by a resistor ladders.
These titanium rings act to shape the electric field, creating a vertical field across the entire active region such that any charge created in the active region will drift upwards through the liquid to the electroluminescence region.
\par
The TPC active region, which is defined as the volume between the cathode grid and the gate grid, is a cylinder of 1.46m diameter and height that contains 7-tonnes of Liquid Xenon.
This active region is a significant size increase compared to previous generation detectors such as LUX (250kg) \cite{lux_ref}, XENON1T (2.2-tonne) \cite{xenon1t_ref} and PandaX (500kg) \cite{pandax_ref}.
Other current generation Xenon TPC experiments include PandaX-4T (3.7-tonne) \cite{pandax_4t_ref} and XENONnT (5.9-tonne) \cite{xenonnt_projected_sensitivty_ref}, with both having recently published results from first runs \cite{pandax_4t_sr1_ref,xenonnt_sr1_er_ref}. 
\par
The electroluminescence region is defined as the distance from the surface of the LXe to the anode grid, an 8mm distance.
The distance between the gate grid and the anode grid is 13mm.
In the region, the electric field can reach 10 kV/cm.
This is typically referred to as the extraction field.
The electric field in the active region can reach 310 V/cm when applying an operating voltage of -50keV to the cathode.
\par
Below the cathode is a fourth grid which acts to shield the bottom array PMTs from the electric field.
Between this grid and the cathode is a reverse-field region (RFR), where the ionisation from interactions cannot be detected.


\subsection{Veto Detectors}
\label{sec:lz_veto_detectors}
\par
In addition to the TPC, the LZ experiment makes use of two additional detectors.
Neither of these are designed to directly detector dark matter, but rather to act to improve the sensitivity of the TPC by identifying events which would otherwise impact the TPCs ability to detector dark matter via anti-coincidence.
The detection principle if shown in \autoref{fig:LZ_Veto_Principle}.

\par
Firstly, surrounding the TPC is a layer of LXe referred to as the Xenon Skin, or Skin detector.
This region of LXe has to exist due to the geometrical challenge of placing the TPC inside of a cryovessel.
As such it has been made active by 38 Hamamatsu R8778 2-inch PMTs in the barrel and bottom dome and 93 R8520 1-inch PMTs in the top barrel region.
Similarly to the TPC, the ICV inner-wall has been coated with PTFE, but is optically de-coupled from the TPC.

\par
Outside of the OCV, the second veto detector is the Outer Detector (OD).
This is made up of a number of acrylic tanks which contain a liquid scintillator doped with gadolinium (GdLS).
Together these provide near 4$\pi$ coverage about the OCV, which act to identify neutron events.
The OD is described in more detail in \autoref{chapter:lz_outer_detector}.

\begin{figure}
    \centering
    \includegraphics[width=0.75\textwidth]{Figures/LZ/lz_veto_plan.png}
    \caption{Depiction of operating principle of the veto detectors and the particles that they are designed to veto.}
    \label{fig:LZ_Veto_Principle}
\end{figure}


\subsubsection{Design Changes}
\par
During the installation phase of the Acrylic tanks about the OCV, the curvature of the top Acrylic tanks were significantly different to that of the OCV, as such, the top 2 acrylic tanks could not be placed around the 


\subsection{Backgrounds}
\label{sec:lz_backgrounds}
\par
As previously discussed, the direct detection dark matter search philosophy is to search for an excess in nuclear recoils.
In order for this to be achievable, a deep understanding of the backgrounds that are expected are required.
In this section, the dominant backgrounds in the standard WIMP region of interest (ROI) are discussed.
Additionally, the backgrounds that become apparent in outside the ROI and affect other dark matter searched as introduced.

\par
The first background of note are detector components.
Everything is radioactive, and given that the detector has to be made of something there is no getting away from the fact that some of what you will see is from the materials that make up your detector.
Prior to the construction of LZ, a comprehensive screening campaign was undertaken to measure all materials which would be used in the detector.

\subsection{Calibrations}
\par
In order for the above experiment to work and have understandable results, a set of calibrations are required to characterise the; energy scale, energy threshold and detection efficiency.
In this section the calibration systems are explained, along with the different sources that are deployable.
A summary of the sources used prior to SR1 are shown in Table XXX. 
Additional sources planed are outlined in the LZ Technical Design Review \cite{LZ_TechnicalDesignReview_ref}

\begin{table}[!htbp]
    \centering
    \begin{tabular}{c|c|c|c}
    \hline
    Isotope       & Interacting particle         & Purpose                    & Deployment \\
    \hline
    ${}^{83m}Kr $ & beta/gamma, 32.1 keV/9.4 keV & TPC (x,y,z)                & Internal  \\
    ${}^{131m}Xe$ & 164 keV gamma                & TPC (x,y,z), Xe skin       & Internal  \\ 
    ${}^{220}Rn $ & various alphas               & xenon skin                 & Internal  \\
    $AmLi       $ & (alpha, n)                   & NR band                    & CSD       \\
    ${}^{252}Cf $ & spontaneous fission          & NR efficiency              & CSD       \\
    ${}^{57}Co  $ & 122 keV gamma                & Xe skin threshold          & CSD       \\
    ${}^{228}Th $ & 2.615 MeV gamma              & OD energy scale            & CSD       \\
    ${}^{22}Na  $ & back-to-back 511 keV gamma’s & TPC and OD sync            & CSD       \\
    ${}^{88}Y Be$ & 152 keV neutron low-energy   & NR response                & External  \\
    $DD         $ & 2,450 keV neutron            & NR light and charge yields & External  \\
    $DD         $ & 272 keV neutron              & NR light and charge yields & External
    \end{tabular}
    \caption{LZ calibration sources that were used for calibration prior to the first science run along with the calibration purpose and deployment method. Table adapted from \cite{LZ_TechnicalDesignReview_ref}.}
    \label{tab:LZ_Used_Calibration_Sources}
\end{table}


\section{Detector Status}
\par
During this thesis, the construction of the LZ experiment was completed, followed by a successful commissioning and calibration phase.
The first Science Run concluded April 2022 with the first result for the WIMP search announced June 2022.
The detector is currently undergoing Pre-SR2 upgrades, with SR2 due to commence September 2022.



%\section{LZ Dark Matter Search}
\begin{enumerate}
    \item What an LZ Event is
    \item What LZap is
\end{enumerate}

\subsection{Triggers}
\par
LZ utilises a number of different triggers to determine when data should be recorded, utilising the logic based upon \cite{lux_trigger_logic_ref}.
The three most significant triggers to analysis are briefly discussed below.


\paragraph{S2 Trigger}
\par
Triggering off an S2 signal in the TPC

\paragraph{OD Trigger}
\par
The OD Trigger is implemented to trigger off of the H-capture.


\paragraph{Random Trigger}
Also often referred to as the heartbeat trigger, data is recorded for a fixed time at a maximum rate of XXX Hz.
It can act as a measurement of backgrounds as it is a fixed amount of data recorded every now and then.


\par
There are two important features to note;
Firstly the logic is hierarchical, meaning that the Random Trigger will only record when none of the other triggers have switched.
Secondly, the DAQ captures an event when it receives a trigger signal unless there is already an event being captured, or the trigger signal arrives during the hold off time of the previous event.
In practical terms for example, it means that when both the OD Trigger and Random Trigger are active, the Random Trigger will be biased to low energy events as higher energy events are likely to be encompassed by the OD Trigger.

\subsubsection*{Summary}
\par
This chapter began with an overview of Xenon physics and how energy deposits can be observed.
This was used to explain how Time-Projection Chambers (TPC) can be used to search from WIMP-nucleon recoils.
The direct dark matter experiment LUX-ZEPLIN, which has at its core a TPC, was then summarised.
Each of the sub-detectors was described along with the backgrounds which are observed by the experiment.
An overview of the dark matter search strategy was then given.
Finally, an overview of the simulation and analysis chain of the LUX-ZEPLIN experiment was presented.