\section{WIMP identification in Xenon}
\label{sec:wimps_with_xenon}

\par
The high atomic mass of Xe is of benefit for spin-independent WIMP-nucleon scattering as can be seen in EQUATION XXX where there is a $A^2$ enhancement.
Additionally, Xe is sensitive to spin-dependent interactions as natural Xe two isotopes which odd-neutrons; ${}^{129}Xe$ and ${}^{131}Xe$ which isotopic abundances of 26.4\% and 21.1\% respectively.
Though the sensitivity to WIMP-proton spin-dependent interactions is less obvious as it requires neutron-proton mixing and is very dependent upon the nuclear model used.

\par
A further benefit of Xe is that it does not naturally contain any long-lived radioactive isotopes, or activation products.
There are however several isotopes than can be produced by cosmogenically: and two metastable isotopes ${}^{129m}Xe$ and ${}^{131m}Xe$.


\par
In a liquid state, Xe has a relatively high density of approximately 3g/cm${}^{3}$\footnote{for reference, aluminium has a density of 2.7g/cm${}^{3}$}.



\subsection{Energy deposits in Xenon}
\par
From an interacting particle to atomic electrons, the energy transfer will result in scintillation photons.
The 



Following the convention of XXX, the energy deposited in a medium from a recoil, $E_R$, is split between the atoms and the liberated electrons from the nuclei. 

\subsection{Dual-phase Time Projection Chamber}
\par


\par
In \autoref{fig:er_nr_discrimination} the discrimination between ER and NR events for an arbitrary detector is shown.

\begin{figure}[]%
\centering
\begin{tikzpicture}
\centering
    \begin{axis}[
            xlabel={S1${}_c$ [phd]},
            ylabel={log${}_{10}$ (S2${}_{c}$ [phd])},
            width=15cm, height=10cm,
            xmin=0, xmax=90,
            %ymin=0, %ymax=100,
            %minor y tick num=4,
            legend pos=south east,
            grid=major,]
            
            % ER Band
            \addplot[blue, opacity = 0.4, name path = er_high, forget plot] table[x=x, y=high_er]
                {Data/tpc/bands.dat};
            \addplot[blue, opacity = 0.4, name path = er_low, forget plot] table[x=x, y=low_er]
                {Data/tpc/bands.dat};
            \addplot[blue, opacity = 0.4, forget plot] fill between[of=er_high and er_low];
            \addplot[blue, opacity = 0.8, name path = er_mid, forget plot] table[x=x, y=mid_er]
                {Data/tpc/bands.dat};
            
            % NR Band
            \addplot[red, opacity = 0.4, name path = nr_high, forget plot] table[x=x, y=high_nr]
                {Data/tpc/bands.dat};
            \addplot[red, opacity = 0.4, name path = nr_low, forget plot] table[x=x, y=low_nr]
                {Data/tpc/bands.dat};
            \addplot[red, opacity = 0.4, forget plot] fill between[of=nr_high and nr_low];
            \addplot[red, opacity = 0.8, name path = nr_mid, forget plot] table[x=x, y=mid_nr]
                {Data/tpc/bands.dat};
            
            \addplot[blue, only marks,]
                   table [x=x,y=y]
              {Data/tpc/er_points.dat};
            \addlegendentry{Electronic Recoil};
            \addplot[red, only marks,]
                   table [x=x,y=y]
              {Data/tpc/nr_points.dat};
            \addlegendentry{Nuclear Recoil};    
            

    \end{axis}
\end{tikzpicture}
    \caption[Discrimination between electronic recoil and nuclear recoil events by the light-to-charge ratio]{Discrimination between electronic recoil (blue) events and nuclear recoil (red) events by the light-to-charge ratio.
    The solid lines are the median response whilst the shading regions are the 10-90\% quantiles.  
    Simulations were performed by G. Rischbieter and additional details can be found in \cite{gregrischbieter_thesis_ref}.}
    \label{fig:er_nr_discrimination}
\end{figure}

\par
The energy of such a scatter can be determined by 
\begin{equation}
sdfsdf    
\end{equation}
