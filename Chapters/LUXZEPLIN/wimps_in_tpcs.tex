\section{WIMP identification in Xenon}
\label{sec:wimps_with_xenon}

\par
The high atomic mass of Xe is of benefit for spin-independent WIMP-nucleon scattering as can be seen in EQUATION XXX where there is a $A^2$ enhancement.
Additionally, Xe is sensitive to spin-dependent interactions as natural Xe two isotopes which odd-neutrons; ${}^{129}Xe$ and ${}^{131}Xe$ which isotopic abundances of 26.4\% and 21.1\% respectively.
Though the sensitivity to WIMP-proton spin-dependent interactions is less obvious as it requires neutron-proton mixing and is very dependent upon the nuclear model used.

\par
A further benefit of Xe is that it does not naturally contain any long-lived radioactive isotopes, or activation products.
There are however several isotopes than can be produced by cosmogenically: and two metastable isotopes ${}^{129m}Xe$ and ${}^{131m}Xe$.


\par
In relation to a detector, the target material will have to be held in place by some other material.
This casing will contribute to the observed interactions in the target material.
Xenon remains an appropriate target particularly in liquid phase due to it's high density (approximately 3g/cm${}^{3}$), and therefore has a high stopping power for both $\beta$ and $\gamma$ radioactive decays\footnote{for reference, aluminium has a density of 2.7g/cm${}^{3}$}.
Therefore a detector of a reasonable size will contain a very low background central volume.


\subsection{Energy deposits in Xenon}
\par
From an interacting particle to atomic electrons, the energy transfer will result in scintillation photons.
The 



Following the convention of XXX, the energy deposited in a medium from a recoil, $E_R$, is split between the atoms and the liberated electrons from the nuclei. 

\subsection{Dual-phase Time Projection Chamber}
\par
One such detector design that can utilise the charge-to-light ratio between ER and NR signals are dual-phase time projection chambers (TPCs).
Modern variants of these detectors contain a single element in a majority liquid state with a region of gas above.
\par
An interaction event in a TPC is characterised by two signals.
Firstly a prompt scintillation signal in the liquid region - referred to as the S1.
This is caused by excimers and recombination previously discussed.
Secondly there is a delayed scintillation signal, or S2.
This is observed in the gas phase via electroluminescence.
Both of these signals produce VUV light ($\approx$175nm) which can be detected by photomultipler tubes (PMTs) or silicon photomulitpliers (SiPMs).
\par
In a typical TPC design, as shown in \autoref{fig:TPC_theory}, an electric field is applied.
This causes electrons that are freed as a result of the interaction to drift upwards towards the gas phase where there is a probability of being extracted into the gas phase and produce and S2 signal.
\par
With PMTs or SiPMs placed at both the top and bottom of the TPC, the size of the interaction signals can be measured.
The preferred unit of this is "photons detected" (phd)\footnote{Historically the signal was measured in photoelectrons (phe) which can be traced back the to first TPC designs of the 1970s \cite{tpc_origins_ref}. However, as there is a non-negligible probability of two photoelectrons being emitted from a single VUV photon in a PMT \cite{pmts_in_xenon_ref}, phd has become more favourable as this can be taken into account.}.

\begin{figure}
    \centering
    \includegraphics[width=0.6\textwidth]{Figures/LZ/tpc_theory.png}
    \caption{An illustration of the LUX dual-phase TPC \cite{lux_ref}.}
    \label{fig:TPC_theory}
\end{figure}

\par
The position of an interaction in the TPC can be reconstructed using only the S1 and S2 signals.
Firstly the x-y position of an interaction can be reconstructed by the hit-pattern of photons on the gas phase PMTs. 
As the electrons will not deviate significantly from the interaction site, the hit-pattern is a good approximation this location, with ~5mm resolution demonstrated \cite{lux_position_reconstruction_ref}.
The z-position can be directly measured by the time difference between the two signals.
This is possible as the electrons reach a terminal velocity rapidly within the LXe, allowing for a resolution ~100$\mu$s \cite{LZ_TechnicalDesignReview_ref}.
This high-resolution position reconstruction allows for the self-shielding of Xenon to be utilised, with the regions of LXe away from the TPC edge experiences a significantly lower background rate.

\par
The energy of an interaction can be energy of the event can be reconstructed using S1 and S2;


\begin{align}
    S1 = n_\gamma && S2 = n_\gamma
\end{align}
where 

\par
The final 
\begin{align}
    S1_c = g_1 n_\gamma && S2_c = g_2 n_\gamma
\end{align}

\par
The energy of the recoil is then defined as
\begin{equation}
    E_R = \frac{W}{L}(\frac{S1_c}{g_1} + \frac{S2_c}{g_2})
\end{equation}


\par
An example of the possible discrimination between ER and NR from a Xenon TPC is shown in \autoref{fig:er_nr_discrimination} for an arbitrary detector.
The viability of using TPCs for direct dark matter searches can be summarised by two requirements: understanding backgrounds and understanding ER and NR bands.

\begin{figure}[]%
\centering
\begin{tikzpicture}
\centering
    \begin{axis}[
            xlabel={S1${}_c$ [phd]},
            ylabel={log${}_{10}$ (S2${}_{c}$ [phd])},
            width=15cm, height=10cm,
            xmin=0, xmax=90,
            %ymin=0, %ymax=100,
            %minor y tick num=4,
            legend pos=south east,
            grid=major,]
            
            % ER Band
            \addplot[blue, opacity = 0.4, name path = er_high, forget plot] table[x=x, y=high_er]
                {Data/tpc/bands.dat};
            \addplot[blue, opacity = 0.4, name path = er_low, forget plot] table[x=x, y=low_er]
                {Data/tpc/bands.dat};
            \addplot[blue, opacity = 0.4, forget plot] fill between[of=er_high and er_low];
            \addplot[blue, opacity = 0.8, name path = er_mid, forget plot] table[x=x, y=mid_er]
                {Data/tpc/bands.dat};
            
            % NR Band
            \addplot[red, opacity = 0.4, name path = nr_high, forget plot] table[x=x, y=high_nr]
                {Data/tpc/bands.dat};
            \addplot[red, opacity = 0.4, name path = nr_low, forget plot] table[x=x, y=low_nr]
                {Data/tpc/bands.dat};
            \addplot[red, opacity = 0.4, forget plot] fill between[of=nr_high and nr_low];
            \addplot[red, opacity = 0.8, name path = nr_mid, forget plot] table[x=x, y=mid_nr]
                {Data/tpc/bands.dat};
            
            \addplot[blue, only marks,]
                   table [x=x,y=y]
              {Data/tpc/er_points.dat};
            \addlegendentry{Electronic Recoil};
            \addplot[red, only marks,]
                   table [x=x,y=y]
              {Data/tpc/nr_points.dat};
            \addlegendentry{Nuclear Recoil};    
            

    \end{axis}
\end{tikzpicture}
    \caption[Discrimination between electronic recoil and nuclear recoil events by the light-to-charge ratio]{Discrimination between electronic recoil (blue) events and nuclear recoil (red) events by the light-to-charge ratio.
    The solid lines are the median response whilst the shading regions are the 10-90\% quantiles.  
    Simulations were performed by G. Rischbieter and additional details can be found in \cite{gregrischbieter_thesis_ref}.}
    \label{fig:er_nr_discrimination}
\end{figure}
