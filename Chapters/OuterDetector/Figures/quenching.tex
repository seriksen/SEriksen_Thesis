\begin{figure}[]%
\begin{tikzpicture}
\centering
    \begin{axis}[
            ylabel=Visible Energy (MeV),
            xlabel=Particle Energy (MeV),
            width=15cm,
            height=8cm,
            grid=major,
            xmin=0, xmax=10,
            ymin=0,
            legend pos=north west,
            ]
            
        % Electron
        \addplot[red]
            coordinates {(0.099999982,0.07882662060000001)(19.79101040560001,17.986646500189995)};
        \addplot[green]
            table [x=Energy,y=Quenched]
            {Data/GdLS_Physics/Quenching/proton.dat};
        \addplot[blue]
            table [x=Energy,y=Quenched]
            {Data/GdLS_Physics/Quenching/alpha.dat};

        \addplot+[black, only marks, mark=*, mark options={fill=red}]
            table [x=x,y=y]
            {Data/GdLS_Physics/Quenching/electron_points.dat};
        \addplot+[black, only marks, mark=*, mark options={fill=green} ]
            table [x=x,y=y]
            {Data/GdLS_Physics/Quenching/proton_points.dat};
        \addplot+[black, only marks, mark=*, mark options={fill=blue}]
            table [x=x,y=y]
            {Data/GdLS_Physics/Quenching/alpha_points.dat};
            
        \legend{$e^-$, $p$, $\alpha$};
    \end{axis}
            
\end{tikzpicture}
    \caption{Quenching of various particles in LS as predicted by Birks' Law.
    Each colour represents a different particle.
    The points are measured energies.
    The $\alpha$ and $e^-$ data points are from \cite{ls_alpha_quenching_ref} and the $p$ points from \cite{ls_proton_quenching_ref}.
    The solid lines are Birks Law fits using the parameters in \autoref{tab:Birks_law_parameters}.}
    \label{fig:gdls_quenching}
\end{figure}