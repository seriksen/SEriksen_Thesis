\section{Interaction Detection}
\label{sec:od_physics}
\par
GdLS, like all LAB-based scintillators, does not have a linear energy response with respect to kinetic energies of charged particles \cite{nonlinear_gdls_ref}.
This is both energy- and particle-dependent and is primarily due to ionisation quenching \cite{lab_quenching_theory_ref}.
Ionisation quenching in GdLS will produce a light yield that can be described by Birk's law \cite{birks_law_ref} with the extension by Chou et al. \cite{generalised_birks_ref}:
\begin{equation} 
    \frac{dL}{dx} = S \frac{\frac{dE}{dx}}{1 + kB\frac{dE}{dx} + C\frac{dE}{dx}^2}
    \label{eq:birkslaw}
\end{equation}
where $L$ is the light yield, $S$ is the scintillation efficiency, $\frac{dE}{dx}$ is the energy loss of the particle per path length, and $k$, $B$ and $C$ are Birk's law parameters which differ depending upon the source.
This suppressed light yield is due to recombination and the quenching of exciting molecules via transfer to surrounding ones due to local saturation.
This results in highly ionised particles producing significantly lower amplitude signals than their energy would indicate.

\par
The parameters of Birk's law have to be measured for each material as the coefficients contain the probability of quenching and link the local density of ionised molecules to the energy loss, a factor that is material-dependent and cannot be isolated.
These parameters have been well studied for cocktails of similar makeup to the LZ GdLS with identical PPO and bis-MSB concentrations \cite{ls_alpha_quenching_ref,ls_proton_quenching_ref}.
The low amount of gadolinium-doping means that these should not differ significantly from those studies.
The parameters for LAB quenching are shown in \autoref{tab:Birks_law_parameters}.
The effect on observed energy using these parameters is shown in \autoref{fig:gdls_quenching} where $\alpha$ particles are the most heavily quenched.
In GdLS, a 100 keV electron will produce roughly the same amount of light as a 2~MeV $\alpha$.
In \autoref{tab:Birks_law_parameters} the photons-per-MeV is a rounded value between two separate measurements in \cite{scotthaselschwardt_thesis_ref}.

\begin{table}[]
    \centering
    \begin{tabular}{c | c | c | c }
                   & $S$ (photons/MeV) & $kB$ (m/MeV) & $C$ (m/MeV)$^{2}$ \\ \hline
    $\alpha$       &                   & 4.63E-3 & 1.77E-6 \\
    $\beta/\gamma$ & 9,000             & 0.03    & 0 \\ 
    proton         &                   & 8.26E-3 & 0
    \end{tabular}
    \caption{Birk's law parameters for LAB. $S$ is a rounded value between two measurements from \cite{scotthaselschwardt_thesis_ref}. The quenching values are from non-Gd doped LS studies \cite{ls_alpha_quenching_ref,ls_proton_quenching_ref}.}
    \label{tab:Birks_law_parameters}
\end{table} 

\begin{figure}[!htbp]%
\centering
\begin{tikzpicture}
\centering
    \begin{axis}[
            ylabel=Visible Energy (MeV),
            xlabel=Particle Energy (MeV),
            width=15cm,
            height=8cm,
            grid=major,
            xmin=0, xmax=10,
            ymin=0,
            legend style={anchor=north west},
            ]
            
        % Electron
        \addplot[red]
            coordinates {(0.099999982,0.07882662060000001)(19.79101040560001,17.986646500189995)};
        \addplot[green]
            table [x=Energy,y=Quenched]
            {Data/GdLS_Physics/Quenching/proton.dat};
        \addplot[blue]
            table [x=Energy,y=Quenched]
            {Data/GdLS_Physics/Quenching/alpha.dat};
            
        \legend{$e^-$, $p$, $\alpha$};
    \end{axis}
            
\end{tikzpicture}
    \caption{Quenching of various particles in LS as predicted by Birks' Law.}
    \label{fig:gdls_quenching}
\end{figure}

\subsection{Neutron Propagation}
\par
A neutron is most readily detectable from the products after a neutron capture.
The probability of a neutron being captured is generally proportional to the amount of time the neutron spends in proximity to any given nuclei.
Effectively, the slower a neutron is travelling, the more readily it will be captured.
Neutrons which enter the detector will be travelling relatively fast, making capture difficult.
They can, however, be slowed down by scattering off particles.
In the lab frame, the ratio between a neutron's initial energy ($E$) and final energy ($E'$) when colliding with an at-rest nucleus of mass A is given by:
\begin{equation}
    \frac{E'}{E} = \frac{A^2 + 1 + 2A\cos{\theta}}{(A + 1)^2}
\end{equation}
where $\theta$ is the scattering angle in the centre of mass.
Now, if the nucleus is a hard sphere, so independent of $\theta$, then an elastic scatter results in:
\begin{equation}
    E^{'} = E\frac{A}{A+1}
\end{equation}
If A = 1 (so is a proton), then this reduces further to $E^{'} = \frac{1}{2}E$, meaning that the neutron loses half its energy on average per collision.
The scattered proton has energy $E - E^{'}$.
Neutrons are most effectively moderated down on Hydrogen atoms \cite{neutron_thermalisation_and_capture_ref}.
If the scatter occurs in the GdLS, the light is quenched as previously described.
\par
Once thermalised, the neutron will be captured.
In GdLS, this will happen on a gadolinium isotope 87\% of the time, and on hydrogen the remaining times\footnote{Other isotopes can capture the neutron, but the probability is sufficiently low that they are omitted here.}.
Despite only making up a small fraction of the GdLS cocktail, the capture cross-section of two gadolinium isotopes (${}^{155}$Gd and ${}^{157}$Gd) is orders of magnitude above that of hydrogen and so dominate this process, as can be seen in \autoref{fig:gdls_capture_cross_sections}.

\begin{figure}
	\begin{tikzpicture}
		\centering
		\begin{axis}[%point meta max=150,
			%point meta min=0.0,
			view={0}{90},
			ylabel={Cross-Section (barns)},
			xlabel={Incident Energy (MeV)},
			width=15cm,
			height=8cm,
			ymode=log,
			xmode=log,
			]
			\addplot[blue] table[x=energy,y=barns] {Data/GdLS_Physics/capture_cross_sections/h.dat};
			\addplot[orange] table[x=energy,y=barns] {Data/GdLS_Physics/capture_cross_sections/gd155.dat};
			\addplot[green] table[x=energy,y=barns] {Data/GdLS_Physics/capture_cross_sections/gd157.dat};
			\legend{H, ${}^{155}$Gd, ${}^{157}$Gd};
		\end{axis}
	\end{tikzpicture}
\caption{Capture cross sections for the most significant capture particles in GdLS cocktail. 
A neutron has an increase capture probability if it's energy is close to that of the isotope energy. 
Values from ENDF/B-VIII.0 \cite{endf_bv3_ref}.}
\label{fig:gdls_capture_cross_sections}
\end{figure}
		

\par
Once a neutron is captured, the new isotope de-excites as either direct $\gamma$ emission or internal conversion resulting in ejected auger electrons or characteristic x-rays.
For an n-H capture, a single 2.2~MeV $\gamma$ is emitted, which may leave the GdLS without scattering, so the neutron is not detected.
However, n-Gd capture produces $\backsim$ 8 MeV worth of particles, typically as several $\gamma$s.
Between the excited and ground states of the gadolinium isotopes, ${}^{156}$Gd and ${}^{158}$Gd, there are in excess of $10^{6}$ nuclear excitation levels.
When the simulation framework was set up for the LZ experiment, GEANT4 was not adequate for modelling high atomic mass de-excitations where there are a huge number of possible cascade combinations \cite{ucsb_gdls_dicebox_simulations_ref}.
Other experiments with GdLS have the same problem, such as DayaBay \cite{dayabay_overview_ref} and RENO \cite{reno_overview_ref}.
However, the quantity of scintillator in those detectors is large enough that the entirety of the de-excitation energy is generally deposited in the medium.
The LZ OD is significantly thinner than those, and so there is a non-negligible probability of not all of the energy being deposited in the GdLS.
As such LZ incorporated DICEBOX \cite{dicebox_simulations_ref} simulations, which are sampled allowing for a more realistic simulation \cite{lz_simulations_ref}.
A detailed comparison of the DICEBOX result against GEANT4 can be found in \cite{ucsb_gdls_dicebox_simulations_ref}.
\par
From DICEBOX simulations, the way the energy is dissipated in the isotope is shown in \autoref{fig:od_gdls_dicebox_deexcitation}, and the resultant $\gamma$ spectrum in \autoref{fig:gd_capture_resulting_gamma_spectrum}.
For both ${}^{156}$Gd and ${}^{158}$Gd, the de-excitation occurs in an average of 4-5 steps, with the vast majority of the energy going directly to $\gamma$ emission.
Because of this, the de-excitation of gadolinium is often quoted as ``4.7 $\gamma$s".
This highlights the importance of doping the scintillator as there is a much higher probability of being able to observe an n-Gd capture with multiple $\gamma$s than it is as a single $\gamma$ from an n-H capture.

\begin{figure}[!htbp]%
\centering
\begin{tikzpicture}
\centering
  \begin{groupplot}[%view={0}{90},
    group style = {group size = 2 by 1}]
    
    \nextgroupplot[
            xlabel=N. Steps,
            ylabel=,
            width=0.45\textwidth,
            height=6cm,
            xmin=0,
            xmax=13,
            ymin=0,
            legend pos=north east,]
            \addplot[red, const plot]
                    table [x=Lower,y=Weight]
                    {Data/GdLS_Physics/DICEBOX/gd156_n_steps.txt};
                %\addlegendentry{$Gd^{156}$};
            \addplot[blue, const plot]
                    table [x=Lower,y=Weight]
                    {Data/GdLS_Physics/DICEBOX/gd158_n_steps.txt};
                %\addlegendentry{$Gd^{158}$};
    
    \nextgroupplot[
            xlabel=Energy Transfer location,
            ylabel=Percentage of steps,
            width=0.45\textwidth,
            height=6cm,
            xtick=data,
            xticklabels from table={Data/GdLS_Physics/DICEBOX/gd158_energy_conversion_to_shells.txt}{Label}]
            \addplot[red, const plot]
                    table [x=Bins, y=Weight]
                    {Data/GdLS_Physics/DICEBOX/gd158_energy_conversion_to_shells.txt};
                %\addlegendentry{$Gd^{156}$};
            \addplot[blue, const plot]
                    table [x=Bins, y=Weight]
                    {Data/GdLS_Physics/DICEBOX/gd156_energy_conversion_to_shells.txt};
                %\addlegendentry{$Gd^{158}$};
        
  \end{groupplot}
\end{tikzpicture}
\caption{Empty caption}
\label{fig:od_gdls_dicebox_deexcitation}
\end{figure}

\begin{figure}[]
    \centering
    \begin{tikzpicture}
        \begin{axis}[
            %title=$\gamma$ energy spectrum from $Gd^{155}$ and $Gd^{157}$ neutron captures,
            xlabel=Energy (KeV),
            ylabel=Emission (Arb.),
            width=15cm,
            height=6cm,
            xmin=0,
            xmax=8000,
            ymin=0,
            legend pos=north east,]
            \addplot[red, const plot]
                    table [x=Lower,y=Weight]
                    {Data/GdLS_Physics/DICEBOX/gd156_gamma_spectrum.txt};
                \addlegendentry{${}^{156}Gd$};
            \addplot[blue, const plot]
                    table [x=Lower,y=Weight]
                    {Data/GdLS_Physics/DICEBOX/gd158_gamma_spectrum.txt};
                \addlegendentry{${}^{158}Gd$};
        \end{axis}
    \end{tikzpicture}
    \caption{DICEBOX simulation of the energy released by $\gamma$s from de-excitation after neutron capture for ${}^{156}Gd$ and ${}^{158}Gd$}
    \label{fig:gd_capture_resulting_gamma_spectrum}
\end{figure}