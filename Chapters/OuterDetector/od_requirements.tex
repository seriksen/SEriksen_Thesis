\section{OD requirements}

\par
The performance requirements for the OD are listed in Table \ref{tab:veto_requirements}

\begin{table}[!htbp]
    \centering
    \begin{tabular}{p{0.2\textwidth}p{0.7\textwidth}} %{ c | c {\textwidth} } 
    \hline
    {Requirement Number} & {Description} \\ \hline
    R-160001             & Detection efficiency of 95\% for a 1 MeV neutron that scatters once in the xenon \\
    R-160005             & Must not veto more than 5\% of the WIMP search live-time
    \end{tabular}
    \caption{Selection of Veto Requirements. Adapted from Table 12.2.6 from \cite{LZ_TechnicalDesignReview_ref}}
    \label{tab:veto_requirements}
\end{table} 

\par
Neutron sources that can enter the TPC can originate from a ($\alpha$,n) reactions.
These are the most dangerous in terms of number expected, but also as only a single neutron will be released.
The primary other expected neutrons originate from spontaneous fission.
However multiple neutrons are released, often with higher energies and they are accompanied by $\gamma$'s and $\beta$'s so vetoing these is of less concern.

\begin{figure}[!htbp]
    \centering
    \begin{tikzpicture}
        \begin{axis}[
            xlabel=Energy (MeV),
            ylabel=Emission (arbitary units),
            width=15cm,
            height=6cm,
            xmin=0,
            xmax=10,
            ymin=0,]
            \addplot[red, const plot]
                    table [x=Energy,y=Rate]
                    {Data/Neutrons/background_neutron_spectrum.dat};
        \end{axis}
    \end{tikzpicture}
    \caption{Expected distribution of neutron energies from ($\alpha$,n) and USF neutrons.}
    \label{fig:simulation_background_neutron_energies}
\end{figure}


\begin{equation}
    \epsilon = \frac{\mathbf{SS, ROI, FID, Vetos}}{\mathbf{SS, ROI, FID}}
    \label{eq:neutron_efficiency}
\end{equation}


\subsection{Simulated Efficiency}
\par
The neutron efficiency has been calculated before, however given that OD geometry has changed significantly since then it has been re-done and additional detector efficiencies have been taken into account.
Previously, energy deposit simulations for background neutrons were used. 
From this, single scatters were isolated by XXX.
The skin cuts were defined as; XXX


\par
The same method has been applied here, however, the neutron source was changed to be 'post'-single scatter. 
As the neutron will not lose a significant amount of it's energy upon the scatter, a 1 MeV neutron will typically remain above XXX MeV...
Additionally, studying the efficiency in this regard allows for the requirement to be directly probed as well as studying how the neutron energy affects the efficiency.

\par
Figure XXX contains the result of this study alongside the previous result.
Firstly, it is important to note that the requirement R-160001 remains satisfied.
Secondly, the energy of the neutron is an important factor in the performance of the veto.
The expected energy spectrum for neutrons is shown in Figure \ref{fig:simulation_background_neutron_energies}.


\begin{tcolorbox}[colback=red!5!white, colframe=red!50!black, title=Key Plots]
\begin{enumerate}
    \item Simulation Efficiency
    \item Neutron capture energy deposits
    \item alpha,n energies
    \item fraction of gammas that get captured in the OD after SS (AmLi sims)
    \item fraction of neutrons that get captured in each volume
    \item time spent in volume vs OD capture time (for all and for Gd-only)
\end{enumerate}
\end{tcolorbox}

\subsection{Simulated Calibration Efficiency}
\par
Although the efficiency shown in energy deposit simulations shows that the requirement will be met, it does not necessarily reflect real-world performance.
Particularly given the propagation effects raised in Section XXX.
Additionally, during the calibration data taking it will not be possible to have such ideal circumstances to test the neutron tagging performance.
Therefore, in order to quantify this, two neutron sources were simulated, AmLi and DD, the deployment methods of each are described in Section XXX.

\par
AmLi is an ($\alpha$,n) source, producing neutrons predominantly under 0.5MeV, the full spectrum is shown in Figure \ref{fig:amli_neutron_energy_spectrum}.

\begin{figure}[!htbp]
    \centering
    \begin{tikzpicture}
        \begin{axis}[
            xlabel=Energy (MeV),
            ylabel=Probability,
            width=15cm,
            height=6cm,
            xmin=0,
            ymin=0,]
            \addplot[red, const plot]
                    table [x=Energy,y=Probability]
                    {Data/Neutrons/amli_neutron_spectrum.dat};
        \end{axis}
    \end{tikzpicture}
    \caption{AmLi neutron energy. Data from \cite{amli_neutron_energy_ref} }
    \label{fig:amli_neutron_energy_spectrum}
\end{figure}


\begin{figure}[!htbp]%
\centering
\begin{tikzpicture}
\centering
  \begin{groupplot}[%view={0}{90},
    group style = {group size = 2 by 1}]
    \nextgroupplot[
    xmin=300,
    xmax=600,
    ylabel={Emission (Arbitrary Units)},]
    \addplot[ybar interval, blue, const plot]
                    table [x=Wavelength,y=Rate]
                    {Data/GdLS_Physics/Light_Collection/gdls_light_output.dat};
    \nextgroupplot[
    xmin=300,
    xmax=600,
    xlabel={Wavelength (nm)}]
    \addplot[ybar interval, red, const plot]
            table [x=Wavelength,y=QE]
            {Data/GdLS_Physics/Light_Collection/od_pmt_qe.dat};
  \end{groupplot}
\end{tikzpicture}
\caption{AmLi simulated efficiency. \textbf{Left:} Energy deposits \textbf{Right:} Complete simulation}
\label{fig:asdasdasd}
\end{figure}

\par
In Chapter \ref{chap:analysis_of_the_od}, the requirements are tested in data.