\section{OD requirements}

\par
There are a number of performance requirements which the OD has been designed to meet, all of which can be found in \cite{LZ_TechnicalDesignReview_ref}, 
a subset of which is in Table \ref{tab:veto_requirements}.

\begin{table}[!htbp]
    \centering
    \begin{tabular}{p{0.2\textwidth}p{0.7\textwidth}} %{ c | c {\textwidth} } 
    \hline
    {Requirement Number} & {Description} \\ \hline
    R-160001             & Detection efficiency of 95\% for a 1 MeV neutron that scatters once in the xenon \\
    R-160005             & Must not veto more than 5\% of the WIMP search live-time
    \end{tabular}
    \caption{Selection of Veto Requirements. Adapted from Table 12.2.6 from \cite{LZ_TechnicalDesignReview_ref}}
    \label{tab:veto_requirements}
\end{table} 

\par
Neutron sources that can enter the TPC can originate from a ($\alpha$,n) reactions.
These are the most dangerous in terms of number expected, but also as only a single neutron will be released.
The primary other expected neutrons originate from spontaneous fission.
However multiple neutrons are released, often with higher energies and they are accompanied by $\gamma$'s and $\beta$'s so vetoing these is of less concern.

\begin{figure}[!htbp]
    \centering
    \begin{tikzpicture}
        \begin{axis}[
            xlabel=Energy (MeV),
            ylabel=Emission (arbitary units),
            width=15cm,
            height=6cm,
            xmin=0,
            xmax=10,
            ymin=0,]
            \addplot[red, const plot]
                    table [x=Energy,y=Rate]
                    {Data/Neutrons/background_neutron_spectrum.dat};
        \end{axis}
    \end{tikzpicture}
    \caption{Expected distribution of neutron energies from ($\alpha$,n) and USF neutrons.}
    \label{fig:simulation_background_neutron_energies}
\end{figure}


\begin{equation}
    \epsilon = \frac{\mathbf{SS, ROI, FID, Vetos}}{\mathbf{SS, ROI, FID}}
    \label{eq:neutron_efficiency}
\end{equation}


\subsection{Simulated Efficiency}
\par
The neutron efficiency has been calculated before, however given that OD geometry has changed significantly since then it has been re-done and additional detector efficiencies have been taken into account.
Previously, energy deposit simulations for background neutrons were used. 
From this, single scatters were isolated by XXX.
The skin cuts were defined as; XXX


\par
The same method has been applied here, however, the neutron source was changed to be 'post'-single scatter. 
As the neutron will not lose a significant amount of it's energy upon the scatter, a 1 MeV neutron will typically remain above XXX MeV...
Additionally, studying the efficiency in this regard allows for the requirement to be directly probed as well as studying how the neutron energy affects the efficiency.

\par
Figure \ref{fig:neutron_eff_energy_dep_tpc_neutrons} contains the result of this study alongside the previous result.
Firstly, it is important to note that the requirement R-160001 remains satisfied.
Secondly, the energy of the neutron is an important factor in the performance of the veto.
The expected energy spectrum for neutrons is shown in Figure \ref{fig:simulation_background_neutron_energies}.


\begin{tcolorbox}[colback=red!5!white, colframe=red!50!black, title=Key Plots]
\begin{enumerate}
    \item Simulation Efficiency
    \item Neutron capture energy deposits
    \item alpha,n energies
    \item fraction of gammas that get captured in the OD after SS (AmLi sims)
    \item fraction of neutrons that get captured in each volume
    \item time spent in volume vs OD capture time (for all and for Gd-only) to show what causes each component
\end{enumerate}
\end{tcolorbox}

\subsection{Simulated Calibration Efficiency}
\par
Although the efficiency shown in energy deposit simulations shows that the requirement will be met, it does not necessarily reflect real-world performance.
Particularly given the propagation effects raised in Section XXX.
Additionally, during the calibration data taking it will not be possible to have such ideal circumstances to test the neutron tagging performance.
Therefore, in order to quantify this, two neutron sources were simulated, AmLi and DD, the deployment methods of each are described in Section XXX.

\par
AmLi is an ($\alpha$,n) source, producing neutrons predominantly under 0.5MeV, the full spectrum is shown in Figure \ref{fig:amli_neutron_energy_spectrum}.

\begin{figure}[!htbp]
    \centering
    \begin{tikzpicture}
        \begin{axis}[
            xlabel=Energy (MeV),
            ylabel=Probability,
            width=15cm,
            height=6cm,
            xmin=0,
            ymin=0,]
            \addplot[red, const plot]
                    table [x=Energy,y=Probability]
                    {Data/Neutrons/amli_neutron_spectrum.dat};
        \end{axis}
    \end{tikzpicture}
    \caption{AmLi neutron energy. Data from \cite{amli_neutron_energy_ref} }
    \label{fig:amli_neutron_energy_spectrum}
\end{figure}


\par
In Chapter \ref{chap:analysis_of_the_od}, the requirements are tested in data.


\begin{figure}[!htbp]%
\centering
\begin{tikzpicture}
\centering
  \begin{groupplot}[%view={0}{90},
    group style = {group size = 2 by 2,vertical sep=1.5cm}]
    \nextgroupplot[
            title=TDR background neutrons,
            ylabel=Efficiency (\%),
            width=0.5\textwidth, height=6cm,
            xmin=0, xmax=1000,
            ymin=85, ymax=100,
            minor y tick num=4,
            grid=major]
            \addplot+[green, mark=none]
                    table [x=Time,y=Efficiency]
                    {Data/Neutron_Efficiency/Simulation/od_efficiency_sally_0kev.dat};
            \addplot[green, only marks, 
                     error bar legend,
                     error bars/.cd,
                     x dir=both, x explicit, error bar style={color=black}]
                    table [x=Time,y=Efficiency, x error=XError]
                    {Data/Neutron_Efficiency/Simulation/od_efficiency_sally_0kev.dat};
                    
            \addplot+[blue, mark=none]
                    table [x=Time,y=Efficiency]
                    {Data/Neutron_Efficiency/Simulation/od_efficiency_sally_100kev.dat};
            \addplot[blue, only marks, 
                     error bar legend,
                     error bars/.cd,
                     x dir=both, x explicit, error bar style={color=black}]
                    table [x=Time,y=Efficiency, x error=XError]
                    {Data/Neutron_Efficiency/Simulation/od_efficiency_sally_100kev.dat};
                    
            \addplot+[red, mark=none]
                    table [x=Time,y=Efficiency]
                    {Data/Neutron_Efficiency/Simulation/od_efficiency_sally_200kev.dat};
            \addplot[red, only marks, 
                     error bar legend,
                     error bars/.cd,
                     x dir=both, x explicit, error bar style={color=black}]
                    table [x=Time,y=Efficiency, x error=XError]
                    {Data/Neutron_Efficiency/Simulation/od_efficiency_sally_200kev.dat};
                    
    \nextgroupplot[
            title=100 keV,
            width=0.5\textwidth, height=6cm,
            xmin=0, xmax=1000,
            ymin=85, ymax=100,
            %yticklabels=\empty,
            yticklabel pos=right,
            minor y tick num=4,
            grid=major]
            \addplot+[green, smooth, mark=none]
                    table [x=Time,y=Efficiency]
                    {Data/Neutron_Efficiency/Simulation/edep_tpc_neutron_eff_100kev_threshold_0kev_smooth_line.dat};
            \addplot[green, only marks, 
                     error bar legend,
                     error bars/.cd,
                     x dir=both, x explicit, error bar style={color=black}]
                    table [x=Time,y=Efficiency, x error=EfficiencyError]
                    {Data/Neutron_Efficiency/Simulation/edep_tpc_neutron_eff_100kev_threshold_0kev_error_bars.dat};
                    
            \addplot+[blue, smooth, mark=none]
                    table [x=Time,y=Efficiency]
                    {Data/Neutron_Efficiency/Simulation/edep_tpc_neutron_eff_100kev_threshold_100kev_smooth_line.dat};
            \addplot[blue, only marks, 
                     error bar legend,
                     error bars/.cd,
                     x dir=both, x explicit, error bar style={color=black}]
                    table [x=Time,y=Efficiency, x error=EfficiencyError]
                    {Data/Neutron_Efficiency/Simulation/edep_tpc_neutron_eff_100kev_threshold_100kev_error_bars.dat};
                    
            \addplot+[red, smooth, mark=none]
                    table [x=Time,y=Efficiency]
                    {Data/Neutron_Efficiency/Simulation/edep_tpc_neutron_eff_100kev_threshold_200kev_smooth_line.dat};
            \addplot[red, only marks, 
                     error bar legend,
                     error bars/.cd,
                     x dir=both, x explicit, error bar style={color=black}]
                    table [x=Time,y=Efficiency, x error=EfficiencyError]
                    {Data/Neutron_Efficiency/Simulation/edep_tpc_neutron_eff_100kev_threshold_200kev_error_bars.dat};

    \nextgroupplot[
            title=1 MeV,
            xlabel=Veto Window ($\mu$s),
            ylabel=Efficiency (\%),
            width=0.5\textwidth, height=6cm,
            xmin=0, xmax=1000,
            ymin=85, ymax=100,
            minor y tick num=4,
            grid=major,
            legend style = { column sep = 10pt, legend columns = -1, legend to name = Simulated_Neutron_Eff_CommonLegend,}]
            \addplot+[green, smooth, mark=none]
                    table [x=Time,y=Efficiency]
                    {Data/Neutron_Efficiency/Simulation/edep_tpc_neutron_eff_1mev_threshold_0kev_smooth_line.dat};
            \addplot[green, only marks, 
                     error bar legend,
                     error bars/.cd,
                     x dir=both, x explicit, error bar style={color=black}]
                    table [x=Time,y=Efficiency, x error=EfficiencyError]
                    {Data/Neutron_Efficiency/Simulation/edep_tpc_neutron_eff_1mev_threshold_0kev_error_bars.dat};
                    
            \addplot+[blue, smooth, mark=none]
                    table [x=Time,y=Efficiency]
                    {Data/Neutron_Efficiency/Simulation/edep_tpc_neutron_eff_1mev_threshold_100kev_smooth_line.dat};
            \addplot[blue, only marks, 
                     error bar legend,
                     error bars/.cd,
                     x dir=both, x explicit, error bar style={color=black}]
                    table [x=Time,y=Efficiency, x error=EfficiencyError]
                    {Data/Neutron_Efficiency/Simulation/edep_tpc_neutron_eff_1mev_threshold_100kev_error_bars.dat};
                    
            \addplot+[red, smooth, mark=none]
                    table [x=Time,y=Efficiency]
                    {Data/Neutron_Efficiency/Simulation/edep_tpc_neutron_eff_1mev_threshold_200kev_smooth_line.dat};
            \addplot[red, only marks, 
                     error bar legend,
                     error bars/.cd,
                     x dir=both, x explicit, error bar style={color=black}]
                    table [x=Time,y=Efficiency, x error=EfficiencyError]
                    {Data/Neutron_Efficiency/Simulation/edep_tpc_neutron_eff_1mev_threshold_200kev_error_bars.dat};
            \legend{,0keV,,100keV,,200keV}                
    \nextgroupplot[
            title=6 MeV,
            xlabel=Veto Window ($\mu$s),
            width=0.5\textwidth, height=6cm,
            xmin=0, xmax=1000,
            ymin=85, ymax=100,
            %yticklabels=\empty,
            yticklabel pos=right,
            minor y tick num=4,
            grid=major,]
            \addplot+[green, smooth, mark=none]
                    table [x=Time,y=Efficiency]
                    {Data/Neutron_Efficiency/Simulation/edep_tpc_neutron_eff_6mev_threshold_0kev_smooth_line.dat};
            \addplot[green, only marks, 
                     error bar legend,
                     error bars/.cd,
                     x dir=both, x explicit, error bar style={color=black}]
                    table [x=Time,y=Efficiency, x error=EfficiencyError]
                    {Data/Neutron_Efficiency/Simulation/edep_tpc_neutron_eff_6mev_threshold_0kev_error_bars.dat};
                    
            \addplot+[blue, smooth, mark=none]
                    table [x=Time,y=Efficiency]
                    {Data/Neutron_Efficiency/Simulation/edep_tpc_neutron_eff_6mev_threshold_100kev_smooth_line.dat};
            \addplot[blue, only marks, 
                     error bar legend,
                     error bars/.cd,
                     x dir=both, x explicit, error bar style={color=black}]
                    table [x=Time,y=Efficiency, x error=EfficiencyError]
                    {Data/Neutron_Efficiency/Simulation/edep_tpc_neutron_eff_6mev_threshold_100kev_error_bars.dat};
                    
            \addplot+[red, smooth, mark=none]
                    table [x=Time,y=Efficiency]
                    {Data/Neutron_Efficiency/Simulation/edep_tpc_neutron_eff_6mev_threshold_200kev_smooth_line.dat};
            \addplot[red, only marks, 
                     error bar legend,
                     error bars/.cd,
                     x dir=both, x explicit, error bar style={color=black}]
                    table [x=Time,y=Efficiency, x error=EfficiencyError]
                    {Data/Neutron_Efficiency/Simulation/edep_tpc_neutron_eff_6mev_threshold_200kev_error_bars.dat};
            \node[] at (axis cs: 600,90) {6 MeV};
  \end{groupplot}
   \node at ($(group c2r1) - (group c1r1) + (-0.5cm, 6.0cm)$) {\ref{Simulated_Neutron_Eff_CommonLegend}};
\end{tikzpicture}
\caption{Neutron Veto Efficiency using energy deposits. TDR-era result with incomplete and simpler geometry taken from \cite{sallyshaw_thesis_ref}.}
\label{fig:neutron_eff_energy_dep_tpc_neutrons}
\end{figure}

\par
Those this shows that there is good neutron efficiency, it does not tell the full story.
In fact only XXX \% of the neutrons get captured in the GdLS (from the 1MeV neutrons that start in the fiducial volume), and even if a neutron is captured, the released $\gamma$'s do not have to scatter within the GdLS.
The energy deposits that are observed come from $\gamma$'s released as a result of captures elsewhere.


\begin{table}[]
    \centering
    \begin{tabular}{c|c|c|c|c|c|c}
         \multirow{2}{*}{Neutron Source} & \multicolumn{5}{c|}{Number}                  & \multirow{2}{*}{Veto Eff. (\%)}  \\ 
                         & Simulated  & Usable     & SS        & FID     & Veto         &                                  \\ \hline
        energy dep       & 19,310,000 & 19,280,766 & 1,474,033 & 325,080 & 323,319      & 99.4                             \\
        full propagation & X          & X          &           &         &              & 
    \end{tabular}
    \caption{Number of events passing various cuts from AmLi simulation in CSD1.}
    \label{tab:neutron_simulation_table_of_cuts}
\end{table}


\begin{figure}[!htbp]%
\centering
\begin{tikzpicture}
\centering
    \begin{groupplot}[%view={0}{90},
    group style = {group size = 2 by 2,vertical sep=1.5cm}]
    \nextgroupplot[
            xlabel=Veto Window ($\mu$s),
            ylabel=Efficiency (\%),
            width=0.5\textwidth, height=6cm,
            xmin=0, xmax=1000,
            ymin=85, ymax=100,
            minor y tick num=4,
            grid=major,
            legend style = { column sep = 10pt, legend columns = -1, legend to name = Simulated_AmLi_CommonLegend,}]
            \addplot+[green, mark=none]
                    table [x=Time,y=Efficiency]
                    {Data/Neutron_Efficiency/Simulation/edep_amli_neutron_eff_amli_edeps_threshold_0kev_error_bars.dat};
            \addplot[green, only marks, 
                     error bar legend,
                     error bars/.cd,
                     x dir=both, x explicit, error bar style={color=black}]
                    table [x=Time,y=Efficiency, x error=EfficiencyError]
                    {Data/Neutron_Efficiency/Simulation/edep_amli_neutron_eff_amli_edeps_threshold_0kev_error_bars.dat};
                    
            \addplot+[blue, mark=none]
                    table [x=Time,y=Efficiency]
                    {Data/Neutron_Efficiency/Simulation/edep_amli_neutron_eff_amli_edeps_threshold_100kev_error_bars.dat};
            \addplot[blue, only marks, 
                     error bar legend,
                     error bars/.cd,
                     x dir=both, x explicit, error bar style={color=black}]
                    table [x=Time,y=Efficiency, x error=EfficiencyError]
                    {Data/Neutron_Efficiency/Simulation/edep_amli_neutron_eff_amli_edeps_threshold_100kev_error_bars.dat};
                    
            \addplot+[red, mark=none]
                    table [x=Time,y=Efficiency]
                    {Data/Neutron_Efficiency/Simulation/edep_amli_neutron_eff_amli_edeps_threshold_200kev_error_bars.dat};
            \addplot[red, only marks, 
                     error bar legend,
                     error bars/.cd,
                     x dir=both, x explicit, error bar style={color=black}]
                    table [x=Time,y=Efficiency, x error=EfficiencyError]
                    {Data/Neutron_Efficiency/Simulation/edep_amli_neutron_eff_amli_edeps_threshold_200kev_error_bars.dat};
            \legend{,0keV,,100keV,,200keV}
        \nextgroupplot[
            xlabel=Veto Window ($\mu$s),
            width=0.5\textwidth, height=6cm,
            xmin=0, xmax=1000,
            ymin=85, ymax=100,
            yticklabel pos=right,
            minor y tick num=4,
            grid=major]
            \addplot+[green, mark=none]
                    table [x=Time,y=Efficiency]
                    {Data/Neutron_Efficiency/Simulation/amli_neutron_eff_0kev_error_bars.dat};
            \addplot[green, only marks, 
                     error bar legend,
                     error bars/.cd,
                     x dir=both, x explicit, error bar style={color=black}]
                    table [x=Time,y=Efficiency, x error=EfficiencyError]
                    {Data/Neutron_Efficiency/Simulation/amli_neutron_eff_0kev_error_bars.dat};
            \addplot+[blue, mark=none]
                    table [x=Time,y=Efficiency]
                    {Data/Neutron_Efficiency/Simulation/amli_neutron_eff_50kev_error_bars.dat};
            \addplot[blue, only marks, 
                     error bar legend,
                     error bars/.cd,
                     x dir=both, x explicit, error bar style={color=black}]
                    table [x=Time,y=Efficiency, x error=EfficiencyError]
                    {Data/Neutron_Efficiency/Simulation/amli_neutron_eff_50kev_error_bars.dat};
            \addplot+[red, mark=none]
                    table [x=Time,y=Efficiency]
                    {Data/Neutron_Efficiency/Simulation/amli_neutron_eff_100kev_error_bars.dat};
            \addplot[red, only marks, 
                     error bar legend,
                     error bars/.cd,
                     x dir=both, x explicit, error bar style={color=black}]
                    table [x=Time,y=Efficiency, x error=EfficiencyError]
                    {Data/Neutron_Efficiency/Simulation/amli_neutron_eff_100kev_error_bars.dat};
    \end{groupplot}
     \node at ($(group c2r1) - (group c1r1) + (-0.5cm, 5.0cm)$) {\ref{Simulated_AmLi_CommonLegend}};
\end{tikzpicture}
    \caption{Simulated AmLi neutron tagging efficiency. \textbf{Left:} energy deposits \textbf{Right:} Complete propagation and event reconstruction}
    \label{fig:simulated_amli_neutron_efficiency.}
\end{figure}
