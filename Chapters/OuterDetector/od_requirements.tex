\section{OD requirements}

\par
The performance requirements for the OD are listed in Table \ref{tab:veto_requirements}

\begin{table}[!htbp]
    \centering
    \begin{tabular}{p{0.2\textwidth}p{0.7\textwidth}} %{ c | c {\textwidth} } 
    \hline
    {Requirement Number} & {Description} \\ \hline
    R-160001             & Detection efficiency of 95\% for a 1 MeV neutron that scatters once in the xenon \\
    R-160005             & Must not veto more than 5\% of the WIMP search live-time
    \end{tabular}
    \caption{Selection of Veto Requirements. Adapted from Table 12.2.6 from \cite{LZ_TechnicalDesignReview_ref}}
    \label{tab:veto_requirements}
\end{table} 

\par
Neutron sources that can enter the TPC can originate from a ($\alpha$,n) reactions.
These are the most dangerous in terms of number expected, but also as only a single neutron will be released.
The primary other expected neutrons originate from spontaneous fission.
However multiple neutrons are released, often with higher energies and they are accompanied by $\gamma$'s and $\beta$'s so vetoing these is of less concern.

\begin{figure}[!htbp]
    \centering
    \begin{tikzpicture}
        \begin{axis}[
            xlabel=Energy (MeV),
            ylabel=Rate (arb),
            width=15cm,
            height=6cm,
            xmin=0,
            xmax=10,
            ymin=0,]
            \addplot[red, const plot]
                    table [x=Energy,y=Rate]
                    {Data/Neutrons/background_neutron_spectrum.dat};
        \end{axis}
    \end{tikzpicture}
    \caption{Expected distribution of neutron energies from ($\alpha$,n) and USF neutrons.}
    \label{fig:simulation_background_neutron_energies}
\end{figure}


\begin{equation}
    \epsilon = \frac{\mathbf{SS, ROI, FID, Vetos}}{\mathbf{SS, ROI, FID}}
    \label{eq:neutron_efficiency}
\end{equation}


\subsection{Simulated Efficiency}
\par
The neutron efficiency has been calculated before, however given that OD geometry has changed significantly since then it has been re-done and additional detector efficiencies have been taken into account.
Previously, energy deposit simulations for background neutrons were used. 
From this, single scatters were isolated by XXX.
The skin cuts were defined as; XXX


\par
The same method has been applied here, however, the neutron source was changed to be 'post'-single scatter. 
As the neutron will not lose a significant amount of it's energy upon the scatter, a 1 MeV neutron will typically remain above XXX MeV...
Additionally, studying the efficiency in this regard allows for the requirement to be directly probed as well as studying how the neutron energy affects the efficiency.

\par
Figure XXX contains the result of this study alongside the previous result.
Firstly, it is important to note that the requirement R-160001 remains satisfied.
Secondly, the energy of the neutron is an important factor in the performance of the veto.
The energy spectrum for ($\alpha$,n) neutrons is shown in Figure XXX.
This is 


\begin{tcolorbox}[colback=red!5!white, colframe=red!50!black, title=Key Plots]
\begin{enumerate}
    \item Simulation Efficiency
    \item Equation of energy lost from a scatter (why neutron loses hardly any energy so can assume it's already single scattered)
    \item Neutron capture energy deposits
    \item alpha,n energies
\end{enumerate}
\end{tcolorbox}

\par
Now


\par
In Chapter \ref{chap:analysis_of_the_od}, the requirements are tested in data.