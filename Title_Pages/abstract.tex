%
% File: abstract.tex
% Author: V?ctor Bre?a-Medina
% Description: Contains the text for thesis abstract
%
% UoB guidelines:
%
% Each copy must include an abstract or summary of the dissertation in not
% more than 300 words, on one side of A4, which should be single-spaced in a
% font size in the range 10 to 12. If the dissertation is in a language other
% than English, an abstract in that language and an abstract in English must
% be included.

\chapter*{Abstract}
\begin{SingleSpace}
\par
There is compelling evidence both astrophysical and cosmological for the bulk of the mass of the universe to be comprised of a non-luminous and near collisionless substance, so-called dark matter.
A favoured dark matter candidate is the Weakly Interacting Massive Particle (WIMP), which may be detectable via WIMP-nucleon scattering.
The LUX-ZEPLIN (LZ) experiment is a second-generation dual-phase Time Projection Chamber (TPC), which searches for these interactions.
The sensitivity of LZ is enhanced by two veto detectors, a liquid xenon Skin detector and a gadolinium-doped liquid scintillator Outer Detector (OD).
\par
To understand the behaviour of the detectors to interactions, large-scale simulations are required.
Presented in this work is a solution to the bottleneck experienced by CPU simulations.
The GPU approach explored in this work showed a decrease in the simulation time of optical simulations of 720-times compared to CPU simulations.
\par
In order to maximise the usefulness of the OD, the backgrounds and neutron veto performance need to understand.
The expected backgrounds have been fitted to the observed spectra, which showed an elevated rate of ${}^{210}$Po $\alpha$-decays.
This has been linked to radon-progeny plate-out inside of the OD during assembly and installation.
The neutron veto efficiency of the OD is measured in this work to be 84.6$\pm$1.2\%, which is lower than the design requirement of 95\%.
The decrease is linked to water ingress between the OD and other detectors, reducing the number of neutrons which enter the OD.
\par
Also described in this thesis are the results of Effective Field Theory (EFT) operator couplings.
The projected sensitivity of LZ to these operators in an extended region of interest, up to $\backsim$270 keV, is shown to be 5-orders of magnitude better than in previous experiments.
Analysis in a reduced region of interest ($\backsim$70 keV) on the first science run data was also performed.
Exclusion limits are set on the coupling of each EFT operator with an improvement upon previous limits of up to 4 orders of magnitude.
World-leading limits are set for all operators in at least some phase-space. 

\end{SingleSpace}
\clearpage